\subsection{Notation and Terminology}
\subsubsection{The Language of Set Theory}
% TODO predpokladame splnovani a tak, z logiky? link na ucebnici kdyztak? 
This text assumes the knowledge of basic terminology and some results from first-order predicate logic.\footnote{todo odkaz na pripadny zdroj? svejdar? neco en?}

We will now shortly review the basic notions that allow us to define the \emph{Zermelo–Fraenkel} set theory.

% theory with the Axiom of Choice ($\sf{ZFC}$), a first-order set theory in the language $\mathscr{L} = \{=, \in\}$, which will be sometimes referred to as \emph{the language of set theory}. In Chapter 3, the Axiom of Choice is assumed. When tlaking about higher-order logic, we will usually denote type 1 variables, which are elements of the domain of discourse, by lowercase letters, mostly $u, v, w, x, y, z, p_1, p_2, p_3,  \ldots$ while type 2 variables, which represent $n$-ary relations of the domain of discourse for any natural number $n$, are usually denoted by uppercase letters $A, B, C, X, Y, Z$. Note that there are exception to convention rules as $f$ usually denotes a function, which is in fact a type 2 variable. On the other hand, $M$ stands for a set.


When we talk about \emph{class}, we have the notion of definable class in mind. 
If $\varphi(x, p_1, \ldots, p_n)$ is a formula in the language of set theory, we call 
\begin{equation}
A = \{x : \varphi(x)\}
\end{equation}
a class of all sets satisfying $\varphi(x)$ in a sense that 
\begin{equation}
x \in A \iff \varphi(x)
\end{equation}
Given classes $A$, $B$, one can easily define the elementary set operations such as $A \cap B$, $A \cup B$, $A \setminus C$, $\bigcup A$, see the first part of \cite{JechBook} for details.
Axioms are the tools by which we can decide whether a particular class is ``small enough'' to be considered a set\footnote{``Small enough'' means that it doesn't introduce a paradox similar to Russell's.}. A class that fails to be considered a set is called a \emph{proper class}.

We will often write ``$M$ is a limit ordinal'', it should always be clear that this can be rewritten as a formula that was introduced earlier.

\


\subsubsection{The Axioms}

\begin{definition}{(The Existence of a Set)}\label{def:existence_of_a_set}
\begin{equation}
\exists x (x = x)
\end{equation}
\end{definition}
% The above axiom is usually omitted because it can be deduced from the axiom of \emph{Infinity} (see below), but since we will be using set theories that omit \emph{Infinity}, this will be useful.

\begin{definition}{(Axiom of Extensionality)}\label{def:extensionality}
\begin{equation}
\forall x, y (x = y \iff \forall z (z \in x \iff z \in y)) %ok
\end{equation}
\end{definition}

\begin{definition}{(Axiom Schema of Specification)}\label{def:specification}\\
The following yields an axiom for every first-order formula $\varphi(x, p_1, \ldots, p_n)$ with no free variables other than $x, p_1, \ldots, p_n$.
\begin{equation}
\forall x, p_1, \ldots, p_n \exists y \forall z ( z \in y \iff z \in x \et \varphi(z, p_1, \ldots, p_n)) %ok
\end{equation}
\end{definition}

We will now provide two definitions that are not axioms, but will be helpful in establishing some axioms in a more comprehensible way.
\begin{definition}{($x \subseteq y$, $x \subset y$)}\label{def:subset}
\begin{equation}
x \subseteq y \iff (\forall z \in x) z \in y
\end{equation}
\begin{equation}
x \subset y \iff x \subseteq y \et x \neq y
\end{equation}
We read $x \subseteq y$ as \emph{x is a subset of y} and $x \subset y$ as \emph{x is a proper subset of y}.
\end{definition}

\begin{definition}{(Empty Set)}\label{def:emptyset}
%Let $\varphi = \neg(x = x)$, $y$ is an arbitrary set, we there exists at least one set $y$ from \ref{def:existence_of_a_set} or \emph{Infinity}
%\begin{equation}
%\emptyset \defeq \{x : x \in y\ \et \varphi(x)\}
%\end{equation}
%We know that $\emptyset$ is a set from \emph{specification} and it is the same set for every $y$ given from \emph{extensionality}.
%TODO
%co radsi ``for an arbitrary $x$''
For an arbitrary set $x$, the empty set, represented by the symbol ``$\emptyset$'', is defined by the following formula:
\beq
(\forall y \in x)(y \in \emptyset \iff \neg(y = y))
\eeq
$\emptyset$ is a set due to \emph{Specification}. While the empty set could also be defined by the formula $\forall y (y \in \empty \iff \neg(y = y))$, the version we use is $\Delta_0$, which we will find useful later. The two definitions yield the same set for every $x$ given because of \emph{Extensionality}.
\end{definition}

\begin{definition}{(Axiom of Pairing)}\label{def:pairing}
\begin{equation}
\forall x, y \exists z \forall q (q \in z \iff q = x \lor q = y) % ok
\end{equation}
\end{definition}

\begin{definition}{(Axiom of Union)}\label{def:union}
\begin{equation}
\forall x \exists y \forall z (z \in y \iff \exists q( z \in q \et q \in x)) % ok
\end{equation}
\end{definition}

%\begin{definition}{(Set Intersection)}\\
%\beq
%x \cap y = \{ z: z \in x \et z \in y \}
%\eeq
%\end{definition}
%
%\begin{definition}{(Set Union)}\\
%\beq
%x \cup y = \{ z: z \in x \lor z \in y \}
%\eeq
%\end{definition}

Now we can introduce more axioms.
\begin{definition}{(Axiom of Foundation)}\label{def:foundation}
\begin{equation}
\forall x (x \neq \emptyset \then (\exists y \in x) (x \cap y = \emptyset)) %ok
\end{equation}
\end{definition}

\begin{definition}{(Axiom of Powerset)}\label{def:powerset}
\begin{equation}
\forall x \exists y \forall z (z \in y \iff z \subseteq x) %ok
\end{equation}
\end{definition}

\begin{definition}{(Axiom of Infinity)}\label{def:infinity} %?
\begin{equation}
\exists x (\emptyset \in x \et (\forall y \in x)(y\cup\{y\} \in x))
\end{equation}
The least set satisfying this is denoted ``$\omega$''.
\end{definition}

Let us introduce a few more definitions that will make the two remaining axioms more comprehensible.
\begin{definition}{(Powerset Function)}\\
Given a set $x$, the \emph{powerset of $x$}, denoted $\power{x}$ and satisfying \ref{def:powerset}, is defined as follows:
\begin{equation}
\power{x} \defeq \{y: y \subseteq x\}
\end{equation}
\end{definition}

\begin{definition}{(Function)}\label{def:function}\\
Given arbitrary first-order formula $\varphi(x, y, p_1, \ldots, p_n)$, we say that $\varphi$ is a function iff
\begin{equation}\label{def:function_formula}
\forall x, y, z, p_1, \ldots, p_n (\varphi(x, y, p_1, \ldots, p_n) \et \varphi(x, z, p_1, \ldots, p_n) \then y = z)
\end{equation}
\end{definition}
When a $\varphi(x, y)$ is a function, we also write the following:
% \footnote{This can also be done for $\varphi$s with more than two free variables by either setting $f(x, p_1, \ldots p_n) = y \iff \varphi(x, y p_1, \ldots, p_n)$ or saying that $\varphi$ represents more functions, determined by the various parameters, so given terms $t_1, \ldots, t_n$, $f(x) = y \iff \varphi(x, y, t_1, \ldots, t_n)$.}
\begin{equation}
f(x) = y \iff \varphi(x, y)
\end{equation}
Alternatively, $f = \{\langle x, y \rangle : \varphi(x, y)\}$ is a class.

% TODO hezci formulace
\begin{definition}{(Domain of a Function)}\label{def:dom}\\
Let $f$ be a function. We call the \emph{domain of $f$} the set of all sets for which $f$ yields a value. We use ``$Dom(f)$'' to refer to this set.
%We read the following as ``$Dom(f)$ is the domain of $f$''.
\begin{equation}
x \in Dom(f) \iff \exists y (f(x) = y)
\end{equation}
\end{definition}
We say ``$f$ is a function on $A$'', $A$ being a class, if $A = dom(f)$.

\begin{definition}{(Range of a Function)}\label{def:rng}\\
Let $f$ be a function. We call the \emph{range of $f$} the set of all sets that are images of other sets via $f$. We use ``$Rng(f)$'' to refer to this set. %We read the following as ``$Rng(f)$ is the range of $f$''.
\begin{equation}
x \in Rng(f) \iff \exists y (f(y) = x)
\end{equation}
\end{definition}
We say that $f$ is \emph{a function into $A$}, $A$ being a class, if $rng(f) \subseteq A$.
We say that $f$ is \emph{a function onto $A$} if $rng(f) = A$%, in other words,
%\begin{equation}
%(\forall y \in A)(\exists x \in dom(f))(f(x) = y)
%\end{equation}
We say a function $f$ is a \emph{one to one function}, iff
\begin{equation}
(\forall x_1, x_2 \in dom(f))(f(x_1) = f(x_2) \then x_1 = x_2)
\end{equation}
We say that $f$ is a bijection iff it is a one to one function that is onto.

Note that \emph{Dom(f)} and \emph{Rng(f)} are not definitions in a strict sense, they are in fact definition schemas that yield definitions for every function $f$ given. Also note that they can be easily modified for $\varphi$ instead of $f$, with the only difference being the fact that it is then defined only for those $\varphi$s that are functions, which must be taken into account. This is worth noting as we will use the notions of \emph{function} and \emph{formula} interchangably.

\begin{definition}{(Function Defined For All Ordinals)}\label{def:function_dfao}\\
We say a function $f$ is \emph{defined for all ordinals}, this is sometimes written $f: Ord \then A$ for any class $A$, if $Dom(f) = Ord$.\
Alternatively,
\begin{equation}
(\forall \alpha \in Ord)(\exists y \in A)(f(\alpha) = y))
\end{equation}
\end{definition}

And now for the axioms.

\begin{definition}{(Axiom Schema of Replacement)}\label{def:replacement}\\
The following is an axiom for every first-order formula $\varphi(x, p_1, \ldots, p_n)$ with no free variables other than $x, p_1, \ldots, p_n$.
\begin{equation}
``\varphi\mbox{ is a function}''\then \forall x \exists y \forall z (z \in y \iff (\exists q \in x)(\varphi(x, y, p_1, \ldots, p_n)))
\end{equation}
\end{definition}


% TODO mas to blbe :(
\begin{definition}{(Choice)}\label{def:choice}\\
\begin{equation}
\begin{gathered}
\forall x \exists f ((f \mbox{ is a choice function with } dom(f) = x \setminus \{\emptyset\})\\
\et \forall y ((y \in y \et y \neq \emptyset) \then f(y) \in y))
\end{gathered}
\end{equation}
\end{definition}
% Choice!!!!

We will refer to the axioms by their name, written in italic type, e.g. \emph{Foundation} refers to the Axiom of Foundation. Now we need to define the set theories to be used in the article. 
% Note that axiomatic set theory is a set of formulas.

\begin{definition}{$(\sf{S})$}\label{def:s}\\ %ok
We call $\sf{S}$ an axiomatic theory in the language $\mathscr{L} = \{=, \in\}$ with exactly the following axioms:
\bce[(i)]
\item \emph{Existence of a set} (see \ref{def:existence_of_a_set})
\item \emph{Extensionality} (see \ref{def:extensionality})
\item \emph{Specification} (see \ref{def:specification})
\item \emph{Foundation} (see \ref{def:foundation})
\item \emph{Pairing} (see \ref{def:pairing})
\item \emph{Union} (see \ref{def:union})
\item \emph{Powerset} (see \ref{def:powerset})
\ece
\end{definition}

\begin{definition}{$(\sf{ZF})$}\label{def:zf}\\ %ok
We call $\sf{ZF}$ an axiomatic theory in the language $\mathscr{L} = \{=, \in\}$ that contains all the axioms of $\sf{S}$ in addition to the following:
\bce[(i)]
\item \emph{Replacement} schema (see \ref{def:replacement})
\item \emph{Infinity} (see \ref{def:infinity})
\ece
\emph{Existence of a set} is usually left out because it is a consequence of \emph{infinity}.
\end{definition}

\begin{definition}{$(\sf{ZFC})$}\label{def:zfc}\\ % ok
$\sf{ZFC}$ is an axiomatic theory in the language $\mathscr{L} = \{=, \in\}$ that contains all the axioms of $\sf{ZF}$ plus \emph{Choice} (\ref{def:choice}).
\end{definition}

\

\subsubsection{The Transitive Universe}
\begin{definition}{(Transitive Class)}\label{def:transitivity}\\ % ok
We say a class $A$ is \emph{transitive} iff
\begin{equation}
(\forall x \in A)(x \subseteq A)
\end{equation}
\end{definition}

\begin{definition}{(Well Ordered Class)}\label{def:well_ordering} % ok
A class $A$ is said to be \emph{well ordered by $\in$} iff the following hold:
\bce[(i)]
\item $(\forall x \in A)(x \not\in x)$ (Antireflexivity)
\item $(\forall x, y, z \in A)(x \in y \et y \in z \then x \in z)$ (Transitivity)
\item $(\forall x, y \in A)(x = y \lor x \in y \lor y \in x)$ (Linearity)
\item $(\forall x \subseteq A)(x \neq \emptyset \then (\exists y \in x)(\forall z \in x)(z = y \lor z \in y)))$ (Existence of the least element)
\ece
\end{definition}

\begin{definition}{(Ordinal Number)}\label{def:ordinal}\\ % ok
A set $x$ is said to be an \emph{ordinal number} if it is \emph{transitive} and \emph{well-ordered by $\in$}. 
\end{definition}
For the sake of brevity, we usually just say ``$x$~is an \emph{ordinal}''. 
Note that ``$x$~is an ordinal'' is a well-defined formula in the language of set theory, since \ref{def:transitivity} is a first-order formula and \ref{def:well_ordering} is in fact a conjunction of four first-order formulas.
Ordinals will be usually denoted by lower case greek letters, starting from the beginning of the alphabet: $\alpha, \beta, \gamma, \ldots$.
Given two different ordinals $\alpha, \beta$, we will write $\alpha < \beta$ for $\alpha \in \beta$, see Lemma 2.11 in \cite{JechBook} for technical details.

\begin{definition}{(Non-Zero Ordinal)} % ok 
We say an ordinal $\alpha$ is \emph{non-zero} iff $\alpha \neq \emptyset$.
\end{definition}

\begin{definition}{(Successor Ordinal)}\label{def:successor_ordinal}\\ % ok 
Consider the following function defined for all ordinals. Let $\beta$ be an arbitrary ordinal. We call $S$ the \emph{successor function}.
\begin{equation}
S(\beta) = \beta \cup \{\beta\}
\end{equation}
An ordinal $\alpha$ is called a \emph{successor ordinal} iff there is an ordinal $\beta$, such that $\alpha = S(\beta)$. We also write $\alpha = \beta+1$.
\end{definition}

\begin{definition}{(Limit Ordinal)}\label{def:limit_ordinal}\\ % ok
A non-zero ordinal $\alpha$ is called a \emph{limit ordinal} iff it is not a successor ordinal.
\end{definition}

\begin{definition}{(Ord)}\label{def:ord}\\  % ok
\emph{The class of all ordinal numbers}, which we will denote ``$Ord$''\footnote{Other authors use ``$On$'', we will stick to the notation used in \cite{JechBook}} is the proper class defined as follows.
\begin{equation}
x \in Ord \iff x\mbox{ is an ordinal}
\end{equation}
\end{definition}

%The following construction will be often referred to as the \emph{Von Neumann's Hierarchy}, sometimes also the \emph{Von Neumann's Universe}. 
%, the former referring more to the construction with the individual levels in mind, the latter referring more to the class $V$, but they can be interchanged with no confusion caused.

\begin{definition}{(Von Neumann's Hierarchy)}\label{def:von_neumann}\\ % ok 
The \emph{Von Neumann's Hierarchy} is a collection of sets indexed by elements of $Ord$, defined recursively in the following way:
\bce[(i)]
\item 
\begin{equation}
V_0 = \emptyset
\end{equation}
\item 
\begin{equation}
V_{\alpha+1} = \power{V_\alpha}\mbox{ for any ordinal $\alpha$}
\end{equation}
\item
\begin{equation} 
V_\lambda = \bigcup_{\beta < \lambda} V_\beta \mbox{ for a limit ordinal $\lambda$}
\end{equation}
\ece
We will also refer to the \emph{Von Neumann's Hierarchy} as \emph{Von Neumann's Universe} or the \emph{Cumulative Hierarchy}. % velky nebo maly c?
\end{definition}

\begin{definition}{(Rank)}\label{def:rank}\\ % ok
Given a set $x$, we say that the rank of $x$ (written as $rank(x)$) is the least ordinal $\alpha$ such that $x \in V_{\alpha+1}$
\end{definition}
Due to \emph{Regularity}, every set has a rank.\footnote{See chapter 6 of \cite{JechBook} for details.}

%\begin{definition}{($\omega$)}\label{def:omega}\\ % ???? bacha na to
%We call $\omega$
%\begin{equation}
%\omega \defeq \bigcap\{x : \mbox{``$x$ is a limit ordinal''})\}
%\end{equation}
%\end{definition}
%$\omega$ is non-empty if \emph{Infinity} or any equivalent holds.

\begin{definition}{(Lévy's Hierarchy)}\\
!!! pozor na konflikt s analytickou (vyres podle kanamoriho)
TODO (potrebujeme ji?)
\end{definition}
% TODO levyho hierarchie?
\

\subsubsection{Cardinal Numbers}

% todo def one.to-one mapping?
\begin{definition}{(Cardinality)}\\
Given a set $x$, let the cardinality of $x$, written $|x|$, be defined as the smallest ordinal number such that there is a one to one mapping from $x$ to $\alpha$.
\end{definition}

\begin{definition}{(Aleph function)}\label{def:aleph}\\
Let $\omega$ be the set defined by \ref{def:omega}.
We will recursively define the function $\aleph$ for all ordinals.
\bce[(i)]
\item $\aleph_0 = \omega$
\item $\aleph_{\alpha+1}$ is the least cardinal larger than $\aleph_\alpha$\footnote{``The least cardinal larger than $\aleph_\alpha$'' is sometimes notated as $\aleph_\alpha^{+}$}
\item $\aleph_\lambda = \bigcup_{\beta < \lambda}\aleph_\beta$ for a limit ordinal $\lambda$
\ece
If $\kappa = \aleph_\alpha$ and $\alpha$ is a successor ordinal, we call $\kappa$ a \emph{successor cardinal}. If $\alpha$ is a limit ordinal, we call $\kappa$ a \emph{limit cardinal}.
\end{definition} % mam def. succ ordinaly?

\begin{definition}{(Cardinal number)}\label{def:cardinal}\\
\bce[(i)]
\item A set $x$ is called a \emph{finite cardinal} iff $x \in \omega$.
\item A set is called an \emph{infinite cardinal} iff there is an ordinal $\alpha$ such that $\aleph_\alpha = x$
\item A set is called a \emph{cardinal} iff it is either a \emph{finite cardinal} or an \emph{infinite cardinal}.
\ece
\end{definition}
We say $\kappa$ is an uncountable cardinal iff it is an infinite ordinal and $\aleph_0 < \kappa$.
Infinite cardinals will be notated by lowercase greek letters from the middle of the alphabet, e.g. $\kappa, \mu, \nu, \ldots$.\footnote{Except $\lambda$ which is preferably used for limit ordinals.}

For formal details as well as why every set can be well-ordered assuming \emph{Choice}, and therefore has a cardinality, see \cite{JechBook}. % proc je to tady?

% todo mame def sequence? !! ================ !!!!
% todo mame def ``cofinal''?
% TODO fix it!!
% todo pozor na supremum, upravit! (ikdyz pro limitni je to stejny)
\begin{definition}{(Cofinality of a Limit Ordinal)}\label{def:cofinality}\\ % a co https://math.berkeley.edu/~jhicks/links/SOTS/cskipper112613.pdf? 
Let $\lambda$ be a limit ordinal. We say that the \emph{cofinality} of $\lambda$ is $\alpha$ iff $\alpha$ is the smallest limit ordinal, such that there is an $\alpha$-sequence $\langle \beta_\xi : \xi < \alpha \rangle$, such that  % todo co funkce z alpha do A?
\begin{equation}
%(\forall x \in \lambda)(\exists y \in \alpha)(x < y) % !!!!!!!!!
sup(\beta_\xi: \xi < \alpha) = \lambda
\end{equation}
We writte $cf(\lambda) = \alpha$.
\end{definition}

\begin{definition}{(Regular Cardinal)}\label{def:regular_cardinal}\\
We say a cardinal $\kappa$ is regular iff $cf(\kappa) = \kappa$
\end{definition}

\begin{definition}{(Limit Cardinal)}\label{def:limit_cardinal}\\
We say that a cardinal $\kappa$ is a \emph{limit cardinal} iff there is a limit ordinal $\lambda$ such that $\kappa = \aleph_\lambda$ % neni in ord redundantni, protoze je V_x def jenom pro ordinalni x?
\end{definition}

\begin{definition}{(Strong Limit Cardinal)}\label{def:strong_limit_cardinal}\\
We say that an ordinal $\kappa$ is a \emph{strong limit cardinal} if it is a \emph{limit cardinal} and 
\begin{equation}
(\forall \alpha \in \kappa)(\power{\alpha} \in \kappa)
\end{equation}
\end{definition}

\begin{definition}{(Generalised Continuum Hypothesis)}\label{def:gch}\\
\begin{equation}
\aleph_{\alpha+1}=\power{\aleph_\alpha}
\end{equation}
\end{definition}
If \emph{GCH} holds (for example in Gödel's $L$, see chapter 3), the notions of limit cardinal and strong limit cardinal are equivalent.

\

\subsubsection{Relativisation and Absoluteness}
\begin{definition}{(Relativization)}\label{def:relativization}\\
Let $M$ be a class, $R \subseteq M\times M$ and let $\varphi(p_1, \ldots, p_n)$ be a first-order formula with no free variables besides $p_1, \ldots, _n$. 
The \emph{relativization of $\varphi$ to $M$ and $R$} is the formula, written as $\varphi^{M, R}(p_1, \ldots, p_n)$, defined in the following inductive manner:
\bce[(i)]
\item $(x \in y)^{M,R} \iff R(x, y)$
\item $(x = y)^{M,R} \iff x = y$
\item $(\neg \varphi)^{M,R} \iff \neg \varphi^{M,R}$
\item $(\varphi \et \psi)^{M,R} \iff \varphi^{M,R} \et \psi^{M,R}$
\item $(\varphi \lor \psi)^{M,R} \iff \varphi^{M,R} \lor \psi^{M,R}$
\item $(\varphi \then \psi)^{M,R} \iff \varphi^{M,R} \then \psi^{M,R}$
\item $(\exists x \varphi(x))^{M,R} \iff (\exists x \in M) \varphi^{M,R}(x)$
\item $(\forall x \varphi(x))^{M,R} \iff (\forall x \in M) \varphi^{M,R}(x)$
\ece
\end{definition}
When $R=\in\cap(M \times M)$, we usually write $\varphi^M$ instead of $\varphi^{M, R}$. When we talk about $\varphi^M(p_1, \ldots, p_n)$, it is understood that $p_1, \ldots, p_n \in M$.
We will also use $M \models \varphi(p_1, \ldots, p_n)$ and $\varphi^M(p_1, \ldots, p_n)$ interchangably.
% TODO definice splnovani!

\begin{definition}{(Absoluteness)}
Given a transitive class $M$, we say a formula $\varphi$ is \emph{absolute in $M$} if for all $p_1, \ldots, p_n \in M$
\begin{equation}
\varphi^M(p_1, \ldots, p_n) \iff \varphi(p_1, \ldots, p_n)
\end{equation}
\end{definition}

\begin{definition}{(Hierarchy of First-Order Formulas)}\\
\bce[(I)]
\item A first-order formula $\varphi$ is $\Delta_0$ iff it is logically equivalent to a first-order formula $\varphi'$ satisfying any of the following:
\bce[(i)]
\item $\varphi'$ contains no quantifiers
\item $y$ is a set, $\psi$ is a $\Delta_0$ formula, and $\varphi'$ is either $(\exists x \in y)\psi(y)$ or $(\forall x \in y)\psi(y)$.
\item $\psi_1, \psi_2$ are $\Delta_0$ formulas and $\varphi'$ is any of the following: $\psi_1 \lor \psi_2$, $\psi_1 \et \psi_2$, $\psi_1 \then \psi_2$, $\neg \psi_2$, 
\ece
\item If a formula is $\Delta_0$ it is also $\Sigma_0$ and $\Pi_0$
\item A formula $\varphi$ is $\Pi_n+1$ if it is logically equivalent to a formula $\varphi'$ such that $\varphi' = \forall x \psi$ where $\psi$ is a $\Sigma_n$-formula for any $n < \omega$.
\item A formula $\varphi$ is $\Sigma_n+1$ if it is logically equivalent to a formula $\varphi'$ such that $\varphi' = \forall x \psi$ where $\psi$ is a $\Pi_n$-formula for any $n < \omega$.
\ece
\end{definition}
Note that we can use the pairing function so that for $\forall p_1, \ldots, p_n \psi(p_1, \ldots, p_n)$, there is a logically equivalent formula of the form $\forall x \psi'(x)$.

\begin{lemma}{($\Delta_0$ absoluteness)}\label{lemma:delta_0_absoluteness}
Let $\varphi$ be a $\Delta_0$ formula, then $\varphi$ is absolute in any transitive class $M$.
\end{lemma}

\begin{proof}
This will be proven by induction over the complexity of a given $\Delta_0$ formula $\varphi$. Let $M$ be an arbitrary transitive class. 

Atomic formulas are always absolute by the definition of relativisation, see \ref{def:relativization}.
Suppose that $\Delta_0$ formulas $\psi_1$ and $\psi_2$ are absolute in $M$. Then from relativization, $(\psi_1 \et \psi_2)^M \iff \psi_1^M \et \psi_2^M$, which is, from the induction hypothesis, equivalent to $\psi_1 \et \psi_2$. The same holds for $\lor, \then, \neg$.

Suppose that a $\Delta_0$ formula $\psi$ is absolute in $M$. Let $y$ be a set and let $\varphi = (\exists x \in y) \psi(x)$. 
From relativization, $(\exists x \psi(x))^M \iff (\exists x \in M) \psi^M(x)$. Since the hypotheses makes it clear that $\psi^M \iff \psi$, we get $((\exists x \in y) \psi(x))^M \iff (\exists x \in y\cap M) \psi(x)$, which is the equivalent of $\varphi^M \iff \varphi$. The same applies to $\varphi = (\forall x \in y) \psi(x)$.
\end{proof}

% todo co Devlin -- p.27 -- downward a upward absoluteness?
\begin{lemma}{(Downward Absoluteness)}\label{lemma:downward_absoluteness}\\
Let $\varphi$ be a $\Pi_1$ formula and $M$ a transitive class. Then the following holds:
\begin{equation}
(\forall p_1, \ldots, p_n \in M)(\varphi(p_1, \ldots, p_n) \then \varphi(p_1, \ldots, p_n)^M)
\end{equation}
\end{lemma}
\begin{proof}
Since $\varphi(p_1, \ldots, p_n)$ is $\Pi_1$, there is a $\Delta_0$ formula $\psi(p_1, \ldots, p_n, x)$ such that $\varphi = \forall x \psi(p_1, \ldots, p_n, x)$. From relativization and lemma \ref{lemma:delta_0_absoluteness}, $\varphi^M(p_1, \ldots, p_n) \iff (\forall x \in M)\psi(p_1, \ldots, p_n, x)$.

Assume that for $p_1, \ldots, p_n \in M$ fixed, that $\forall x \psi(p_1, \ldots, p_n, x)$ holds, but $(\forall x \in M)\psi(p_1, \ldots, p_n, x)$ does not. 
Therefore $\exists x \neg \psi(p_1, \ldots, p_n, x)$, which contradicts $\forall x \psi(p_1, \ldots, p_n, x)$.
\end{proof}


\begin{lemma}{(Upward Absoluteness)}\label{lemma:upward_absoluteness}\\
Let $\varphi$ be a $\Sigma_1$ formula and $M$ a transitive class. Then the following holds:
\begin{equation}
(\forall p_1, \ldots, p_n \in M)(\varphi^M(p_1, \ldots, p_n) \then \varphi(p_1, \ldots, p_n))
\end{equation}
\end{lemma}
\begin{proof}
Since $\varphi(p_1, \ldots, p_n)$ is $\Sigma_1$, there is a $\Delta_0$ formula $\psi(p_1, \ldots, p_n, x)$ such that $\varphi = \exists x \psi(p_1, \ldots, p_n, x)$. From relativization and lemma \ref{lemma:delta_0_absoluteness}, $\varphi^M(p_1, \ldots, p_n) \iff (\exists x \in M)\psi(p_1, \ldots, p_n, x)$.

Assume that for $p_1, \ldots, p_n \in M$ fixed, that $(\exists x \in M)\psi(p_1, \ldots, p_n, x)$ holds, but $\exists x \psi(p_1, \ldots, p_n, x)$ does not. This is an obvious contradiction.
\end{proof}


\subsubsection{More Functions}

\begin{definition}{(Strictly Increasing Function)}\label{def:increasing_function}\\
A function $f: Ord \then Ord$ is said to be \emph{strictly increasing} iff
\begin{equation}
\forall \alpha, \beta \in Ord (\alpha < \beta \then f(\alpha) < f(\beta)).
\end{equation}
\end{definition}

\begin{definition}{(Continuous Function)}\label{def:continuous_function}\\
A function $f: Ord \then Ord$ is said to be \emph{continuous} iff
\begin{equation}
\lambda\mbox{ is limit} \then f(\lambda) = \bigcup_{\alpha < \lambda} f(\alpha).
\end{equation}
\end{definition}

\begin{definition}{(Normal Function)}\label{def:normal_function}\\
A function $f: Ord \then Ord$ is said to be \emph{normal} iff it is \emph{strictly increasing} and \emph{continuous}.
\end{definition}

\begin{definition}{(Fixed Point)}\label{def:fixed_point}\\
We say $x$ is a fixed point of a function $f$ iff $x=f(x)$.
\end{definition}

\begin{definition}{(Unbounded Class)}\label{def:unbounded_class}\\
We say a class $A$ is unbounded iff
\begin{equation}
\forall x (\exists y \in A) (x < y)
\end{equation}
\end{definition}

\begin{definition}{(Limit Point)}\label{def:limit_point}\\
Given a class $x \subseteq Ord$, we say that $\alpha \neq \emptyset$ is a limit point of $x$ iff 
\begin{equation}
\alpha = \bigcup(x \cap \alpha)
\end{equation}
\end{definition}

\begin{definition}{(Closed Class)}\label{def:closed_class}\\
We say a class $A \subseteq Ord$ is closed iff it contains all of its limit points.
\end{definition}

\begin{definition}{(Club set)}\label{def:club_set}\\
For a regular uncountable cardinal $\kappa$, a set $x \subset \kappa$ is a \emph{closed unbounded} subset, abbreviated as a \emph{club set}, iff $x$ is both closed and unbounded in $\kappa$.
\end{definition}

\begin{definition}{(Stationary set)}\label{def:stationary_set}\\
For a regular uncountable cardinal $\kappa$, we say a set $A \subset \kappa$ is stationary in $\kappa$ iff it intersects every club subset of $\kappa$.
\end{definition}

\subsubsection{Structure, Substructure and Embedding}

Structures will be denoted $\langle M, \in, R \rangle$ where $M$ is a domain, $\in$ stands for the standard membership relation, it is assumed to be restricted to the domain\footnote{To be totally explicit, we should write $\langle M, \in \cap M \times M, R \rangle$}, $R \subseteq M$ is a relation on the domain. When $R$ is not needed, we can as well only write $M$ instead of $\langle M, \in \rangle$.

\begin{definition}{(Elementary Embedding)}\label{def:elementary_embedding}\\
Given the structures $\langle M_0, \in, R \rangle$, $\langle M_1, \in, R \rangle$ and a one-to-one function $j: M_0 \then M_1$, we say $j$ is an \emph{elementary embedding} of $M_0$ into $M_1$, we write $j: M_0 \prec M_1$, when the following holds for every formula $\varphi(p_1, \ldots, p_n)$ and every $p_1, \ldots, p_n \in M_0$:
\begin{equation}
\langle M_0, \in, R \rangle \models \varphi(p_1, \ldots, p_n) \iff \langle M_1, \in, R \rangle  \models \varphi(j(p_1), \ldots, j(p_n))
\end{equation}
\end{definition}


\begin{definition}{(Elementary Substructure)}\label{def:elementary_substructure}\\
Given the structures $\langle M_0, \in, R \rangle$, $\langle M_1, \in, R \rangle$ and a one-to-one function $j: M_0 \then M_1$ such that $j: M_0 \prec M_1$, we say that $M_0$ is an \emph{elementary substructure} of $M_1$, denoted as $M_0 \prec M_1$, iff $j$ is an identity on $M_0$. In other words
\begin{equation}
\langle M_0, \in, R \rangle \models \varphi(p_1, \ldots, p_n) \iff \langle M_1, \in, R \rangle  \models \varphi(p_1, \ldots, p_n)
\end{equation}
for $p_1, \ldots, p_n \in M_0$
\end{definition}

% While higher-order satisfaction relation for proper classes is unformalizable\footnote{TODO CITE KDE? Tarski nebo tak neco?},we can formalize satisfaction in a structure. For the rest of this chapter, let $D$ be a domain of such structure.

