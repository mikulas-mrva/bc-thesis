\bce
\item $\emph{Reflection}$ je obecne reflexe (jaka presne?)
\item $\emph{Reflection}_1$ je reflexe prvoradovych formuli TODO presna formulace!
\item etc...
\ece

\

$V$ a $V_\alpha$ odkazuji k Von Neumannove hierarchii (pro jistotu)

\

zakladni definice

\begin{definition}{(Ord)}\\
TODO Ord je trida vsech ordinalu
\end{definition}

\begin{definition}{(Function)}\label{def:function}\\
We say that a first-order formula $\varphi(x, y, u_1, \ldots, u_n)$ with no free variable besides $x, y, u_1, \ldots, u_n$ is a \emph{function} iff
\begin{equation}
\forall x, y, z (\varphi(x, y, u_1, \ldots, u_n)\ \&\ \varphi(x, z, u_1, \ldots, u_n) \then y=z)
\end{equation}
\end{definition}

TODO kofinalita
\begin{definition}{(Cofinality)}\label{def:extensionality}\\
Let $\kappa$ be a cardinal. The {cofinality} of $\kappa$, written as $cf(\kappa)$ is defined as follows
\begin{equation}
TODO
\end{equation}
\end{definition}

\begin{definition}{(Limit ordinal)}\label{def:limit_ordinal}\\
We say that an ordinal $\alpha$ is a \emph{limit ordinal} iff
\begin{equation}
\alpha \in Ord\ \&\ \exists x (x\in\alpha)\ \&\ \forall x(x \in \alpha \then x+1 \in \alpha)
\end{equation}
\end{definition}

TODO def $\aleph_\alpha$ ?

\begin{definition}{(Limit Cardinal)}\label{def:limit_ordinal}\\
We say that a cardinal $\kappa$ is a \emph{limit cardinal} iff
\begin{equation}
\exists \alpha (\alpha \in Ord\ \&\ \kappa = \aleph_\alpha)
\end{equation}
\end{definition}

\begin{definition}{(Strong Limit Cardinal)}\label{def:limit_ordinal}\\
We say that an ordinal $\kappa$ is a \emph{strong limit cardinal} iff it is a \emph{limit cardinal} and 
\begin{equation}
\forall \alpha (\alpha \in \kappa \then \power(\alpha) \in \kappa)
\end{equation}
\end{definition}
\

Vypsat axiomy ZFC a jake formulace pouzivam\\
\emph{Replacement}, $Replacement_2$ atd
\emph{Subsets}

\begin{definition}{(\emph{Extensionality})}\label{def:extensionality}\\
\begin{equation}
\emph{Extensionality} \iff \forall x, y(\forall z (z \in x \iff z \in y) \then x = y)
\end{equation}
\end{definition}

\begin{definition}{(\emph{Foundation})}\label{def:foundation}\\
\begin{equation}
\emph{Foundation} \iff \forall x (\exists z (z \in x) \then \exists z (z \in x\ \&\ \forall u \neg (u \in z\ \&\ u \in x)))
\end{equation}
\end{definition}

\begin{definition}{(\emph{Pairing})}\label{def:pairing}\\
\begin{equation}
\emph{Pairing} \iff \forall x, y \exists z (x \in z\ \&\ y \in z)
\end{equation}
\end{definition}

\begin{definition}{(\emph{Union})}\label{def:union}\\
\begin{equation}
\emph{Union} \iff \forall x \,\exists y \, \forall z\, \forall w ((w \in z \land z \in x) \then w \in y)
\end{equation}
\end{definition}

\begin{definition}{(\emph{Powerset})}\label{def:powerset}\\
\begin{equation}
\emph{Powerset} \iff \forall x \exists y \forall z (z \subseteq x \then z \in y)
\end{equation}
\end{definition}

\begin{definition}{(\emph{Specification})}\label{def:specification}\\
The following is a schema for every first-order formula $\varphi$.
\begin{equation}
\emph{Specification} \iff \forall x \exists y \forall z \, ( z \in y \iff ( z \in x \& \varphi(z, x)))
\end{equation}
\end{definition}

\begin{definition}{(\emph{Infinity})}\label{def:infinity}\\
\begin{equation}
\emph{Infinity} \iff \exists x (\exists y (y \in x)\ \&\ \forall y(y \in x \then y+1 \in x))
\end{equation}
\end{definition}

\begin{definition}{(\emph{Replacement})}\label{def:replacement}\\
The following is a schema for every first-order formula $\varphi$.
\begin{equation}
\begin{gathered}
\emph{Replacement} \iff\\
\forall x, y, z (\varphi(x, y) \& \varphi(x, z) \then y\ =\ z) \then \\
\then (\forall x \exists y \forall z (z \in y \iff \exists w \in x (\varphi(w, z))))
\end{gathered}
\end{equation}
\end{definition}

\begin{definition}{(\emph{Choice})}\label{def:choice}\\
\begin{equation}
\emph{Choice} \iff TODO
\end{equation}
\end{definition}

\

\begin{definition}{$(\sf{S})$}\label{def:s}\\
TODO
\end{definition}

\begin{definition}{$(\sf{ZF})$}\label{def:zf}\\
TODO
\end{definition}

\begin{definition}{$(\sf{ZFC})$}\label{def:zfc}\\
TODO
\end{definition}

\begin{definition}{$(\sf{ZFC\textsubscript{2}})$}\label{def:zfc_2}\\
TODO
\end{definition}

\

TODO definice druhoradoveho splnovani

TODO analyticka (ne aritmeticka!) hierarchie

\begin{definition}{(\emph{Reflection\textsubscript{1}})}\label{def:reflection_1}\\
\begin{equation}
ASD
\end{equation}
\end{definition}
\

Asi vsechno budeme delat v ZFC, nic bychom neziskali, pokud ne.
