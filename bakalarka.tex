\documentclass[12pt,a4paper]{article}
% konvence:
% ``''
% (\ref)
% (i) ((ani emph ani bold)
% tuple $\langle x, y \rangle$
% Ord
% {x : \varphi(x)}

% TODOS:
% nbsp ~ !! s/$x$ /$x$~/gi !!
% definice splnovani do uvodu!!
% projet jeste definice
% projet dukaz axiomu ZF v 2
% pozor na ruzne definice delta_0, zkontroluj jestli to pouzivas jako Jech (jestli to rikame o te fli nebo tim myslime ze je ekvivalentni s nejakou takovou)
% kofinalita je spatne!!
% \emph{Infinity} uppercase first!
% supremum! PODLE JECHA
% napsat nekam ze V je vzdycky sjednoceni kumulativni hierarchie a ne proste univerzalni trida!!
% Zmenily se citace, zkontrolovat ze nepisu Jech [Jech]
% TODO Reclection_1 vs. Reflection --- spravne je \emph{First-order reflection}!!!
% todo def axiomatic set theory jako (RS?) mnozinu formulí v jazyce teorie mnozit, tedy \{\in, =\}
% zmenil se replacement, v Levyho stare verzi opraveno, zkontrolovat v restatementu

% MAHLO:
% viz ``Interestingly enough, Gaifman [67] showed that in a concrete sense a weakly Mahlo cardinal is the least upper bound of diagonalizing limit processes from below'' (kanamori)

\usepackage{mathrsfs}
\usepackage{amssymb}
\usepackage{amsmath}
\usepackage{amsfonts}
\usepackage{longtable}
\usepackage{paralist}
\usepackage{lineno}
\usepackage{verbatim}
\linenumbers

% \usepackage{mathrsfs}
\usepackage{amssymb}
\usepackage{amsmath}
\usepackage{amsfonts}
\usepackage{longtable}
\usepackage{paralist}
\usepackage{lineno}
\usepackage{verbatim}
\linenumbers

% \usepackage{mathrsfs}
\usepackage{amssymb}
\usepackage{amsmath}
\usepackage{amsfonts}
\usepackage{longtable}
\usepackage{paralist}
\usepackage{lineno}
\usepackage{verbatim}
\linenumbers

% \include{00_headers.tex}
\usepackage{color} %pro barevné odkazy, příp. nadpisy
\definecolor{odkazy}{rgb}{0.21,0.27,0.53} %tmavì modrá
\definecolor{nadpisy}{rgb}{0.5812,0.0665,0.0659} %cihlová
%
% Parametry prevodu do pdf
\providecommand{\hypersetup}[1]{}%
\hypersetup{%
unicode,% ? Pravdepodobne bezvyznamne
pdfauthor={Mikuláš Mrva},
pdftitle={Reflection principles and large cardinals},
pdfsubject={Reflection principles and large cardinals},
pdfkeywords={set theory, large cardinals, reflection principle, ZFC, Azriel Lévy},
pdffitwindow=false,% Inicialni umisteni textu v okne Readeru
bookmarksopen=true,% Panel zalozek inicialne zobrazen
% Je-li tohle nastaveno jinak, nektere odkazy nekdy nefunguji
hypertexnames=false,
plainpages=false,
%pdfpagelabels,
%
breaklinks=true,% Radkovy lom smi prijit do klikatelneho odkazu
linkcolor=odkazy,% Graficka podoba odkazu
citecolor=odkazy,% ...
colorlinks=true,% ...
pdfhighlight=/O% ... (vzhled odkazu pri stisknuti)
}%
% Inputenc je asi zbytecne.
% Option 'split' ovlivnuje deleni slov obsahujicich v sobe rozdelovnik
\usepackage[utf8x]{inputenc} % UTF-8 ?
%\usepackage[czech]{babel} %dnes už je však hotová integrace èeštiny do babelu
%\usepackage[split]{czech} %dnes už je však hotová integrace èeštiny do babelu
%
%\usepackage{logdp} %užiteèné drobnosti
%\usepackage{amsthm} %lepšší práce s větami
%\usepackage{amsmath} %nová prostøedí pro matematiku a vylepšení tìch stávajících
%\usepackage{latexsym,amsfonts,amssymb} % nová písmenka
\usepackage{fancyhdr} % zápatí a záhlaví
%\usepackage[nottoc]{tocbibind} % přidá do obsahu položky Literatura a Rejstřík
\usepackage{csquotes}
\pagestyle{plain}
%pøedbìžné nastavení hlavièky (balík fancyhdr)
%\headheight 13.6pt %možná ji bude tøeba zvednout, fancyhdr si pak stìžuje: \headheight
% too small, make it at least Xpt
\headheight 14.5pt %možná ji bude tøeba zvednout, fancyhdr si pak stìžuje: \headheight too \fancyhead{}
\fancyhead[R]{\leftmark}
\fancyfoot{}
\fancyfoot[C]{\thepage}


\newtheorem{theorem}{Theorem}[section]
\newtheorem{Claim}[theorem]{Claim}
\newtheorem{definition}[theorem]{Definition}
\newtheorem{Cor}[theorem]{Corollary}
\newtheorem{Fact}[theorem]{Fact}
\newtheorem{lemma}[theorem]{Lemma}
\newtheorem{sublemma}[theorem]{Sublemma}
\newtheorem{ex}[theorem]{Example}
\newtheorem{remark}[theorem]{Remark}
\newtheorem{obs}[theorem]{Observation}
\newtheorem{que}[theorem]{Question}
\newtheorem{conjecture}[theorem]{Conjecture}

\renewcommand{\theequation}{\thesection.\arabic{equation}}

\newenvironment{proof}
{\noindent \textit{Proof.}}
{\hspace*{\fill} $\Box$}

\newcommand{\toch}{\fbox{\small {\bf ??}}}
\newcommand{\bt}[1]{{\underset{\widetilde{}}{#1}}}
\newcommand{\trcl}[1]{\ensuremath{\mathrm{trcl}(\{#1\})}}
\newcommand{\cf}[1]{\ensuremath{\mathrm{cf}(#1)}}
\newcommand{\cl}[1]{\ensuremath{\mathrm{cl}}(#1)}
\newcommand{\ord}[1]{\ensuremath{\mathrm{ORD}}(#1)}
\newcommand{\dom}[1]{\ensuremath{\mathrm{dom}}(#1)}
\newcommand{\rng}[1]{\ensuremath{\mathrm{rng}}(#1)}
\newcommand{\power}[1]{\ensuremath{\mathscr{P}} (#1)}
\newcommand{\set}[2]{\ensuremath{\{#1 \,:\, #2 \}}}
\newcommand{\seq}[2]{\ensuremath{\langle #1 \,:\, #2 \rangle}}
\newcommand{\singl}[1]{\ensuremath{\{#1\}}}
\newcommand{\pair}[2]{\ensuremath{\{ #1, #2 \}}}
\newcommand{\restr}[2]{\ensuremath{#1 \! \upharpoonright \! #2}}
\renewcommand{\iff}{\leftrightarrow}
\newcommand{\Iff}{\Leftrightarrow}
\newcommand{\el}{\prec}
\newcommand{\iso}{\cong}
\newcommand{\sub}{\subseteq}
\newcommand{\super}{\supseteq}
\newcommand{\la}{\langle}
\newcommand{\ra}{\rangle}
\newcommand{\embed}{\rightarrow}
\newcommand{\mc}{\mathcal}
\newcommand{\supr}[1]{\mathrm{sup}\,#1}
\newcommand{\then}{\rightarrow}
\newcommand{\conc}{^{\smallfrown}}
\newcommand{\bb}{\mathbb}
\newcommand{\supp}[1]{\mathrm{supp}(#1)}
\newcommand{\beq}{\begin{equation}}
\newcommand{\eeq}{\end{equation}}
\newcommand{\brm}{\begin{remark}\begin{rm}}
\newcommand{\erm}{\end{rm}\end{remark}}
\newcommand{\mx}{\mathrm}
\newcommand{\bce}{\begin{compactenum}}
\newcommand{\ece}{\end{compactenum}}
\newcommand{\op}[2]{\la #1, #2 \ra}
\newcommand{\treq}{\trianglelefteq}
\newcommand{\et}{\mathrel{\&}}
\newcommand{\proves}{\vdash}

\newcommand\defeq{\mathrel{\overset{\makebox[0pt]{\mbox{\normalfont\tiny\sffamily def}}}{=}}}

\begin{document}
%titulní stránka
\begin{titlepage}
%\fontsize{16.16pt}{25pt}\selectfont
\Large
\begin{center}
Univerzita Karlova v~Praze, Filozofick{\/á} fakulta\\
Katedra logiky

\vspace{8.5em}
\textsc{Mikuláš Mrva}\\[1.4em]
{REFLECTION PRINCIPLES AND LARGE CARDINALS}\\
Bakalářská práce\\[6.8em]
Vedoucí práce: Mgr. Radek Honzík, Ph.D.\\[6.8em]
2016
\end{center}
\end{titlepage}\


\vspace{\fill}
\noindent 
Prohlašuji, že jsem bakalářskou práci vypracoval samostatně a~že jsem uvedl všechny použité prameny a~literaturu.

\bigskip
\noindent V~Praze 22.~května 2016\\[3em]
\hspace*{\fill}Mikuláš Mrva\hspace*{3em}
\clearpage

\begin{abstract}
\noindent Práce zkoumá vztah tzv. principů reflexe a velkých kardinálů. Lévy ukázal, že v ZFC platí tzv. věta o reflexi~a dokonce, že věta o reflexi je ekvivalentní schématu nahrazení a~axiomu nekonečna nad teorií ZFC bez axiomu nekonečna a~schématu nahrazení. Tedy lze na větu o~reflexi pohlížet jako na svého druhu axiom nekonečna. Práce zkoumá do jaké míry a~jakým způsobem lze větu o reflexi zobecnit a~jaký to má vliv na existenci tzv. velkých kardinálů. Práce definuje nedosažitelné, Mahlovy a nepopsatelné kardinály a ukáže, jak je lze zavést pomocí reflexe. Přirozenou limitou kardinálů získaných reflexí jsou kardinály nekonzistentní s L. Práce nabídne intuitivní zdůvodněn, proč tomu tak je.

\end{abstract}
\bigskip
\renewcommand{\abstractname}{Abstract}
\begin{abstract}
\noindent This thesis aims to examine relations between the so called Reflection Principles and Large cardinals. Lévy has shown that the Reflection Theorem is a sound theorem of ZF and it is equivalent to the Replacement Scheme and the Axiom of Infinity. From this point of view, Reflection theorem can be seen a~specific version of an Axiom of Infinity. This paper aims to examine the Reflection Principle and its generalisations with respect to the existence of Large Cardinals. This thesis will establish the Inaccessible, Mahlo and Indescribable cardinals and show how can those be defined via reflection. A natural limit of Large Cardinals obtained via reflection are cardinals inconsistent with L. This thesis will offer an intuitive explanation of why this holds.
\end{abstract}
\clearpage

\tableofcontents
\clearpage

% podekovani firme co vyrabi club mate -- Loscher gmbh?
\pagestyle{fancy} %detailní definice chování záhlaví
\renewcommand{\sectionmark}[1]{\markboth{\slshape\thesection.\ #1}{}}


\usepackage{color} %pro barevné odkazy, příp. nadpisy
\definecolor{odkazy}{rgb}{0.21,0.27,0.53} %tmavì modrá
\definecolor{nadpisy}{rgb}{0.5812,0.0665,0.0659} %cihlová
%
% Parametry prevodu do pdf
\providecommand{\hypersetup}[1]{}%
\hypersetup{%
unicode,% ? Pravdepodobne bezvyznamne
pdfauthor={Mikuláš Mrva},
pdftitle={Reflection principles and large cardinals},
pdfsubject={Reflection principles and large cardinals},
pdfkeywords={set theory, large cardinals, reflection principle, ZFC, Azriel Lévy},
pdffitwindow=false,% Inicialni umisteni textu v okne Readeru
bookmarksopen=true,% Panel zalozek inicialne zobrazen
% Je-li tohle nastaveno jinak, nektere odkazy nekdy nefunguji
hypertexnames=false,
plainpages=false,
%pdfpagelabels,
%
breaklinks=true,% Radkovy lom smi prijit do klikatelneho odkazu
linkcolor=odkazy,% Graficka podoba odkazu
citecolor=odkazy,% ...
colorlinks=true,% ...
pdfhighlight=/O% ... (vzhled odkazu pri stisknuti)
}%
% Inputenc je asi zbytecne.
% Option 'split' ovlivnuje deleni slov obsahujicich v sobe rozdelovnik
\usepackage[utf8x]{inputenc} % UTF-8 ?
%\usepackage[czech]{babel} %dnes už je však hotová integrace èeštiny do babelu
%\usepackage[split]{czech} %dnes už je však hotová integrace èeštiny do babelu
%
%\usepackage{logdp} %užiteèné drobnosti
%\usepackage{amsthm} %lepšší práce s větami
%\usepackage{amsmath} %nová prostøedí pro matematiku a vylepšení tìch stávajících
%\usepackage{latexsym,amsfonts,amssymb} % nová písmenka
\usepackage{fancyhdr} % zápatí a záhlaví
%\usepackage[nottoc]{tocbibind} % přidá do obsahu položky Literatura a Rejstřík
\usepackage{csquotes}
\pagestyle{plain}
%pøedbìžné nastavení hlavièky (balík fancyhdr)
%\headheight 13.6pt %možná ji bude tøeba zvednout, fancyhdr si pak stìžuje: \headheight
% too small, make it at least Xpt
\headheight 14.5pt %možná ji bude tøeba zvednout, fancyhdr si pak stìžuje: \headheight too \fancyhead{}
\fancyhead[R]{\leftmark}
\fancyfoot{}
\fancyfoot[C]{\thepage}


\newtheorem{theorem}{Theorem}[section]
\newtheorem{Claim}[theorem]{Claim}
\newtheorem{definition}[theorem]{Definition}
\newtheorem{Cor}[theorem]{Corollary}
\newtheorem{Fact}[theorem]{Fact}
\newtheorem{lemma}[theorem]{Lemma}
\newtheorem{sublemma}[theorem]{Sublemma}
\newtheorem{ex}[theorem]{Example}
\newtheorem{remark}[theorem]{Remark}
\newtheorem{obs}[theorem]{Observation}
\newtheorem{que}[theorem]{Question}
\newtheorem{conjecture}[theorem]{Conjecture}

\renewcommand{\theequation}{\thesection.\arabic{equation}}

\newenvironment{proof}
{\noindent \textit{Proof.}}
{\hspace*{\fill} $\Box$}

\newcommand{\toch}{\fbox{\small {\bf ??}}}
\newcommand{\bt}[1]{{\underset{\widetilde{}}{#1}}}
\newcommand{\trcl}[1]{\ensuremath{\mathrm{trcl}(\{#1\})}}
\newcommand{\cf}[1]{\ensuremath{\mathrm{cf}(#1)}}
\newcommand{\cl}[1]{\ensuremath{\mathrm{cl}}(#1)}
\newcommand{\ord}[1]{\ensuremath{\mathrm{ORD}}(#1)}
\newcommand{\dom}[1]{\ensuremath{\mathrm{dom}}(#1)}
\newcommand{\rng}[1]{\ensuremath{\mathrm{rng}}(#1)}
\newcommand{\power}[1]{\ensuremath{\mathscr{P}} (#1)}
\newcommand{\set}[2]{\ensuremath{\{#1 \,:\, #2 \}}}
\newcommand{\seq}[2]{\ensuremath{\langle #1 \,:\, #2 \rangle}}
\newcommand{\singl}[1]{\ensuremath{\{#1\}}}
\newcommand{\pair}[2]{\ensuremath{\{ #1, #2 \}}}
\newcommand{\restr}[2]{\ensuremath{#1 \! \upharpoonright \! #2}}
\renewcommand{\iff}{\leftrightarrow}
\newcommand{\Iff}{\Leftrightarrow}
\newcommand{\el}{\prec}
\newcommand{\iso}{\cong}
\newcommand{\sub}{\subseteq}
\newcommand{\super}{\supseteq}
\newcommand{\la}{\langle}
\newcommand{\ra}{\rangle}
\newcommand{\embed}{\rightarrow}
\newcommand{\mc}{\mathcal}
\newcommand{\supr}[1]{\mathrm{sup}\,#1}
\newcommand{\then}{\rightarrow}
\newcommand{\conc}{^{\smallfrown}}
\newcommand{\bb}{\mathbb}
\newcommand{\supp}[1]{\mathrm{supp}(#1)}
\newcommand{\beq}{\begin{equation}}
\newcommand{\eeq}{\end{equation}}
\newcommand{\brm}{\begin{remark}\begin{rm}}
\newcommand{\erm}{\end{rm}\end{remark}}
\newcommand{\mx}{\mathrm}
\newcommand{\bce}{\begin{compactenum}}
\newcommand{\ece}{\end{compactenum}}
\newcommand{\op}[2]{\la #1, #2 \ra}
\newcommand{\treq}{\trianglelefteq}
\newcommand{\et}{\mathrel{\&}}
\newcommand{\proves}{\vdash}

\newcommand\defeq{\mathrel{\overset{\makebox[0pt]{\mbox{\normalfont\tiny\sffamily def}}}{=}}}

\begin{document}
%titulní stránka
\begin{titlepage}
%\fontsize{16.16pt}{25pt}\selectfont
\Large
\begin{center}
Univerzita Karlova v~Praze, Filozofick{\/á} fakulta\\
Katedra logiky

\vspace{8.5em}
\textsc{Mikuláš Mrva}\\[1.4em]
{REFLECTION PRINCIPLES AND LARGE CARDINALS}\\
Bakalářská práce\\[6.8em]
Vedoucí práce: Mgr. Radek Honzík, Ph.D.\\[6.8em]
2016
\end{center}
\end{titlepage}\


\vspace{\fill}
\noindent 
Prohlašuji, že jsem bakalářskou práci vypracoval samostatně a~že jsem uvedl všechny použité prameny a~literaturu.

\bigskip
\noindent V~Praze 22.~května 2016\\[3em]
\hspace*{\fill}Mikuláš Mrva\hspace*{3em}
\clearpage

\begin{abstract}
\noindent Práce zkoumá vztah tzv. principů reflexe a velkých kardinálů. Lévy ukázal, že v ZFC platí tzv. věta o reflexi~a dokonce, že věta o reflexi je ekvivalentní schématu nahrazení a~axiomu nekonečna nad teorií ZFC bez axiomu nekonečna a~schématu nahrazení. Tedy lze na větu o~reflexi pohlížet jako na svého druhu axiom nekonečna. Práce zkoumá do jaké míry a~jakým způsobem lze větu o reflexi zobecnit a~jaký to má vliv na existenci tzv. velkých kardinálů. Práce definuje nedosažitelné, Mahlovy a nepopsatelné kardinály a ukáže, jak je lze zavést pomocí reflexe. Přirozenou limitou kardinálů získaných reflexí jsou kardinály nekonzistentní s L. Práce nabídne intuitivní zdůvodněn, proč tomu tak je.

\end{abstract}
\bigskip
\renewcommand{\abstractname}{Abstract}
\begin{abstract}
\noindent This thesis aims to examine relations between the so called Reflection Principles and Large cardinals. Lévy has shown that the Reflection Theorem is a sound theorem of ZF and it is equivalent to the Replacement Scheme and the Axiom of Infinity. From this point of view, Reflection theorem can be seen a~specific version of an Axiom of Infinity. This paper aims to examine the Reflection Principle and its generalisations with respect to the existence of Large Cardinals. This thesis will establish the Inaccessible, Mahlo and Indescribable cardinals and show how can those be defined via reflection. A natural limit of Large Cardinals obtained via reflection are cardinals inconsistent with L. This thesis will offer an intuitive explanation of why this holds.
\end{abstract}
\clearpage

\tableofcontents
\clearpage

% podekovani firme co vyrabi club mate -- Loscher gmbh?
\pagestyle{fancy} %detailní definice chování záhlaví
\renewcommand{\sectionmark}[1]{\markboth{\slshape\thesection.\ #1}{}}


\usepackage{color} %pro barevné odkazy, příp. nadpisy
\definecolor{odkazy}{rgb}{0.21,0.27,0.53} %tmavì modrá
\definecolor{nadpisy}{rgb}{0.5812,0.0665,0.0659} %cihlová
%
% Parametry prevodu do pdf
\providecommand{\hypersetup}[1]{}%
\hypersetup{%
unicode,% ? Pravdepodobne bezvyznamne
pdfauthor={Mikuláš Mrva},
pdftitle={Reflection principles and large cardinals},
pdfsubject={Reflection principles and large cardinals},
pdfkeywords={set theory, large cardinals, reflection principle, ZFC, Azriel Lévy},
pdffitwindow=false,% Inicialni umisteni textu v okne Readeru
bookmarksopen=true,% Panel zalozek inicialne zobrazen
% Je-li tohle nastaveno jinak, nektere odkazy nekdy nefunguji
hypertexnames=false,
plainpages=false,
%pdfpagelabels,
%
breaklinks=true,% Radkovy lom smi prijit do klikatelneho odkazu
linkcolor=odkazy,% Graficka podoba odkazu
citecolor=odkazy,% ...
colorlinks=true,% ...
pdfhighlight=/O% ... (vzhled odkazu pri stisknuti)
}%
% Inputenc je asi zbytecne.
% Option 'split' ovlivnuje deleni slov obsahujicich v sobe rozdelovnik
\usepackage[utf8x]{inputenc} % UTF-8 ?
%\usepackage[czech]{babel} %dnes už je však hotová integrace èeštiny do babelu
%\usepackage[split]{czech} %dnes už je však hotová integrace èeštiny do babelu
%
%\usepackage{logdp} %užiteèné drobnosti
%\usepackage{amsthm} %lepšší práce s větami
%\usepackage{amsmath} %nová prostøedí pro matematiku a vylepšení tìch stávajících
%\usepackage{latexsym,amsfonts,amssymb} % nová písmenka
\usepackage{fancyhdr} % zápatí a záhlaví
%\usepackage[nottoc]{tocbibind} % přidá do obsahu položky Literatura a Rejstřík
\usepackage{csquotes}
\pagestyle{plain}
%pøedbìžné nastavení hlavièky (balík fancyhdr)
%\headheight 13.6pt %možná ji bude tøeba zvednout, fancyhdr si pak stìžuje: \headheight
% too small, make it at least Xpt
\headheight 14.5pt %možná ji bude tøeba zvednout, fancyhdr si pak stìžuje: \headheight too \fancyhead{}
\fancyhead[R]{\leftmark}
\fancyfoot{}
\fancyfoot[C]{\thepage}


\newtheorem{theorem}{Theorem}[section]
\newtheorem{Claim}[theorem]{Claim}
\newtheorem{definition}[theorem]{Definition}
\newtheorem{Cor}[theorem]{Corollary}
\newtheorem{Fact}[theorem]{Fact}
\newtheorem{lemma}[theorem]{Lemma}
\newtheorem{sublemma}[theorem]{Sublemma}
\newtheorem{ex}[theorem]{Example}
\newtheorem{remark}[theorem]{Remark}
\newtheorem{obs}[theorem]{Observation}
\newtheorem{que}[theorem]{Question}
\newtheorem{conjecture}[theorem]{Conjecture}

\renewcommand{\theequation}{\thesection.\arabic{equation}}

\newenvironment{proof}
{\noindent \textit{Proof.}}
{\hspace*{\fill} $\Box$}

\newcommand{\toch}{\fbox{\small {\bf ??}}}
\newcommand{\bt}[1]{{\underset{\widetilde{}}{#1}}}
\newcommand{\trcl}[1]{\ensuremath{\mathrm{trcl}(\{#1\})}}
\newcommand{\cf}[1]{\ensuremath{\mathrm{cf}(#1)}}
\newcommand{\cl}[1]{\ensuremath{\mathrm{cl}}(#1)}
\newcommand{\ord}[1]{\ensuremath{\mathrm{ORD}}(#1)}
\newcommand{\dom}[1]{\ensuremath{\mathrm{dom}}(#1)}
\newcommand{\rng}[1]{\ensuremath{\mathrm{rng}}(#1)}
\newcommand{\power}[1]{\ensuremath{\mathscr{P}} (#1)}
\newcommand{\set}[2]{\ensuremath{\{#1 \,:\, #2 \}}}
\newcommand{\seq}[2]{\ensuremath{\langle #1 \,:\, #2 \rangle}}
\newcommand{\singl}[1]{\ensuremath{\{#1\}}}
\newcommand{\pair}[2]{\ensuremath{\{ #1, #2 \}}}
\newcommand{\restr}[2]{\ensuremath{#1 \! \upharpoonright \! #2}}
\renewcommand{\iff}{\leftrightarrow}
\newcommand{\Iff}{\Leftrightarrow}
\newcommand{\el}{\prec}
\newcommand{\iso}{\cong}
\newcommand{\sub}{\subseteq}
\newcommand{\super}{\supseteq}
\newcommand{\la}{\langle}
\newcommand{\ra}{\rangle}
\newcommand{\embed}{\rightarrow}
\newcommand{\mc}{\mathcal}
\newcommand{\supr}[1]{\mathrm{sup}\,#1}
\newcommand{\then}{\rightarrow}
\newcommand{\conc}{^{\smallfrown}}
\newcommand{\bb}{\mathbb}
\newcommand{\supp}[1]{\mathrm{supp}(#1)}
\newcommand{\beq}{\begin{equation}}
\newcommand{\eeq}{\end{equation}}
\newcommand{\brm}{\begin{remark}\begin{rm}}
\newcommand{\erm}{\end{rm}\end{remark}}
\newcommand{\mx}{\mathrm}
\newcommand{\bce}{\begin{compactenum}}
\newcommand{\ece}{\end{compactenum}}
\newcommand{\op}[2]{\la #1, #2 \ra}
\newcommand{\treq}{\trianglelefteq}
\newcommand{\et}{\mathrel{\&}}
\newcommand{\proves}{\vdash}

\newcommand\defeq{\mathrel{\overset{\makebox[0pt]{\mbox{\normalfont\tiny\sffamily def}}}{=}}}

\begin{document}
%titulní stránka
\begin{titlepage}
%\fontsize{16.16pt}{25pt}\selectfont
\Large
\begin{center}
Univerzita Karlova v~Praze, Filozofick{\/á} fakulta\\
Katedra logiky

\vspace{8.5em}
\textsc{Mikuláš Mrva}\\[1.4em]
{REFLECTION PRINCIPLES AND LARGE CARDINALS}\\
Bakalářská práce\\[6.8em]
Vedoucí práce: Mgr. Radek Honzík, Ph.D.\\[6.8em]
2016
\end{center}
\end{titlepage}\


\vspace{\fill}
\noindent 
Prohlašuji, že jsem bakalářskou práci vypracoval samostatně a~že jsem uvedl všechny použité prameny a~literaturu.

\bigskip
\noindent V~Praze 22.~května 2016\\[3em]
\hspace*{\fill}Mikuláš Mrva\hspace*{3em}
\clearpage

\begin{abstract}
\noindent Práce zkoumá vztah tzv. principů reflexe a velkých kardinálů. Lévy ukázal, že v ZFC platí tzv. věta o reflexi~a dokonce, že věta o reflexi je ekvivalentní schématu nahrazení a~axiomu nekonečna nad teorií ZFC bez axiomu nekonečna a~schématu nahrazení. Tedy lze na větu o~reflexi pohlížet jako na svého druhu axiom nekonečna. Práce zkoumá do jaké míry a~jakým způsobem lze větu o reflexi zobecnit a~jaký to má vliv na existenci tzv. velkých kardinálů. Práce definuje nedosažitelné, Mahlovy a nepopsatelné kardinály a ukáže, jak je lze zavést pomocí reflexe. Přirozenou limitou kardinálů získaných reflexí jsou kardinály nekonzistentní s L. Práce nabídne intuitivní zdůvodněn, proč tomu tak je.

\end{abstract}
\bigskip
\renewcommand{\abstractname}{Abstract}
\begin{abstract}
\noindent This thesis aims to examine relations between the so called Reflection Principles and Large cardinals. Lévy has shown that the Reflection Theorem is a sound theorem of ZF and it is equivalent to the Replacement Scheme and the Axiom of Infinity. From this point of view, Reflection theorem can be seen a~specific version of an Axiom of Infinity. This paper aims to examine the Reflection Principle and its generalisations with respect to the existence of Large Cardinals. This thesis will establish the Inaccessible, Mahlo and Indescribable cardinals and show how can those be defined via reflection. A natural limit of Large Cardinals obtained via reflection are cardinals inconsistent with L. This thesis will offer an intuitive explanation of why this holds.
\end{abstract}
\clearpage

\tableofcontents
\clearpage

% podekovani firme co vyrabi club mate -- Loscher gmbh?
\pagestyle{fancy} %detailní definice chování záhlaví
\renewcommand{\sectionmark}[1]{\markboth{\slshape\thesection.\ #1}{}}


\section{Introduction}\label{sec:introduction}
The central point of this thesis is the so called \emph{reflection principle}, which could be informally expressed like this:
\begin{displayquote}
For every property that holds in the universe of all sets, there is a set in which this property holds.
\end{displayquote}

Clearly, this formulation is rather vague and we should be extremely cautious when dealing with the word ``property''. 
One problem that immediately comes to mind is that ``being the set of all sets'' must not be considered a property in this sense, otherwise we run into the well known paradox of Russell.
This is a well-known problem that exemplifies the fact that reflection is a phenomenon that is closely connected to the very foundations of mathematics.
This is also emphasised by the fact that the very first explicit use of reflection in a mathematical proof can be found in Gödel's paper \emph{The Consistency of the Axiom of Choice and of the Generalized Continuum-hypothesis with the Axioms of Set Theory}\footnote{See \cite{Godel1940consistency}.}
that deals with the consistency of the \emph{generalised continuum hypothesis}, which is a question that was central to the development of set theory in the 20\textsuperscript{th} century.
Furthermore, Lévy's article \emph{Axiom Schemata of Strong Infinity in Axiomatic Set Theory}, that is a cornerstone of this thesis is concerned primarily with the so called \emph{strong axioms of infinity}, which are axioms that imply the existence of the set of all natural numbers, this assertion is called the \emph{Axiom of Infinity}\footnote{For a rigorous definition, see definition \bref{def:infinity} later in this section.}, but they also imply the existence of larger sets whose existence can not be proven in the current theory\footnote{For the purposes of this thesis, unless stated otherwise, this will be the \emph{Zermelo–Fraenkel set theory}, that is formally established in definition \bref{def:zfc}.}.

After introducing the elementary theoretical tools required for this task in the rest of this chapter, in chapter 2, we will review the \emph{Reflection theorem} that originally formulated by Richard Montague in 1961\footnote{Note that Lévy's paper was published in 1960, a year before Montague's, but Lévy refers to Montague and not vice versa. While this may seem confusing, it is because Montague gave a lecture on this topic at a conference at the Cornell University in 1957. It is also interesting that Lévy's article refers for Montague's reflection to a publication by Montague and Solomon Feferman called \emph{The method of arithmetization and some of its applications} which was never finished. This is explained by Solomon Feferman in \cite{Feferman2008}.} and extended by Azriel Lévy in his aforementioned article and then restate it in a way that is more in line with today's set theory. 
Chapter 2 deals with the fact that when the term ``property'' is restricted to first-order formulas in the language of set theory, it does not behave like a axiom of strong infinity, but it is equivalent to the \emph{Axiom of Infinity} and \emph{Replacement Schema}, which is one of the key set-forming principles in the \emph{Zermelo–Fraenkel set theory}.

It is in chapter 3 where we get to examine the large cardinals and in a manner similar to Lévy in his article, we introduce a various axioms schemata that come from reflection and lead towards \emph{inaccessible} and \emph{Mahlo cardinals}. We will briefly argue that Mahlo's operation exhausts large cardinals reachable via reflection from below and introduce indescribable cardinals, which are also based on reflection, but lead us into higher–order logic. We will introduce \emph{weakly inaccessible cardinals} and show that they are also based on reflection and examine their relation to the cardinals presented earlier. Finally, we will examine Gödel's constructible universe and see whether the large cardinals we have introduced are compatible with the \emph{axiom of constructibility}.

%\begin{displayquote}
%``The Universe of sets cannot be uniquely characterised (i. e. distinguished from all its initial elements) by any internal structural property of the membership relation in it, which is expressible in any logic of finite of transfinite type, including infinitary logics of any cardinal order.''
%\end{displayquote}
%\rightline{{\rm --- Kurt Gödel \cite{GodelWang}}}

% TODO kapitola 3 resi velky kardinaly a L=V
% TODO proc jsou nektery velky kardinaly jako omega
% TODO nejakej Godeluv citat o strong infinity, the next theory above set theory
% TODO collected works: \cite{Godel2003collected}

% poprve pouzil godel v dukazu 
% axiomy ktere ho inspirovaly (najdi zneni?)
% kontext (doba, kdy teorie mnozin jeste nebyla usazena?) proto i Levy navrhuje axiomy?
\begin{comment}
Kanamori: (p.55)
The heuristic of generalization from $\aleph_0$, like reflection, also came to be used to motivate various large cardinals. Recalling Cantor’s unitary view of the finite and the transfinite, large cardinal properties satisfied by $\aleph_0$ would be too accidental if they were not also ascribable to higher cardinals in an eternal recurrence.

kan (58): This principle is implicit in remarks of Gödel made in 1946 (see Gödel [90:146]). 


To understand why need reflection in the first place, let's think about infinity for a moment. In the intuitive sense, infinity is an upper limit of all numbers. 
But for centuries, this was merely a philosophical concept of limitlessness. % zdroj?
Probably the best-known classic problems involving infinity are the famous paradoxes of Zeno. % zdroj?
In response to those, Aristotle introduced the distinction between an actual and a potential infinity\footnote{See Aristotle’s Physics, Book III}. % presun do citaci
By potential infinity we understand that concept of a process does in unbounded in a sense that it could continue for an arbitrary amount of time, but is also never complete. % prepsat, asi citace z aristotela?
Actual infinity, is, on the other hand, the concept of infinity contained in a bounded space, just like the number of fractions between 0 and 1. Aristotle argued, that the potential infinity is (in today's words) well defined, as opposed to the actual infinity, which is a vague and incoherent concept. He didn't think it's possible for infinite amount of entities to inhabit a bounded place in space or time, rejecting Zeno's thought experiments as a whole. But it's not our aim to get into much detail.  % radsi citace?
The aspect of infinity that is relevant to our interests is the human inability to directly experience limitlessness in contrast to how easily can one talk about infinity and limitlessness in the natural language. 
The short trip into history hopefully served as an example of the fact that certain statements can easily be considered either meaningful or meaningless.  % co?
But while infinity of any kind can't be experienced directly through senses, much effort has been made by philosophers to find a way to meaningfully talk about infinite. 
To see how this leads to reflection, see what Aquinas wrote in his Summa Theologica \footnote{Part I, Question 7, Article 3, Reply to Objection 1}:
\begin{displayquote}
A geometrician does not need to assume a~line actually infinite, but takes some actually finite line, from which he subtracts whatever he finds necessary; which line he calls infinite.
\end{displayquote}
He seems to acknowledge, that infinity can not be reached directly, but for practical purposes it is enough to take a limited part of the whole. One can that act as if it was the whole because the part has all the properties needed at the moment. This, as we shall see in a moment, is in fact an instance of reflection, even though the term itself was introduced centuries later.
% dava to vubec smysl?

To illustrate this elusiveness of infinity, let us remember the early days of set theory. 
When Cantor proved that there are at least two distinct infinite quantities, this effectively turned what previously was an abstract, unreachable absolute, into a mathematical object, a set. 
But just as one infinity was seemingly tamed, about 10 years later, Russell's paradox uncovered the fact that there is another absolute, the paradoxical collection of all sets. 
% tohle asi neni uplne pravda krome ZFC:
Mathematicians have decided to focus on axiomatic set theories so that the paradoxical collection was kept out of sets, being considered a class instead \footnote{When we use the words "class" and "property" in this section, "property" refers to statement in natural or formal language that can be meaningfully stated for sets, the notion of class then refers to the collection of all sets holding that particular property. For all practical purposes, the two are synonyms. They will be later properly redefined for use in formal context.}
This is where reflection comes in again. 

The original idea behind reflection principles probably comes from what could be informally called \textquote{universality of the universe}.
Trying express the universe as a~set $\{x  |  x = x\}$, we either decide to make such statement on a meta-level, or directly in a theory that formalizes the concept of class\footnote{Like the Bernays–Gödel set theory, for example.}. 
But since it is practical to consider sets formed by a property, we must carefully formalize the notion of property of that we stay within the formal framework of a given theory. % to asi neni pravda, viz comprehension a repalcement, vlastnr 
Reflection can be seen as reverting this approach. Even thought we have, in a sense, included the infinity into the set theory in the form of $\omega$, the set of all natural numbers, as well as the hierarchy of larger infinite sets constructed from $\omega$, there is still an unreachable absolute.
Since we have weakened the notion of property so that it only yields sets, there is obviously no way to directly describe the whole universe, every attempt to do so inevitably fails.

% precti si neco poradne o platonismu
If one was was to hold a platonistic view on the philosophy of mathematics, assuming that the sets themselves objectively exist, reflection, the fact that every description of the universe collapses to a bounded object within the universe, can be percieved as the imperfection of formal systems. % nekdo to rikal, NAJDI TO! Drake?
Similarly, while Gödel's second theorem implies that no formal system\footnote{To be more precise, no formal system satisfying specific properties, see Gödel's results for details.} proves everything, one might argue that some the independed statements objectively hold, but the system is not strong enough to verify the fact. 
Speaking of Gödel, it is worth noting that reflection made its first in set-theoretical appearance in Gödel's proof of the Generalised Continuum Hypothesis in the constructible universe L, but it was around even earlier as a~concept. 
Gödel himself regarded it as very close to Russell's reducibility axiom (an earlier equivalent of the axiom schema of separation proposed by Zermelo). %
Richard Montague then studied reflection properties as a~tool for verifying that Replacement is not finitely axiomatizable. a~few years later Lévy proved in \cite{Levy60a} the equivalence of reflection with Axiom of infinity together with Replacement in proof we shall examine closely in chaper 2.
% konec korektury
From this point of view, we can argue that $\omega$ was established as an object satisfying a property attributed informally to the universe of all sets. That is, the property of "being the collection of all natural numbers". But since there was no was to reach it from below, it had to be explicitly brought into existence\footnote{Existence as in "the theory knows that $\omega$ exists".} by the axiom of infinity. 

\

The purpose of the previous paragraphs were formulated from a naive platonistic view, which makes it easy to talk about the universe of all sets, even though the formulations are very informal. Now it should be made clear, that one does not need to informally talk about the universe of all sets or any other ideal objects as if they inhabited an ideal world. We will look at the theory from a structuralist point of view, inspired by Hilbert, Shapiro and Geoffrey Hellman. This allows us to dismiss questions about objective existence of objects beyond mathematical theories. Instead, we can consider objects  meaningful if and only if there is a consistent\footnote{Thanks to Gödel, we only need to care whether it's consistent relatively to the axiomatic set theory of Zermelo and Fraenkel.} formal system which admits the existence of such objects. Starting with $\omega$, a set of all finite ordinals\footnote{See the next section for the exact definition. Until then, finite ordinals are synonymous to natural numbers.}, which is then extended to the Von Neumann's hierarchy $V$, we will examine axioms that can consistently be added to the Zermelo–Fraenkel theory so that the hierarchy is extended to a larger model that contains the previous one as an initial segment. To see why this is also reflection, one should bear in mind that we will create models that reflect certain properties.

Later we will see that large cardinals discussed in this paper are natural extension of this process. We can informally obtain the inaccessible cardinal by arguing the the axioms of the Zermelo–Fraenkel set theory hold in the universe and establishing an object, let's call it $\kappa_I$ for now, that satisfies this property. But then this either leads to a stronger set theory exntended in order to be able to talk about $\kappa_I$. But this process iterates for the new set theory to yield Mahlo cardinals, but it is also clear that even by arbitrary iterations of this principle, the universe still can't be reached. On the final pages of this thesis, we will argue, that while this iterative process seems to lead to a very large object, it is in fact not strong enough to be inconsistent with Gödel's L, which was designed to be the minimal model of set the Zermelo–Fraenkel set theory in terms of width\footnote{The model minimal in terms if height of the universe is the inaccessible cardinal.}. Present day set theory is able to consider many large cardinals far above the hierarchy introduced here. Even though those are beyond the scope of this work, we will briefly mention the measurable cardinal later on.

\end{comment}
\subsection{Notation and Terminology}
\subsubsection{The Language of Set Theory}
% TODO predpokladame splnovani a tak, z logiky? link na ucebnici kdyztak? 
This text assumes the knowledge of basic terminology and some results from first-order predicate logic.\footnote{todo odkaz na pripadny zdroj? svejdar? neco en?} All proofs are based on \cite{JechBook} unless explicitly stated otherwise. Notable amount of material is also drawn from \cite{KanamoriBook} and \cite{DrakeBook}.

We will now shortly review the basic notions that allow us to define the \emph{Zermelo–Fraenkel} set theory.

% theory with the Axiom of Choice ($\sf{ZFC}$), a first-order set theory in the language $\mathscr{L} = \{=, \in\}$, which will be sometimes referred to as \emph{the language of set theory}. In Chapter 3, the Axiom of Choice is assumed. When tlaking about higher-order logic, we will usually denote type 1 variables, which are elements of the domain of discourse, by lowercase letters, mostly $u, v, w, x, y, z, p_1, p_2, p_3,  \ldots$ while type 2 variables, which represent $n$-ary relations of the domain of discourse for any natural number $n$, are usually denoted by uppercase letters $A, B, C, X, Y, Z$. Note that there are exception to convention rules as $f$ usually denotes a function, which is in fact a type 2 variable. On the other hand, $M$ stands for a set.


When we talk about \emph{class}, we have the notion of definable class in mind. 
If $\varphi(x, p_1, \ldots, p_n)$ is a formula in the language of set theory, we call 
\begin{equation}
A = \{x : \varphi(x)\}
\end{equation}
a class of all sets satisfying $\varphi(x)$ in a sense that 
\begin{equation}
x \in A \iff \varphi(x)
\end{equation}
Given classes $A$, $B$, one can easily define the elementary set operations such as $A \cap B$, $A \cup B$, $A \setminus B$, $\bigcup A$, see the first part of \cite{JechBook} for details.
Axioms are the tools by which we can decide whether a particular class is ``small enough'' to be considered a set\footnote{``Small enough'' means that it doesn't introduce a paradox similar to Russell's.}. A class that fails to be considered a set is called a \emph{proper class}.

We will often write ``$M$ is a limit ordinal'', it should always be clear that this can be rewritten as a formula that was introduced earlier.

\


\subsubsection{The Axioms}

\begin{definition}{(The Existence of a Set)}\label{def:existence_of_a_set}
\begin{equation}
\exists x (x = x)
\end{equation}
\end{definition}
% The above axiom is usually omitted because it can be deduced from the axiom of \emph{Infinity} (see below), but since we will be using set theories that omit \emph{Infinity}, this will be useful.

\begin{definition}{(Axiom of Extensionality)}\label{def:extensionality}
\begin{equation}
\forall x, y (x = y \iff \forall z (z \in x \iff z \in y)) %ok
\end{equation}
\end{definition}

\begin{definition}{(Axiom Schema of Specification)}\label{def:specification}\\
The following yields an axiom for every first-order formula $\varphi(x, p_1, \ldots, p_n)$ with no free variables other than $x, p_1, \ldots, p_n$.
\begin{equation}
\forall x, p_1, \ldots, p_n \exists y \forall z ( z \in y \iff z \in x \et \varphi(z, p_1, \ldots, p_n)) %ok
\end{equation}
\end{definition}

We will now provide two definitions that are not axioms, but will be helpful in establishing some axioms in a more comprehensible way.
\begin{definition}{($x \subseteq y$, $x \subset y$)}\label{def:subset}
\begin{equation}
x \subseteq y \iff (\forall z \in x) z \in y
\end{equation}
\begin{equation}
x \subset y \iff x \subseteq y \et x \neq y
\end{equation}
We read $x \subseteq y$ as \emph{x is a subset of y} and $x \subset y$ as \emph{x is a proper subset of y}.
\end{definition}

\begin{definition}{(Empty Set)}\label{def:emptyset}
%Let $\varphi = \neg(x = x)$, $y$ is an arbitrary set, we there exists at least one set $y$ from \ref{def:existence_of_a_set} or \emph{Infinity}
%\begin{equation}
%\emptyset \defeq \{x : x \in y\ \et \varphi(x)\}
%\end{equation}
%We know that $\emptyset$ is a set from \emph{specification} and it is the same set for every $y$ given from \emph{extensionality}.
%TODO
%co radsi ``for an arbitrary $x$''
For an arbitrary set $x$, the empty set, represented by the symbol ``$\emptyset$'', is defined by the following formula:
\beq
(\forall y \in x)(y \in \emptyset \iff \neg(y = y))
\eeq
$\emptyset$ is a set due to \emph{Specification}. While the empty set could also be defined by the formula $\forall y (y \in \empty \iff \neg(y = y))$, the version we use is $\Delta_0$, which we will find useful later. The two definitions yield the same set for every $x$ given because of \emph{Extensionality}.
\end{definition}

\begin{definition}{(Axiom of Pairing)}\label{def:pairing}
\begin{equation}
\forall x, y \exists z \forall q (q \in z \iff q = x \lor q = y) % ok
\end{equation}
\end{definition}

\begin{definition}{(Axiom of Union)}\label{def:union}
\begin{equation}
\forall x \exists y \forall z (z \in y \iff \exists q( z \in q \et q \in x)) % ok
\end{equation}
\end{definition}

%\begin{definition}{(Set Intersection)}\\
%\beq
%x \cap y = \{ z: z \in x \et z \in y \}
%\eeq
%\end{definition}
%
%\begin{definition}{(Set Union)}\\
%\beq
%x \cup y = \{ z: z \in x \lor z \in y \}
%\eeq
%\end{definition}

Now we can introduce more axioms.
\begin{definition}{(Axiom of Foundation)}\label{def:foundation}
\begin{equation}
\forall x (x \neq \emptyset \then (\exists y \in x) (x \cap y = \emptyset)) %ok
\end{equation}
\end{definition}

\begin{definition}{(Axiom of Powerset)}\label{def:powerset}
\begin{equation}
\forall x \exists y \forall z (z \in y \iff z \subseteq x) %ok
\end{equation}
\end{definition}

\begin{definition}{(Axiom of Infinity)}\label{def:infinity} %?
\begin{equation}
\exists x (\emptyset \in x \et (\forall y \in x)(y\cup\{y\} \in x))
\end{equation}
The least set satisfying this is denoted ``$\omega$''.
\end{definition}

Let us introduce a few more definitions that will make the two remaining axioms more comprehensible.
\begin{definition}{(Powerset Function)}\\
Given a set $x$, the \emph{powerset of $x$}, denoted $\power{x}$ and satisfying \ref{def:powerset}, is defined as follows:
\begin{equation}
\power{x} \defeq \{y: y \subseteq x\}
\end{equation}
\end{definition}

\begin{definition}{(Function)}\label{def:function}\\
Given arbitrary first-order formula $\varphi(x, y, p_1, \ldots, p_n)$, we say that $\varphi$ is a function iff
\begin{equation}\label{def:function_formula}
\forall x, y, z, p_1, \ldots, p_n (\varphi(x, y, p_1, \ldots, p_n) \et \varphi(x, z, p_1, \ldots, p_n) \then y = z)
\end{equation}
\end{definition}
When a $\varphi(x, y)$ is a function, we also write the following:
% \footnote{This can also be done for $\varphi$s with more than two free variables by either setting $f(x, p_1, \ldots p_n) = y \iff \varphi(x, y p_1, \ldots, p_n)$ or saying that $\varphi$ represents more functions, determined by the various parameters, so given terms $t_1, \ldots, t_n$, $f(x) = y \iff \varphi(x, y, t_1, \ldots, t_n)$.}
\begin{equation}
f(x) = y \iff \varphi(x, y)
\end{equation}
Alternatively, $f = \{\langle x, y \rangle : \varphi(x, y)\}$ is a class.

% TODO hezci formulace
\begin{definition}{(Domain of a Function)}\label{def:dom}\\
Let $f$ be a function. We call the \emph{domain of $f$} the set of all sets for which $f$ yields a value. We use ``$Dom(f)$'' to refer to this set.
%We read the following as ``$Dom(f)$ is the domain of $f$''.
\begin{equation}
x \in Dom(f) \iff \exists y (f(x) = y)
\end{equation}
\end{definition}
We say ``$f$ is a function on $A$'', $A$ being a class, if $A = dom(f)$.

\begin{definition}{(Range of a Function)}\label{def:rng}\\
Let $f$ be a function. We call the \emph{range of $f$} the set of all sets that are images of other sets via $f$. We use ``$Rng(f)$'' to refer to this set. %We read the following as ``$Rng(f)$ is the range of $f$''.
\begin{equation}
x \in Rng(f) \iff \exists y (f(y) = x)
\end{equation}
\end{definition}
We say that $f$ is \emph{a function into $A$}, $A$ being a class, if $rng(f) \subseteq A$.
We say that $f$ is \emph{a function onto $A$} if $rng(f) = A$%, in other words,
%\begin{equation}
%(\forall y \in A)(\exists x \in dom(f))(f(x) = y)
%\end{equation}
We say a function $f$ is a \emph{one to one function}, iff
\begin{equation}
(\forall x_1, x_2 \in dom(f))(f(x_1) = f(x_2) \then x_1 = x_2)
\end{equation}
We say that $f$ is a bijection iff it is a one to one function that is onto.

Note that \emph{Dom(f)} and \emph{Rng(f)} are not definitions in a strict sense, they are in fact definition schemas that yield definitions for every function $f$ given. Also note that they can be easily modified for $\varphi$ instead of $f$, with the only difference being the fact that it is then defined only for those $\varphi$s that are functions, which must be taken into account. This is worth noting as we will use the notions of \emph{function} and \emph{formula} interchangably.

\begin{definition}{(Function Defined For All Ordinals)}\label{def:function_dfao}\\
We say a function $f$ is \emph{defined for all ordinals}, this is sometimes written $f: Ord \then A$ for any class $A$, if $Dom(f) = Ord$.\
Alternatively,
\begin{equation}
(\forall \alpha \in Ord)(\exists y \in A)(f(\alpha) = y))
\end{equation}
\end{definition}

And now for the axioms.

\begin{definition}{(Axiom Schema of Replacement)}\label{def:replacement}\\
The following is an axiom for every first-order formula $\varphi(x, p_1, \ldots, p_n)$ with no free variables other than $x, p_1, \ldots, p_n$.
\begin{equation}
``\varphi\mbox{ is a function}''\then \forall x \exists y \forall z (z \in y \iff (\exists q \in x)(\varphi(x, y, p_1, \ldots, p_n)))
\end{equation}
\end{definition}


% TODO mas to blbe :(
\begin{definition}{(Choice)}\label{def:choice}\\
\begin{equation}
\begin{gathered}
\forall x \exists f ((f \mbox{ is a choice function with } dom(f) = x \setminus \{\emptyset\})\\
\et \forall y ((y \in y \et y \neq \emptyset) \then f(y) \in y))
\end{gathered}
\end{equation}
\end{definition}
% Choice!!!!

We will refer to the axioms by their name, written in italic type, e.g. \emph{Foundation} refers to the Axiom of Foundation. Now we need to define the set theories to be used in the article. 
% Note that axiomatic set theory is a set of formulas.

\begin{definition}{$(\sf{S})$}\label{def:s}\\ %ok
We call $\sf{S}$ an axiomatic theory in the language $\mathscr{L} = \{=, \in\}$ with exactly the following axioms:
\bce[(i)]
\item \emph{Existence of a set} (see \ref{def:existence_of_a_set})
\item \emph{Extensionality} (see \ref{def:extensionality})
\item \emph{Specification} (see \ref{def:specification})
\item \emph{Foundation} (see \ref{def:foundation})
\item \emph{Pairing} (see \ref{def:pairing})
\item \emph{Union} (see \ref{def:union})
\item \emph{Powerset} (see \ref{def:powerset})
\ece
\end{definition}

\begin{definition}{$(\sf{ZF})$}\label{def:zf}\\ %ok
We call $\sf{ZF}$ an axiomatic theory in the language $\mathscr{L} = \{=, \in\}$ that contains all the axioms of $\sf{S}$ in addition to the following:
\bce[(i)]
\item \emph{Replacement} schema (see \ref{def:replacement})
\item \emph{Infinity} (see \ref{def:infinity})
\ece
\emph{Existence of a set} is usually left out because it is a consequence of \emph{infinity}.
\end{definition}

\begin{definition}{$(\sf{ZFC})$}\label{def:zfc}\\ % ok
$\sf{ZFC}$ is an axiomatic theory in the language $\mathscr{L} = \{=, \in\}$ that contains all the axioms of $\sf{ZF}$ plus \emph{Choice} (\ref{def:choice}).
\end{definition}

\

\subsubsection{The Transitive Universe}
\begin{definition}{(Transitive Class)}\label{def:transitivity}\\ % ok
We say a class $A$ is \emph{transitive} iff
\begin{equation}
(\forall x \in A)(x \subseteq A)
\end{equation}
\end{definition}

\begin{definition}{(Well Ordered Class)}\label{def:well_ordering} % ok
A class $A$ is said to be \emph{well ordered by $\in$} iff the following hold:
\bce[(i)]
\item $(\forall x \in A)(x \not\in x)$ (Antireflexivity)
\item $(\forall x, y, z \in A)(x \in y \et y \in z \then x \in z)$ (Transitivity)
\item $(\forall x, y \in A)(x = y \lor x \in y \lor y \in x)$ (Linearity)
\item $(\forall x \subseteq A)(x \neq \emptyset \then (\exists y \in x)(\forall z \in x)(z = y \lor z \in y)))$ (Existence of the least element)
\ece
\end{definition}

\begin{definition}{(Ordinal Number)}\label{def:ordinal}\\ % ok
A set $x$ is said to be an \emph{ordinal number} if it is \emph{transitive} and \emph{well-ordered by $\in$}. 
\end{definition}
For the sake of brevity, we usually just say ``$x$~is an \emph{ordinal}''. 
Note that ``$x$~is an ordinal'' is a well-defined formula in the language of set theory, since \ref{def:transitivity} is a first-order formula and \ref{def:well_ordering} is in fact a conjunction of four first-order formulas.
Ordinals will be usually denoted by lower case greek letters, starting from the beginning of the alphabet: $\alpha, \beta, \gamma, \ldots$.
Given two different ordinals $\alpha, \beta$, we will write $\alpha < \beta$ for $\alpha \in \beta$, see Lemma 2.11 in \cite{JechBook} for technical details.

\begin{definition}{(Non-Zero Ordinal)} % ok 
We say an ordinal $\alpha$ is \emph{non-zero} iff $\alpha \neq \emptyset$.
\end{definition}

\begin{definition}{(Successor Ordinal)}\label{def:successor_ordinal}\\ % ok 
Consider the following function defined for all ordinals. Let $\beta$ be an arbitrary ordinal. We call $S$ the \emph{successor function}.
\begin{equation}
S(\beta) = \beta \cup \{\beta\}
\end{equation}
An ordinal $\alpha$ is called a \emph{successor ordinal} iff there is an ordinal $\beta$, such that $\alpha = S(\beta)$. We also write $\alpha = \beta+1$.
\end{definition}

\begin{definition}{(Limit Ordinal)}\label{def:limit_ordinal}\\ % ok
A non-zero ordinal $\alpha$ is called a \emph{limit ordinal} iff it is not a successor ordinal.
\end{definition}

\begin{definition}{(Ord)}\label{def:ord}\\  % ok
\emph{The class of all ordinal numbers}, which we will denote ``$Ord$''\footnote{Other authors use ``$On$'', we will stick to the notation used in \cite{JechBook}} is the proper class defined as follows.
\begin{equation}
x \in Ord \iff x\mbox{ is an ordinal}
\end{equation}
\end{definition}

%The following construction will be often referred to as the \emph{Von Neumann's Hierarchy}, sometimes also the \emph{Von Neumann's Universe}. 
%, the former referring more to the construction with the individual levels in mind, the latter referring more to the class $V$, but they can be interchanged with no confusion caused.

\begin{definition}{(Von Neumann's Hierarchy)}\label{def:von_neumann}\\ % ok 
The \emph{Von Neumann's Hierarchy} is a collection of sets indexed by elements of $Ord$, defined recursively in the following way:
\bce[(i)]
\item 
\begin{equation}
V_0 = \emptyset
\end{equation}
\item 
\begin{equation}
V_{\alpha+1} = \power{V_\alpha}\mbox{ for any ordinal $\alpha$}
\end{equation}
\item
\begin{equation} 
V_\lambda = \bigcup_{\beta < \lambda} V_\beta \mbox{ for a limit ordinal $\lambda$}
\end{equation}
\ece
We will also refer to the \emph{Von Neumann's Hierarchy} as \emph{Von Neumann's Universe} or the \emph{Cumulative Hierarchy}. % velky nebo maly c?
\end{definition}

\begin{definition}{(Rank)}\label{def:rank}\\ % ok
Given a set $x$, we say that the rank of $x$ (written as $rank(x)$) is the least ordinal $\alpha$ such that $x \in V_{\alpha+1}$
\end{definition}
Due to \emph{Regularity}, every set has a rank.\footnote{See chapter 6 of \cite{JechBook} for details.}

\begin{definition}{(Order-type)}\label{def:order_type}\\ % TODO check
Given an arbitrary well-ordered set $x$, we say that an ordinal $\alpha$ is the \emph{order-type} of $x$ iff $x$ and $\alpha$ are isomorphic. % TODO definovat izomorfismus?
\end{definition}

%\begin{definition}{($\omega$)}\label{def:omega}\\ % ???? bacha na to
%We call $\omega$
%\begin{equation}
%\omega \defeq \bigcap\{x : \mbox{``$x$ is a limit ordinal''})\}
%\end{equation}
%\end{definition}
%$\omega$ is non-empty if \emph{Infinity} or any equivalent holds.

%\begin{definition}{(Lévy's Hierarchy)}\\
%!!! pozor na konflikt s analytickou (vyres podle kanamoriho)
%TODO (potrebujeme ji?)
%\end{definition}
% TODO levyho hierarchie?
\

\subsubsection{Cardinal Numbers}

% todo def one.to-one mapping?
\begin{definition}{(Cardinality)}\\
Given a set $x$, let the cardinality of $x$, written $|x|$, be defined as the smallest ordinal number such that there is a one to one mapping from $x$ to $\alpha$.
\end{definition}

\begin{definition}{(Aleph function)}\label{def:aleph}\\
Let $\omega$ be the set defined by \ref{def:omega}.
We will recursively define the function $\aleph$ for all ordinals.
\bce[(i)]
\item $\aleph_0 = \omega$
\item $\aleph_{\alpha+1}$ is the least cardinal larger than $\aleph_\alpha$\footnote{``The least cardinal larger than $\aleph_\alpha$'' is sometimes notated as $\aleph_\alpha^{+}$}
\item $\aleph_\lambda = \bigcup_{\beta < \lambda}\aleph_\beta$ for a limit ordinal $\lambda$
\ece
If $\kappa = \aleph_\alpha$ and $\alpha$ is a successor ordinal, we call $\kappa$ a \emph{successor cardinal}. If $\alpha$ is a limit ordinal, we call $\kappa$ a \emph{limit cardinal}.
\end{definition} % mam def. succ ordinaly?

\begin{definition}{(Cardinal number)}\label{def:cardinal}\\
\bce[(i)]
\item A set $x$ is called a \emph{finite cardinal} iff $x \in \omega$.
\item A set is called an \emph{infinite cardinal} iff there is an ordinal $\alpha$ such that $\aleph_\alpha = x$
\item A set is called a \emph{cardinal} iff it is either a \emph{finite cardinal} or an \emph{infinite cardinal}.
\ece
\end{definition}
We say $\kappa$ is an uncountable cardinal iff it is an infinite ordinal and $\aleph_0 < \kappa$.
Infinite cardinals will be notated by lowercase greek letters from the middle of the alphabet, e.g. $\kappa, \mu, \nu, \ldots$.\footnote{Except $\lambda$ which is preferably used for limit ordinals.}

For formal details as well as why every set can be well-ordered assuming \emph{Choice}, and therefore has a cardinality, see \cite{JechBook}. % proc je to tady?

% todo mame def sequence? !! ================ !!!!
% todo mame def ``cofinal''?
% TODO fix it!!
% todo pozor na supremum, upravit! (ikdyz pro limitni je to stejny)

\begin{definition}{(Sequence)}\label{def:sequence}\\
We say that a function $\varphi(x, y)$ is a \emph{sequence} iff there is an ordinal $\alpha$ such that $dom(\varphi) = \alpha$. In other words, a function is called a sequence if it is defined exactly for every ordinal from $0$ to some $\alpha$. We then say it is an $\alpha$-sequence. We usually write $\langle \beta_i : i \in \alpha \rangle$ or $\langle \beta_0, \beta_1, \ldots \rangle$ when referring to a sequence, $\xi_i$ denote the elements of $rng(\varphi)$ for every $i \in dom(\varphi)$.
\end{definition}

\begin{definition}{(Cofinal Subset)}\label{def:cofinal_subset}\\
Given a class $A$, we say that $B \subseteq A$ is \emph{cofinal in $A$} iff
\beq
(\forall x \in A)(\exists y \in B)(x \in y)
\eeq
\end{definition}

\begin{definition}{(Cofinality of a Limit Ordinal)}\label{def:cofinality}\\ % a co https://math.berkeley.edu/~jhicks/links/SOTS/cskipper112613.pdf? 
Let $\lambda$ be a limit ordinal. We say that the \emph{cofinality} of $\lambda$ is $\kappa$ iff $\kappa$ is the least cardinal, such that there is a cofinal $\kappa$-sequence $\langle \beta_\xi : \xi < \kappa \rangle$, such that
\begin{equation}
sup(\{\beta_\xi: \xi < \kappa\}) = \lambda
\end{equation}
We write $cf(\lambda) = \kappa$.
\end{definition}

\begin{definition}{(Regular Cardinal)}\label{def:regular_cardinal}\\
We say a cardinal $\kappa$ is regular iff $cf(\kappa) = \kappa$
\end{definition}

%\begin{definition}{(Limit Cardinal)}\label{def:limit_cardinal}\\
%We say that a cardinal $\kappa$ is a \emph{limit cardinal} iff there is a limit ordinal $\lambda$ such that $\kappa = \aleph_\lambda$ 
% neni in ord redundantni, protoze je V_x def jenom pro ordinalni x?
%\end{definition}

\begin{definition}{(Strong Limit Cardinal)}\label{def:strong_limit_cardinal}\\
We say that an ordinal $\kappa$ is a \emph{strong limit cardinal} if it is a \emph{limit cardinal} and 
\begin{equation}
(\forall \alpha \in \kappa)(\power{\alpha} \in \kappa)
\end{equation}
\end{definition}

\begin{definition}{(Generalised Continuum Hypothesis)}\label{def:gch}\\
\begin{equation}
\aleph_{\alpha+1}=\power{\aleph_\alpha}
\end{equation}
\end{definition}
If \emph{GCH} holds (for example in Gödel's $L$, see chapter 3), the notions of limit cardinal and strong limit cardinal are equivalent.

\

\subsubsection{Relativisation and Absoluteness}
\begin{definition}{(Relativization)}\label{def:relativization}\\
Let $M$ be a class, $R \subseteq M\times M$ and let $\varphi(p_1, \ldots, p_n)$ be a first-order formula with no free variables besides $p_1, \ldots, _n$. 
The \emph{relativization of $\varphi$ to $M$ and $R$} is the formula, written as $\varphi^{M, R}(p_1, \ldots, p_n)$, defined in the following inductive manner:
\bce[(i)]
\item $(x \in y)^{M,R} \iff R(x, y)$
\item $(x = y)^{M,R} \iff x = y$
\item $(\neg \varphi)^{M,R} \iff \neg \varphi^{M,R}$
\item $(\varphi \et \psi)^{M,R} \iff \varphi^{M,R} \et \psi^{M,R}$
\item $(\varphi \lor \psi)^{M,R} \iff \varphi^{M,R} \lor \psi^{M,R}$
\item $(\varphi \then \psi)^{M,R} \iff \varphi^{M,R} \then \psi^{M,R}$
\item $(\exists x \varphi(x))^{M,R} \iff (\exists x \in M) \varphi^{M,R}(x)$
\item $(\forall x \varphi(x))^{M,R} \iff (\forall x \in M) \varphi^{M,R}(x)$
\ece
\end{definition}
When $R=\in\cap(M \times M)$, we usually write $\varphi^M$ instead of $\varphi^{M, R}$. When we talk about $\varphi^M(p_1, \ldots, p_n)$, it is understood that $p_1, \ldots, p_n \in M$.
We will also use $M \models \varphi(p_1, \ldots, p_n)$ and $\varphi^M(p_1, \ldots, p_n)$ interchangably.
% TODO definice splnovani!

\begin{definition}{(Absoluteness)}
Given a transitive class $M$, we say a formula $\varphi$ is \emph{absolute in $M$} if for all $p_1, \ldots, p_n \in M$
\begin{equation}
\varphi^M(p_1, \ldots, p_n) \iff \varphi(p_1, \ldots, p_n)
\end{equation}
\end{definition}

\begin{definition}{(Hierarchy of First-Order Formulas)}\\
\bce[(I)]
\item A first-order formula $\varphi$ is $\Delta_0$ iff it is logically equivalent to a first-order formula $\varphi'$ satisfying any of the following:
\bce[(i)]
\item $\varphi'$ contains no quantifiers
\item $y$ is a set, $\psi$ is a $\Delta_0$ formula, and $\varphi'$ is either $(\exists x \in y)\psi(y)$ or $(\forall x \in y)\psi(y)$.
\item $\psi_1, \psi_2$ are $\Delta_0$ formulas and $\varphi'$ is any of the following: $\psi_1 \lor \psi_2$, $\psi_1 \et \psi_2$, $\psi_1 \then \psi_2$, $\neg \psi_2$, 
\ece
\item If a formula is $\Delta_0$ it is also $\Sigma_0$ and $\Pi_0$
\item A formula $\varphi$ is $\Pi_n+1$ if it is logically equivalent to a formula $\varphi'$ such that $\varphi' = \forall x \psi$ where $\psi$ is a $\Sigma_n$-formula for any $n < \omega$.
\item A formula $\varphi$ is $\Sigma_n+1$ if it is logically equivalent to a formula $\varphi'$ such that $\varphi' = \forall x \psi$ where $\psi$ is a $\Pi_n$-formula for any $n < \omega$.
\ece
\end{definition}
Note that we can use the pairing function so that for $\forall p_1, \ldots, p_n \psi(p_1, \ldots, p_n)$, there is a logically equivalent formula of the form $\forall x \psi'(x)$.

\begin{lemma}{($\Delta_0$ absoluteness)}\label{lemma:delta_0_absoluteness}
Let $\varphi$ be a $\Delta_0$ formula, then $\varphi$ is absolute in any transitive class $M$.
\end{lemma}

\begin{proof}
This will be proven by induction over the complexity of a given $\Delta_0$ formula $\varphi$. Let $M$ be an arbitrary transitive class. 

Atomic formulas are always absolute by the definition of relativisation, see \ref{def:relativization}.
Suppose that $\Delta_0$ formulas $\psi_1$ and $\psi_2$ are absolute in $M$. Then from relativization, $(\psi_1 \et \psi_2)^M \iff \psi_1^M \et \psi_2^M$, which is, from the induction hypothesis, equivalent to $\psi_1 \et \psi_2$. The same holds for $\lor, \then, \neg$.

Suppose that a $\Delta_0$ formula $\psi$ is absolute in $M$. Let $y$ be a set and let $\varphi = (\exists x \in y) \psi(x)$. 
From relativization, $(\exists x \psi(x))^M \iff (\exists x \in M) \psi^M(x)$. Since the hypotheses makes it clear that $\psi^M \iff \psi$, we get $((\exists x \in y) \psi(x))^M \iff (\exists x \in y\cap M) \psi(x)$, which is the equivalent of $\varphi^M \iff \varphi$. The same applies to $\varphi = (\forall x \in y) \psi(x)$.
\end{proof}

% todo co Devlin -- p.27 -- downward a upward absoluteness?
\begin{lemma}{(Downward Absoluteness)}\label{lemma:downward_absoluteness}\\
Let $\varphi$ be a $\Pi_1$ formula and $M$ a transitive class. Then the following holds:
\begin{equation}
(\forall p_1, \ldots, p_n \in M)(\varphi(p_1, \ldots, p_n) \then \varphi(p_1, \ldots, p_n)^M)
\end{equation}
\end{lemma}
\begin{proof}
Since $\varphi(p_1, \ldots, p_n)$ is $\Pi_1$, there is a $\Delta_0$ formula $\psi(p_1, \ldots, p_n, x)$ such that $\varphi = \forall x \psi(p_1, \ldots, p_n, x)$. From relativization and lemma \ref{lemma:delta_0_absoluteness}, $\varphi^M(p_1, \ldots, p_n) \iff (\forall x \in M)\psi(p_1, \ldots, p_n, x)$.

Assume that for $p_1, \ldots, p_n \in M$ fixed, that $\forall x \psi(p_1, \ldots, p_n, x)$ holds, but $(\forall x \in M)\psi(p_1, \ldots, p_n, x)$ does not. 
Therefore $\exists x \neg \psi(p_1, \ldots, p_n, x)$, which contradicts $\forall x \psi(p_1, \ldots, p_n, x)$.
\end{proof}


\begin{lemma}{(Upward Absoluteness)}\label{lemma:upward_absoluteness}\\
Let $\varphi$ be a $\Sigma_1$ formula and $M$ a transitive class. Then the following holds:
\begin{equation}
(\forall p_1, \ldots, p_n \in M)(\varphi^M(p_1, \ldots, p_n) \then \varphi(p_1, \ldots, p_n))
\end{equation}
\end{lemma}
\begin{proof}
Since $\varphi(p_1, \ldots, p_n)$ is $\Sigma_1$, there is a $\Delta_0$ formula $\psi(p_1, \ldots, p_n, x)$ such that $\varphi = \exists x \psi(p_1, \ldots, p_n, x)$. From relativization and lemma \ref{lemma:delta_0_absoluteness}, $\varphi^M(p_1, \ldots, p_n) \iff (\exists x \in M)\psi(p_1, \ldots, p_n, x)$.

Assume that for $p_1, \ldots, p_n \in M$ fixed, that $(\exists x \in M)\psi(p_1, \ldots, p_n, x)$ holds, but $\exists x \psi(p_1, \ldots, p_n, x)$ does not. This is an obvious contradiction.
\end{proof}


\subsubsection{More Functions}

\begin{definition}{(Strictly Increasing Function)}\label{def:increasing_function}\\
A function $f: Ord \then Ord$ is said to be \emph{strictly increasing} iff
\begin{equation}
\forall \alpha, \beta \in Ord (\alpha < \beta \then f(\alpha) < f(\beta)).
\end{equation}
\end{definition}

\begin{definition}{(Continuous Function)}\label{def:continuous_function}\\
A function $f: Ord \then Ord$ is said to be \emph{continuous} iff
\begin{equation}
\lambda\mbox{ is limit } \then f(\lambda) = \bigcup_{\alpha < \lambda} f(\alpha).
\end{equation}
\end{definition}

\begin{definition}{(Normal Function)}\label{def:normal_function}\\
A function $f: Ord \then Ord$ is said to be \emph{normal} iff it is \emph{strictly increasing} and \emph{continuous}.
\end{definition}

\begin{definition}{(Fixed Point)}\label{def:fixed_point}\\
We say $x$ is a fixed point of a function $f$ iff $x=f(x)$.
\end{definition}

\begin{definition}{(Unbounded Class)}\label{def:unbounded_class}\\
We say a class $A$ is unbounded iff
\begin{equation}
\forall x (\exists y \in A) (x < y)
\end{equation}
\end{definition}

\begin{definition}{(Limit Point)}\label{def:limit_point}\\
Given a class $x \subseteq Ord$, we say that $\alpha \neq \emptyset$ is a limit point of $x$ iff 
\begin{equation}
\alpha = \bigcup(x \cap \alpha)
\end{equation}
\end{definition}

\begin{definition}{(Closed Class)}\label{def:closed_class}\\
We say a class $A \subseteq Ord$ is closed iff it contains all its limit points.
\end{definition}

\begin{definition}{(Club set)}\label{def:club_set}\\
For a regular uncountable cardinal $\kappa$, a set $x \subset \kappa$ is a \emph{closed unbounded} subset, abbreviated as a \emph{club set}, iff $x$ is both closed and unbounded in $\kappa$.
\end{definition}

\begin{definition}{(Stationary set)}\label{def:stationary_set}\\
For a regular uncountable cardinal $\kappa$, we say a set $A \subset \kappa$ is stationary in $\kappa$ iff it intersects every club subset of $\kappa$.
\end{definition}

\subsubsection{Structure, Substructure and Embedding}

Structures will be denoted $\langle M, \in, R \rangle$ where $M$ is a domain, $\in$ stands for the standard membership relation, it is assumed to be restricted to the domain\footnote{To be totally explicit, we should write $\langle M, \in \cap M \times M, R \rangle$}, $R \subseteq M$ is a relation on the domain. When $R$ is not needed, we can as well only write $M$ instead of $\langle M, \in \rangle$.

\begin{definition}{(Elementary Embedding)}\label{def:elementary_embedding}\\
Given the structures $\langle M_0, \in, R \rangle$, $\langle M_1, \in, R \rangle$ and a one-to-one function $j: M_0 \then M_1$, we say $j$ is an \emph{elementary embedding} of $M_0$ into $M_1$, we write $j: M_0 \prec M_1$, when the following holds for every formula $\varphi(p_1, \ldots, p_n)$ and every $p_1, \ldots, p_n \in M_0$:
\begin{equation}
\langle M_0, \in, R \rangle \models \varphi(p_1, \ldots, p_n) \iff \langle M_1, \in, R \rangle  \models \varphi(j(p_1), \ldots, j(p_n))
\end{equation}
\end{definition}


\begin{definition}{(Elementary Substructure)}\label{def:elementary_substructure}\\
Given the structures $\langle M_0, \in, R \rangle$, $\langle M_1, \in, R \rangle$ and a one-to-one function $j: M_0 \then M_1$ such that $j: M_0 \prec M_1$, we say that $M_0$ is an \emph{elementary substructure} of $M_1$, denoted as $M_0 \prec M_1$, iff $j$ is an identity on $M_0$. In other words
\begin{equation}
\langle M_0, \in, R \rangle \models \varphi(p_1, \ldots, p_n) \iff \langle M_1, \in, R \rangle  \models \varphi(p_1, \ldots, p_n)
\end{equation}
for $p_1, \ldots, p_n \in M_0$
\end{definition}

% While higher-order satisfaction relation for proper classes is unformalizable\footnote{TODO CITE KDE? Tarski nebo tak neco?},we can formalize satisfaction in a structure. For the rest of this chapter, let $D$ be a domain of such structure.



%\newpage

\section{Levy's First-Order Reflection}\label{sec:first_order}

\subsection{Lévy's Original Paper}\label{sec:levy1960}
This section is based on Lévy's paper \emph{Axiom Schemata of Strong Infinity in Axiomatic Set Theory}, \cite{Levy60a}. It presents Lévy's general reflection principle and its equivalence to \emph{Replacement} and \emph{Infinity} under $\sf{S}$\footnote{See definition (\ref{def:s}).}.

First, we should point out that set theory has changed over the last 66 years and show a few notable, albeit only formal, differences. Firstly, when reading Lévy's article, one should bear in mind that while the author often speaks about a~model of $\sf{ZF}$, usually denoted $u$, it doesn't necessarily mean that there is a set $u$ that is a model of $\sf{ZF}$\footnote{This is indeed impossible to prove in $\sf{ZF}$ due to Gödel's Incompleteness.}, we are nowadays used to using the notion of universal class $V$ in similar sense, even though independently from a particular axiomatic set theory. %We will review the exact meaning of the notion of a standard complete model in a moment.
The theory $\sf{ZF}$ is practically identical to the theory we have established in (\ref{def:zf}), the differences are only formal.
One might be confused by the fact that Lévy treats the \emph{Subsets} axiom, which we call \emph{Specification}, as a single axiom rather than a schema. He even takes the conjunction of all axioms of $\sf{ZF}$ and treats it like a formula. This is possible because the underlying logic calculus is different. Lévy works with set theories formulated in the \emph{non-simple applied first order functional calculus}, see Chapter IV in \cite{church1996introduction} for details. For now, we only need to know that the calculus contains a substitution rule for functional variables. This way, \emph{Subsets} is de facto a schema even though it sometimes treated as a single formula\footnote{This way, the conjunction of all axioms is then in fact an axiom schema.}.
% todo koukni do churche jak se to s tim ma
It should also be noted that the logical connectives look different. The now usual symbol for an universal quantifier does not appear, $\forall x \varphi (x)$ would be written as $(x) \varphi (x)$. The symbol for negation is ``$\sim$'', implication is written as ``$\supset$'' and equivalence is ``$\equiv$''. We will use standard notation with ``$\neg$'', ``$\then$'' and ``$\iff$'' respectively when presenting Lévy's results.

%The following definitions are not used in contemporary set theory, but they illustrate 1960's set theory mind-set and they are used heavily in Lévy's text, so we will include and explain them for clarity. 
%Generally in this chapter, $\sf{Q}$ stands for an arbitrary axiomatic set theory. % used for general definitions, $u$ is usually a model of $\sf{Q}$, counterpart of today's the universal class $V$.

\

This subsection uses $\sf{ZF}$ instead of the usual $\sf{ZFC}$ as the underlying theory. % neni to zbytecny?

\begin{definition}{(Standard Complete Model of a Set Theory)}\label{def:scm_q}\\
Let $\sf{Q}$ be an arbitrary axiomatic set theory.
We say that $u$ is a standard complete model of $\sf{Q}$ iff
\bce[(i)]
\item $(\forall \sigma \in \sf{Q})(u \models \sigma)$
\item $\forall y (y \in u \then y \subset u)$
% \item $\forall e \langle x, y \rangle \in e \iff (y \in u \et x \in y)$ % co je $e$? omg je to jinak, je to model vsech axiomu pro vsechny $e$!!! relatiizace do u, e?
\ece 
We write $Scm^{\sf{Q}}(u)$.
\end{definition}

\begin{definition}{(Cardinals Inaccessible With Respect to $\sf{Q}$)}\label{def:levy_inaccessible_q}\\
Let $\sf{Q}$ be an arbitrary axiomatic set theory. We say that a cardinal $\kappa$ is inaccessible with respect to theory $\sf{Q}$ iff
\begin{equation}
Scm^{\sf{Q}}(V_\kappa)
\end{equation}
We write $In^{\sf{Q}}(\kappa)$
\end{definition}

\begin{definition}{(Inaccessible Cardinal With Respect to $\sf{ZF}$)}\label{def:levy_inaccessible}\\
When a cardinal $\kappa$ is inaccessible with respect to $\sf{ZF}$, we only say that it is inaccessible. We write $In(\kappa)$.
\begin{equation}
In(\kappa) \iff In^{\sf{ZF}}(\kappa)
\end{equation}
\end{definition}
The above definition of inaccessibles is used because it doesn't require \emph{Choice}.

For the definition of relativization, see (\ref{def:relativization}). The notation used by Lévy is ``$Rel(u, \varphi)$'', we will stick to ``$\varphi^{u}$''.
\begin{definition}{($N$)}\label{def:levy_axiom_n}\\
The following is an axiom schema of complete reflection over $\sf{ZF}$, denoted as $N$. For every first-order formula $\varphi$ in the language of set theory with no free variables except for $p_1, \ldots , p_n$, the following is an instance of schema $N$.
\begin{equation}
\exists u (Scm^{\sf{ZF}}(u) \et (\forall p_1, \ldots, p_n \in u)(\varphi \iff \varphi^{u}))
\end{equation}
%where $\varphi$ is a~formula which contains no free variables except for $p_1, \ldots , p_n$.
\end{definition}

Let $\sf{S}$ be an axiomatic set theory defined in (\ref{def:s}). 

\begin{definition}{($N_0$)}\label{def:levy_axiom_n0}\\
Axiom schema $N_0$ is similar to $N$ defined above, but with $\sf{S}$ instead of $\sf{ZF}$. For every $\varphi$, a first-order fomula in the language of set theory with no free variables except $p_1, \ldots , p_n$, the following is an instance of $N_0$.
\begin{equation}
\exists u (Scm^{\sf{S}}(u) \et (\forall p_1, \ldots , p_n \in u)(\varphi \iff \varphi^{u}))
\end{equation}
%where $\varphi$ is a~formula which contains no free variables except for $p_1, \ldots , p_n$.
\end{definition}

We will now show that in $\sf{S}$, $N_0$ implies both \emph{Replacement} and \emph{Infinity}.

\

Let $N_0$ be defined as in (\ref{def:levy_axiom_n0}), for \emph{Infinity} see (\ref{def:infinity}).
\begin{theorem}\label{theorem:n0_implies_infinity}\
In $\sf{S}$, the axiom schema $N_0$ implies \emph{Infinity}.
\end{theorem}

\begin{proof} % Levy to bere pres V_\lambda \models 
Let $\varphi = \forall x \exists y (y = x \cup \{x\})$. 
This clearly holds in $\sf{S}$ because given a set $x$, there is a set $y = x \cup \{x\}$ obtained via \emph{Pairing} and \emph{Union}. %\emph{Powerset} and \emph{Specification}.
From $N_0$, there is a set $u$ such that $\varphi^{u}$ holds. This $u$ satisfies the conditions required by \emph{Infinity}.
\end{proof}

Lévy proves this theorem in a different way. He argues that for an arbitrary formula $\varphi$, $N_0$ gives us $\exists u Scm^{\sf{S}}(u)$ and this $u$ already satisfies \emph{Infinity}. 
To do this, we would need to prove lemma (\ref{lemma:scm_s_is_limit}) now, which would make second half of this chapter quite confusing.
%We would need to prove (\ref{lemma:scm_s_is_limit}), which will happen later in this chapter, but we don't know that yet. % mozna taky trochu preformulovat

\

Let $\sf{S}$ be a set theory defined in (\ref{def:s}), $N_0$ a schema defined in (\ref{def:levy_axiom_n0}) and \emph{Replacement} a schema defined in (\ref{def:replacement}).
\begin{theorem} 
%In $\sf{S}$ with $N_0$ implies \emph{Replacement}.\\
%TODO jedno nebo druhe % nebo treti
In $\sf{S}$, axiom the schema $N_0$ implies \emph{Replacement}.
%\beq
%\sf{S}, N_0 \vdash \mbox{\emph{Replacement}}
%\sf{S} \vdash N_0 \then \mbox{\emph{Replacement}}
%\eeq % nebo S \vdash N_0 \then Replacement % (veta o dedukci?) nebo \sf{S} + N_0 \vdash Repl
\end{theorem}
% todo zkontrolovat ==========================================================================================
\begin{proof}
Let $\varphi(x, y, p_1, \ldots, p_n)$ be a~formula with no free variables except $x, y, p_1, \ldots, p_n$.
%Out goal is to prove
%\beq
%``\varphi\mbox{ is a function}''\then \forall x \exists y \forall z (z \in y \iff (\exists q \in x)(\varphi(x, y, p_1, \ldots, p_n)))
%\eeq
Let $\chi$ be an instance of the \emph{Replacement} schema for the $\varphi$ given. We want to verify that $\chi$ holds in $\sf{S}$ with $N_0$.
\begin{equation}
\begin{gathered}
\chi = \forall x, y, z(\varphi(x, y, p_1, \ldots, p_n) \et \varphi(x, z, p_1, \ldots, p_n) \then y = z) \\
\then \forall x \exists y \forall z (z \in y \iff (\exists q \in x)(\varphi(x, y, p_1, \ldots, p_n)))
\end{gathered}
\end{equation}

Now consider the following formulas. % and $\forall x, p_1, \ldots, p_n \chi$ respectively: 
% TODO CO JE TO POSLEDNI? UZAVER?
\bce[(i)]
\item $(\forall x, y, p_1, \ldots, p_n \in u)(\varphi \iff \varphi^{u})$
\item $(\forall x, p_1, \ldots, p_n \in u)(\exists y \varphi \iff (\exists y \varphi)^{u})$
\item $(\forall x, p_1, \ldots, p_n \in u)(\chi \iff \chi^{u})$
\item $\forall x, p_1, \ldots, p_n \chi \iff (\forall x, p_1, \ldots, p_n \chi)^{u}$ % co ho pridat pozdeji?
\ece
The above formulas are instances of the $N_0$ schema for $\varphi$, $\exists y \varphi$, $\chi$ and the universal closure of $\chi$ respectively.
By $N_0$, there exists a set $u$ where all four formulas hold.\footnote{Despite the fact that $N_0$ is defined for one formula, we have just used it for four at once. To make this formally possible, we can either prove that $N_0$ is equivalent to a more general version for any finite number of formulas or we can reflect their conjunction and argue that if $u \models \varphi_1 \et \ldots \et \varphi_n$, then $(u \models \varphi_1), \ldots, (u \models \varphi_n)$.}
From relativization, $(\exists y \varphi)^{u}$ is equivalent to $(\exists y \in u) \varphi^{u}$, together with (i) and (ii), we get
\begin{equation}
(\forall x, p_1, \ldots, p_n \in u)((\exists y \in u)\varphi \iff \exists y \varphi)
\end{equation}

If $\varphi$ is a~function, then for every $x \in u$, which is also $x \subset u$ since $Scm^{\sf{S}}(u)$ and therefore $u$ is transitive,
it maps elements of $x$ into $u$. From the \emph{Specification}, we can find $y$, a~set of all images of the elements of $x$.
That gives us $x, p_1, \ldots, p_n \in u \then \chi$. By (iii) we get that $x, p_1, \ldots, p_n \in u \then \chi^{u}$ holds. The universal closure of this formula is $\forall x, p_1, \ldots, p_n (x, p_1, \ldots, p_n \in u \then \chi^{u})$ which is equivalent to $(\forall x, p_1, \ldots, p_n \in u)(\chi)^{u}$, which is exactly $(\forall x, p_1, \ldots, p_n \chi)^{u}$. 
From (iv), $\forall x, p_1, \ldots, p_n \chi$ holds. 
% Via universal instantiation, we end up with $\chi$, which is an instance of \emph{Replacement} for an arbitrary formula $\varphi$ given.
\end{proof}

What we have just proven is only a single theorem from Lévy's aforementioned article, we will introduce other interesting results, mostly related to Mahlo and inaccessible cardinals, later in their appropriate context in chapter 3. % porad trochu kostrbaty?
% ale fakt to pak udelej!

% =====================================================================================================================================

\subsection{Contemporary Restatement}
We will now introduce and prove a theorem that is called Lévy's Reflection in contemporary set theory. The only difference is that while Lévy originally reflects a formula $\varphi$ from $V$ to a set $u$ which is a \emph{standard complete model of $\sf{S}$}, we say that there is a $V_\lambda$ for a limit $\lambda$ that reflects $\varphi$. Those two conditions are equivalent due to lemma (\ref{lemma:scm_s_is_limit}).

%\begin{definition}{(Reflection\textsubscript{1})}\label{def:reflection_1}\\ % co spis Levy's reflection principle?
%Let $\varphi(p_1, \ldots, p_n)$ be a first-order formula in the language of set theory. Than the following holds for any such $\varphi$.
%\begin{equation}
%\forall M_0 \exists M (M_0 \subseteq M \et (\varphi^M(p_1, \ldots, p_n) \iff \varphi(p_1, \ldots, p_n)))
%\end{equation}
%\end{definition}

% Note that this is a restatement of both Lévy's $N$ and $N_0$ from the previous chapter\footnote{see (\ref{def:levy_axiom_n}) and (\ref{def:levy_axiom_n0}}). We prefer to call it \emph{First-order reflection} so it complies with how other axioms and schemata are named. \footnote{We will not use the name $N_0$, because it might be confusing to work $N_0$ and $M_0$ where $M_0$ is a set and $N_0$ is an axiom schema.} 
% Note that the subscript 1 refers to the fact that $\varphi(p_1, \ldots, p_n)$ is a first-order formula, and since we're using the word ``reflection'' in less strict meaning throughout this thesis, distinguishing between the two just by using italic font face for the schema might cause confusion.

% We will now prove the equivalence of \emph{First-order reflection} with \emph{Replacement} and \emph{Infinity} in $\sf{S}$ in two parts. First, we will show that \emph{First-order reflection} is a theorem of $\sf{ZFC}$, then we shall show that the second implication, which proves \emph{Infinity} and \emph{Replacement} from \emph{First-order reflection}, also holds.

% The following lemma is usually done in two parts, the first being for one formula, the other for $n$ formulas. We will only state and prove the more general version.

\begin{lemma}\label{lemma:reflection_lemma}\
Let $\varphi_1, \ldots, \varphi_n$ be first-order formulas in the language of set theory, all with $m$ free variables
\footnote{For formulas with a different number of free variables, take for $m$ the highest number of parameters among those formulas. Add spare parameters to every formula that has less than $m$ parameters in a way that preserves the last parameter, which we will denote $x$.
E.g. let $\varphi'_i$ be the a~formula with $k$ parameters, $k < m$. Let us set $\varphi_i(p_1, \ldots, p_{m-1}, x) = \varphi'_i(p_1, \ldots, p_{k-1}, x)$, notice that the parameters $p_k, \ldots, p_{m-1}$ are not used.}.
\bce[(i)]
\item For each set $M_0$ there is such set $M$ that $M_0 \subset M$ and the following holds for every $i$, $1 \leq i \leq n$:
\begin{equation}\label{equation:refl_lemma_i}
\exists x \varphi_i(p_1, \ldots, p_{m-1}, x) \then (\exists x \in M) \varphi_i(p_1, \ldots, p_{m-1}, x)
\end{equation}
for every $p_1, \ldots, p_{m-1} \in M$.

\item Furthermore, there is a limit ordinal $\lambda$ such that $M_0 \subset V_\lambda$ and the following holds for each $i$, $1 \leq i \leq n$:
\begin{equation}\label{equation:refl_lemma_ii}
\exists x \varphi_i(p_1, \ldots, p_{m-1}, x) \then (\exists x \in V_\lambda) \varphi_i(p_1, \ldots, p_{m-1}, x)
\end{equation}
for every $p_1, \ldots, p_{m-1} \in M$.

\item Assuming \emph{Choice}, there is $M$, $M_0 \subset M$ such that (\ref{equation:refl_lemma_i}) holds for every $M,\ i \leq n$ and $|M| \leq |M_0| \cdot \aleph_0$.
\ece
\end{lemma}

\begin{proof}
We will simultaneously prove statements (i) and (ii), denoting $M^T$ the transitive set required by part (ii).
Steps in the construction of $M^T$ that are not explicitly included are equivalent to steps for $M$.

Let us first define an operation $H_i(p_1, \ldots, p_{m-1})$ that yields the set of $x$'s with minimal rank\footnote{Rank is defined in (\ref{def:rank})} satisfying $\varphi_i(p_1, \ldots, p_{m-1}, x)$ for $p_1, \ldots, p_{m-1}$ and for every $i$, $1 \leq i \leq n$.

\begin{equation}
H_i(p_1, \ldots, p_n) = \{x \in C_i: (\forall z \in C)(rank(x) \leq rank(z))\}
\end{equation}
for each $1 \leq i \leq n$, where
\begin{equation}
C_i = \{x: \varphi_i(p_1, \ldots, p_{m-1}, x)\} \mbox{ for $1 \leq i \leq n$}
\end{equation}

\

Next, let's construct $M$ from given $M_0$ by induction. 
\begin{equation}
M_{i+1} = M_i \cup \bigcup_{j=0}^{n} \bigcup \{H_j(p_1, \ldots, p_{m-1}): p_1, \ldots, p_{m-1} \in M_i\}
\end{equation}
In other words, in each step we include into the construction the elements satisfying $\varphi(p_1, \ldots, p_{m-1}, x)$ for $p_1, \ldots, p_{m-1}$ from the previous step.
For statement (ii), this is the only part that differs from (i). To end up with a transitive $M$, we need to extend every step to it's transitive closure transitive closure of $M_{i+1}$ from (i). In other words, let $\gamma$ be the smallest ordinal such that 
\begin{equation}
(M^T_i \cup \bigcup_{j=0}^{n} \{\bigcup\{H_j(p_1, \ldots, p_{m-1}): p_1, \ldots, p_{m-1} \in M_i\}\}) \subset V_\gamma
\end{equation}
Then the incremental step is
\begin{equation}
M^T_{i+1} = V_\gamma
\end{equation}
and the final $M$ is obtained by joining the previous steps.
\begin{equation}
M = \bigcup_{i=0}^{\infty} M_i, \mbox{  }M^T = \bigcup_{i=0}^{\infty} M^T_i = V_\lambda\mbox{ for some limit }\lambda\mbox{.}
\end{equation}

\

We have yet to finish part (iii).
Let's try to construct a~set $M'$ that satisfies the same conditions like $M$ but is kept as small as possible. Assuming the Axiom of Choice, we can modify the construction so that the cardinality of $M'$ is at most $|M_0| \cdot \aleph_0$. Note that the size of $M$ in the previous construction is determined by the size of $M_0$ and, most importantly, by the size of $H_i(p_1, \ldots, p_{m-1})$ for every $i$, $1 \leq i \leq n$ in individual iterations of the construction. Since (i) only ensures the existence of an $x$ that satisfies $\varphi_i(p_1, \ldots, p_{m-1}, x)$ for any $i$, $1 \leq i \leq n$, we only need to add one $x$ for every set of parameters but $H_i(u_1, \dots, u_{m-1})$ can be arbitrarily large. Let $F$ be a~choice function on $\power{M'}$. Also let $h_i(p_1, \ldots, p_{m-1}) = F(H_i(p_1, \ldots, p_{m-1}))$ for $i$, where $1 \leq i \leq n$, which means that $h$ is a~function that outputs an $x$ that satisfies $\varphi_i(p_1, \ldots, p_{m-1}, x)$ for $i$ such that $1 \leq i \leq n$ and has minimal rank among all such sets. The induction step needs to be redefined to
\begin{equation}
M'_{i+1} = M'_i \cup \bigcup_{j=0}^n \{ h_j(p_1, \ldots, p_{m-1}): p_1, \ldots, p_{m-1} \in M'_i \}
\end{equation}
This way, the amount of elements added to $M'_{i+1}$ in each step of the construction is the same as the amount of $m$-tuples of parameters that yielded elements not included in $M'_i$. It is easy to see that if $M_0$ is finite, $M'$ is countable because it was constructed as a countable union of sets that are themselves at most countable. If $M_0$ is countable or larger, the cardinality of $M'$ is equal to the cardinality of $M_0$.\footnote{It can not be smaller because $|M'_{i+1}| \geq |M'_i|$ for every $i$. It may not be significantly larger because the maximum of elements added is the number of $n$-tuples in $M'_i$, which is of the same cardinality as $M'_i$.}
Therefore $|M'| \leq |M_0| \cdot \aleph_0$
\end{proof}

\begin{theorem}{(Lévy's first-order reflection theorem)}\label{theorem:first_order_reflection}\\
Let $\varphi(p_1, \ldots, p_n)$ be a~first-order formula.
\bce[(i)]
\item For every set $M_0$ there exists a set $M$ such that $M_0 \subset M$ and the following holds:
\begin{equation}
\varphi^M(p_1, \ldots, p_n) \iff \varphi(p_1, \ldots, p_n)\label{equation:levy_theorem_i}
\end{equation}
for every $p_1, \ldots, p_n \in M$.

\item For every set $M_0$ there is a~transitive set $M$, $M_0 \subset M$ such that the following holds:
\begin{equation}
\varphi^M(p_1, \ldots, p_n) \iff \varphi(p_1, \ldots, p_n)
\end{equation}
for every $p_1, \ldots, p_n \in M$.

\item For every set $M_0$ there is a limit ordinal $\lambda$ such that $M_0 \subset V_{\lambda}$ and the following holds:
\begin{equation}
\varphi^{V_{\lambda}}(p_1, \ldots, p_n) \iff \varphi(p_1, \ldots, p_n)
\end{equation}
for every $p_1, \ldots, p_n \in M$.

\item Assuming \emph{Choice}, for every set $M_0$ there is $M$ such that $M_0 \subset M$ and $|M| \leq |M_0| \cdot \aleph_0$ and the following holds:
\begin{equation}
\varphi^M(p_1, \ldots, p_n) \iff \varphi(p_1, \ldots, p_n)
\end{equation}
for every $p_1, \ldots, p_n \in M$.
\ece
\end{theorem}

\begin{proof}
% CO ty parametry? \varphi(p_1, \ldots, p_n) nebo jen \varphi ?
%Before we start, note that the following holds for any set $M$ if $\varphi$ is an atomic formula(x1 \in x2 or x1 = x2), as a direct consequence of relativisation to $M, \in$\footnote{See (\ref{def:relativization}). Also note that this only holds for relativization to $M, \in$, not $M, E$ for arbitrary $E$.}. 
%\begin{equation}
%\varphi \iff \varphi^M
%\end{equation}
Let's now prove (i) for given $\varphi$ via induction by complexity. We can safely assume that $\varphi$ contains no quantifiers besides ``$\exists$'' and no logical connectives other than ``$\neg$'' and ``$\et$''.
Let $\varphi_1, \ldots, \varphi_n$ be all subformulas of $\varphi$. Then there is a set $M$, obtained by the means of lemma (\ref{lemma:reflection_lemma}), for all of the formulas $\varphi_1, \ldots, \varphi_n$. 

Let's first consider atomic formulas in the form of either $x_1 = x_2$ or $x_1 \in x_2$. % preformulovat trochu
It is clear from relativisation\footnote{See (\ref{def:relativization}). This only holds for relativization to $M, \in \cap M \times M$, not $M, R$ for an arbitrary $R$.} that (\ref{equation:levy_theorem_i}) holds for both cases, $(x_1 = x_2)^M \iff (x_1 = x_2)$ and $(x_1 \in x_2)^M \iff (x_1 \in x_2)$.

\

We now want to verify the inductive step. First, take $\varphi = \neg \varphi'$. From relativization, we get $(\neg \varphi')^M \iff \neg (\varphi'^M)$.
Because the induction hypothesis tells us that $\varphi'^M \iff \varphi'$, the following holds:
\begin{equation}
(\neg \varphi')^{M} \iff \neg (\varphi'^M) \iff \neg \varphi'
\end{equation}

The same holds for $\varphi = \varphi_1 \et \varphi_2$. From the induction hypothesis, we know that $\varphi_1^M \iff \varphi_1$ and $\varphi_2^M \iff \varphi_2$, which together with relativization for formulas in the form of $\varphi_1 \et \varphi_2$ gives us
\begin{equation}
(\varphi_1 \et \varphi_2)^M \iff \varphi_1^M \et \varphi_2^M \iff \varphi_1 \et \varphi_2
\end{equation}

\
% kde jsem tu najednou vzal parametry? TODO
Let's now examine the case when $\varphi = \exists x \varphi'(p_1, \ldots, p_n, x)$. The induction hypothesis tells us that $\varphi'^M(p_1, \ldots, p_n, x) \iff \varphi'(p_1, \ldots, p_n, x)$,
so, together with above lemma (\ref{lemma:reflection_lemma}), the following holds:
\begin{equation}
\begin{gathered}
\varphi(p_1, \ldots, p_n, x) \\
\iff \exists x \varphi'(p_1, \ldots, p_n, x) \\
\iff (\exists x \in M) \varphi'(p_1, \ldots, p_n, x) \\
\iff (\exists x \in M) \varphi'^M (p_1, \ldots, p_n, x) \\
\iff (\exists x \varphi'(p_1, \ldots, p_n, x))^M \\
\iff \varphi^M(p_1, \ldots, p_n, x)
\end{gathered}
\end{equation}
Which is what we wanted to prove for part (i). %\ref{equation:levy_theorem_i} holds for all subformulas $\varphi_1, \ldots, \varphi_n$ of a given formula $\varphi$.

\

%So far we have proven part $\bold{(i)}$ of this theorem for one formula $\varphi$. 
We now need to verify that the same holds for any finite number of formulas $\varphi_1, \ldots, \varphi_n$. 
This has in fact been already done since lemma (\ref{lemma:reflection_lemma}) gives us a set $M$ for any finite amount of formulas and given $M_0$. We can therefore find a set $M$ for the union of all of their subformulas. When we obtain such $M$, it should be clear that it also reflects every formula in $\varphi_1, \ldots, \varphi_n$.

\

Since $V_\lambda$ is a~transitive set, by proving $\bold{(iii)}$ we also satisfy $\bold{(ii)}$. To do so, we only need to look at part $\bold{(ii)}$ of lemma (\ref{lemma:reflection_lemma}). All of the above proof also holds for $M = V_lambda$. 

To finish part $\bold{(iv)}$, we take $M$ of size $\leq |M_0| \cdot \aleph_0$, which exists due to part $\bold{(iii)}$ of lemma (\ref{lemma:reflection_lemma}), the rest being identical.
\end{proof}

% TODO existuje jich dokonce club set!
% viz http://ozark.hendrix.edu/~yorgey/settheory/15-reflection-principle.pdf


\

Let $\sf{S}$ be a set theory defined in (\ref{def:s}), for $\sf{ZFC}$ see definition (\ref{def:zfc}).

% ACHTUNG
% Sm je definovany jako konjunkce vsech relativizovanych axiomu, takze to je kruh...
% presunout do contemporary restatement?
% viz Levy str. (224)
% todo s/u/V_lambda
% DRAKE!!! ch.3, ch.4 dukaz v_alfa models ZFC pro limitni alfa bez nekonecna etc
% Drake dokazuje reflexi v ch.3 par.6.3

The two following lemmas are based on \cite{DrakeBook}[Chapter 3, Theorem 1.2].
\begin{lemma}\label{lemma:extensionality_in_transitive} % Drake ch.3 Theorem 1.2
If $M$ is a transitive set, then $M \models \mbox{\emph{Extensionality}}$.
\end{lemma}

\begin{proof}
Given a transitive set $M$, we want to show that the following holds.
\beq
M \models \forall x, y (x = y \iff \forall z (z \in x \iff z \in y))
\eeq % TODO pozor na definici splnovani!
% From satisfaction, we get that for every $x$, $y$, the following holds % neni v definici splnovani, ze volny musej platit vsechny?
Given arbitrary $x, y \in M$, we want to prove that $M \models (x = y \iff \forall z (z \in x \iff z \in y))$.
This is equivalent to % ref na definici pravdy
$M \models x = y \mbox{ iff } M \models \forall z(z \in x \iff z \in y)$, which is the same as $x = y \mbox{ iff } M \models \forall z(z \in x \iff z \in y)$.

So all elements of $x$ are also elements of $y$ in $M$, and vice versa. Because $M$ is transitive, all elements of $x$ and $y$ are in $M$, so $M \models \forall z(z \in x \iff z \in y)$ holds iff $x$ and $y$ contain the same elements and are therefore equal.
\end{proof}

\begin{lemma}\label{lemma:foundation_in_transitive}
If $M$ is a transitive set, then $M \models \mbox{\emph{Foundation}}$.
\end{lemma}

\begin{proof}
We want to prove the following:
\beq
M \models \forall x (x \neq \emptyset \then (\exists y \in x) (x \cap y = \emptyset))
\eeq

Given an arbitrary non-empty $x \in M$ let's show that $M \models (\exists y \in x) (x \cap y = \emptyset)$.

Because $M$ is transitive, every element of $x$ is an element of $M$. Take for $y$ the element of $x$ with the lowest rank\footnote{Rank is defined in (\ref{def:rank}).}. It should be clear that there is no $z \in y$ such that $z \in x$, because then $rank(z) < rank(y)$, which would be a contradiction.
\end{proof}

Let $\sf{S}$ be a set theory as defined in (\ref{def:s}). 
\begin{lemma}\label{lemma:scm_s_is_limit}\
The following holds for every $\lambda$.
\begin{equation}
\mbox{``$\lambda$ is a limit ordinal''} \then V_\lambda \models \sf{S} % proc ne?
\end{equation}
\end{lemma}

\begin{proof}
Given an arbitrary limit ordinal $\lambda$, we will verify the axioms of \sf{S} one by one.
\bce[(i)]
\item \emph{The existence of a set} comes from the fact that $V_\lambda$ is a non-empty set because limit ordinal is non-zero by definition.
% Jech 12.10, 12.11

\item \emph{Extensionality} holds from (\ref{lemma:extensionality_in_transitive}).

\item \emph{Foundation} holds from (\ref{lemma:foundation_in_transitive}).

\item \emph{Union}:\\ % TODO kontrola % separatni lemma, ze plati v kazdem V_\alpha?
% (see (\ref{def:union}))
%\begin{equation}
%\forall x \exists y \forall z (z \in y \iff \exists q( z \in q \et q \in x))
%\end{equation}
Given any $x \in V_\lambda$, we want verify that $y = \bigcup x$ is also in $V_\lambda$. Note that $y = \bigcup x$ is a $\Delta_0$-formula.
\beq
y = \bigcup x \iff (\forall z \in y)(\exists q \in x) z \in q \et (\forall z \in x)(\forall q \in z) q \in y
\eeq
So by lemma (\ref{lemma:delta_0_absoluteness})
\beq
% y = \bigcup x \iff (y = \bigcup x)^{V_\lambda}
y = \bigcup x \iff V_\lambda \models y = \bigcup x
\eeq
% asi ok ^

\item \emph{Pairing}: \\ % TODO kontrola
% (see (\ref{def:pairing}))
%\beq
%\forall x, y \exists z \forall q (q \in z \iff q = x \lor q = y)
%\eeq
Given two sets $x, y \in V_\lambda$, we want to show that $z = \{x, y\}$ is also an element of $V_\lambda$.
\beq
z = \{x, y\} \iff x \in z \et y \in z \et (\forall q \in z)(q = x \lor q = y)
\eeq
So $z = \{x, y\}$ is a $\Delta_0$-formula, and thus by lemma (\ref{lemma:delta_0_absoluteness}) it holds that
\beq
%z = \{x, y\} \iff (z = \{x, y\})^{V_\lambda}
z = \{x, y\} \iff V_\lambda \models z = \{x, y\}
\eeq
% asi ok ^


\item \emph{Powerset}: \\
%\begin{equation}
%\forall x \exists y \forall z (z \subseteq x \iff z \in y)
%\end{equation}
Given any $x \in V_\lambda$, we want to make sure that $\power{x} \in V_\lambda$. Let $\varphi(y)$ denote the formula $y \in \power{x} \iff y \subset x$.
according to definition of subset (\ref{def:subset}), $y \subset x$ is $\Delta_0$, so for any given $x, y \in V_\lambda$, $y = \power{x} \iff V_\lambda \models y = \power{x}$.
Because $\lambda$ is limit and $rank(\power{x}) = rank(x)+1$, if $\power{x} \in V_\lambda$ for every $x \in V_\lambda$.
% cajk

\item \emph{Specification}: \\ %zkontrolovat, podle Draka, pripadne Jech 12.11
Given a first-order formula $\varphi$, we want to show the following:
\beq
V_\lambda \models \forall x, p_1, \ldots, p_n, \exists y \forall z (z \in y \iff z \in x \et \varphi(z, p_1, \ldots, p_n))
\eeq
Given any $x$ along with parameters $p_1, \ldots, p_n$ in $V_\lambda$, we set
\beq
y~=~\{z~\in~x~:~\varphi^{V_\lambda}(z, p_1, \ldots, p_n)\}
\eeq
From transitivity of $V_\lambda$ and the fact that $y \subset x$ and $x \in V_\lambda$, we know that $y \in V_\lambda$, 
so $V_\lambda \models \forall z (z \in y \iff z \in x \et \varphi(z, p_1, \dots, p_n))$.
\ece
\end{proof}

% The lemma we've just proven shows how important are the axioms of \emph{Infinity} and \emph{Replacement} for \sf{ZFC}. If we only add \emph{Infinity}, we don't get above $V_{\omega+\omega}$\footnote{TODO citation needed!}, so replacements if needed to 

% ==================================== POTUD zkontrolovano ======================================================


%TODO citace! http://ozark.hendrix.edu/~yorgey/settheory/13-SI.pdf
% http://ozark.hendrix.edu/~yorgey/settheory/12-relative-consistency-2.pdf 
% -- jako konzistence ZF-reg -- ukazeme ze V jakozto sjednoveni V_alpha je model?

% TODO v kanamorim? co dukaz?
%\begin{definition}{(First-Order Reflection Schema)}\\ % wtf, pro jedu nebo kolik formuli?
%Let $\varphi_1, \ldots, \varphi_n$ be first-order formulas in the language of set theory.
%For each set $M_0$ there is such set $M$ that $M_0 \subset M$ and the following holds for every $i$, $1 \leq i \leq n$:
%\begin{equation}\label{equation:refl_lemma_i}
%\exists x \varphi_i(p_1, \ldots, p_{m-1}, x) \then (\exists x \in M) \varphi_i(p_1, \ldots, p_{m-1}, x)
%\end{equation}
%for every $p_1, \ldots, p_{m-1} \in M$.
%\end{definition}
% elegantnejsi by bylo formulovat to pro jednu formuli a pak ukazat, ze se to totez.

\begin{definition}{(First-Order Reflection Schema)}\label{def:first_order_reflection}\\ % Jednu
%Let $\varphi$ be a first-order formula in the language of set theory.
% and a set $M_0$ there is such set $M$ that $M_0 \subseteq M$ and the following holds for every $p_1, \ldots, p_n \in M$:
For every first-order formula $\varphi$, the following is an axiom:
\begin{equation}
\forall M_0 \exists M (M_0 \subseteq M \et (\varphi(p_1, \ldots, p_n) \iff \varphi(p_1, \ldots, p_n)^M)) % drive M \models \varphi, ale to vypada divne
\end{equation}
We will refer to this axiom schema as \emph{First-order reflection}.
\end{definition}

% TODO lemma, obecnejsi reflexe pro vic formuli?
% TODO staci vzit uzaver obou fli a udelat konjunkci

% pozor na to, ze jsme dokazali zdanlive silnejsi tvrzeni pro n formuli a ted predpokladame jenom jednu.
% zkontroluj ze to tak je a napis ze silnejsi vysledek a slabsi predpoklad jsou cool, jsou ve skutecnostni ekvivalentni

Let \emph{Infinity} and \emph{Replacement} be as defined in (\ref{def:infinity}) and (\ref{def:replacement}) respectively.

\begin{theorem}\label{theorem:levy_equivalence_contemporary}
\emph{First-order reflection} is equivalent to \emph{Infinity} $ \et $ \emph{Replacement} under $\sf{S}$.
\end{theorem}
% lemma:model_of_s?
\begin{proof}
Since (\ref{theorem:first_order_reflection}) already gives us one side of the implication, we are only interested in showing the converse which we shall do in two parts:

$\bold{\emph{First-order reflection} \then \emph{Infinity}}$
This is done exactly like (\ref{theorem:n0_implies_infinity}). We pick for $\varphi$ the formula $(\forall y \in x)(y \cup \{y\} \in x)$, $M_0 = \{\emptyset\}$. From (\ref{def:first_order_reflection}), there is a set $M$ that satisfies $\varphi$, so there is an inductive set. We have picked $M_0$ so that $\emptyset \in M$ obviously holds and $M$ is the witness for 
\beq
\exists x(\emptyset \in x \et (\forall y \in x)(y \cup \{y\} \in x))
\eeq
which is exactly (\ref{def:infinity}).

\

$\bold{\emph{First-order reflection} \then \emph{Replacement}}$

%Given a~formula $\varphi(x, y, p_1, \ldots, p_n)$, we can suppose that if it holds for given $x, y, p_1, \ldots, p_n$, it is reflected in a set $M$ \footnote{Which means that for $x, y, p_1, \ldots, p_n \in M$, $\varphi^M(x, y, p_1, \ldots, p_n) \iff \varphi(x, y, p_1, \ldots, p_n)$.}
%What we want to obtain is the following:
%\begin{equation}
%\begin{gathered}
%\forall x, y, z (\varphi(x, y, p_1, \ldots, p_n) \et \varphi(x, z, p_1, \ldots, p_n) \then y = z) \then\\
%\then \forall X \exists Y \forall y\ (y \in Y \iff \exists x (\varphi(x, y, p_1, \ldots, p_n) \et x \in X ))
%\end{gathered}
%\end{equation}

%TODO OMG FIX! Drake nebo jech nebo Kanamori!
Let's first point out that while \emph{First-order reflection} gives us a set for one formula, we can generalize it to hold for any finite number of formulas. We will show how is it done for two formulas, which is what we will use in this proof. Given two first-order formulas $\varphi, \psi$, we can suppose that there are formulas $\varphi'$ and $\psi'$ that are equivalent to $\varphi$ and $\psi$ respectively, but their free variables are different \footnote{This is plausible since we can for example substitute all free variables in $\varphi'$ for $x_0, x_2, x_4, \ldots$ and use $x_1, x_3, x_5, \ldots$ for free variables in $\psi'$, the resulting formulas will be equivalent.}. Let $\xi = \varphi \et \psi$, given any $M_0$, we can find a $M$ such that $\xi \iff \xi^M$. It is easy to see that from relativisation, the following holds:
\beq
\varphi \et \psi \iff \varphi' \et \psi' \iff \xi \iff \xi^M \iff (\varphi' \et \psi')^M \iff \varphi'^M \et \psi'^M \iff \varphi^M \et \psi^M
\eeq

Now given a function $\varphi(x, y)$, we know from \emph{First-order reflection} that for every $M_0$, there is a set $M$ such that $M_0 \subseteq M$ and both
\beq
(\forall x,y \in M)(\varphi(x, y) \iff \varphi^M(x, y))
\eeq 
and
\beq
(\forall x, y \in M)(\exists y \varphi(x, y) \iff (\exists y \varphi(x, y))^M)
\eeq 
hold, the latter being equivalent to 
\beq
(\forall x, y \in M)(\exists y \varphi(x, y) \iff (\exists y \in M) \varphi^M(x, y))
\eeq
Therefore 
\beq
(\forall x, y \in M)(\exists y \varphi(x, y) \iff (\exists y \in M) \varphi(x, y))
\eeq
holds too.
That means that we have a set $M$ such that for every $x \in M$, if $\varphi$ is defined for $x$, $(\exists y \in M) \varphi(x, y)$. 

To show that \emph{Replacement} holds for this particular $\varphi$, we need to verify that given a set $M_0$, $M'_0 = \{ y : (\exists x \in M_0) \varphi(x, y)\}$ is also a set. But since $M_0 \subseteq M$ and because given any $x \in M$, there is $y \in M$ satisfying $\varphi(x, y)$, the following is a set due to \emph{Specification}:
\beq
M'_0 = \{ y : (\exists x \in M_0) \varphi(x, y)\} = \{ y \in M : (\exists x \in M_0) \varphi(x, y)\}
\eeq
% tranzitivita? V_\lambda?

%We also know that $x, y \in M$, in other words for every $X$, $Y = \{y : \varphi(x, y, p_1, \ldots, p_n)\}$ and we know that $X \subset M$ and $Y \subset M$, which, together with the specification schema implies that $Y$, the image of $X$ over $\varphi$, is a~set.
\end{proof}

\

% Stronger reflection: 
% a) any number of formulas
% b) there is a club set of $M$s.
% asi do dalsi kapitoly

% co s tema $\emph{Reflection}$ / reflection / Reflection ??
We have shown that $\emph{Reflection}$ for first-order formulas, \emph{First-order reflection} is a~theorem of $\sf{ZFC}$.%, which means that it won't yield us any large cardinals. 
We have also shown that it can be used instead of the \emph{Infinity} and \emph{Replacement} scheme, but $\sf{ZFC}\ +\ \emph{First-order reflection}$ is a~conservative extension of $\sf{ZF}$. Besides being a~starting point for more general and powerful statements, it can be used to show that $\sf{ZF}$ is not finitely axiomatizable. This follows from the fact that \emph{Reflection} gives a~model to any consistent finite set of formulas. % or their conjunction? % nedokazal jsem verzi pro n formuli!
So if $\varphi_1, \ldots, \varphi_n$ would be the axioms of $\sf{ZFC}$, $\emph{Reflection}$ would prove that every model of $\sf{ZFC}$ contains a smaller model of $\sf{ZFC}$, which would in turn contradict the Second Gödel's Theorem\footnote{See chapter \ref{section:inaccessibility} for further details.}.

It is also worthwhile to note that, in a~way, Reflection is dual to compactness. 
% ref http://www.helsinki.fi/sls2015/materials/Fontanella%20Scandinavian%20Summer%20School.pdf
Compactness says that given a set of sentences, if every finite subset yields a~model, so does the whole set. Reflection, on the other hand, says that while the whole set has no model in the underlying theory, every finite subset has a model.

Furthemore, $\emph{Reflection}$ can be used in ways similar to upward Löwenheim–Skolem theorem.
Since Reflection extends any set $M_0$ into a~model of given formulas $\varphi_1, \ldots, \varphi_n$, we can choose the lower bound of the size of $M$ by appropriately choosing $M_0$.

In the next section, we will try to generalize \emph{Reflection} in a~way that transcends $\sf{ZF}$ and finally yields some large cardinals.
\newpage
% =============================================================

% \section{Reflection And Large Cardinals}
\begin{comment} % ======================================================================== //
In this chapter we aim to examine stronger reflection properties in order to reach cardinals unavailable in $\sf{ZFC}$. Like we said in the first chapter, 
the variety of reflection principles comes from the fact that there are many way to formalize ``properties of the universal class''. It is not always obvious what properties hold for $V$ because, as Tarski
has shown, there is no way to formalize satisfaction for proper classes. We have shown that reflecting properties as first-order formulas doesn't allow us to leave $\sf{ZFC}$. We will broaden the class of admissible properties to be reflected and see whether there is a~natural limit in the height or width on the reflected universe and also see that no matter how far we go, the universal class is still as elusive as it is when seen from $\sf{S}$. That is because for every process for obtaining larger sets such as for example the powerset operation in $\sf{ZFC}$, this process can't reach $V$ and thus, from reflection, there is an initial segment of $V$ that can't be reached via said process.

To see why this is important, let's dedicate a few lines to the intuition behind the notions of limitness, regularity and inaccessibility in a manner strongly influenced by \cite{Infinity_in_mind}. To see why limit and strongly limit cardinals are worth mentioning, note that they are ``limit'' not only in a sense of being a supremum of an ordinal sequence, they also show that a certain way of obtaining larger sets from smaller ones is limited. We will see that all of the alternatives offered in this thesis are in a sense limited. 
$\aleph_\lambda$ is a limit cardinal if there is no $\alpha$ such that $\aleph_{\alpha+1}=\aleph_\lambda$. Strongly limit cardinals point to the limits of the powerset operation. It has been too obvious so far, so let's look at the regular cardinals in this manner. Regular cardinals are those that cannot be\footnote{Assuming $\emph{Choice}$.}, expressed as a supremum of smaller amount of smaller objects\footnote{Just like $\omega$ can not be expressed as a supremum of a finite set consisting solely of finite numbers.}. More precisely, $\kappa$ is regular if there is no way to define it as a union of less than $\kappa$ ordinals, all smaller than $\kappa$. So unless there already is a set of size $\kappa$, \emph{Replacement} is useless in determining whether $\kappa$ is really a set. Note that assuming \emph{Choice}, successor cardinals are always regular, so most\footnote{All provable to exist in $\sf{ZFC}$.} limit cardinals are singular cardinals. So if one is traversing the class of all cardinals upwards, successor steps are still sets thanks to the powerset axiom while singular limit cardinals are not proper classes because they are suprema of images of smaller sets via \emph{Replacement}. Regular cardinals are, in a way, limits of how far can we get by taking limits of increasing sequences of ordinals obtained via $\emph{Replacement}$. 

In order to reach an inaccessible cardinal of size $\kappa$, one has to pass at least $\kappa$ limit ordinals. Them, to get to a Mahlo cardinal of size $\kappa$, one has to move past $\kappa$ inaccessible cardinals. This concept is then iterable for hyper-Mahlo cardinals, as we will see later in this section.

% That all being said, it is easy to see that no cardinals in $\sf{ZFC}$ are both strongly limit and regular because there is no way to ensure they are sets and not proper classes in $\sf{ZFC}$. The only exception to this rule is $\aleph_0$ which needs \emph{Infinity} to exist. % nase otazka je: proc omega a ne jine kardinaly?
% It should now be obvious why the fact that $\kappa$ is inaccessible implies that $\kappa = aleph_\kappa$.\footnote{This doesn't work backwards, the least fixed point of the $\aleph$ function is the limit of $\{\aleph_0,\ \aleph_{\aleph_0},\ \aleph_{\aleph_{\aleph_0}},\ \ldots \}$, it is singular since the sequence has countably many elements.}

% We will first examine the connection between reflection principles and (regular) fixed points of ordinal functions in a manner proposed by Lévy in \cite{Levy60a}. %We will also see that, like Lévy has proposed in the same paper, there is a meaningful way to extend the relation between $\sf{S}$ and $\sf{ZFC}$ into a hierarchy of stronger axiomatic set theories. 
% Those are the three lines of thinking that we will find are in fact different facets of the same gem, especially in the section devoted to Inaccessible and Mahlo cardinals.
% viz Shapiro, Stewart. 1987. “Principles of Reflection and Second-order Logic”. Journal of Philosophical Logic 16 (3). Springer: 309–33. http://www.jstor.org/stable/30227043.
% Reflections on \emph{Replacement} and Reflection: The axioms in a~structuralist setting (Geoffrey Hellman)
%TODO neco o tom, ze kdyz je reflexe formule, da se sama reflektovat?
% The above should make a clear picture of why $\emph{Infinity}$ is a specific case of $\emph{Reflection}$.
%TODO proc je Refl zaroven zobecneny replacement?

% TODO ze ``uplne totalni'' reflexe se zacykli a rozbije? nebo ne?


\end{comment} % ======================================================================== //

\subsection{Regular Fixed-Point Axioms}\label{sec:regular_fixed_points}
Lévy's article mentions various schemata that are not instances of reflection per se, but deal with fixed points of normal ordinal functions. We will introduce them and show that they are equivalent to \emph{First-Order Reflection}\footnote{For definition, see (\ref{def:first_order_reflection}).}.

% 
%
% This small chapter is dedicated to ?
%
% TODO regularni pevny body

\begin{lemma}{(Fixed-point lemma for normal functions)}\label{lemma:normal_fixed_point}\\
Let $f$ be a normal function defined for all ordinals\footnote{For the definition of normal function, see (\ref{def:normal_function}).}. Then all of the following hold:
\bce[(i)]
\item $\forall \lambda(\mbox{``$\lambda$ is a limit ordinal''} \then \mbox{``f($\lambda$) is a limit ordinal''})$
\item $\forall \alpha (\alpha \leq f(\alpha))$
\item $\forall \alpha \exists \beta (\alpha < \beta \et f(\beta) = \beta)$
\item The fixed points of $f$ form a closed unbounded class.\footnote{See (\ref{def:closed_class}.% TODO spis club class?
) for the definition of closed class, (\ref{def:unbounded_class}) for the definition of unboundedness.}
\ece
\end{lemma}

\begin{proof}
Let $f$ be a normal function defined for all ordinals.
\bce[(i)]
\item % Proof of (i)):\\
Suppose $\lambda$ is a limit ordinal. 
For an arbitrary ordinal $\alpha < \lambda$, the fact that $f$ is strictly increasing means that $f(\alpha) < f(\lambda)$ and for any ordinal $\beta$, 
satisfying $\alpha < \beta < \lambda$, $f(\alpha) < f(\beta) < f(\lambda)$. 
We know that there is such $\beta$ from limitness of $\lambda$.
Because $f$ is continuous and $\lambda$ is limit, $f(\lambda) = \bigcup_{\gamma < \lambda} f(\gamma)$.% and since $\beta < \lambda$, $f(\beta) < f(\lambda)$. 
That means that if $\lambda$ is limit, so is $f(\lambda)$.
%So we have found $f(\beta)$ such that $f(\alpha) < f(\beta) < f(\lambda)$, therefore $f(\lambda)$ is a limit ordinal.\\

\item This step will be proven using the transfinite induction.
Since $f$ is defined for all ordinals, there is an ordinal $\alpha$ such that $f(\emptyset) = \alpha$ and because $\emptyset$ is the least ordinal, (ii) holds for $\emptyset$.

Suppose (ii) holds for some $\beta$ form the induction hypothesis. It the holds for $\beta+1$ because $f$ is strictly increasing. 

For a limit ordinal $\lambda$, suppose (ii) holds for every $\alpha < \lambda$. (i) implies that $f(\lambda)$ is also limit, 
so there is a strictly increasing $\kappa$-sequence $\langle \alpha_0, \alpha_1, \ldots \rangle$ for some $\kappa$ such that $\lambda = \bigcup_{i<\kappa} \alpha_i$. Because $f$ is stricly increasing, the $\kappa$-sequence $\langle f(\alpha_0), f(\alpha_1), \ldots$ is also strictly increasing, the induction hypothesis implies that $\alpha_i \leq f(\alpha_i)$ for each $i \leq \kappa$. Thus, $\lambda \leq f(\lambda)$.

\item For a given ordinal $\alpha$, let there be an $\omega$-sequence $\langle \alpha_0, \alpha_1, \ldots \rangle$, 
such that $\alpha_0 = \alpha$ and $\alpha_{i+1} = f(\alpha_i)$ for each $i < \omega$.
This sequence is stricly increasing because so is $f$. 
Now, there's a limit ordinal $\beta = \bigcup_{i < \omega} \alpha_i$, we want to show that this is the fixed point. 
So $f(\beta) = f(\bigcup_{i < \omega} \alpha_i) = \bigcup_{i < \omega} f(\alpha)$ because $f$ is continuous. 
We have defined the above sequence so that $\beta$, $\bigcup_{i < \omega} f(\alpha) = \bigcup_{i < \omega} \alpha_{i+1}$, 
which means we are done, since $\bigcup_{i < \omega} \alpha_{i+1} = \bigcup_{i < \omega} \alpha_{i}  = \beta$.

% Todo http://math.stackexchange.com/questions/1865519/when-is-the-union-of-a-set-of-ordinals-a-limit-ordinal/1865527#1865527
\item The class of fixed points of $f$ is obviously unbounded by (iii).
It remains to show that it is closed, this is based on \cite{DrakeBook}, chapter 4. Let $Y$ be a non-empty set of fixed points of $f$ such that $\bigcup Y \not\in Y$. Since $f$ is defined on ordinals, $Y$ is a set of ordinals, so $\bigcup Y$ is an ordinal because a supremum of a set of ordinals is an ordinal%\footnote{TODO aspon dvema slovy?}
. $\bigcup Y$ is a limit ordinal. If it were a successor ordinal, suppose that $\alpha+1 = \bigcup Y$, then $\alpha \in \bigcup Y$, which means that there is some $x$ such that $\alpha \in x \in Y$. But the least such $x$ is $\alpha+1$, so $\bigcup Y \in Y$.
%We will show that $\bigcup Y$ is a limit ordinal because $Y$ doesn't have a maximal element.

Note that $\alpha < \bigcup Y iff \exists \xi \in Y (\alpha < \xi)$. Since $f$ is defined for all ordinals and $\bigcup Y$ is a limit ordinal, $f(\bigcup Y) = \bigcup_\alpha \in Y f(\alpha)$, but because $Y$ is a set of fixed points of $f$, $f(\bigcup Y) = \bigcup_\alpha \in Y f(\alpha) = \bigcup Y$, so $\bigcup Y$ is also a limit point of $Y$.
\ece
\end{proof}

% Lévy proposes in \cite{Levy60a} those axioms as equivalent to \emph{Reflection\textsubscript{1}}.

\begin{definition}{(\emph{Axiom Schema $M$\textsubscript{1}})}\label{def:levy_m}\\
``Every normal function defined for all ordinals has at least one inaccessible number in its range.''
\end{definition}
Lévy uses ``$M$'' to refer to this axiom but since we also use ``$M$'' for sets and models, for example in (\ref{def:first_order_reflection}), we will call the above axiom ``\emph{Axiom Schema $M$\textsubscript{1}}'' to avoid confusion.

%Now we will express \emph{Axiom $M$\textsubscript{1}} as a formula to make it clear that it is an axiom scheme and the same can be done with \emph{Axiom $M$\textsubscript{2}} as well as \emph{Axiom Schema $M$} introduced immediately afterwards. Since it is an axiom schema and we will later dive into second-order logic, we may also want to refer to \emph{Axiom $M$\textsubscript{2}} as opposed \emph{Axiom $M$\textsubscript{1}}, the former being a single second-order sentence obtained by the obvious modification of \emph{Axiom $M$\textsubscript{1}}.\footnote{Second-order set theory will be introduced in the next subsection.}

Let $\varphi(x, y, p_1, \ldots, p_n)$ be a first-order formula with no free variables besides $x, y, p_1, \ldots, p_n$. The following is equivalent to \emph{Axiom $M$\textsubscript{1}}.
\begin{equation}
\begin{gathered}
\mbox{``$\varphi$ is a normal function''} \et \forall x (x \in Ord \then \exists y(\varphi(x, y, p_1, \ldots, p_n))) \then\\
\then \exists y (\exists x \varphi(x, y, p_1, \ldots, p_n) \et cf(y) = y \et (\forall x \in \kappa)(\exists y \in \kappa)(x > y))
\end{gathered}
\end{equation}

\begin{definition}{(Axiom Schema $M$\textsubscript{2})}\\
``Every normal function defined for all ordinals has at least one fixed point which is inaccessible.''
\end{definition}

\begin{definition}{(Axiom Schema $M$\textsubscript{3})}\\
``Every normal function defined for all ordinals has arbitrarily great fixed points which are inaccessible.''
\end{definition}

Similar axiom is proposed in \cite{DrakeBook}.
\begin{definition}{(Axiom Schema $F$)}\label{def:axiom_f}\\
``Every normal function has a regular fixed point.''
\end{definition}

\begin{theorem}
\begin{equation}
\emph{Axiom $M$\textsubscript{1}} \iff \emph{Axiom $M$\textsubscript{2}} \iff \emph{Axiom $M$\textsubscript{3}} \iff \emph{Axiom Schema $F$}
\end{equation}
\end{theorem}
\
This is \emph{Theorem 1} in \cite{Levy60a}.\\
\begin{proof} % TODO check!
% It is clear that \emph{Axiom $M$\textsubscript{3}} is a stronger version of \emph{Axiom $M$\textsubscript{2}}, which is in turn a stronger version of both \emph{Axiom $M$\textsubscript{1}} and \emph{Axiom $F$\textsubscript{1}}, so the implication \emph{Axiom $M$\textsubscript{3}} $\then$ \emph{Axiom $M$\textsubscript{2}} $\then$ \emph{Axiom $M$\textsubscript{1}} is satisfied and \emph{Axiom $M$\textsubscript{2}} $\then$ \emph{Axiom $F$\textsubscript{1}} holds too.

% We will now make sure that  \emph{Axiom $M$\textsubscript{1}} $\then$  \emph{Axiom $M$\textsubscript{3}} also holds. 
%Let $f$ be a normal function defined for all ordinals. % such that there is $\varphi$, $f(x) = y \iff \varphi$ that satisfies \emph{Axiom $M$\textsubscript{1}}.
% Let $g$ be a normal function that counts the fixed points of $f$. Lemma (\ref{lemma:normal_fixed_point}) implies that there arbitrarily many fixed points of $f$, therefore $g$ is defined for all ordinals. Let there be another family of functions, $h_\alpha(\beta) = g(\alpha+\beta)$, obviously $h_\alpha$ is defined for all ordinals for every $\alpha \in Ord$ because so is $g$. Given an arbitrary ordinal $\gamma$, from \emph{Axiom $M$\textsubscript{1}} we can assume that there is an ordinal $\delta$ such that such that $h_\alpha(\delta) = \kappa$, where $\kappa$ is inaccessible. 
%But since $\kappa = g(\alpha+\delta)$, $\kappa$ is a fixed point of $f$. To show that there are arbitrarily many fixed points of $f$, notice that $\gamma$ is arbitrary and $h_\gamma$ is a normal function, so, by lemma (\ref{lemma:normal_fixed_point}), $(\forall \alpha \in Ord)(\alpha \leq f(\alpha)$, therefore $\gamma \leq \gamma + \alpha \leq \kappa)$, in other words, there is $\kappa$ above an arbitrary ordinal $\gamma$.
\end{proof}

But how do those schemata relate to reflection? 
Let's introduce a stronger version of \emph{First-order reflection schema} from the previous chapter to see it more clearly. % TODO ref na minuly refl schema
But in order to do this, we must first establish the inaccessible cardinal.


% TODO nevime co je inaccessible! 
% zacit kapitolu tady?
\subsection{Inaccessible Cardinal}\label{sec:inaccessible}
\begin{definition}
An uncountable cardinal $\kappa$ is \emph{inaccessible} iff it is \emph{regular} and \emph{strongly limit}. We write $In(\kappa)$ to say that $\kappa$ is an inaccessible cardinal.
\end{definition}

An uncountable cardinal that is regular and limit is called a \emph{weakly limit cardinal}, we will only use the (strongly) inaccessible cardinal, but most of the results are similar, including higher types of ordinals that will be presented later in this chapter.

\begin{theorem}
Let $\kappa$ be an inaccessible cardinal.
\beq
V_\kappa \models \sf{ZFC}
\eeq
\end{theorem}

We will prove this theorem in a way similar to \cite{KanamoriBook}.
\begin{proof}
Most of this is already done in lemma (\ref{lemma:scm_s_is_limit}), we only need to verify that \emph{Replacement} and \emph{Infinity} axioms hold in $V_\kappa$.

\emph{Infinity} holds because $\kappa$ is uncountable, so $\omega \in V_\kappa$.

To verify \emph{Replacement}, let $x$ be an element of $V_\kappa$ and $f$ a function from $x$ to $V_\kappa$. Let $y = \{z \in V_\kappa : (\exists q \in x) f(q) = z \}$, so $y \subset V_\kappa$, it remains to show that $y \in V_\kappa$. Because $f$ is a function, we know that $|y| \leq |x| \leq \kappa$. But since $\kappa$ is regular, $\{rank(z) : z \in y\} \subseteq \alpha$ for some $\alpha < \kappa$, and so $x \in V_{\alpha+1} \subseteq V_\kappa$. Therefore $y \in V_\kappa$.
\end{proof}

\begin{definition}{(Inaccessible Reflection Schema)}\label{def:inaccessible_reflection}\\ % Jednu
For every first-order formula $\varphi$, the following is an axiom:
\beq
\forall M_0 \exists \kappa (M_0 \subseteq V_\kappa \et In(\kappa) \et (\varphi(p_1, \ldots, p_n) \iff \varphi(p_1, \ldots, p_n)^{V_\kappa}))
\eeq
We will refer to this axiom schema as \emph{Inaccessible reflection schema}.
\end{definition}
% Uvozovky?

We have added the requirement that $\alpha$ is inaccessible, which trivially means that there is an inaccessible cardinal. By taking appropriate $M_0$, it can be shown that in a theory that includes the \emph{Inaccessible reflection schema}, there is a closed unbounded class of inaccessible cardinals. Since we know that for an inaccessible $\kappa$, $V_\kappa$ is a model of \sf{ZFC}, \emph{Inaccessible reflection schema} is equivalent to
\beq
\forall M_0 \exists \kappa (M_0 \subseteq V_\kappa \et V_\kappa \models \sf{ZFC} \et (\varphi(p_1, \ldots, p_n) \iff \varphi(p_1, \ldots, p_n)^{V_\kappa}))
\eeq
because we have proven in the last section that for an inaccessible $\kappa$, $V_\kappa \models \sf{ZFC}$. % TODO ref

\begin{theorem}
\emph{Inaccessible reflection schema} is equivalent to \emph{Axiom schema $F$}.
\end{theorem}

This is Theorem 4.1 in chapter four of \cite{DrakeBook}.
\begin{proof}
Let's start by showing that \emph{Inaccessible reflection schema} implies \emph{Axiom schema $F$}. 
It should be clear that we can reflect two formulas to a single set, just form a new formula as a conjunction of universal closures of the two.

Given a normal function $f$ defined for all ordinals, we want to show that it has a regular fixed point. 
%Let $\varphi_1$ be the formula $f(\gamma) = \delta$ and let $\varphi_2$ be $\forall \gamma \exists \delta f(\gamma) = \delta$. 
For any ordinal $\alpha$, there is an ordinal $\kappa$ such that 
\beq
\alpha < \kappa \et In(\kappa) \et (\forall \gamma, \delta \in V_\kappa)(f(\gamma) = \delta \iff (f(\gamma) = \delta)^{V_\kappa})
\eeq
and
\beq
\alpha < \kappa \et In(\kappa) \et \forall \gamma \exists \delta (f(\gamma) = \delta) \iff (\forall \gamma \exists \delta f(\gamma) = \delta)^{V_\kappa}
\eeq
Since $V_\kappa$ is the set of all sets of rank less than $\kappa$ and since every ordinal is the rank of itself, there is an inaccessible ordinal $\kappa$ such that
\beq
\forall \gamma < \kappa \exists \delta < \kappa (f^{V_\kappa} (\gamma) = \delta)\label{eq:reflected_function}
\eeq
We also know that $f(\gamma) = \delta \iff (f(\gamma) = \delta)^{V_\kappa}$. 
Now since $\kappa$ is a limit ordinal and $f$ is continuous we get
\beq
f(\kappa) = \bigcup_{\gamma < \kappa} f^{V_\kappa}(\gamma) = \bigcup_{\gamma < \kappa} f(\gamma)\mbox{.}
\eeq
From (\ref{eq:reflected_function}) and the fact that $f$ is increasing, we know that $\kappa \leq \bigcup_{\gamma < \kappa} f(\gamma) \leq \kappa$. Therefore $\kappa$ is an inaccessible fixed point of $f$.

For the opposite direction, it is enough to show that since there is an inaccessible cardinal from \emph{Axiom schema $F$}, there is a set $V_\kappa$ for some inaccessible cardinal $\kappa$, which is thus a model of of \sf{ZFC}.  TODO
% TODO  reflexe mi nepomuze, ma to platit prave ve V_\kappa. nebo staci vzit limitu, kdyz existuje club set takovych modelu?
\end{proof}


\begin{comment} % ============================================================

\begin{definition}{($\sf{ZMC}$)}\\
We will call $\sf{ZMC}$ an axiomatic set theory that contains all axioms and schemas of $\sf{ZFC}$ together with the schema \emph{Axiom $M$\textsubscript{1}}.
\end{definition}
We have decided to call it $\sf{ZMC}$, because Lévy uses $\sf{ZM}$, derived from $\sf{ZF}$, which is more intuitive, but we also need the axiom of choice, thus, $\sf{ZMC}$.


The fact, that in $\sf{ZFC}$, the above \emph{Axiom M} is equivalent to \emph{Reflection\textsubscript{1}} as defined in (\ref{def:first_order_reflection}) is proven in \cite{Levy60a}[Theorem 3].

\begin{theorem}\label{theorem:levy_m_iff_reflection}
\begin{equation}
\sf{ZFC} \models \mbox{\emph{Axiom M}} \iff \mbox{\emph{Reflection\textsubscript{1}}}
\end{equation}
\end{theorem}

\subsection{Inaccessibility}\label{section:inaccessibility}

\begin{definition}{(Weak Inaccessibility)}\label{def:weakly_inaccessible}
An uncountable cardinal $\kappa$ is \emph{weakly inaccessible} iff it is \emph{regular} and \emph{limit}.
\end{definition}
\begin{definition}(Inaccessibility)\label{def:inaccessible}
An uncountable cardinal $\kappa$ is \emph{inaccessible} iff it is \emph{regular} and \emph{strongly limit}.
\end{definition}

\

We will now show that the above notion is equivalent to the definition Lévy uses in \cite{Levy60a}, which is, in more contemporary notation, the following:
\begin{theorem}\label{theorem:inaccessible_models_zfc}
The following are equivalent:
\bce
\item $\kappa$ in inaccessible
\item $\langle V_\kappa, \in \rangle \models \sf{ZFC}$
\ece
\end{theorem}
% false theorem? :(
% 

\begin{proof}
We know that all the axioms except for \emph{Replacement} and \emph{Infinity} are satisfied in $V_\lambda$ for any limit ordinal $\lambda$ from lemma (\ref{lemma:scm_s_is_limit}).

Obviously, \emph{Infinity} holds in $V_\kappa$, since $\omega < \kappa$, so $V_\omega \in V_\kappa$.

To see how for a given formula $\varphi$, an instance replacement is obtained from an instance of reflection, refer to the appropriate part of theorem (\ref{theorem:levy_equivalence_contemporary}).

\

We will now show that if a~set is a~model of $\sf{ZFC}$, it is in fact an inaccessible cardinal. So let $V_\kappa$ be a~model of $\sf{ZFC}$ which means that it is closed under the powerset operation, in other words:
\begin{equation}
\forall \lambda (\lambda < \kappa \then 2^{\lambda} < \kappa)
\end{equation}
which is exactly the definition of strong limitness. $\kappa$ is regular from the following argument by contradiction:\\
Let us suppose for a~moment that $\kappa$ is singular. Therefore there is an ordinal $\alpha < \kappa$ and a~function $F:\ \alpha \then \kappa$ such that the range of $F$ in unbounded in $\kappa$, in other words, $F[\alpha] \subseteq V_\kappa$ and $sup(F[\alpha]) = kappa$. In order to achieve the desired contradiction, we need to see that it is the case that $F[\alpha] \in V_\kappa$. Let $\varphi(x, y)$ be the following first-order formula:
\begin{equation}
F(x)\ =\ y
\end{equation}
Then there is an instance of \emph{Replacement} that states the following:
\begin{equation}
\begin{gathered}
(\forall x, y, z(\varphi(x, y) \et \varphi(x, z) \then y\ =\ z)) \then \\
\then (\forall x \exists y \forall z (z \in y \iff \exists w (\varphi(w, z))))
\end{gathered}
\end{equation}
Which in turn means that there is a~set $y = F[\alpha]$ and $y \in V_\kappa$, which is the contradiction with $sup(y) = \kappa$ we are looking for.
\end{proof}

\

% zkontorluj jeslti jsme to dokazali
% pak si rekneme, ze jsme to dokazali abychom videli ze existence nedosazitelneho kardinalu neni dokazatelna v ZFC
We have transcended $\sf{ZFC}$, but that is just a~start. Naturally, we could go on and consider the next inaccessible cardinal, which is inaccessible with respect to the theory $\sf{ZFC} + \exists \kappa (V_\kappa \models \sf{ZFC})$. But let's try to find a faster way up, informally at first. 

Since we can find an inaccessible set larger than any chosen set $M_0$, it is clear that there are arbitrarily large inaccessible cardinals in $V$, they are ``unbounded''\footnote{The notion is formally defined for sets, but the meaning should be obvious.} in $V$. If $V$ were a cardinal, we could say that there are $V$ inaccessible cardinals less than $V$, but this statement of course makes no sense in set theory as is because $V$ is not a set. But being more careful, we could find a property that can be formalized in second-order logic and reflect it to an initial segment of $V$. That would allow us to construct large cardinals more efficiently than by adding inaccessibles one by one. The property we are looking for ought to look like something like this (the following statement is not a mathematical  statement in a strict sense):
\begin{equation}
\begin{gathered}
\kappa \mbox{ is an inaccessible cardinal and}\\
\mbox{there are }\kappa\mbox{ inaccessible cardinals }\mu\ <\ \kappa
\end{gathered}
\end{equation}
This is in fact a fixed-point type of statement. We shall call those cardinals hyper-inaccessible. Now consider the following definition.

\begin{definition}{$0$-inaccessible Cardinal}\\
A cardinal $\kappa$ is $0$-inaccessible if it is inaccessible.
\end{definition}
We can define \emph{$\alpha$-weakly-inaccessible} cardinals analogously with the only difference that those are limit, not strongly limit.

\begin{definition}{$\alpha$-Inaccessible Cardinal}\label{def:alpha_inaccessible}\\
For any ordinal $\alpha$, $\kappa$ is called $\alpha$-inaccessible, if $\kappa$ is inaccessible and for each $\beta$ < $\alpha$, the set of $\beta$-inaccessible cardinals less than $\kappa$ is unbounded in $\kappa$.
\end{definition}

Because $\kappa$ is inaccessible and therefore regular, the number of $\beta$-inaccessibles below $\kappa$ is equal to $\kappa$. We have therefore successfully formalized the above vague notion of hyper-inaccessible cardinal into a hierarchy of $\alpha$-inaccessibles.

\

Let's now consider iterating this process over again. Since, informally, $V$ would be $\alpha$-inaccessible for any $\alpha$, this property of the universal class could possibly be reflected to an initial segment, the smallest of those will be the first hyper-inaccessible cardinal. Such $\kappa$ is larger than any $\alpha$-inaccessible since from regularity of $\kappa$, for given $\alpha\ <\ \kappa$, $\kappa$ is $\kappa$-th $\alpha$-hyper-inaccessible cardinal. It is in fact ``inaccessible'' via $\alpha$-inaccessibility.

\begin{definition}{Hyper-Inaccessible Cardinal}\\
$\kappa$ is called the hyper-inaccessible, also $0$-hyper-inaccessible, cardinal if it is $\alpha$-inaccessible for every $\alpha\ <\ \kappa$.
\end{definition}

\begin{definition}{$\alpha$-Hyper-Inaccessible Cardinal}\\
For any ordinal $\alpha$, $\kappa$ is called $\alpha$-hyper-inaccessible cardinal if for each ordinal $\beta\ <\ \alpha$, the set of $\beta$-hyper-inaccessible cardinals less the $\kappa$ is inbounded in $\kappa$.
\end{definition}

Obviously we could go on and iterate it ad libitum, yielding $\alpha$-hyper-$\ldots$-hyper-inaccessibles, but the nomenclature would be increasingly confusing. A smarter way to accomplish the same goal is carried out in the following section.

% =====================================================================================================================================

\subsection{Mahlo Cardinals}

While the previous chapter introduced us to a notion of inaccessibility and the possibility of iterating it ad libitum in new theories, there is an even faster way to travel upwards in the cumulative hierarchy, that was proposed by Paul Mahlo in his articles (see \cite{Mahlo11}, \cite{Mahlo12} and \cite{Mahlo13}) at the very beginning of the 20th century, and which can be easily reformulated using reflection.

\begin{theorem}\label{club_intersection} 
Let $\kappa$ be a regular uncountable cardinal. The intersection of fewer than $\kappa$ club subsets of $\kappa$ is a club set.
\end{theorem}
For the proof, see \cite[Theorem 8.3]{JechBook}

\begin{definition}{Weakly Mahlo Cardinal}\label{def:weakly_mahlo}\\
$\kappa$ is \emph{weakly Mahlo} $\iff$ it is a~weakly-inaccessible ordinal and the set of all regular ordinals less then $\kappa$ is stationary in $\kappa$
\end{definition}

\begin{definition}{Mahlo Cardinal}\label{def:mahlo_cardinal}\\
$\kappa$ is a \emph{Mahlo Cardinal} iff it is an inaccessible cardinal and the set of all inaccessible ordinals less then $\kappa$ is stationary in $\kappa$.
\end{definition}
% It is interesting to note, that weakly-Mahlo cardinals are fixed points of $\alpha$-weakly inaccessible cardinals, so if $\kappa$ is weakly mahlo,  .. viz Kanamori Proposition 1.1

It should be clear that a cardinal $\kappa$ is Mahlo iff $V_\kappa$ is a models of $\sf{ZFC} + \mbox{\emph{Axiom Schema $M$}}$.

Analogously, 
\begin{definition}{$\alpha$-Mahlo Cardinal}\label{def:alpha_mahlo_cardinal}\\
$\kappa$ is a \emph{$\alpha$-Mahlo Cardinal} iff it is an $\alpha$-inaccessible cardinal and the set of all $\alpha$-inaccessible ordinals less then $\kappa$ is stationary in $\kappa$.
\end{definition}

In other words, $\kappa$ is a (weakly-)Mahlo cardinal if it is (weakly-)inaccessible and every club set in $\kappa$ contains an (weakly-)inaccessible cardinal. Alternatively, a cardinal is (weakly-)Mahlo if it is (weakly-)inaccesible and there are $\kappa$ (weakly-)inaccessibles below $\kappa$.
% viz http://euclid.colorado.edu/~monkd/m6730/gradsets12.pdf
%Thus a~Mahlo cardinal $\kappa$ is not only inaccessible, but also has $\kappa$ inaccessibles below it.

%\cite{DrakeBook}



In a fashion similar to hyper-inaccessible cardinals, one can define hyper-Mahlo cardinals as well as hyper-hyper-Mahlo cardinals and so on.

To se why we need to mention Mahlo Cardinals, notice that while an inaccessible cardinal reflects any first-order formula, a Mahlo cardinal reflects inaccessibility, so it, in a sense, reflects reflection. Hyper-Mahlo cardinals then stand for reflecting reflecting reflection and so on.

Mahlo cardinals are also interesting from a different point of view. If we wanted to reach large cardinal from below via fixed-point argument, we don't get any higher.
% TODO proc se vys nedostaneme pevnyma bodama?
%TODO co s nima dela Jech?
% TODO Drake p.121!!

% TODO $\kappa$ is hyper-Mahlo iff $\kappa$ is inaccessible and the set $\{\lambda < \kappa : \lambda\mbox{ is Mahlo}\}$ is stationary in $\kappa$. to je to samy jako $\alpha$-Mahlo, ne?

% TODO viz https://en.wikipedia.org/wiki/Mahlo_cardinal#Mahlo_cardinals_and_reflection_principles

% Note that Mahlo cardinals were first described in 1911, almost 50 years before Lévy's reflection, which was heavily inspired by them.

% `` We also state the appropriate generalization for greatly Mahlo cardinals.'' % viz http://arxiv.org/abs/math/9204218

%TODO veta na zaver, shrnuti

%sjednotil \then a~\implies
% =====================================================================================================================================
% \newpage
\subsection{Second-Order Reflection}
Let's try a different approach in formalizing reflection. We have seen that reflecting individual first-order formulas doesn't even transcend $\sf{ZFC}$, we have examined what can be done with axiom schemas. The aim of this chapter is to examine second-order formulas as possible axioms. Note that second-order variables (which will be established as type 2 variables later in the text) are subcollections of the universal class, but so are functions and relations. So first-order axiom schemata can also be interpreted as formulas with free second-order variables, which quantify over first-order variables only, we only need to customize the underlying theory accordingly. For example, the satisfaction relation was so far defined for first-order formulas only, but we will deal with that in a moment. Also note that by rewriting \emph{\emph{Replacement}} and \emph{Specification} to single axioms, $\sf{ZFC}$ becomes finitely axiomatizable, which in turn means that the reflection theorem as stated in section (\ref{sec:first_order}) does not hold for higher-order theories because of Gödel's second incompleteness theorem. We will explore stronger axioms of reflection instead.

Let us establish a formal background first. We will now introduce higher-order formulas.

\begin{definition}{(Higher-Order Variables)}\label{def:higher_order_variables}\\
Let $M$ be a structure and $D$ it's domain. In first-order logic, variables range over individuals, that is, over elements of $D$. We shall call those \emph{type 1 variables} for the purposes of higher-order logic. Type 2 variables then range over collections, that is, the elements of $\power{D}$. Generally, type $n$ variables are defined for any $n \in \omega$ such that they range over $\mathscr{P}^{n-1}(D)$.
\end{definition}
We will use lowercase latin letters for type 1 variables for backwards compatibility with first-order logic, type 2 variables will be represented by uppercase letters, mostly $P, X, Y, Z$. If we ever stumble upon type 3 variables in this text, they shall be represented as $\mathscr{X}, \mathscr{Y}, \mathscr{Z}$ or in a similar font.

\begin{definition}{(Full Prenex Normal Form)}\label{def:pnf}\\
We say a formula is in the \emph{prenex normal form} if it is written as a block of quantifiers followed by a quantifier-free part.\\
We say a formula is in the \emph{full prenex normal form} if it is written in \emph{prenex normal form} and if there are type $n+1$ quantifiers, they are written before type $n$ quantifiers.
\end{definition}
It is an elementary that every formula is equivalent to a formula in the prenex normal form.


\begin{definition}{(Hierarchy of Formulas)}\label{def:analytical_hierarchy}\\
Let $\varphi$ be a formula in the prenex formal form.
\bce[(i)]
\item We say $\varphi$ is a $\Delta^0_0$-formula if it contains only bounded quantifiers.
\item We say $\varphi$ is a $\Sigma^0_0$-formula or a $\Pi^0_0$-formula if it is a $\Delta^0_0$-formula.
\item We say $\varphi$ is a $\Pi^{m+1}_0$-formula if it is a $\Pi^m_n$- or $\Sigma^m_n$-formula for any $n \in \omega$ or if it is a $\Pi^m_n$- or $\Sigma^m_n$-formula with additional free variables of type $m+1$.
\item We say $\varphi$ is a $\Sigma^m_0$-formula if it is a $\Pi^m_0$-formula.
\item We say $\varphi$ is a $\Sigma^m_n+1$-formula if it is of a form $\exists P_1, \ldots, P_i \psi$ for any non-zero $i$, where $\psi$ is a $\Pi^m_n$-formula and $P_1, \ldots, P_i$ are type $m+1$ variables.
\item We say $\varphi$ is a $\Pi^m_n+1$-formula if it is of a form $\forall P_1, \ldots, P_i \psi$ for any non-zero $i$, where $\psi$ is a $\Sigma^m_n$-formula and $P_1, \ldots, P_i$ are type $m+1$ variables.
\ece
\end{definition}

Now that we have introduced higher types of quantifiers, we will use it to formulate reflection. But first, let's make it clear how relativization works for higher-order quantifiers and type 2 parameters. Let $\alpha, \kappa$ be ordinals such that $\alpha < \kappa$, $R \subseteq V_\kappa$.
\begin{equation}
R^{V_\alpha} \defeq R \cap V_\alpha
\end{equation}
And let $\exists^{m}$ be a quantifier that ranges over type $m$ variables, let $P$ represent a type $m$ variable, let $\varphi$ be a type $m$ formula with the only free variable $P$.
\begin{equation}
(\exists P \varphi(P))^{V_\alpha} \defeq (\exists \power^(m-1){V_\alpha})\varphi^{V_\alpha}(P))
\end{equation}


\begin{definition}{(Reflection)}\label{def:reflection_2}\\
Let $\varphi(R)$ be a $\Pi^n_m$-formula with one free variable of type type 2 denoted $P$. We say $\varphi(R)$ reflects in $V_\kappa$ if for every $R \sub V_\kappa$ there is an ordinal $\alpha<\kappa$ such that the following holds:
\begin{equation}
\begin{gathered}
\mbox{If }(V_\kappa,\in, R)\models \varphi(R),\\
\mbox{ then }(V_\alpha,\in, R\cap V_\alpha) \models \varphi(R\cap V_\alpha).
\end{gathered}
\end{equation}
\end{definition}

This formalization of the notion of reflection allows us to describe Inaccessible and Mahlo cardinals more easily, which we will do in the following section. 

It is important to see, that while we can now reflect $\Pi^m_n$-formulas for arbitrary $m, n \in \omega$, they can only have type 2 free variables. 
This formalization of reflection can not be extended to higher-order parameters as is. This will be briefly reviewed in the next paragraph.

In order to extend reflection as a stated above in (\ref{def:reflection_2}), we need to make sure that given the domain of the structure, $V_\kappa$, we know what relativization to $V_\alpha$, $\alpha < \kappa$, means.
Since a type 3 parameters are collections of subcollections of $V_\kappa$ and we can already relativize subcollections of $V_\kappa$, this seems to be a reasonable way to extend relativization to type 3 parameters:
\begin{equation}
\mathscr{R}^{V_\alpha} = \{R^{V_\alpha} : R \in \mathscr{R} \}
\end{equation}
Where $R^{V_\alpha}$ is type 2 relativization, which is $R \cap V_\alpha$.

For an infinite ordinal $\kappa$, let
\begin{equation}
\mathscr{S} \defeq \{\{x \in \kappa : x \in \alpha \}:\alpha < \kappa \}
\end{equation}
then consider the following formula $\varphi(\mathscr{R})$ with one type 3 parameter $\mathscr{R}$:
\begin{equation}
\varphi(\mathscr{R}) = (\forall R \in \mathscr{R})(\mbox{``$R$ is unbounded in $\kappa$''})
\end{equation}

Even though $V_\kappa \models \varphi(\mathscr{S})$ holds, there's no $\alpha < \kappa$ for which $V_{\alpha} \models \varphi(\mathscr{S})$.

We will therefore stick to formulas with type 2 parameters. While there are ways to extend reflection for higher orders, it is beyond the scope of this thesis.
% ========================================================
\subsection{Indescribality}

Since this section talks about indescribability, this is how an ordinal is described according to Drake \cite[Chapter 9]{DrakeBook}.
\begin{definition}\
We say an ordinal $\alpha$ is described by a formula $\varphi(P_1, \ldots, P_n)$ with type 2 parameters $P_1, \ldots, P_n$ given iff
\begin{equation}
\langle V_\alpha, \in \rangle \models \langle \varphi(P_1, \ldots, P_n)
\end{equation}
but for every $\beta < \alpha$
\begin{equation}
\langle V_\beta, \in \rangle \not\models \varphi(P_1 \cap V_\beta, \ldots, P_n \cap V_\beta)
\end{equation}
\end{definition}

Drake then notes that the same notion can be established for sentences if the corresponding type 2 parameters are added to the language. Since the this approach is used by Kanamori in \cite{KanamoriBook}, we will stick to that too.\footnote{The first definition is included because it is more intuitive.}
\begin{definition}{(Describability)}\label{def:describability}\\
We say an ordinal $\alpha$ is described by a sentence $\varphi$ in the language $\mathscr{L}$ with relation symbols $P_1, \ldots, P_n$ given iff
\begin{equation}
\langle V_\alpha, \in, P_1, \ldots, P_n \rangle \models \varphi
\end{equation}
but for every $\beta < \alpha$
\begin{equation}
\langle V_\beta, \in, P_1 \cap V_\beta, \ldots, P_n \cap V_\beta \rangle \not\models \varphi
\end{equation}
\end{definition}

\begin{definition}{($\Pi^m_n$-Indescribable Cardinal)}\label{def:pi_mn_indescribable}
We say that $\kappa$ is $\Pi^m_n$-indescribable iff it is not described by any $\Pi^m_n$-formula.
\end{definition}
\begin{definition}{($\Sigma^m_n$-Indescribable Cardinal)}\label{def:sigma_mn_indescribable}
We say that $\kappa$ is $\Sigma^m_n$-indescribable iff it is not described by any $\Sigma^m_n$-formula.
\end{definition}

To see that this notion is based in reflection, note that for $\Pi^m_n$-formulas\footnote{This holds for $\Sigma^m_n$-formulas alike.}, a cardinal $\kappa$ is $\Pi^m_n$-indescribable iff every $\Pi^m_n$-formula reflects in $\kappa$ in the sense of definition (\ref{def:reflection_2}). Informally, can also view indescribability as a property held by the universe $V$, in the sense that every formula aiming to describe it in fact describes an initial segment, which is similar to a reflection principle, albeit stated informally.\footnote{Formally, we have to be once again careful with ``properties of $V$'' for the reasons mentioned in the introduction of this thesis. That's why this chapter only reflects sentences to models with additional relations.}

\

Since we are interested accessing cardinals from below via fixed points of normal functions, we will limit ourselves to $\Pi^1_n$-formulas, with the exception of measurable cardinal, that is included for context.

\

\begin{lemma}
Let $\kappa$ be a cardinal, the following holds for any $n \in \omega$. $\kappa$ is $\Pi^1_n$-indescribable iff $\kappa$ is $\Sigma^1_n+1$-indescribable
\end{lemma}

\begin{proof}
The forward direction is obvious, we can always add a spare quantifier over a type 2 variable to turn a $\Pi^1_n$ formula $\varphi$ into a $\exists P \varphi$ which is obviously a $\Sigma^1_n+1$ formula.\footnote{Note that unlike in previous sections, it is worth noting that $\varphi$ is now a sentence so we don't have to worry whether $P$ is free in $\varphi$.}

To prove the opposite direction, suppose that $V_\kappa \models \exists X \varphi(X)$ where $X$ is a type 2 variable and $\varphi$ is a $\Pi^1_n$ formula with one free variable of type 2. This means that there is a set $S \subseteq V_\kappa$ that is a witness of $\exists X \varphi(X)$, in other words, $\varphi(S)$ holds. We can replace every occurence of $X$ in $\varphi$ by a new predicate symbol $S$, this allows us to say that $\kappa$ is $\Pi^1_n$-indescribable (with respect to $\langle V_\kappa, \in, R, S \rangle$).
\footnote{A different yet interesting approach is taken by Tate in \cite{Tait_constructingcardinals}. He states that for $n\geq 0$, a formula of order $\leq n$ is called a $\Pi^n_0$ and a $\Sigma^n_0$ formula. Then a $\Pi^n_{m+1}$ is a formula of form $\forall Y \psi(Y)$ where $\psi$ is a $\Sigma^n_m$ formula and $Y$ is a variable of type $n$. Finally, a $\Sigma^n_{m+1}$ is the negation of a $\Pi^n_m$ formula. So the above holds ad definitio.}
\end{proof}

The above lemma makes it clear that we can suppose that all formulas with no higher than type 2 variables are $\Pi^1_n$-formulas, $n \in \omega$, without the loss of generality.

\begin{lemma}\label{lemma:inaccessible_clubset}\
If $\kappa$ is an inaccessible cardinal and given $R \subseteq V_\kappa$, then the following is a club set in $\kappa$:
\begin{equation}
\{\alpha : \alpha < \kappa \et \langle V_\alpha, \in, R \cap V_\alpha \rangle \prec \langle V_\kappa, \in, R \rangle \}\label{eq:inacc_lemma_set}
\end{equation}
\end{lemma}

\begin{proof}
To see that (\ref{eq:inacc_lemma_set}) is closed, let us recall that a $A \subseteq \kappa$ is closed iff for every ordinal $\alpha < \lambda$, $\alpha \neq \emptyset$: if $A \cap \alpha$ is unbounded in $\alpha$ then $\alpha \in A$. Since $\kappa$ is an inaccessible cardinal, thus strong limit, it is closed under limits of sequences of ordinals lesser than $\kappa$.  

%TODO neco s $V_\kappa$, ze je tranzitivni a tak jso vsechny $V_\alpha$ pro $\alpha<\kappa$ $V_\alpha \in V_\kappa$

We want to verify that it is unbounded, we will use a recursively defined sequence $\alpha_0, \alpha_1, \ldots$
to build an elementary substructure of $\langle V_\kappa, \in, R \rangle$ that is built above an arbitrary $\alpha_0 <\kappa$ .
Let us fix an arbitrary $\alpha_0 < \kappa$. Given $\alpha_n$, $\alpha_n+1$ is defined as the least $\beta$, $\alpha_n \leq \beta$ that satisfies 
the following for any formula $\varphi$, $p_1, \ldots, p_m \in V_{\alpha_{n}}, m \in \omega$:
\begin{equation}
\begin{gathered}
\mbox{If }\langle V_\kappa, \in, R \rangle \models \exists x \varphi(p_1, \ldots, p_n)\mbox{,}\
\mbox{then }\langle V_\kappa, \in, R \rangle \models \varphi(x, p_1, \ldots, p_n)
\end{gathered}
\end{equation}

Let $\alpha = \bigcup_{n < \omega} \alpha_n$. 

Then $\langle V_\alpha, \in, R \cap V_\alpha \rangle \prec \langle V_\kappa, \in, R \rangle$, in other words, for any $\varphi$ with given arbitrary parameters $p_1, \ldots, p_n \in V_\alpha$, it holds that
\begin{equation}
\langle V_\alpha, \in, R \cap V_\alpha \rangle \models \varphi(p_1, \ldots, p_n) \iff \langle V_\kappa, \in, R \rangle \models \varphi(p_1, \ldots, p_n)
\end{equation}
Which should be clear from the construction of $\alpha$
\end{proof}

\begin{theorem}
Let $\kappa$ be an ordinal. The following are equivalent.
\bce[(i)]
\item $\kappa$ is inaccessible
\item $\kappa$ is $\Pi^1_0$-indescribable.
\ece
\end{theorem}

%Note that $\Pi^1_0$ formulas are those that contain zero unbound quantifiers over type-2 variables, they are in fact first-order formulas, only with additional free type 2 variables allowed. For an example of a formula with type 1 quantifiers and a type 2 free variable, the axiom schemas used in previous parts, e.g. \emph{Replacement\textsubscript{1}}.


\begin{proof}
Since $\Pi^1_0$-sentences are first-order sentences, we want to prove that $\kappa$ is an inaccessible cardinal iff whenever a first-order tries to describe $\kappa$ in the sense of definition (\ref{def:describability}), the formula fails to do so and describes a initial segment thereof instead.
We have already shown in (\ref{theorem:inaccessible_models_zfc}) that there is no way to reach an inaccesible cardinal via first-order formulas in $\sf{ZFC}$. We will now prove it again in for formal clarity.

For (i)$\then$(ii), suppose that $\kappa$ is inaccessible.

Then there is, by lemma (\ref{lemma:inaccessible_clubset}) a club set of ordinals $\alpha$ such that $V_\alpha$ is an elementary substructures of $V_\kappa$. For $\kappa$ to be $\Pi^1_0$inderscribable, we need to make sure that given an arbitrary first-order sentence $\varphi$ satisfied in the structure $\langle V_\kappa, \in, R \rangle$, there is an ordinal $\alpha < \kappa$, such that $\langle V_\alpha, \in, R \cap V_\alpha \rangle \models \varphi$. But this follows from the definition of elementary substructure.

For (ii)$\then$(i), suppose $\kappa$ is not inaccessible, so it is either singular, or there is a cardinal $\nu < \kappa$ such that $\kappa \leq \power{\nu}$ or $\kappa=\omega$. 


%For the successor case, there is some $\nu$ so that $\nu+1=\kappa$. 
%Let's take $R = \{\nu\}$ and $\varphi = \exists x \psi(x)$ such that
%\begin{eqaution}
Suppose $\kappa$ is singular. Then there is a cardinal $\nu < \kappa$ and a function $f: \nu \then \kappa$ such that $rng(f)$ is cofinal in $\kappa$. Since $f \subseteq V_\kappa$, we can add $f$ as a relation to the language. We can do the same with $\{\nu\}$. That means $\langle V_\kappa, \in, P_1, P_1$ with $P_1 = f, P_2 = \{\nu\}$ is a structure, 
let $\varphi = P_1 \neq \emptyset \et rng(P_1) = P_2$\footnote{$rng(x)=y$ is a first-order formula, see (\ref{def:rng}).}. Since for every $\alpha < \nu$, $P_1 \cap V_\alpha = \emptyset$, $\varphi$ is false and therefore describes $\kappa$. That contradicts the fact that $\kappa$ was supposed to be $\Pi^1_0$-indescribable, but $\varphi$ is a first-order formula.

Suppose there a cardinal $\nu$ satisfying $\kappa \leq \power{\nu}$. Let there be a function $f: \power{\nu} \then \kappa$ that is onto. Then, like in the previous paragraph, we can obtain a structure $\langle V_\kappa, \in, P_1, P_2 \rangle$, where $P_1 = f$ like before, but this time $P_2 = \power{\nu}$. Again, $\varphi = P_1 \neq \emptyset \et rng(P_1) = P_2$ describes $\kappa$.

Finally, suppose $\kappa = \omega$, then the sentence $\varphi = \forall x \exists y (x \in y)$ describes $\kappa$, there is obviously no $\alpha < \omega$ such that $\langle V_\alpha, \in \rangle \models \varphi$.

\end{proof}

Generally, it should be clear that it a cardinal $\kappa$ is $\Pi^m_n$-indescribable, it is also $\Pi^{m'}_{n'}$-indescribable for every $m'<m, n'<n$. By the same line of thought, if a cardinal $\kappa$ satisfies property implied by $\Pi^m_n$-indescribability, it satisfies all properties implied by $\Pi^{m'}_{n'}$-indescribability for $m'<m, n'<n$, for example $\kappa$ is $\Pi^m_n$-indescribable for $m \geq 1, n \geq 0$, it is also an inaccessible cardinal.

% TODO pozorovani ze kdyz je $\kappa$ $\Pi$

\begin{theorem}\
If a cardinal $\kappa$ is $\Pi^1_1$-indescribable, then it is a Mahlo cardinal.
\end{theorem}

% todo kappa a ne v-kappa?
\begin{proof}
Assuming that $\kappa$ is $\Pi^1_1$-indescribable, we want to prove that every club set in $\kappa$ contains an inaccessible cardinal. 

Consider the following $\Pi^1_1$-sentence:
\begin{equation}
\begin{gathered}\label{eq:inac}
\forall P (\mbox{``$P$ is a function''} \et \exists x(x = dom(P) \lor \power{x} = dom(P)) \then\
\then \exists y(y = rng(P)))
\end{gathered}
\end{equation}
where $P$ is a type 2 variable and $x, y$ are type 1 variables, $rng(P)$ is defined in (\ref{def:rng}), $dom(P)$ in (\ref{def:dom}) and ``$P$ is a function'' is a first-order formula defined in (\ref{def:function}).
We will call this sentence \emph{Inac}, as in ``inaccessible'', because, given a cardinal $\mu$, the following holds if and only if $\mu$ is inaccessible:
\begin{equation}
\langle V_\mu, \in \rangle \models Inac
\end{equation}

So let's fix an arbitrary $C \subset \kappa$, club set in $\kappa$. We want to show that it contains an inaccessible cardinal. Since $C$ is a subset of $V_\kappa$, let's add it to the structure $\langle V_\kappa, \in \rangle$, turning it into $\langle V_\kappa, \in, C \rangle$. Then the following holds:
\begin{equation}
\langle V_\kappa, \in, C \rangle \models Inac \et \mbox{``$C$ in unbounded''}
\end{equation}
Note that this is correct, because, as we have noted just before introducing the statement now being proven, if $\kappa$ is $\Pi^1_1$-indescribable, it is also $\Pi^1_0$-indescribable. So $\kappa$ is itself inaccessible and therefore $\langle V_\kappa, \in, C \rangle \models Inac$. $C$ is obviously picked so that it is unbounded in $\kappa$\footnote{``$C$ in unbounded'' is a first-order formula defined in (\ref{def:unbounded_class})}.

Now because we have assumed that $\kappa$ is $\Pi^1_1$-indescribable and $Inac$ is a $\Pi^1_1$-formula, so $Inac \et \mbox{``$C$ in unbounded''}$ is equivalent to a $\Pi^1_1$-formula, there must be an ordinal $\alpha$ that satisfies
\begin{equation}
\langle V_\alpha, \in, C \cap V_\alpha \rangle \models Inac \et \mbox{``$C$ in unbounded''}
\end{equation}
which implies that $\alpha$ is inaccessible. 

To be finished, we need to verify that $\alpha \in C$. Since $\kappa = V_\kappa$ for inaccessible $\kappa$, $C \cap V_\alpha = C \cap \alpha$, from unboundedness of $C \cap \alpha$ in $\alpha$, $\bigcup(C \cap \alpha) = \alpha$, which, together with the fact that $C$ is a club set in $\kappa$ and therefore closed in $\kappa$, yields that $\alpha \in C$.
\end{proof}

For a proof, see \cite{KanamoriBook}[Theorem 6.4]

\begin{definition}{(Totally Indescribable Cardinal)}\label{def:totally_indescribable_cardinal}\\
We say a cardinal $\kappa$ is a \emph{totally indescribable cardinal} iff it is $\Pi^m_n$-indescribable for every $m, n < \omega$.
\end{definition}

\subsection{Measurable Cardinal}

\begin{definition}{(Ultrafilter)}\\
Given a set $x$, we say $U \subset \power{x}$ is an \emph{ultrafilter} over $x$ iff all of the following hold:
\bce[(i)]
\item $\emptyset \not\in U$
\item $\forall y, z (\subset x \et y \subset z \et y \in U \then z \in U)$
\item $\forall y, z \in U (y \cap z) \in U$
\item $\forall y (y \subset x \then (y \in U \lor (x \setminus y) \in U))$
\ece
\end{definition}

\begin{definition}{($\kappa$-Complete Ultrafilter)}\\
We say that an ultrafilter $U$ is $\kappa$-complete iff
\end{definition}

\begin{definition}{(Measurable Cardinal)}\\
Let $\kappa$ be a caridnal. We say $\kappa$ is a \emph{measurable cardinal} iff there is a $\kappa$-complete ultrafilter over $\kappa$.
\end{definition}

\begin{theorem}
Let $\kappa$ be a cardinal. If $\kappa$ is a measurable cardinal then the following hold:
\bce[(i)]
\item $\kappa$ is $\Pi^2_1$-indescribable.
\item Given $U$, a normal ultrafilter over $\kappa$, a relation $R \subseteq V_\kappa$ and a $\Pi^2_1$-formula $\varphi$ such that $\langle V_\kappa, \in, R \rangle \models \varphi$, then
\begin{equation}
\{ \alpha < \kappa : \langle V_\alpha, \in, R \cap V_\alpha \rangle \models \varphi \} \in U
\end{equation}
\ece
\end{theorem}
For a proof, see \cite{KanamoriBook}[Proposition 6.5]

\begin{theorem}
If $\kappa$ is a measurable cardinal and $U$ is a normal ultrafilter over $\kappa$, the following holds:
\begin{equation}
\{ \alpha < \kappa: \mbox{``$\alpha$ is totally indescribable''}\} \in U
\end{equation}
\end{theorem}
For a proof, see \cite{KanamoriBook}[Proposition 6.6].

This is interesting because if shows, that while we have a hierarchy of sets and a hierarchy of formulas, their relation is more complex than it might seem on the first sight. 
TODO trochu rozepsat.

%\newpage
% =====================================================================================================================================

\subsection{The Constructible Universe}

The constructible universe, denoted $L$, is a cumulative hierarchy of sets, presented by Kurt Gödel in his 1938 paper \emph{The Consistency of the Axiom of Choice and of the Generalised Continuum Hypothesis}. For a technical description, see below. Assertion of their equality, $V=L$, is called the \emph{axiom of constructibility}. The axiom implies GCH and therefore also AC and contradicts the existence of some of the large cardinals, our goal is to decide whether those introduced earlier are among them.

On order to formally establish this class, we need to formalize the notion of definability first. 
\begin{definition}{(Definability)}\label{def:definability}\\
We say that a set $X$ is \emph{definable} over a model $\langle M, \in \rangle$ if there is a first-order formula $\varphi$ together with parameters $p_1, \ldots, p_n \in M$ such that
\begin{equation}
X = \{x: x \in M \et \langle M, \in \rangle \models \varphi(x, p_1, \ldots, p_n)\}
\end{equation}
\end{definition}

\begin{definition}{(The Set of Definable Subsets)}\label{def:definable_powerset}\\
The following is a set of all definable subsets of a given set $M$, denoted Def($M$).
\begin{equation}
\begin{gathered}
Def(M) = \{\{y : x \in M \land \langle M, \in \rangle \models \varphi(y, u_1, \ldots, i_n) \} |\\
\mbox{ $\varphi$ is a~first-order formula, }p_1, \ldots, p_n \in M \}
\end{gathered}
\end{equation}
\end{definition}

We will use $Def(M)$ in the following construction in the way the powerset operation is used when constructing the usual Von Neumann's hierarchy of sets\footnote{For that reason, some authors use $\power^{\*}{M}$ instead of $Def(M)$, see section 11 of \cite{PinterBook} for one such example.}.

Now we can recursively build $L$.
\begin{definition}{(The Constructible Universe)}\label{def:constructible_universe}\\
\bce[(i)]
\item
\begin{equation}
L_0 \defeq  \emptyset
\end{equation}

\item
\begin{equation}
L_{\alpha+1} \defeq  Def(L_{\alpha})
\end{equation}
\item
\begin{equation}
L_{\lambda} = \bigcup_{\alpha < \lambda} L_{\alpha}\mbox{ If }\lambda\mbox{ is a~limit ordinal }
\end{equation}
\item
\begin{equation}\label{eq:def_l}
L = \bigcup_{\alpha\in Ord} L_{\alpha}
\end{equation}
\ece
\end{definition}

Note that while $L$ bears very close resemblance to $V$, the difference is, that in every successor step of constructing $V$, we take every subset of $V_\alpha$ to be $V_{\alpha+1}$, whereas $L_{\alpha+1}$ consists only of definable subsets of $L_\alpha$. Also note that $L$ is transitive.

In order to 

\begin{theorem}
Let $L$ be as in (\ref{def:constructible_universe}).
\begin{equation}
L \models \sf{ZFC}
\end{equation}
\end{theorem}
For details, refer to \cite{JechBook}[Theorem 13.3].

\begin{definition}{(Constructibility)}\\
The axiom of constructibility say that every set is constructible. It is usually denoted as $L = V$.
\end{definition}

Without providing a proof, we will introduce two important results established by Gödel in TODO citace!

\begin{theorem}{(Constructibility $\then$ Choice)}
\begin{equation}
\sf{ZF} \models \mbox{\emph{Constructibility}} \then \mbox{\emph{Choice}} 
\end{equation}
\end{theorem}

The $GCH$ refers to the \emph{Generalised Continuum Hypothesis}, see (\ref{def:gch}).
\begin{theorem}{(Constructibility $\then$ Generalised Continuum Hypothesis)}\label{theorem:l_then_gch}
\begin{equation}
\sf{ZF} \models \mbox{\emph{Constructibility}} \then \mbox{\emph{GCH}} 
\end{equation}
\end{theorem}
It is worth mentioning that Gödel's proof of \emph{Construcibility} $\then$ \emph{GCH} featured the first formal use of a reflection principle. 
For the actual proofs, see for example \cite{Kunen_independence},

Since \emph{GCH} implies that $\kappa$ is a limit cardinal iff $\kappa$ is a strong limit cardinal for every $\kappa$, the distinctions between inaccessible and weakly inaccessible cardinals as well as between Mahlo and weakly Mahlo cardinals vanish.

% -------------------

\begin{theorem}{(Inaccessibility in $L$)}\label{theorem:inaccessible_in_l}\\
Let $\kappa$ be an inaccessible cardinal. Then $\mbox{``$\kappa$ is inaccessible''}^L$.
\end{theorem}
\begin{proof}
We want to show that the following are all true for an inaccessible cardinal $\kappa$:
\bce[(i)] 
\item $\mbox{``$\kappa$ is a cardinal''}^L$
\item $(\omega < \kappa)^L$
\item $\mbox{``$\kappa$ is regular''}^L$
\item $\mbox{``$\kappa$ is limit''}^L$ .\footnote{While inaccessible cardinals are strong limit cardinals, since \emph{GCH} holds in $L$, $\mbox{``$\kappa$ is limit''}^L$ 
implies $\mbox{``$\kappa$ is strong limit''}^L$.}
\ece

Suppose $\mbox{``$\kappa$ is not a cardinal''}^L$ holds, then there is a cardinal $\mu$, $\mu < \kappa$ and a function $f:\mu\then\kappa$, $f \in L$, such that $\mbox{``$f:\mu\then\kappa$ is onto''}^L$. But since ``$f$ is onto'' is a $\Delta_0$ formula and $\Delta_0$ formulas are are absolute in transitive structures\footnote{see lemma (\ref{lemma:delta_0_absoluteness})} and $L$ is a transitive class, $\mbox{``$f$ is onto''}^M \iff \mbox{``$f$ is onto''}$, this contradicts the fact that $\kappa$ is a cardinal.
$(\omega < \kappa)^L$ holds because $\omega \in \kappa$ and because ordinals remain ordinals in $L$, so $(\omega \in \kappa)^L$.

In order to see that $\mbox{``$\kappa$ is regular''}^L$, we can repeat the argument by contradiction used to show that $\kappa$ is a cardinal in $L$. If $\kappa$ was singular, there is a $\mu < \kappa$ together with a function $f: \mu \then \kappa$ that is onto, but since ``$f$ is onto'' implies $\mbox{``$f$ is onto''}^L$, we have reached a contradiction with the fact that $\kappa$ is regular, but singular in $L$.

It now suffices to show that $\mbox{``$\kappa$ is a limit cardinal''}^L$. That means, that for any given $\lambda<\kappa$, we need to find an ordinal $\mu$ such that $\lambda < \mu < \kappa$ that is also a cardinal in $L$. But since cardinals remain cardinals in $L$ by an argument with surjective functions just like above, we are done.\end{proof}

\begin{theorem}{(Mahloness in $L$)}\label{theorem:mahlo_in_l}\\
Let $\kappa$ be a Mahlo cardinal. Then $\mbox{``$\kappa$ is Mahlo''}^L$.
\end{theorem}
% http://math.stackexchange.com/questions/1791631/reference-mahlo-cardinals-remain-mahlo-in-l/1792486#1792486
% TODO citace webu?

\begin{proof}
Let $\kappa$ be a Mahlo cardinal. From the definition of Mahloness in (\ref{def:mahlo_cardinal}), it should be clear that we want prove that $\kappa$ is inaccessible in $L$ and 
\begin{equation}
\mbox{``The set }\{\alpha : \alpha \in \kappa \et \mbox{'$\alpha$ is inaccessible'}\}\mbox{ is stationary in $\kappa$''}^L
\end{equation}

Since we have shown that an inaccessible cardinals remain inaccessible in $L$ in the previous theorem, $L\mbox{``$\kappa$ is inaccessible''}^L$ holds.

Now consider the two following sets:
\bce[(i)]
\item \begin{equation}
S \defeq \{\alpha : \alpha \in \kappa \et \mbox{``$\alpha$ is inaccessible''}\}
\end{equation}
\item \begin{equation}
T \defeq \{\alpha : \alpha \in \kappa \et \mbox{``$\alpha$ is inaccessible''}^L\}
\end{equation} 
\ece 
Since inaccessible cardinals are inaccessible in $L$ from theorem (\ref{theorem:inaccessible_in_l}), $S \subseteq T$.
So if $T$ is stationary in $\kappa$, we are done. Suppose for contradiction that it is not the case. 
Therefore there is a $C \subset \kappa$ satisfying $\mbox{``$C$ is a club set in $\kappa$''}^L$, but it is the case that $T \cap C = \emptyset$.
But because $\mbox{``$C$ is a club set in $\kappa$''}$ is equivalent to a $\Delta_0$ formula, $\mbox{``$C$ is a club set in $\kappa$''}^M \iff \mbox{``$C$ is a club set in $\kappa$''}$, ergo $C$ is a club set in $\kappa$. But since it has o intersection with $T$, it can't have an intersection with a subset thereof, which contradicts the fact that $S$ is stationary in $\kappa$.

$\kappa$ remains Mahlo in $L$.
\end{proof}

It should be clear that the above process can be iterated over again. Since Mahlo cardinals are absolute in $L$, the same argument using stationary sets can be carried out for hyper-Mahlo cardinals and so on. It is clear that since a regular and an inaccessible cardinal in consistent with \emph{Constructibility}, so should be the higher properties acquired from assuring the existence of regular, inaccessible and Mahlo fixed points of normal functions.

\

Let's discuss the relation of $L$ and large cardinals on a more general level. One might ask: ``Why should they interfere with each other?''. This is an interesting question. It is easy to see, that the recursive definition of $L$ is very similar to the hierarchy $V$, the only difference being, that on successor steps, $V_{\alpha+1}$ includes every subset of $V_\alpha$, while $L_{\alpha+1}$ includes the definable subsets of $L_\alpha$. Therefore, each level of $L$, $L_\alpha$ is at most as large as $V_\alpha$. We can therefore say that $V = L$ is a statement about the width of the universe. Large cardinal axioms, on the other hand, talk about the height of the universe, the take the existing hierarchy $V$ and add steps that wouldn't have been possible without them, because all means of travelling upwards (that is \emph{Union}, \emph{Powerset}, and \emph{Replacement} when speaking of $\sf{ZFC}$) are already exhausted. 

From a naive point of view, those two should be separate parameters of the universe. It turns out, due to a result by Dana Scott\footnote{See \cite{Scott_Measurable} for the proof.}, 
that there are large cardinals that, if taken into consideration, conclude that the width of the universe containing them is bigger than $L$ can offer.

\

To see whether reflection per se implies transcendence over $L$, we need to return to the question stated at the very beginning. What is a ``property''? From a structuralist point of view and considering tools for extending structures presented in this thesis, we can conclude that it's not the case. However, we have by no means exhausted possible formalizations of the reflection principles. There are ways to reflect higher-order formulas with higher-order parameters\footnote{See \cite{Welch12globalreflection}, for example.}. We can also leave the structuralist mindset and try to find 
a way to justify the fact, that the universal class is measurable, then, also by a reflection, there would a measurable initial segment of $V$, contradicting \emph{Constructibility}.
% kdo rikal ze V je meritelne
\end{comment} % =============================================================


%\newpage
\section{Conclusion}
\begin{comment}
After establishing an intuitive concept of reflection, we have reviewed Lévy's original proof of the equivalence of Replacement Schema and the Axiom of Infinity with his first-order reflection principle, we have then reformulated and proved the same result in contemporary terms. We have also shown that the same results can be obtained via axiom schemas stating the existence of regular fixed points on normal ordinal functions. After examining the concept of regular fixed points and seeing how it relates to stationary sets and Mahlo cardinals, we have introduced to notion of indescribable cardinal to see that Inaccessible, Mahlo and even Hyper-inaccessible cardinals are still significantly smaller than measurable cardinals, therefore concluding that this application of the reflection principle does not lead to transcendence over $L$.
\end{comment}
\newpage
\bibliographystyle{apalike}
\bibliography{bc_biblio}

\end{document}