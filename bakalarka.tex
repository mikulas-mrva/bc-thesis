\documentclass[12pt,a4paper]{article}
%\renewcommand{\baselinestretch}{1.2}   % in case of emergency .)

\usepackage{mathrsfs}
\usepackage{amssymb}
\usepackage{amsmath}
\usepackage{amsfonts}
\usepackage{longtable}
\usepackage{paralist}
\usepackage{lineno}
\usepackage{verbatim}
\linenumbers

% \usepackage{mathrsfs}
\usepackage{amssymb}
\usepackage{amsmath}
\usepackage{amsfonts}
\usepackage{longtable}
\usepackage{paralist}
\usepackage{lineno}
\usepackage{verbatim}
\linenumbers

% \usepackage{mathrsfs}
\usepackage{amssymb}
\usepackage{amsmath}
\usepackage{amsfonts}
\usepackage{longtable}
\usepackage{paralist}
\usepackage{lineno}
\usepackage{verbatim}
\linenumbers

% \include{00_headers.tex}
\usepackage{color} %pro barevné odkazy, příp. nadpisy
\definecolor{odkazy}{rgb}{0.21,0.27,0.53} %tmavì modrá
\definecolor{nadpisy}{rgb}{0.5812,0.0665,0.0659} %cihlová
%
% Parametry prevodu do pdf
\providecommand{\hypersetup}[1]{}%
\hypersetup{%
unicode,% ? Pravdepodobne bezvyznamne
pdfauthor={Mikuláš Mrva},
pdftitle={Reflection principles and large cardinals},
pdfsubject={Reflection principles and large cardinals},
pdfkeywords={set theory, large cardinals, reflection principle, ZFC, Azriel Lévy},
pdffitwindow=false,% Inicialni umisteni textu v okne Readeru
bookmarksopen=true,% Panel zalozek inicialne zobrazen
% Je-li tohle nastaveno jinak, nektere odkazy nekdy nefunguji
hypertexnames=false,
plainpages=false,
%pdfpagelabels,
%
breaklinks=true,% Radkovy lom smi prijit do klikatelneho odkazu
linkcolor=odkazy,% Graficka podoba odkazu
citecolor=odkazy,% ...
colorlinks=true,% ...
pdfhighlight=/O% ... (vzhled odkazu pri stisknuti)
}%
% Inputenc je asi zbytecne.
% Option 'split' ovlivnuje deleni slov obsahujicich v sobe rozdelovnik
\usepackage[utf8x]{inputenc} % UTF-8 ?
%\usepackage[czech]{babel} %dnes už je však hotová integrace èeštiny do babelu
%\usepackage[split]{czech} %dnes už je však hotová integrace èeštiny do babelu
%
%\usepackage{logdp} %užiteèné drobnosti
%\usepackage{amsthm} %lepšší práce s větami
%\usepackage{amsmath} %nová prostøedí pro matematiku a vylepšení tìch stávajících
%\usepackage{latexsym,amsfonts,amssymb} % nová písmenka
\usepackage{fancyhdr} % zápatí a záhlaví
%\usepackage[nottoc]{tocbibind} % přidá do obsahu položky Literatura a Rejstřík
\usepackage{csquotes}
\pagestyle{plain}
%pøedbìžné nastavení hlavièky (balík fancyhdr)
%\headheight 13.6pt %možná ji bude tøeba zvednout, fancyhdr si pak stìžuje: \headheight
% too small, make it at least Xpt
\headheight 14.5pt %možná ji bude tøeba zvednout, fancyhdr si pak stìžuje: \headheight too \fancyhead{}
\fancyhead[R]{\leftmark}
\fancyfoot{}
\fancyfoot[C]{\thepage}


\newtheorem{theorem}{Theorem}[section]
\newtheorem{Claim}[theorem]{Claim}
\newtheorem{definition}[theorem]{Definition}
\newtheorem{Cor}[theorem]{Corollary}
\newtheorem{Fact}[theorem]{Fact}
\newtheorem{lemma}[theorem]{Lemma}
\newtheorem{sublemma}[theorem]{Sublemma}
\newtheorem{ex}[theorem]{Example}
\newtheorem{remark}[theorem]{Remark}
\newtheorem{obs}[theorem]{Observation}
\newtheorem{que}[theorem]{Question}
\newtheorem{conjecture}[theorem]{Conjecture}

\renewcommand{\theequation}{\thesection.\arabic{equation}}

\newenvironment{proof}
{\noindent \textit{Proof.}}
{\hspace*{\fill} $\Box$}

\newcommand{\toch}{\fbox{\small {\bf ??}}}
\newcommand{\bt}[1]{{\underset{\widetilde{}}{#1}}}
\newcommand{\trcl}[1]{\ensuremath{\mathrm{trcl}(\{#1\})}}
\newcommand{\cf}[1]{\ensuremath{\mathrm{cf}(#1)}}
\newcommand{\cl}[1]{\ensuremath{\mathrm{cl}}(#1)}
\newcommand{\ord}[1]{\ensuremath{\mathrm{ORD}}(#1)}
\newcommand{\dom}[1]{\ensuremath{\mathrm{dom}}(#1)}
\newcommand{\rng}[1]{\ensuremath{\mathrm{rng}}(#1)}
\newcommand{\power}[1]{\ensuremath{\mathscr{P}} (#1)}
\newcommand{\set}[2]{\ensuremath{\{#1 \,:\, #2 \}}}
\newcommand{\seq}[2]{\ensuremath{\langle #1 \,:\, #2 \rangle}}
\newcommand{\singl}[1]{\ensuremath{\{#1\}}}
\newcommand{\pair}[2]{\ensuremath{\{ #1, #2 \}}}
\newcommand{\restr}[2]{\ensuremath{#1 \! \upharpoonright \! #2}}
\renewcommand{\iff}{\leftrightarrow}
\newcommand{\Iff}{\Leftrightarrow}
\newcommand{\el}{\prec}
\newcommand{\iso}{\cong}
\newcommand{\sub}{\subseteq}
\newcommand{\super}{\supseteq}
\newcommand{\la}{\langle}
\newcommand{\ra}{\rangle}
\newcommand{\embed}{\rightarrow}
\newcommand{\mc}{\mathcal}
\newcommand{\supr}[1]{\mathrm{sup}\,#1}
\newcommand{\then}{\rightarrow}
\newcommand{\conc}{^{\smallfrown}}
\newcommand{\bb}{\mathbb}
\newcommand{\supp}[1]{\mathrm{supp}(#1)}
\newcommand{\beq}{\begin{equation}}
\newcommand{\eeq}{\end{equation}}
\newcommand{\brm}{\begin{remark}\begin{rm}}
\newcommand{\erm}{\end{rm}\end{remark}}
\newcommand{\mx}{\mathrm}
\newcommand{\bce}{\begin{compactenum}}
\newcommand{\ece}{\end{compactenum}}
\newcommand{\op}[2]{\la #1, #2 \ra}
\newcommand{\treq}{\trianglelefteq}
\newcommand{\et}{\mathrel{\&}}
\newcommand{\proves}{\vdash}

\newcommand\defeq{\mathrel{\overset{\makebox[0pt]{\mbox{\normalfont\tiny\sffamily def}}}{=}}}

\begin{document}
%titulní stránka
\begin{titlepage}
%\fontsize{16.16pt}{25pt}\selectfont
\Large
\begin{center}
Univerzita Karlova v~Praze, Filozofick{\/á} fakulta\\
Katedra logiky

\vspace{8.5em}
\textsc{Mikuláš Mrva}\\[1.4em]
{REFLECTION PRINCIPLES AND LARGE CARDINALS}\\
Bakalářská práce\\[6.8em]
Vedoucí práce: Mgr. Radek Honzík, Ph.D.\\[6.8em]
2016
\end{center}
\end{titlepage}\


\vspace{\fill}
\noindent 
Prohlašuji, že jsem bakalářskou práci vypracoval samostatně a~že jsem uvedl všechny použité prameny a~literaturu.

\bigskip
\noindent V~Praze 22.~května 2016\\[3em]
\hspace*{\fill}Mikuláš Mrva\hspace*{3em}
\clearpage

\begin{abstract}
\noindent Práce zkoumá vztah tzv. principů reflexe a velkých kardinálů. Lévy ukázal, že v ZFC platí tzv. věta o reflexi~a dokonce, že věta o reflexi je ekvivalentní schématu nahrazení a~axiomu nekonečna nad teorií ZFC bez axiomu nekonečna a~schématu nahrazení. Tedy lze na větu o~reflexi pohlížet jako na svého druhu axiom nekonečna. Práce zkoumá do jaké míry a~jakým způsobem lze větu o reflexi zobecnit a~jaký to má vliv na existenci tzv. velkých kardinálů. Práce definuje nedosažitelné, Mahlovy a nepopsatelné kardinály a ukáže, jak je lze zavést pomocí reflexe. Přirozenou limitou kardinálů získaných reflexí jsou kardinály nekonzistentní s L. Práce nabídne intuitivní zdůvodněn, proč tomu tak je.

\end{abstract}
\bigskip
\renewcommand{\abstractname}{Abstract}
\begin{abstract}
\noindent This thesis aims to examine relations between the so called Reflection Principles and Large cardinals. Lévy has shown that the Reflection Theorem is a sound theorem of ZF and it is equivalent to the Replacement Scheme and the Axiom of Infinity. From this point of view, Reflection theorem can be seen a~specific version of an Axiom of Infinity. This paper aims to examine the Reflection Principle and its generalisations with respect to the existence of Large Cardinals. This thesis will establish the Inaccessible, Mahlo and Indescribable cardinals and show how can those be defined via reflection. A natural limit of Large Cardinals obtained via reflection are cardinals inconsistent with L. This thesis will offer an intuitive explanation of why this holds.
\end{abstract}
\clearpage

\tableofcontents
\clearpage

% podekovani firme co vyrabi club mate -- Loscher gmbh?
\pagestyle{fancy} %detailní definice chování záhlaví
\renewcommand{\sectionmark}[1]{\markboth{\slshape\thesection.\ #1}{}}


\usepackage{color} %pro barevné odkazy, příp. nadpisy
\definecolor{odkazy}{rgb}{0.21,0.27,0.53} %tmavì modrá
\definecolor{nadpisy}{rgb}{0.5812,0.0665,0.0659} %cihlová
%
% Parametry prevodu do pdf
\providecommand{\hypersetup}[1]{}%
\hypersetup{%
unicode,% ? Pravdepodobne bezvyznamne
pdfauthor={Mikuláš Mrva},
pdftitle={Reflection principles and large cardinals},
pdfsubject={Reflection principles and large cardinals},
pdfkeywords={set theory, large cardinals, reflection principle, ZFC, Azriel Lévy},
pdffitwindow=false,% Inicialni umisteni textu v okne Readeru
bookmarksopen=true,% Panel zalozek inicialne zobrazen
% Je-li tohle nastaveno jinak, nektere odkazy nekdy nefunguji
hypertexnames=false,
plainpages=false,
%pdfpagelabels,
%
breaklinks=true,% Radkovy lom smi prijit do klikatelneho odkazu
linkcolor=odkazy,% Graficka podoba odkazu
citecolor=odkazy,% ...
colorlinks=true,% ...
pdfhighlight=/O% ... (vzhled odkazu pri stisknuti)
}%
% Inputenc je asi zbytecne.
% Option 'split' ovlivnuje deleni slov obsahujicich v sobe rozdelovnik
\usepackage[utf8x]{inputenc} % UTF-8 ?
%\usepackage[czech]{babel} %dnes už je však hotová integrace èeštiny do babelu
%\usepackage[split]{czech} %dnes už je však hotová integrace èeštiny do babelu
%
%\usepackage{logdp} %užiteèné drobnosti
%\usepackage{amsthm} %lepšší práce s větami
%\usepackage{amsmath} %nová prostøedí pro matematiku a vylepšení tìch stávajících
%\usepackage{latexsym,amsfonts,amssymb} % nová písmenka
\usepackage{fancyhdr} % zápatí a záhlaví
%\usepackage[nottoc]{tocbibind} % přidá do obsahu položky Literatura a Rejstřík
\usepackage{csquotes}
\pagestyle{plain}
%pøedbìžné nastavení hlavièky (balík fancyhdr)
%\headheight 13.6pt %možná ji bude tøeba zvednout, fancyhdr si pak stìžuje: \headheight
% too small, make it at least Xpt
\headheight 14.5pt %možná ji bude tøeba zvednout, fancyhdr si pak stìžuje: \headheight too \fancyhead{}
\fancyhead[R]{\leftmark}
\fancyfoot{}
\fancyfoot[C]{\thepage}


\newtheorem{theorem}{Theorem}[section]
\newtheorem{Claim}[theorem]{Claim}
\newtheorem{definition}[theorem]{Definition}
\newtheorem{Cor}[theorem]{Corollary}
\newtheorem{Fact}[theorem]{Fact}
\newtheorem{lemma}[theorem]{Lemma}
\newtheorem{sublemma}[theorem]{Sublemma}
\newtheorem{ex}[theorem]{Example}
\newtheorem{remark}[theorem]{Remark}
\newtheorem{obs}[theorem]{Observation}
\newtheorem{que}[theorem]{Question}
\newtheorem{conjecture}[theorem]{Conjecture}

\renewcommand{\theequation}{\thesection.\arabic{equation}}

\newenvironment{proof}
{\noindent \textit{Proof.}}
{\hspace*{\fill} $\Box$}

\newcommand{\toch}{\fbox{\small {\bf ??}}}
\newcommand{\bt}[1]{{\underset{\widetilde{}}{#1}}}
\newcommand{\trcl}[1]{\ensuremath{\mathrm{trcl}(\{#1\})}}
\newcommand{\cf}[1]{\ensuremath{\mathrm{cf}(#1)}}
\newcommand{\cl}[1]{\ensuremath{\mathrm{cl}}(#1)}
\newcommand{\ord}[1]{\ensuremath{\mathrm{ORD}}(#1)}
\newcommand{\dom}[1]{\ensuremath{\mathrm{dom}}(#1)}
\newcommand{\rng}[1]{\ensuremath{\mathrm{rng}}(#1)}
\newcommand{\power}[1]{\ensuremath{\mathscr{P}} (#1)}
\newcommand{\set}[2]{\ensuremath{\{#1 \,:\, #2 \}}}
\newcommand{\seq}[2]{\ensuremath{\langle #1 \,:\, #2 \rangle}}
\newcommand{\singl}[1]{\ensuremath{\{#1\}}}
\newcommand{\pair}[2]{\ensuremath{\{ #1, #2 \}}}
\newcommand{\restr}[2]{\ensuremath{#1 \! \upharpoonright \! #2}}
\renewcommand{\iff}{\leftrightarrow}
\newcommand{\Iff}{\Leftrightarrow}
\newcommand{\el}{\prec}
\newcommand{\iso}{\cong}
\newcommand{\sub}{\subseteq}
\newcommand{\super}{\supseteq}
\newcommand{\la}{\langle}
\newcommand{\ra}{\rangle}
\newcommand{\embed}{\rightarrow}
\newcommand{\mc}{\mathcal}
\newcommand{\supr}[1]{\mathrm{sup}\,#1}
\newcommand{\then}{\rightarrow}
\newcommand{\conc}{^{\smallfrown}}
\newcommand{\bb}{\mathbb}
\newcommand{\supp}[1]{\mathrm{supp}(#1)}
\newcommand{\beq}{\begin{equation}}
\newcommand{\eeq}{\end{equation}}
\newcommand{\brm}{\begin{remark}\begin{rm}}
\newcommand{\erm}{\end{rm}\end{remark}}
\newcommand{\mx}{\mathrm}
\newcommand{\bce}{\begin{compactenum}}
\newcommand{\ece}{\end{compactenum}}
\newcommand{\op}[2]{\la #1, #2 \ra}
\newcommand{\treq}{\trianglelefteq}
\newcommand{\et}{\mathrel{\&}}
\newcommand{\proves}{\vdash}

\newcommand\defeq{\mathrel{\overset{\makebox[0pt]{\mbox{\normalfont\tiny\sffamily def}}}{=}}}

\begin{document}
%titulní stránka
\begin{titlepage}
%\fontsize{16.16pt}{25pt}\selectfont
\Large
\begin{center}
Univerzita Karlova v~Praze, Filozofick{\/á} fakulta\\
Katedra logiky

\vspace{8.5em}
\textsc{Mikuláš Mrva}\\[1.4em]
{REFLECTION PRINCIPLES AND LARGE CARDINALS}\\
Bakalářská práce\\[6.8em]
Vedoucí práce: Mgr. Radek Honzík, Ph.D.\\[6.8em]
2016
\end{center}
\end{titlepage}\


\vspace{\fill}
\noindent 
Prohlašuji, že jsem bakalářskou práci vypracoval samostatně a~že jsem uvedl všechny použité prameny a~literaturu.

\bigskip
\noindent V~Praze 22.~května 2016\\[3em]
\hspace*{\fill}Mikuláš Mrva\hspace*{3em}
\clearpage

\begin{abstract}
\noindent Práce zkoumá vztah tzv. principů reflexe a velkých kardinálů. Lévy ukázal, že v ZFC platí tzv. věta o reflexi~a dokonce, že věta o reflexi je ekvivalentní schématu nahrazení a~axiomu nekonečna nad teorií ZFC bez axiomu nekonečna a~schématu nahrazení. Tedy lze na větu o~reflexi pohlížet jako na svého druhu axiom nekonečna. Práce zkoumá do jaké míry a~jakým způsobem lze větu o reflexi zobecnit a~jaký to má vliv na existenci tzv. velkých kardinálů. Práce definuje nedosažitelné, Mahlovy a nepopsatelné kardinály a ukáže, jak je lze zavést pomocí reflexe. Přirozenou limitou kardinálů získaných reflexí jsou kardinály nekonzistentní s L. Práce nabídne intuitivní zdůvodněn, proč tomu tak je.

\end{abstract}
\bigskip
\renewcommand{\abstractname}{Abstract}
\begin{abstract}
\noindent This thesis aims to examine relations between the so called Reflection Principles and Large cardinals. Lévy has shown that the Reflection Theorem is a sound theorem of ZF and it is equivalent to the Replacement Scheme and the Axiom of Infinity. From this point of view, Reflection theorem can be seen a~specific version of an Axiom of Infinity. This paper aims to examine the Reflection Principle and its generalisations with respect to the existence of Large Cardinals. This thesis will establish the Inaccessible, Mahlo and Indescribable cardinals and show how can those be defined via reflection. A natural limit of Large Cardinals obtained via reflection are cardinals inconsistent with L. This thesis will offer an intuitive explanation of why this holds.
\end{abstract}
\clearpage

\tableofcontents
\clearpage

% podekovani firme co vyrabi club mate -- Loscher gmbh?
\pagestyle{fancy} %detailní definice chování záhlaví
\renewcommand{\sectionmark}[1]{\markboth{\slshape\thesection.\ #1}{}}


\usepackage{color} %pro barevné odkazy, příp. nadpisy
\definecolor{odkazy}{rgb}{0.21,0.27,0.53} %tmavì modrá
\definecolor{nadpisy}{rgb}{0.5812,0.0665,0.0659} %cihlová
%
% Parametry prevodu do pdf
\providecommand{\hypersetup}[1]{}%
\hypersetup{%
unicode,% ? Pravdepodobne bezvyznamne
pdfauthor={Mikuláš Mrva},
pdftitle={Reflection principles and large cardinals},
pdfsubject={Reflection principles and large cardinals},
pdfkeywords={set theory, large cardinals, reflection principle, ZFC, Azriel Lévy},
pdffitwindow=false,% Inicialni umisteni textu v okne Readeru
bookmarksopen=true,% Panel zalozek inicialne zobrazen
% Je-li tohle nastaveno jinak, nektere odkazy nekdy nefunguji
hypertexnames=false,
plainpages=false,
%pdfpagelabels,
%
breaklinks=true,% Radkovy lom smi prijit do klikatelneho odkazu
linkcolor=odkazy,% Graficka podoba odkazu
citecolor=odkazy,% ...
colorlinks=true,% ...
pdfhighlight=/O% ... (vzhled odkazu pri stisknuti)
}%
% Inputenc je asi zbytecne.
% Option 'split' ovlivnuje deleni slov obsahujicich v sobe rozdelovnik
\usepackage[utf8x]{inputenc} % UTF-8 ?
%\usepackage[czech]{babel} %dnes už je však hotová integrace èeštiny do babelu
%\usepackage[split]{czech} %dnes už je však hotová integrace èeštiny do babelu
%
%\usepackage{logdp} %užiteèné drobnosti
%\usepackage{amsthm} %lepšší práce s větami
%\usepackage{amsmath} %nová prostøedí pro matematiku a vylepšení tìch stávajících
%\usepackage{latexsym,amsfonts,amssymb} % nová písmenka
\usepackage{fancyhdr} % zápatí a záhlaví
%\usepackage[nottoc]{tocbibind} % přidá do obsahu položky Literatura a Rejstřík
\usepackage{csquotes}
\pagestyle{plain}
%pøedbìžné nastavení hlavièky (balík fancyhdr)
%\headheight 13.6pt %možná ji bude tøeba zvednout, fancyhdr si pak stìžuje: \headheight
% too small, make it at least Xpt
\headheight 14.5pt %možná ji bude tøeba zvednout, fancyhdr si pak stìžuje: \headheight too \fancyhead{}
\fancyhead[R]{\leftmark}
\fancyfoot{}
\fancyfoot[C]{\thepage}


\newtheorem{theorem}{Theorem}[section]
\newtheorem{Claim}[theorem]{Claim}
\newtheorem{definition}[theorem]{Definition}
\newtheorem{Cor}[theorem]{Corollary}
\newtheorem{Fact}[theorem]{Fact}
\newtheorem{lemma}[theorem]{Lemma}
\newtheorem{sublemma}[theorem]{Sublemma}
\newtheorem{ex}[theorem]{Example}
\newtheorem{remark}[theorem]{Remark}
\newtheorem{obs}[theorem]{Observation}
\newtheorem{que}[theorem]{Question}
\newtheorem{conjecture}[theorem]{Conjecture}

\renewcommand{\theequation}{\thesection.\arabic{equation}}

\newenvironment{proof}
{\noindent \textit{Proof.}}
{\hspace*{\fill} $\Box$}

\newcommand{\toch}{\fbox{\small {\bf ??}}}
\newcommand{\bt}[1]{{\underset{\widetilde{}}{#1}}}
\newcommand{\trcl}[1]{\ensuremath{\mathrm{trcl}(\{#1\})}}
\newcommand{\cf}[1]{\ensuremath{\mathrm{cf}(#1)}}
\newcommand{\cl}[1]{\ensuremath{\mathrm{cl}}(#1)}
\newcommand{\ord}[1]{\ensuremath{\mathrm{ORD}}(#1)}
\newcommand{\dom}[1]{\ensuremath{\mathrm{dom}}(#1)}
\newcommand{\rng}[1]{\ensuremath{\mathrm{rng}}(#1)}
\newcommand{\power}[1]{\ensuremath{\mathscr{P}} (#1)}
\newcommand{\set}[2]{\ensuremath{\{#1 \,:\, #2 \}}}
\newcommand{\seq}[2]{\ensuremath{\langle #1 \,:\, #2 \rangle}}
\newcommand{\singl}[1]{\ensuremath{\{#1\}}}
\newcommand{\pair}[2]{\ensuremath{\{ #1, #2 \}}}
\newcommand{\restr}[2]{\ensuremath{#1 \! \upharpoonright \! #2}}
\renewcommand{\iff}{\leftrightarrow}
\newcommand{\Iff}{\Leftrightarrow}
\newcommand{\el}{\prec}
\newcommand{\iso}{\cong}
\newcommand{\sub}{\subseteq}
\newcommand{\super}{\supseteq}
\newcommand{\la}{\langle}
\newcommand{\ra}{\rangle}
\newcommand{\embed}{\rightarrow}
\newcommand{\mc}{\mathcal}
\newcommand{\supr}[1]{\mathrm{sup}\,#1}
\newcommand{\then}{\rightarrow}
\newcommand{\conc}{^{\smallfrown}}
\newcommand{\bb}{\mathbb}
\newcommand{\supp}[1]{\mathrm{supp}(#1)}
\newcommand{\beq}{\begin{equation}}
\newcommand{\eeq}{\end{equation}}
\newcommand{\brm}{\begin{remark}\begin{rm}}
\newcommand{\erm}{\end{rm}\end{remark}}
\newcommand{\mx}{\mathrm}
\newcommand{\bce}{\begin{compactenum}}
\newcommand{\ece}{\end{compactenum}}
\newcommand{\op}[2]{\la #1, #2 \ra}
\newcommand{\treq}{\trianglelefteq}
\newcommand{\et}{\mathrel{\&}}
\newcommand{\proves}{\vdash}

\newcommand\defeq{\mathrel{\overset{\makebox[0pt]{\mbox{\normalfont\tiny\sffamily def}}}{=}}}

\begin{document}
%titulní stránka
\begin{titlepage}
%\fontsize{16.16pt}{25pt}\selectfont
\Large
\begin{center}
Univerzita Karlova v~Praze, Filozofick{\/á} fakulta\\
Katedra logiky

\vspace{8.5em}
\textsc{Mikuláš Mrva}\\[1.4em]
{REFLECTION PRINCIPLES AND LARGE CARDINALS}\\
Bakalářská práce\\[6.8em]
Vedoucí práce: Mgr. Radek Honzík, Ph.D.\\[6.8em]
2016
\end{center}
\end{titlepage}\


\vspace{\fill}
\noindent 
Prohlašuji, že jsem bakalářskou práci vypracoval samostatně a~že jsem uvedl všechny použité prameny a~literaturu.

\bigskip
\noindent V~Praze 22.~května 2016\\[3em]
\hspace*{\fill}Mikuláš Mrva\hspace*{3em}
\clearpage

\begin{abstract}
\noindent Práce zkoumá vztah tzv. principů reflexe a velkých kardinálů. Lévy ukázal, že v ZFC platí tzv. věta o reflexi~a dokonce, že věta o reflexi je ekvivalentní schématu nahrazení a~axiomu nekonečna nad teorií ZFC bez axiomu nekonečna a~schématu nahrazení. Tedy lze na větu o~reflexi pohlížet jako na svého druhu axiom nekonečna. Práce zkoumá do jaké míry a~jakým způsobem lze větu o reflexi zobecnit a~jaký to má vliv na existenci tzv. velkých kardinálů. Práce definuje nedosažitelné, Mahlovy a nepopsatelné kardinály a ukáže, jak je lze zavést pomocí reflexe. Přirozenou limitou kardinálů získaných reflexí jsou kardinály nekonzistentní s L. Práce nabídne intuitivní zdůvodněn, proč tomu tak je.

\end{abstract}
\bigskip
\renewcommand{\abstractname}{Abstract}
\begin{abstract}
\noindent This thesis aims to examine relations between the so called Reflection Principles and Large cardinals. Lévy has shown that the Reflection Theorem is a sound theorem of ZF and it is equivalent to the Replacement Scheme and the Axiom of Infinity. From this point of view, Reflection theorem can be seen a~specific version of an Axiom of Infinity. This paper aims to examine the Reflection Principle and its generalisations with respect to the existence of Large Cardinals. This thesis will establish the Inaccessible, Mahlo and Indescribable cardinals and show how can those be defined via reflection. A natural limit of Large Cardinals obtained via reflection are cardinals inconsistent with L. This thesis will offer an intuitive explanation of why this holds.
\end{abstract}
\clearpage

\tableofcontents
\clearpage

% podekovani firme co vyrabi club mate -- Loscher gmbh?
\pagestyle{fancy} %detailní definice chování záhlaví
\renewcommand{\sectionmark}[1]{\markboth{\slshape\thesection.\ #1}{}}



\section{Introduction}\label{sec:introduction}

%Reflection principle is a kind of a theorem scheme stating the following:
% taky debilni formulace, ale co uz
\subsection{Motivation and Origin}
\begin{displayquote}
The Universe of sets cannot be uniquely characterized (i. e. distinguished from all its initial elements) by any internal structural property of the membership relation in it, which is expressible in any logic of finite of transfinite type, including infinitary logics of any cardinal order.
\end{displayquote}
\rightline{{\rm --- Kurt Gödel \cite{GodelWang}}}

To understand why do need reflection in the first place, let's think about infinity for a moment. In the intuitive sense, infinity is an upper limit of all numbers. But for centuries, this was merely a philosophical concept, closely bound to religious and metaphysical way of thinking, 
% aquinas -- summa, question 7
considered separate from numbers used for calculations or geometry. It was a rather vague concept. 
% neco o nekonecnech?
In ancient Greece, Aristotle's response to famous Zeno's paradoxes introduced the distinction between actual and potential infinity.
% zdroje
% http://www.iep.utm.edu/infinite/#SH1a
% (See Aristotle’s Physics, Book III, for his account of infinity.)
% viz A.W.Moore The Infinite -- je tam nejak reflexe?
He argued, that potential infinity is (in today's words) well defined, as opposed to actual infinity, which remained a vague incoherent concept. He didn't think it's possible for infinity to inhabit a bounded place in space or time, rejecting Zeno's thought experiments as a whole. 
% co Archimedes?
Aristotle's thoughts shaped western thinking partly due to Aquinas, who himself believed actual infinity to be more of a~metaphysical concept for describing God than a~mathematical property attributed to any other entity. In his Summa Theologica \footnote{Part I, Question 7, Article 3, Reply to Objection 1} he argues:
\begin{displayquote}
A geometrician does not need to assume a~line actually infinite, but takes some actually finite line, from which he subtracts whatever he finds necessary; which line he calls infinite.
\end{displayquote} % formalni citace?

Less than hundred years later, Gregory of Rimini wrote
\begin{displayquote}
If God can endlessly add a~cubic foot to a~stone–which He can–then He can create an infinitely big stone. For He need only add one cubic foot at some time, another half an hour later, another a~quarter of an hour later than that, and so on ad infinitum. He would then have before Him an infinite stone at the end of the hour.
\end{displayquote}
% citace (Moore 2001, 53)  "anti-Aristotelian backlash among the medievals"
Which is basically a~Zeno's Paradox made plausible with God being the actor. In contrast to Aquinas' position, Gregory of Rimini theoretically constructs an object with actual infinite magnitude that is essentially different from God.

\

% byl prvni cantor? jak presne to definoval Leibniz a~Newton?
Even later, in the 17th century, pushing the property of infinitness from the Creator to his creation, Nature, Leibniz wrote to Foucher in 1962:
% citace? zdroj: http://www.humanities.mcmaster.ca/~rarthur/papers/LeibCant.pdf
\begin{displayquote}
I am so in favor of the actual infinite that instead of admitting that Nature abhors
it, as is commonly said, I hold that Nature makes frequent use of it everywhere,
in order to show more effectively the perfections of its Author. Thus I believe that
there is no part of matter which is not, I do not say divisible, but actually divided;
and consequently the least particle ought to be considered as a~world full of an
infinity of different creatures.
\end{displayquote}
But even though he used potential infinity in what would become foundations of modern Calculus and argued for actual infinity in Nature, Leibniz refused the existence of an infinite, thinking that Galileo's Paradoxon\footnote{zneni galileova paradoxu} is in fact a~contradiction. The so called Galileo's Paradoxon is an observation Galileo Galilei made in his final book "Discourses and Mathematical Demonstrations Relating to Two New Sciences".
He states that if all numbers are either squares and non–squares, there seem to be less squares than there is all numbers. On the other hand, every number can be squared and every square has it's square root. Therefore, there seem to be as many squares as there are all numbers. Galileo concludes, that the idea of comparing sizes makes sense only in the finite realm.
\begin{displayquote}
Salviati: So far as I see we can only infer that the totality of all numbers is infinite, that the number of squares is infinite, and that the number of their roots is infinite; neither is the number of squares less than the totality of all the numbers, nor the latter greater than the former; and finally the attributes "equal," "greater," and "less," are not applicable to infinite, but only to finite, quantities. When therefore Simplicio introduces several lines of different lengths and asks me how it is possible that the longer ones do not contain more points than the shorter, I answer him that one line does not contain more or less or just as many points as another, but that each line contains an infinite number.
\end{displayquote}
%citace?  Galilei, Galileo (1954) [1638]. Dialogues concerning two new sciences. Transl. Crew and de Salvio. New York: Dover. pp. 31–33.
% btw: http://www.humanities.mcmaster.ca/~rarthur/papers/Lairam.pdf  Leibniz’s Actual Infinite in Relation to his Analysis of Matter
Leibniz insists in part being smaller than the whole saying
\begin{displayquote}
Among numbers there are infinite roots, infinite squares, infinite cubes. Moreover, there are
as many roots as numbers. And there are as many squares as roots. Therefore there are as
many squares as numbers, that is to say, there are as many square numbers as there are
numbers in the universe. Which is impossible. Hence it follows either that in the infinite the
whole is not greater than the part, which is the opinion of Galileo and Gregory of St.
Vincent, and which I cannot accept; or that infinity itself is nothing, i.e. that it is not one and
not a~whole. % (LoC 9) je co?
\end{displayquote}
% viz G.W. Leibniz, Interrelations between Mathematics and Philosophy

TODO  Hegel--strucne?

TODO Cantor

TODO mene teologie, vice matematiky

TODO definovat pojmy (trida etc)

TODO neni V v nejakem smyslu porad potencialni nekonecno, zatimco mnoziny vetsi nez omega jsou aktualni? nebo jsou potencialni protoze se staveji pres indukci, od spoda.

In his work, he defined transfinite numbers to extend existing natural number % existovala fakt? kdo to udelal? asi peano
structure so it contains more objects that behave like natural numbers and are based on an object (rather a~meta-object) that doesn't explicitly exist in the structure, but is closely related to it. This is the first instance of reflection. 
This paper will focus on taking this principle a~step further, extending Cantor's (or Zermelo–Fraenkel's, to be more precise) universe so it includes objects so big, they could be considered the universe itself, in a~certain sense. % dost vagni?


TODO dal asi smazat


% nemuzeme nikdy zachytit cele univerzum
% snaha udelat to stala za vznikem pojmu nekonecno, naivni teorie mnozin i jejich formalizaci
The original idea behind reflection principles probably comes from what could be informally called \textquote{universality of the universe}.
The effort to precisely describe the universe of sets was natural and could be regarded as one of the impulses for formalization of naive set theory.
If we try to express the universe as a~set $\{x  |  x = x\}$, a~paradox appears, because either our set is contained in itself and therefore is contained in a~set (itself again), which contradicts the intuitive notion of a~universe that contains everything but is not contained itself.

TODO ???

If there is an object containing all sets, it must not be a~set itself. The notion of class seems inevitable. Either directly the ways for example the Bernays–Gödel set theory, we will also discuss later in this paper, does in, or on a~meta–level like the Zermelo–Fraenkel set theory, that doesn't refer to them in the axioms but often works with the notion of a~universal class.
duet
Another obstacle of constructing a~set of all sets comes from Georg Cantor, who proved that the set of all subsets of a~set (let $x$ be the set and $\power (x)$ its powerset) is strictly larger that $x$. That would turn every aspiration to finally establish an universal set into a~contradictory infinite regression.\footnote{An intuitive analogy of this \emph{reductio ad infinitum} is the status of $\omega$, which was originally thought to be an unreachable absolute, only to become starting point of Cantor's hierarchy of sets growing beyond all boundaries around the end of the $19^{th}$ century}. We will use $V$ to denote the class of all sets. %Riemann!
%\newpage
From previous thoughts we can easily argue, that it is impossible to construct a~property that holds for $V$ and no set and is neither paradoxical like $\{x  |  x = x\}$ nor trivial. Previous observation can be transposed to a~rather naive formulation of the reflection principle:


\

$\emph{Reflection}$ Any property which holds in $V$ already holds in some initial segment of $V$. 

\

To avoid vagueness of the term "property", we could informally reformulate the above statement into a~schema: 

\

For every first-order formula\footnote{this also works for finite sets of formulas \cite[p.~168]{JechBook}} $\varphi$ holds in $V \iff \varphi$ holds in some initial segment of $V$.

\

Interested reader should note that this is a~theorem scheme rather than a~single theorem. \footnote{If there were a~single theorem stating "for any formula $\varphi$ that holds in $V$ there is an initial segment of $V$ where $\varphi$ also holds", we would obtain the following contradiction with the second G{\"o}del's theorem: In ZFC, any finite group of axioms of ZFC holds in some initial segment of the universe. If we take the largest of those initial segments it is still strictly smaller than the universe and thus we have, via compactness, constructed a~model of ZFC within ZFC. That is, of course a~harsh contradiction. This also leads to an elegant way to prove that ZFC is not finitely axiomatizable.}

%Reflection is, among others, a~tool for constructing models of finitely axiomatizable theories within stronger frameworks of proving impossibility thereof. ?!

\subsection{A few historical remarks on reflection}\label{sec:History}  % mozna reflection itself?
% co motivace? (jako treba ze kdyz existuje inaccessible cardinal tak je ZFC relativne konzistentni, protoze $V_\kappa$ je jeji model)
% viz kanamori
% jak to bylo s tim Godelem?
% nejdriv Montague ukazal ze ZFC neni konecne axiomatizovatelna "in a~strong sense"
%The first notion of reflection in our sense was formally stated and proved as a~theorem of ZF by Azriel L{\'e}vy in his 1960 article \emph{Axiomatic schemata of strong infinity in axiomatic set theory}\cite{Levy60a}. Only a~year later, this was followed by Richard Montague's \emph{Fraenkel's addition to the axioms of Zermelo}.
 %citace?
 % konecna axiomatizovatelnost? reflexe spolu se 2. Godelovou vetou, ~Montague
 
 
 % motivace ke kardinalum:
 % velke kardinaly vznikly v podstate reflektovanim zajimavych vlastnosti absolutniho nekonecna, mnoho z nich ma i omega
 % popis vyvoje axiomu nekonecna: zadny->nejaky->silnejsi??
Reflection made its first in set-theoretical appearance in G{\"o}del's proof of GCH in L  (citace Kanamori ? Lévy and set theory), but it was around even earlier as a~concept. G{\"o}del himself regarded it as very close to Russel's reducibility axiom (an earlier equivalent of the axiom schema of Zermelo's separation). Richard Montague then studied reflection properties as a~tool for verifying that Replacement is not finitely axiomatizable (citace?). a~few years later Lévy proved in \cite{Levy60a} the equivalence of reflection with Axiom of infinity together with Replacement in proof we shall examine closely in chaper 2.

\

TODO co dal? recent results?

\subsection{Reflection in Platonism and Structuralism}
TODO cite "reflection in a structuralist setting"

TODO veci o tom, ze reflexe je ok protoze reflektuje veci ktere objektivne plati, protoze plati pro V ...

TODO souvislost s kompaktnosti, hranice formalich systemu nebo alespon ZFC

\subsection{Notation and terminology}
\subsubsection{The Language of Set Theory}
% TODO predpokladame splnovani a tak, z logiky? link na ucebnici kdyztak? 
% 

We are about to define basic set-theoretical terminology on which the rest of this thesis will be built. For Chapter 2, the underlying theory will be the \emph{Zermelo –Fraenkel} set theory with the Axiom of Choice ($\sf{ZFC}$), a first-order set theory in the language $\mathscr{L} = \{=, \in\}$, which will be sometimes referred to as \emph{the language of set theory}. In Chapter 3\footnote{TODO bude jich vic? Chapter 4 taky?}, we shall always make it clear whether we are in first-order $\sf{ZFC}$ or second-order $\sf{ZFC}_2$, which will be precisely defined later in this chapter. When in second-order theory, we will usually denote type 1 variables, which are elements of the domain of discourse\footnote{co je "domain of discourse"?} by lower-case letters, mostly $u, v, w, x, y, z, p_1, p_2, p_3,  \ldots$ while type 2 variables, which represent $n$-ary relations of the domain of discourse for any natural number $n$, are usually denoted by upper-case letters $A, B, C, X, Y, Z$. Note that those may be used both as relations and functions, see the definition of a function below.\footnote{TODO ref?}

The informal notions of \emph{class} and \emph{property} will be used throughout this thesis. They both represent formulas with respect to the domain of discourse. If $\varphi(x, p_1, \ldots, p_n)$ is a formula in the language of set theory, we call 
\begin{equation}
A = \{x : \varphi(x, p_1, \ldots, p_n)\}
\end{equation}
a class of all sets satisfying $\varphi(x, p_1, \ldots, p_n)$ in a sense that 
\begin{equation}
x \in A \iff \varphi(x, p_1, \ldots, p_n)
\end{equation}
One can easily define for classes $A$, $B$ the operations like $A \cap B$, $A \cup B$, $A \setminus C$, $\bigcup A$, but it is elementary and we won't do it here, see the first part of \cite{JechBook} for technical details. The following axioms are the tools by which decide whether particular classes are in fact sets. A class that fails to be considered a set is called a \emph{proper class}.
% taky diag -- class/property/formula je totez, formula je totiz trida n-tic ktere ji splnuji
\


\subsubsection{The Axioms}

\begin{definition}{(The existence of a set)}\label{def:existence_of_a_set}
\begin{equation}
\exists x (x = x)
\end{equation}
\end{definition}
The above axiom is usually not used because it can be deduced from the axiom of \emph{Infinity} (see below), but since we will be using set theories that omit \emph{Infinity}, this will be useful.

\begin{definition}{(Extensionality)}\label{def:extensionality}
\begin{equation}
\forall x, y(\forall z (z \in x \iff z \in y) \iff x = y)
\end{equation}
\end{definition}

\begin{definition}{(Specification)}\label{def:specification}\\
The following is a schema for every first-order formula $\varphi(x, p_1, \ldots, p_n)$ with no free variables other than $x, p_1, \ldots, p_n$.
\begin{equation}
\forall x, p_1, \ldots, p_n \exists y \forall z ( z \in y \iff ( z \in x \et \varphi(z, p_1, \ldots, p_n)))
\end{equation}
\end{definition}

We will now provide two definitions that are not axioms, but will be helpful in establishing some of the other axioms in a more intuitive way.
\begin{definition}{($x \subseteq y$, $x \subset y$)}
\begin{equation}
x \subseteq y \iff \forall z(z \in x \then z \in y)
\end{equation}
\begin{equation}
x \subset y \iff x \subseteq y \et x \neq y
\end{equation}
\end{definition}

\begin{definition}{(Empty set)}\label{def:emptyset}
\begin{equation}
\emptyset \defeq \{x : x \neq x\}
\end{equation}
\end{definition}
To make sure that $\emptyset$ is a set, note that there exists at least one set $y$ from \ref{def:existence_of_a_set}, then consider the following alternative definition.

\begin{equation}
\emptyset' \defeq \{x : \varphi(x) \et x \in y\}\mbox{ where $y$ $\varphi$ is the formula "$x \neq x$".}
\end{equation}
It should be clear that $\emptyset' = \emptyset$.\footnote{For details, see page 8 in \cite{JechBook}.}

Now we can introduce more axioms.
\begin{definition}{(Foundation)}\label{def:foundation}
\begin{equation}
\forall x (x \neq \emptyset \then \exists z (z \in x \et \forall u \neg (u \in z \et u \in x)))
\end{equation}
\end{definition}

\begin{definition}{(Pairing)}\label{def:pairing}
\begin{equation}
\forall x, y \exists z \forall q (q \in z \iff q \in z \lor q \in y)
\end{equation}
\end{definition}

\begin{definition}{(Union)}\label{def:union}
\begin{equation}
\forall x \exists y \forall z (z \in x \iff \exists q( z \in q \et q \in x))
\end{equation}
\end{definition}

\begin{definition}{(Powerset)}\label{def:powerset}
\begin{equation}
\forall x \exists y \forall z (z \subseteq x \iff z \in y)
\end{equation}
\end{definition}

\begin{definition}{(Infinity)}\label{def:infinity}
\begin{equation}
\exists x (\forall y \in x)(y\cup\{y\} \in x)
\end{equation}
\end{definition}

Let us introduce a few more definitions that will make the two remaining axioms more comprehensible.
\begin{definition}{(Function)}\label{def:function}\\
Given arbitrary first-order formula $\varphi(x, y, p_1, \ldots, p_n)$, we say that $\varphi$ is a function iff
\begin{equation}\label{def:function_formula}
\forall x, y, z, p_1, \ldots, p_n (\varphi(x, y, p_1, \ldots, p_n) \et \varphi(x, z, p_1, \ldots, p_n) \then y = z)
\end{equation}
\end{definition}
When a $\varphi(x, y)$ is a function, we also write the following:
\begin{equation}
f(x) = y \iff \varphi(x, y)
\end{equation}
Note that this $f$ is in fact  a formula 

TODO $f = \{(x, y) : \varphi(x, y)\}$ !!! f muze byt mnozina i trida!
\footnote{This can also be done for $\varphi$s with more than two free variables by either setting $f(x, p_1, \ldots p_n) = y \iff \varphi(x, y p_1, \ldots, p_n)$}

\begin{definition}{(Dom(f))}\label{def:dom}\\
Let $f$ be a function. We read the following as "Dom(f) is the domain of f".
\begin{equation}
Dom(f) \defeq \{x : \exists y (f(x) = y)\}
\end{equation}
\end{definition}
We say "$f$ is a function on $A$", $A$ being a class, if $A = dom(f)$.

\begin{definition}{(Rng(f))}\label{def:rng}\\
Let $f$ be a function. We read the following as "Rng(f) is the range of f".
\begin{equation}
Rng(f) \defeq \{x : \exists y (f(x) = y)\}
\end{equation}
\end{definition}
We say that $f$ is i function into $A$, $A$ being a class, if $rng(f) \subseteq A$.

Note that \emph{Dom(f)} and \emph{Rng(f)} are not definitions in a strict sense, they are in fact definition schemas that yield definitions for every function $f$ given. Also note that they can be easily modified for $\varphi$ instead of $f$, with the only difference that then it is defined only for those $\varphi$s that are functions.

\begin{definition}{(Powerset)}\\
TODO
\end{definition}

And now for the axioms.
\begin{definition}{(Replacement)}\label{def:replacement}\\
The following is a schema for every first-order formula $\varphi(x, p_1, \ldots, p_n)$ with no free variables other than $x, p_1, \ldots, p_n$.
\begin{equation}
"\varphi\mbox{ is a function}"\then \forall x \exists y \forall z (z \in y \iff (\exists q \in x)(\varphi(x, y, p_1, \ldots, p_n))
\end{equation}
\end{definition}

\begin{definition}{(Choice)}\label{def:choice}\\
This is also a schema. For every $A$, a family of non-empty sets\footnote{We say a class $A$ is a "family of non-empty sets" iff there is $B$ such that $A \subseteq \power{B}$}, such that $\emptyset \not\in S$, there is a function $f$ such that for every $x \in A$
\begin{equation}
f(x) \in x
\end{equation}
\end{definition}

We will refer the axioms by their name, written in italic type, e.g. \emph{Foundation} refers to the Axiom of Foundation. Now we need to define some basic set theories to be used in the article. There will be others introduce in Chapter 3, but those will usually be defined just by appending additional axioms or schemata to one of the following.

\begin{definition}{$(\sf{S})$}\label{def:s}\\
We call $\sf{S}$ a set theory with the following axioms:
\bce[(i)]
\item \emph{Existence of a set} (see \ref{def:existence_of_a_set})
\item \emph{Extensionality} (see \ref{def:extensionality})
\item \emph{Specification} (see \ref{def:specification})
\item \emph{Foundation} (see \ref{def:foundation})
\item \emph{Pairing} (see \ref{def:pairing})
\item \emph{Union} (see \ref{def:union})
\item \emph{Powerset} (see \ref{def:powerset})
\ece
\end{definition}

\begin{definition}{$(\sf{ZF})$}\label{def:zf}\\
We call $\sf{ZF}$ a set theory that contains all the axioms of the theory $\sf{S}$\footnote{With the exception of \emph{Existence of a set}} in addition to the following
\bce[(i)]
\item \emph{Replacement} schema (see \ref{def:replacement})
\item \emph{Infinity} (see \ref{def:infinity})
\ece
\end{definition}

\begin{definition}{$(\sf{ZFC})$}\label{def:zfc}\\
$\sf{ZFC}$ is a theory that contains all the axioms of $\sf{ZF}$ plus \emph{Choice} (\ref{def:choice}).
\end{definition}

\

\subsubsection{The transitive universe}
% $V$ a $V_\alpha$ odkazuji k Von Neumannove hierarchii (pro jistotu)
\begin{definition}{(Transitive class)}\label{def:transitivity}\\
We say a class $A$ is \emph{transitive} iff
\begin{equation}
\forall x(x \in A \then x \subseteq A)
\end{equation}
\end{definition}

\begin{definition}{Well Ordered Class}\label{def:well_ordering}
A class $A$ is said to be \emph{well ordered by $\in$} iff the following hold:
\bce[(i)]
\item $(\forall x \in A)(x \not\in x)$ (Antireflexivity)
\item $(\forall x, y, z \in A)(x \in y \et y \in z \then x \in z)$ (Transitivity)
\item $(\forall x, y \in A)(x = y \lor x \in y \lor y \in x)$ (Linearity)
\item $(\forall x)(x \subseteq A \et x \neq \emptyset \then (\exists y \in x)(\forall z \in x)(z = y \lor z \in y)))$ % (every nonempty subclass has a least element)
\ece
\end{definition}

\begin{definition}{(Ordinal number)}\label{def:ordinal}\\
A set $x$ is said to be an \emph{ordinal number}, also known as an \emph{ordinal}, if it is \emph{transitive} and \emph{well-ordered by $\in$}. 
\end{definition}
For the sake of brevity, we usually just say "$x$ is an \emph{ordinal}". 
Note that "$x$ is an ordinal" is a well-defined formula, since \ref{def:transitivity} is a formula and \ref{def:well_ordering} is in fact a conjunction of four formulas.
Ordinals will be usually denoted by lower case greek letters, starting from the beginning: $\alpha, \beta, \gamma, \ldots$.
Given two different ordinals $\alpha, \beta$, we will write $\alpha < \beta$ for $\alpha \in \beta$, see \cite{JechBook}{Lemma 2.11} for technical details.

\begin{definition}{(Successor Ordinal)}\label{def:successor_ordinal}\\
Consider the following operation
\begin{equation}
\beta + 1 \defeq \beta \cup \{\beta\}
\end{equation}
An ordinal $\alpha$ is called a \emph{successor ordinal} iff there is an ordinal $\beta$, such that $\alpha = \beta+1$
\end{definition}

\begin{definition}{(Limit Ordinal)}\label{def:limit_ordinal}\\
A non-zero ordinal $\alpha$\footnote{$\alpha \neq \emptyset$} is called a \emph{limit ordinal} iff it is not a successor ordinal.
\end{definition}

\begin{definition}{(Ord)}\label{def:ord}\\
\emph{The class of all ordinal numbers}, which we will denote $Ord$\footnote{It is sometimes denoted $On$, but we will stick to the notation in \cite{JechBook}} be the following class:
\begin{equation}
Ord \defeq \{x : x\mbox{ is an ordinal}\}
\end{equation}
\end{definition}

The following construction will be often referred to as the \emph{Von Neumann's Hierarchy}, sometimes also the \emph{Von Neumann's Universe}. %, the former referring more to the construction with the individual levels in mind, the latter referring more to the class $V$, but they can be interchanged with no confusion caused.

\begin{definition}{(Von Neumann's Hierarchy)}\label{def:von_neumann}\\
The \emph{Von Neumann's Hierarchy} is a collection of sets indexed by elements of $Ord$, defined recursively in the following way:
\bce[(i)]
\item 
\begin{equation}
V_0 = \emptyset
\end{equation}
\item 
\begin{equation}
V_{\alpha+1} = \power{V_\alpha}\mbox{ for any ordinal $\alpha$}
\end{equation}
\item
\begin{equation} 
V_\lambda = \bigcup_{\beta < \lambda} V_\beta \mbox{ for a limit ordinal $\lambda$}
\end{equation}
\ece
\end{definition}
% TODO mame union pro tridy? (asi cajk)

\begin{definition}{(Rank)}\label{def:rank}\\
Given a set $x$, we say that the rank of $x$ (written as $rank(x)$) is the least ordinal $\alpha$ such that
\begin{equation}
x \in V_{\alpha+1}
\end{equation}
\end{definition}
Due to \emph{Regularity}, every set has a rank.\footnote{See chapter 6 of \cite{JechBook} for details.}
% TODO abchc mozna prepsat na fli?

\begin{definition}{($\omega$)}\\
\begin{equation}
\omega \defeq \bigcap\{x : x is a limit ordinal)\}
\end{equation}
\end{definition}

\

\subsubsection{Cardinal numbers}

\begin{definition}{(Cardinality)}\\
Given a set $x$, let the cardinality of $x$, written $|x|$, be defined as the smallest ordinal number such that there is an injective mapping from $x$ to $\alpha$.
\end{definition}
For formal details as well as why every set can be well-ordered assuming \emph{Choice}, see \cite{JechBook}.

\begin{definition}{(Aleph function)}\label{def:aleph}\\
Let $\omega$ be the set defined by \ref{def:omega}.
We will recursively define the function $\aleph$ for all ordinals.
\bce[(i)]
\item $\aleph_0 = \omega$
\item $\aleph_{\alpha+1}$ is the least cardinal larger than $\aleph_\alpha$\footnote{"The least cardinal larger than $\aleph_\alpha$" is sometimes notated as $\aleph_\alpha^{+}$}
\item $\aleph_\lambda = \bigcup_{\beta < \lambda}\aleph_\beta$ for a limit ordinal $\lambda$
\ece
\end{definition}

\begin{definition}{(Cardinal number)}\label{def:cardinal}\\
We say a set $x$ is a \emph{cardinal number}, usually called a \emph{cardinal}, if either $x \in \omega$
\end{definition}
Cardinals will be notated by lower-case greek letters starting from $\kappa, \lambda, \mu, \ni, \ldots$\footnote{$\lambda$ is also sometimes used for limit ordinals, the distinction should be clear from the context.}.

\begin{definition}{(Cofinality)}\label{def:cofinality}\\
Let $\lambda$ be a limit ordinal. The \emph{cofinality} of $\lambda$, written $cf(\lambda)$, is the least limit ordinal $\alpha$ such that there is an increasing $\alpha$-sequence\footnote{TODO def $\alpha$-sequence} $\langle \lambda_\beta : \beta < \alpha \rangle$ with $lim_{\beta \then \alpha} \lambda_\beta = \lambda$.
\end{definition}

\begin{definition}{(Limit Cardinal)}\label{def:limit_ordinal}\\
We say that a cardinal $\kappa$ is a \emph{limit cardinal} if
\begin{equation}
(\exists \alpha \in Ord)(\kappa = \aleph_\alpha)
\end{equation}
\end{definition}

\begin{definition}{(Strong Limit Cardinal)}\label{def:limit_ordinal}\\
We say that an ordinal $\kappa$ is a \emph{strong limit cardinal} if it is a \emph{limit cardinal} and 
\begin{equation}
\forall \alpha (\alpha \in \kappa \then \power{\alpha} \in \kappa)
\end{equation}
\end{definition}

\begin{definition}{(Generalised Continuum Hypothesis)}\\
\begin{equation}
\aleph_{\alpha+1}=2^{\aleph_\alpha}
\end{equation}
\end{definition}
If \emph{GCH} holds (for example in Gödel's $L$, see chapter 3), the notions of a limit cardinal and a strong limit cardinal are equivalent.

\

\subsubsection{Relativisation} % and absoluteness?
\begin{definition}{(Relativization)}\label{def:relativization}\\ %\cite[Definition 12.6]{JechBook} 
Let $M$ be a class, $R$ a binary relation on $M$ and let $\varphi(p_1, \ldots, p_n)$ be a first-order formula with $n$ parameters. 
The \emph{relativization of $\varphi$ to $M$ and $R$} is the formula, written as $\varphi^{M, R}(p_1, \ldots, p_n)$, defined in the following inductive manner:
\bce[(i)]
\item $(x \in y)^{M,R} \iff R(x, y)$
\item $(x = y)^{M,R} \iff x = y$
\item $(\neg \varphi)^{M,R} \iff \neg \varphi^{M,R}$
\item $(\varphi\ \&\ \psi)^{M,R} \iff \varphi^{M,R}\ \&\ \psi^{M,R}$
\item $(\exists x \varphi)^{M,R} \iff (\exists x \in M) \varphi^{M,R}$
\ece
\end{definition}

\

\subsubsection{Higher-Order Logic}
Since we will utilise some basic tools of set theories formalized in second- and occassionally higher-order logic, we need to establish the basics here. This part is heavily inspired by Preliminaries from \cite{KanBook}.

TODO viz kanamori p. 6

TODO proc se neda formalizovat obecne splnovani ve vyssich radech? cite?

While higher-order satisfaction relation for proper classes is unformalizable\footnote{TODO CITE KDE? Tarski nebo tak neco?},we can formalize satisfaction in a structure. For the rest of this chapter, let $D$ be a domain of such structure.

TODO druhoradove splnovani?\\

% todo lepsi slovo nez variables?
% For the following definition, we need variables and quantifiers of higher orders. Let \emph{type 1} variables be usual variables of first-order set theory. % spis logic?
\begin{definition}{(Hierarchy of formulas)}\label{def:analytical_hierarchy}\\
Let $\varphi$ be a formula. ((v logice radu $n$)) $\Pi^m_n$ und $\Sigma^m_n$

\end{definition}

\begin{lemma}\label{lemma:delta_0_absolute}
$\Delta_0$ formulas are absolute in transitive sets, in other words, let $\varphi$ be a first-order $\Delta_0$ formula and let $M$ be a transitive class.
\begin{equation}
\varphi \iff \varphi^M
\end{equation}
\end{lemma}


\begin{definition}{$(\sf{ZFC\textsubscript{2}})$}\label{def:zfc_2}\\
TODO ?
\end{definition}

TODO nenechat do patricne kapitoly? asi jo.

\begin{definition}{(\emph{Reflection\textsubscript{1}})}\label{def:reflection_1}\\
\begin{equation}
ASD
\end{equation}
\end{definition}

\

% TODO mozna zminit Levyho "sentential reflection"? pouzivame to v indescribable
% Kanamori v reflexi rika, ze pro $\varphi$ a libovolny $\beta \in Ord$ existuje $\alpha$ ze $\varphi^{V_\alpha} \iff \varphi$ (s parametrama $x_1, \ldots, x_n \in V_\alpha$)
% Prepsat modelovy veci do "kanamoriho notace" $\langle V_\kappa, \in, R \rangle$ kde $R$ je nejaka relace




\newpage
\section{Levy's first-order reflection}\label{sec:fixed}

\subsection{Introduction}
This section will try to present Lévy's proof of a~general reflection principle being equivalent to Replacement and Infinity under ZF minus Replacement and Infinity.
We will first introduce a~few axioms and definitions that were a~different in Lévy's paper\cite{Levy60a}, but are equivalent to today's terms. We will write them in contemporary notation, our aim is the result, not history of set theory notation. 

Please note that Lévy's paper was written in a~period when Set theory was oriented towards semantics, which means that everything was done in a~model. 
All proofs were a~model that of $\sf{ZF(C)}$ was $V_\kappa$ (notated as $R(\kappa)$ at the time) for some cardinal $\kappa$, which means that $\kappa$ is a~inaccessible cadinal.  % fakt? mozna ze my to proste delame ciste pres semanticky dusledek, ale jinak to vyslo nastejno.
Please bear in mind that this is vastly different from saying that there is an inaccessible $\kappa$ inside the model. This $V_\kappa$ is also referred to as $Scm^{\sf{Q}}(u)$, which means that $u$ is a~standard complete model of an undisclosed axiomatic set theory $\sf{Q}$ formulated in the "non-simple applied first order functional calculus", which is second-order theory is today's terminology, we are allowed to quantify over functions and thus get rid of axiom schemes. (Note that Lévy always speaks of "the axiom of replacement"). Besides placeholder set theory $\sf{Q}$, and $\sf{ZF}$, which the reader should be familiar with, theories $\sf{Z}$, $\sf{S}$, and $\sf{SF}$ are used in the text. $\sf{Z}$ is $\sf{ZF}$ minus replacement, $\sf{S}$ is $\sf{ZF}$ minus replacement and infinity, and finally $\sf{SF}$ is $\sf{ZF}$ minus infinity. The axiom of \emph{Subsets} is an older name for the axiom scheme of specification (and it's not a~scheme since we are now working in second order logic). Also note that universal quantifier does not appear, $\forall x \varphi (x)$ would be written as $(x) \varphi (x)$, the symbol for negation is "$\sim$", we will use "$\neg$" the whole time.



% TODO mozna neco o simple nebo non-simple logic, viz mathoverflow

%\subsection{Preliminaries}
\subsection{Lévy's Original Paper}\label{sec:Levy1960}
This chapter uses $\sf{ZF}$ instead of the usual $\sf{ZFC}$, unless explicitly stated otherwise.

The following are a few definitions that are used in Lévy's original article. \footnote{While some of them won't be of much use in this paper, they will provide extremely helpful when reading the original article as set theory notation and terminology has evolved in the last 50 years considerably.}

TODO relativizace -- notace u levy a u nas -- \ref{def:relativisation}

Next two definitions are not used in contemporary set theory, but they illustrate 1960's set theory mind-set and they are used heavily in Lévy's text, so we will include and explain them for clarity. Generally in this chapter, $\sf{Q}$ stands for an undisclosed axiomatic set theory, $u$ is usually a model, counterpart of today's $V$\footnote{Which is of course not referred to as a model, but it is used in a similar fashion, in this case the term "model" was a metamathematical notion because it was not based on any underlying structure of theory. It can be easily formalized in any set theory, but it's not helpful for our case.}, $E$ is a relation that serves as $\in$ in the given model.
% TODO to uz jsi rikal nahore

% TODO je to relativizovany, jak rika shepherdson?
\begin{definition}{(Standard model of a set theory)}\label{def:sm_q}\\
Let $\sf{Q}$ be a axiomatic set theory in first-order logic. We say the the a class $u$ is a standard model of $\sf{Q}$ with respect to a membership relation $E$, written as $Sm^{\sf{Q}}(u)$, iff both of the following hold
\bce[(i)]
\item $(x, y) \in E \iff y \in u\ \&\ x \in y$
\item $y \in u\ \&\ x \in y \then x \in u$
\ece
\end{definition}
\begin{definition}{Standard complete model of a set theory}\label{def:scm_q}\\
Let $\sf{Q}$ and $E$ be like in \ref{def:sm_q}. We say that that $u$ is a standard complete model of $\sf{Q}$ with respect to a membership relation $E$ iff both of the following hold
\bce[(i)]
\item $u$ is a transitive set with respect to $\in$
\item $\forall E ((x, y) \in E \iff (y \in u\ \&\ x \in y)\ \&\ Sm^{\sf{Q}}(u, E))$
\ece
this is written as $Scm^{\sf{Q}}(u)$.
\end{definition}

% TODO what is "simple first-order functional calculus" a "non-simple first-order functional calculus"? math.stackovrflow?

\begin{definition}{(Inaccessible cardinal with respect to $\sf{Q}$)}\label{def:levy_inaccessible_q}\\
Let $\sf{Q}$ be an axiomatic first-order set theory. We say that a cardinal $\kappa$ is inaccessible with respect to $\sf{Q}$, we write $In^{\sf{Q}}(\kappa)$, iff
\begin{equation}
Scm^{\sf{Q}}(V_\kappa).
\end{equation}
\end{definition}

\begin{definition}{(Inaccessible cardinal with respect to $\sf{ZF}$)}\label{def:levy_inaccessible}\\
When a cardinal $\kappa$ is inaccessible with respect to $\sf{ZF}$, we only say that it is inaccessible. In the abbreviated version, we just leave out the superscript.
\begin{equation}
In(\kappa) \iff In^{\sf{ZF}}(\kappa)
\end{equation}
\end{definition}
%Note that this definition is more general than the usual one\footnote{Which says that a cardinal $\kappa$ is inaccessible iff it is a strong limit regular cardinal and only .}, we will often write $In(\kappa)$ as a shorthand for $In^{\sf{ZF}}(\kappa)$.
% TODO ekvivalence s beznou definici?
% TODO tohle je lepsi protoze nepotrebujes AC

% The following is a principle of complete reflection over $\sf{ZF}$.
\begin{definition}{($N$)}\label{def:levy_axiom_n}\\
The following is an axiom schema of complete reflection over $\sf{ZF}$, denoted as $N$.
\begin{equation}
N \iff \exists u (Scm^{\sf{ZF}}(u)\ \&\ \forall x_1, \ldots , x_n (x_1, \ldots , x_n \in u \then \varphi \iff \varphi^{u}))
\end{equation}
where $\varphi$ is a~formula which contains no free variables except for $x_1, \ldots , x_n$.
\end{definition}

\begin{definition}{($N_0$)}\label{def:levy_axiom_n0}\\
If we substitute $\sf{ZF}$ for $\sf{S}$, which is $\sf{ZF}$ minus \emph{Replacement} and \emph{Infinity}, we obtain what will now be called $N_0$.
\begin{equation}
N_0 \iff \exists u (Scm^{\sf{S}}(u)\ \&\ \forall x_1, \ldots , x_n (x_1, \ldots , x_n \in u \then \varphi \iff \varphi^{u}))
\end{equation}
where $\varphi$ is a~formula which contains no free variables except for $x_1, \ldots , x_n$.
\end{definition}

Once we have established the definitions, it's time to prove something interesting.
% Note that this by (\ref{def:levy_inaccessible}) equivalent to $\exists u (In^{\sf{ZF}}(u)\ \&\ \forall x_1, \ldots , x_n (x_1, \ldots , x_n \in u \then \varphi \iff \varphi^{u}))$, where $In(\alpha)$ is equivalent to the standard notion of inaccessibility.
\subsection{$\sf{S} \models (N_0\ \iff\ \emph{Replacement}\ \&\ \emph{Infinity})$} 
% nebyl by to lepsi nadpis bez S?

Let $N_0$ be defined as in \ref{def:levy_axiom_n0}, for \emph{Infinity} see \ref{def:infinity}.
\begin{theorem}
In $\sf{S}$, the schema $N_0$ implies \emph{Infinity}.
\end{theorem}

\begin{proof}
For any $\varphi$, $N_0$ gives us $\exists u Scm^{\sf{S}}(u)$, which means that there is a~set $u$ that is identical to $V_\alpha$ for some alpha, so $\exists \alpha Scm^{\sf{S}}(V_\alpha)$.
We don't know the exact size of this $\alpha$, but we know that $\alpha \geq \omega$, otherwise $\alpha$ would be finite, therefore not closed under the powerset operation, which would contradict \emph{Powerset}. 
In order to prove that it is a~model of $\sf{S}$, we would need to verify all axioms of $\sf{S}$. We have already shown that $\omega$ is closed under the powerset operation. Foundation, extensionality and comprehension are clear from the fact that we work in $\sf{ZF}$\footnote{We only need to verify axioms that provide means of constructing larger sets from smaller to make sure they don't exceed $\omega$. Since $\omega$ is an initial segment of $ZF$, the axiom scheme of specification can't be broken, the same holds for foundation and extensionality.}, pairing is clear from the fact, that given two sets $x$, $y$, they have ranks $\alpha$, $\beta$, without loss of generality we can assume that $\alpha \leq \beta$, which means that $x \in V_\alpha \in V_\beta$, therefore $V_\beta$ is a~set that satisfies the paring axiom: it contains both $x$ and $B$.

Note that this implies that any (strong) limit cardinal is a model of $\sf{S}$.

We now want to prove that $V_\alpha$ leads to existence of an inductive set, which is a~set that satisfies $\exists A(\emptyset \in A~\& \forall  x \in A~((x \cup \{ x\}) \in A))$. If we can find a~way to construct $V_\omega$ from any $V_\alpha$ satisfying $\alpha \geq \omega$, we are done. Since $\omega$ is the least limit ordinal, all we need is the following

\begin{equation}
\bigcap \{V_\kappa\ |\  \forall \lambda(\lambda < \kappa \then \exists \mu(\lambda < \mu < \kappa))\}
\end{equation}
because $V_\kappa$ is a~transitive set for every $\kappa$, thus the intersection is non-empty unless empty set satisfies the property or the set of $V_\kappa$s is itself empty.
\end{proof}

\

Let $N_0$ be defined as in \ref{def:levy_axiom_n0}, for \emph{Replacement} see \ref{def:replacement}.
\begin{theorem}
In $\sf{S}$, the schema $N_0$ implies \emph{Replacement}.
\end{theorem}

\begin{proof}
Let $\varphi(v, w, x_1, \ldots, x_n)$ be a~formula wth no free variables except $v, w, x_1, \ldots, x_n$ where $n$ is any natural number.
Let $\chi$ be an instance of replacement schema for this $\varphi$ which is what we want to prove:
\begin{equation}
\begin{gathered}
\chi = \forall r, s, t(\varphi(r, s, x_1, \ldots, x_n) \& \varphi(r, t, x_1, \ldots, x_n) \then s = t) \\
\then \forall x \exists y \forall w (w \in y \iff \exists v (v \in x \& \varphi(v, w, x_1, \ldots, x_n)))
\end{gathered}
\end{equation}

We can deduce the following from $N_0$: 
\bce[(i)]
\item $x_1, \ldots, x_n, v, w \in u \then (\varphi \iff \varphi^{u}) $
\item $x_1, \ldots, x_n, v \in u \then (\exists w \varphi \iff (\exists w \varphi)^{u})$
\item $x_1, \ldots, x_n, x \in u \then (\chi \iff \chi^{u})$
\item $\forall x_1, \ldots, x_n \forall x (\chi \iff (\forall x_1, \ldots, x_n \forall x \chi)^{u})$
\ece

It is easy to see that $\bold{(i)}$, $\bold{(ii)}$, $\bold{(iii)}$ are the instances of $N_0$ for $\varphi$, $\exists w \varphi$ and $\chi$ respectively.  
From relativization we also know that $(\exists w \varphi)^{u}$ is equivalent to $\exists w (w \in u \& \varphi^{u})$.
Therefore $\bold{(ii)}$ is equivalent to
\begin{equation}
x_1, \ldots, x_n, v \in u \then (\exists w (w \in u\ \&\ \varphi^{u})). 
\end{equation}

If $\varphi$ is a~function\footnote{$ \forall r, s, t(\varphi(r, s)\ \&\ \varphi(r, t) \then r=t $)}, then for every $x \in u$, which is also $x \subset u$ by the transitivity of $Scm^{\sf{S}}(u)$,
it maps elements of $x$ onto $u$. From the axiom scheme of comprehension\footnote{Lévy's uses its equivalent, axiom of subsets}, we can find $y$, a~set of all images of elements of $x$.
That gives us $x_1, \ldots, x_n, x \in u \then \chi$. By (iii) we get $x_1, \ldots, x_n, x \in u \then \chi^{u}$, the universal closure of this formula is $(\forall x_1, \ldots, x_n \forall x \chi)^{u}$, 
which together with (iv) yields $\forall x_1, \ldots, x_n \forall x \chi$. By the means of specification we end up with $\chi$, Q.E.D. 
\end{proof}


What we have just proven is just a single theorem from said article, we will introduce other interesting propositions, mostly related to the existence of large cardinals, later in their appropriate context in chapter 3.

% =====================================================================================================================================
% \newpage
\subsection{Contemporary restatement}
We will now prove what is also Lévy's reflection theorem, but a little stronger, rephrased with more up to date set theory. The main difference is, that while Lévy reflects $\varphi$ from $V$ into a set $u$ that is a "standard complete model of $\sf{S}$"\footnote{Any limit ordinal is in fact a model of $\sf{S}$, we shall pay more attention to that in a moment.}, we say that there is a $V_\alpha$ that reflects $\varphi$. In other words, we don't need $\alpha$ to be an inaccessible cardinal like Lévy does.

We will prove the equivalence of $N_0$ with \emph{Replacement} and \emph{Infinity} in $\sf{S}$ in two parts. First, we will show that \emph{$Reflection_1$} is a theorem of $\sf{ZF}$, then the second implication which proves \emph{Infinity} and \emph{Replacement} from $N_0$.

The following lemma is usually done in more parts, the first being with one formula and the other with $n$. We will only state and prove the generalised version for $n$ formulas, knowing that $n=1$ is just a~specific case and the proof is exactly the same.

% TODO Pouzivame ZFC!

\begin{lemma}{ }\label{lemma:reflection_lemma}
Let $\varphi_1, \ldots, \varphi_n$ be formulas with $m$ parameters\footnote{For formulas with a different number of parameters, take for $m$ the highest number of parameters among given formulas. Add spare parameters to the other formulas so that $x$ remains the last parameter. That can be done in a~following manner: Let $\varphi'_i$ be the a~formula with k parameters, $k < m$. Let us set $\varphi_i(u_1, \ldots, u_{m-1}, x) = \varphi'_i(u_1, \ldots, u_{k-1}, u_k, \ldots, u_{m-1}, x)$, notice that $u_k, \ldots, u_{m-1}$ are the aforementioned spare variables.}. 
\bce[(i)]
\item For each set $M_0$ there is such $M$ that $M_0 \subset M$ and the following holds for every $i \leq n$:
\begin{equation}\label{equation:refl_lemma_i}
\exists x \varphi_i(u_1, \ldots, u_{m-1}, x) \then (\exists x \in M) \varphi_i(u_1, \ldots, u_{m-1}, x)
\end{equation}
for every $u_1, \ldots, u_{m-1} \in M$.

\item Furthermore there is an ordinal $\alpha$ such that $M_0 \subset V_\alpha$ and the following holds for each $i \leq n$:
\begin{equation}\label{equation:refl_lemma_ii}
\exists x \varphi_i(u_1, \ldots, u_{m-1}, x) \then (\exists x \in V_\alpha) \varphi_i(u_1, \ldots, u_{m-1}, x)
\end{equation}
for every $u_1, \ldots, u_{m-1} \in M$.

\item Assuming \emph{Choice}, there is $M$, $M_0 \subset M$ such that \ref{equation:refl_lemma_i} holds for every $M,\ i \leq n$ and $|M| \leq |M_0| \cdot \aleph_0$.
\ece
\end{lemma}

\begin{proof}
We will simultaneously prove statements $\bold{(i)}$ and $\bold{(ii)}$, denoting $M^T$ the transitive set required by part $\bold{(ii)}$. Unless explicitly stated otherwise for specific steps, it is thought to be equivalent to $M$.

Let us first define operation $H(u_1, \ldots, u_{m-1})$ that gives us the set of $x$'s with minimal rank satisfying $\varphi_i(u_1, \ldots, u_{m-1}, x)$ for given parameters $u_1, \ldots, u_{m-1}$ for every $i \leq n$.

\begin{equation}
H_i(u_1, \ldots, u_n) = \{x \in C_i: (\forall z \in C)(rank(x) \leq rank(z))\}
\end{equation}
for each $i \leq n$, where
\begin{equation}
C_i = \{x: \varphi_i(u_1, \ldots, u_{m-1}, x)\} \mbox{ for $i \leq n$}
\end{equation}

\

Next, let's construct $M$ from given $M_0$ by induction. 
\begin{equation}
M_{i+1} = M_i \cup \bigcup_{j=0}^{n} \bigcup \{H_j(u_1, \ldots, u_{m-1}): u_1, \ldots, u_{m-1} \in M_i\}
\end{equation}
In other words, in each step we add the elements satisfying $\varphi(u_1, \ldots, u_{m-1}, x)$ for all parameters that were either available earlier or were added in the previous step. 
For statement $\bold{(ii)}$, this is the only part that differs from (i). Let us take for each step transitive closure of $M_{i+1}$ from (i). In other words, let $\gamma$ be the smallest ordinal such that 
\begin{equation}
(M^T_i \cup \bigcup_{j=0}^{n} \{\bigcup\{H_j(u_1, \ldots, u_{m-1}): u_1, \ldots, u_{m-1} \in M_i\}\}) \subset V_\gamma
\end{equation}
Then the incremetal step is like so:
\begin{equation}
M^T_{i+1} = V_\gamma
\end{equation}
The final $M$ is obtained by joining all incremental steps together. 
\begin{equation}
M = \bigcup_{i=0}^{\infty} M_i, \mbox{  }M^T = \bigcup_{i=0}^{\infty} M^T_i
\end{equation}

\

We have yet to finish part $\bold{(iii)}$.
Let's try to construct a~set $M'$ that satisfies the same conditions like $M$ but is kept as small as possible. Assuming the Axiom of Choice, we can modify the process so that cardinality of $M'$ is at most $|M_0| \cdot \aleph_0$. Note that the size of $M'$ is determined by the size of $M_0$ an, most importantly, by the size of $H_i(u_1, \ldots, u_{m-1})$ for any $i \leq n$ in individual levels of the construction. Since the lemma only states existence of some $x$ that satisfies $\varphi_i(u_1, \ldots, u_{m-1}, x)$ for any $i \leq n$, we only need to add one $x$ for every set of parameters but $H_i(u_1, \dots, u_{m-1})$ can be arbitrarily large. Since Axiom of Choice ensures that there is a~choice function, let $F$ be a~choice function on $\power{M'}$. Also let $h_i(u_1, \ldots, u_{m-1}) = F(H_i(u_1, \ldots, u_{m-1}))$ for $i \leq n$, which means that $h$ is a~function that outputs an $x$ that satisfies $\varphi_i(u_1, \ldots, u_{m-1}, x)$ for $i \leq n$ and has minimal rank among all such witnesses. The induction step needs to be redefined to
\begin{equation}
M'_{i+1} = M'_i \cup \bigcup_{j=0}^n \{ H_j(u_1, \ldots, u_{m-1}): u_1, \ldots, u_{m-1} \in M'_i \}
\end{equation}
In every step, the amount of elements added in $M'_{i+1}$ is equivalent to the amount of sets of parameters the yielded elements not included in $M'_i$. So the cardinality of $M'_{i+1}$ exceeds the cardinality of $M'_i$ only for finite $M'_i$. It is easy to see that if $M_0$ is finite, $M'$ is countable because it was built from countable union of finite sets. If $M_0$ is countable or larger, cardinaly of $M'$ is equal to the cardinality of $M_0$.\footnote{It can not be smaller because $|M'_{i+1}|  \geq |M'_i|$ for every $i$. It may not be significantly larger because the maximum of elements added is the number of $n$-tuples in $M'_i$, which is of the same cardinality is $M'_i$.}
Therefore $|M'| \leq |M_0| \cdot \aleph_0$
\end{proof}

% TODO proc $\leq$ a ne $=$?
And now for the theorem itself

\begin{theorem}{(Lévy's first-order reflection theorem)}\label{theorem:first_order_reflection}\\
Let $\varphi(x_1, \ldots, x_n)$ be a~first-order formula.
\bce[(i)]
\item For every set $M_0$ there exists $M$ such that $M_0 \subset M$ and the following holds:
\begin{equation}
\varphi^M(x_1, \ldots, x_n) \iff \varphi(x_1, \ldots, x_n)
\end{equation}
for every $x_1, \ldots, x_n \in M$.

\item For every set $M_0$  there is a~transitive set $M$, $M_0 \subset M$ such that the following holds:
\begin{equation}
\varphi^M(x_1, \ldots, x_n) \iff \varphi(x_1, \ldots, x_n)
\end{equation}
for every $x_1, \ldots, x_n \in M$.

\item For every set $M_0$ there is $\alpha$ such that $M_0 \subset V_{\alpha}$ and the following holds:
\begin{equation}
\varphi^{V_{\alpha}}(x_1, \ldots, x_n) \iff \varphi(x_1, \ldots, x_n)
\end{equation}
for every $x_1, \ldots, x_n \in M$.

\item Assuming \emph{Choice}, for every set $M_0$ there is $M$ such that $M_0 \subset M$ and $|M| \leq |M_0| \cdot \aleph_0$ and the following holds:
\begin{equation}
\varphi^M(x_1, \ldots, x_n) \iff \varphi(x_1, \ldots, x_n)
\end{equation}
for every $x_1, \ldots, x_n \in M$.
\ece
\end{theorem}

\begin{proof}
Let's prove $\bold{(i)}$ for one formula $\varphi$ via induction by complexity first. We can safely assume that $\varphi$ contains no quantifiers besides $\exists$ and no logical connectives other than $\neg$ and $\&$.
Assume that this $M$ is obtained from lemma \ref{lemma:reflection_lemma}. The fact, that atomic formulas are reflected in every $M$ comes directly from definition of relativization and the fact that they contain no quantifiers.\footnote{Note that this does not hold generally for relativizations to $M, E$, but only for relativization to $M, \in$, which is our case.}
The same holds for formulas in the form of $\varphi = \neg \varphi'$. Let us recall the definition of relativization for those formulas in \ref{def:relativization}.
\begin{equation}
(\neg \varphi_1)^M \iff \neg (\varphi_1^M)
\end{equation}
Because we can assume from induction that $\varphi'^M \iff \varphi'$, the following holds:
\begin{equation}
(\neg \varphi')^M \iff \neg (\varphi'^M) \iff \neg \varphi'
\end{equation}

The same holds for $\varphi = \varphi_1\mbox{ }\&\mbox{ }\varphi_2$. From the induction hypothesis we know that $\varphi_1^M \iff \varphi_1$ and $\varphi_2^M \iff \varphi_2$, which together with relativization for formulas in the form of $\varphi_1 \& \varphi_2$ gives us
\begin{equation}
(\varphi_1\mbox{ }\&\mbox{ }\varphi_2)^M \iff \varphi_1^M\mbox{ }\&\mbox{ }\varphi_2^M \iff \varphi_1\mbox{ }\&\mbox{ }\varphi_2
\end{equation}

\

Let's now examine the case when from the induction hypethesis, $M$ reflects $\varphi'(u_1, \ldots, u_n, x)$ and we are interested in $\varphi = \exists x \varphi'(u_1, \ldots, u_n, x)$.
The induction hypothesis tells us that 
\begin{equation}
\varphi'^M(u_1, \ldots, u_n, x) \iff \varphi'(u_1, \ldots, u_n, x)
\end{equation}
so, together with above lemma \ref{lemma:reflection_lemma}, the following holds:
\begin{equation}
\begin{gathered}
\varphi(u_1, \ldots, u_n, x) \\
\iff \exists x \varphi'(u_1, \ldots, u_n, x) \\
\iff (\exists x \in M) \varphi'(u_1, \ldots, u_n, x) \\
\iff (\exists x \in M) \varphi'^M (u_1, \ldots, u_n, x) \\
\iff (\exists x \varphi'(u_1, \ldots, u_n, x))^M \\
\iff \varphi^M(u_1, \ldots, u_n, x)
\end{gathered}
\end{equation}
Which is what we have needed to prove:

\

So far we have proven part $\bold{(i)}$ of this theorem for one formula $\varphi$, we only need to verify that the same holds for any finite number of formulas. This has in fact been already done since lemma \ref{lemma:reflection_lemma} gives us $M$ for any (finite) amount of formulas. We can than use the induction above  to verify that it reflects each of the formulas individually.

\

Now we want to verify other parts of our theorem. Since $V_\alpha$ is a~transitive set, by proving $\bold{(iii)}$ we also satisfy $\bold{(ii)}$. To do so, we only need to look at part $\bold{(ii)}$ of lemma \ref{lemma:reflection_lemma}. All of the above proof also holds for $M = V_\alpha$. 

To finish part $\bold{(iv)}$, we take $M$ of size $\leq |M_0| \cdot \aleph_0$, which exists due to part $\bold{(iii)}$ of lemma \ref{lemma:reflection_lemma}, the rest being identical.
\end{proof}

\

\begin{theorem}\label{levy_equivalence_contemporary}
$\emph{Reflection}$ is equivalent to $\emph{Infinity}\ \&\ \emph{Replacement}$ under ZFC minus $\emph{Infinity}\ \&\ \emph{Replacement}$
\end{theorem}

\

\begin{proof}
Since \ref{theorem:first_order_reflection} already gives one side of the implication, we are only interested in showing the converse which we shall do in two parts:

$\bold{\emph{Reflection} \then \emph{Infinity}}$

Let us first find a~formula to be reflected that requires a~set $M$ at least as large as $V_\omega$.
Let us consider the following formula:
\begin{equation}
\varphi'(x) = \forall \lambda(\lambda < x \then \exists \mu(\lambda < \mu < x))
\end{equation}
Because $\varphi$ says "there is a~limit ordinal", if it holds for some $x$, the Infinity axiom is very easy to satisfy. But we know that there are limit ordinals in $\sf{ZF}$, therefore $\varphi = \exists x \varphi'(x)$ is a~valid statement. $\emph{Reflection}$ then gives us a~set $M$ in which $\varphi^M$ holds, that is, a~set that contains a~limit ordinal. So the set of off limit ordinals is non-empty and because ordinals are well-founded, it has a~minimal element. Let's call it $\mu$.
\begin{equation} % todo mozna sjednotit s definici limitnoho ordinalu?
\mu = \bigcap \{V_\kappa\ :\  \forall \lambda(\lambda < \kappa \then \exists \mu(\lambda < \mu < \kappa))\}
\end{equation}
We can see that $\mu$ is the least limit ordinal and therefore it satisfies $\emph{Infinity}$.

\

$\bold{\emph{Reflection} \then \emph{Replacement}}$

Given a~formula $\varphi(x, y, u_1, \ldots, u_n)$, we can suppose that it is reflected in any $M$ \footnote{Which means that for $x, y, u_1, \ldots, u_n \in M$, $\varphi^M(x, y, u_1, \ldots, u_n) \iff \varphi(x, y, u_1, \ldots, u_n)$.}
What we want to obtain is the following:
\begin{equation}
\forall x, y, z (\varphi(x, y, u_1, \ldots, u_n) \& \varphi(x, z, u_1, \ldots, u_n) \then y = z) \then
\end{equation}
\begin{equation}
\then \forall X \exists Y \forall y\ (y \in Y \iff \exists x (\varphi(x, y, u_1, \ldots, u_n)\ \&\ x \in X ))
\end{equation}

We do also know that $x, y \in M$, in other words for every $X$, $Y\ =\ \{y\ |\ \varphi(x, y, u_1, \ldots, u_n)\}$ we know that $X \subset M$ and $Y \subset M$, which, together with the comprehension schema\footnote{Called the axiom of subsets in Lévy's proof.} implies that $Y$, the image of $X$ over $\varphi$, is a~set.
Which is exactly the Replacement Schema we hoped to obtain.
\end{proof}

\

We have shown that $\emph{Reflection}$ for first-order formulas, $\emph{Reflection}_1$ is a~theorem of $\sf{ZF}$, which means that it won't yield us any large cardinals. We have also shown that it can be used instead of the Axiom of Infinity and Replacement Scheme, but $\sf{ZF}\ +\ \emph{Reflection}_1$ is a~conservative extension of $\sf{ZF}$. Besides being a~starting point for more general and powerful statements, it can be used to show that $\sf{ZF}$ is not finitely axiomatizable. That is because $\emph{Reflection}$ gives a~model to any finite number of (consistent) formulas. So if $\varphi_1, \ldots, \varphi_n$ for any finite $n$ would be the axioms of $\sf{ZF}$, $\emph{Reflection}$ would always contain a~model of itself, which would in turn contradict the Second Gödel's Theorem\footnote{See chapter \ref{section:inaccessibility} for further details.}.
Notice that, in a~way, reflection is complementary to compactness. Compactness argues that given a set of sentences, if every finite subset yields a~model, so does the whole set. Reflection, on the other hand, says that while the whole set has no model in the underlying theory, every finite subset does have one.

Also, notice how reflection can be used in ways similar to upward Löwenheim–Skolem theorem. Since Reflection extends any set $M_0$ into a~model of given formulas $\varphi_1, \ldots, \varphi_n$, we can choose the lower bound of the size of $M$ by appropriately chocing $M_0$.

In the next section, we will try to generalize \emph{Reflection} in a~way that transcends $\sf{ZF}$ and finally yields some large cardinals.

\newpage
% =====================================================================================================================================
\section{Reflection And Large Cardinals}

In this chapter we aim to examine stronger reflection properties in order to reach cardinals unavailable in $\sf{ZFC}$. Like we said in the first chapter, % [TODO odkaz na uvod]
the variety of reflection principles comes from the fact that there are many way to formalize "properties of the universal class". It is not always obvious what properties hold for $V$ because, 
%unlike Lévy's approach, not much attention is paid to what exactly is this $V$, and, more importantly, there are many ways to formalize the notion of property. 
We have shown that reflecting properties as first-order formulas doesn't allow us to leave $\sf{ZFC}$. We will broaden the class of admissible properties to be reflected and see whether there is a~natural limit in the height or width on the reflected universe and also see that no matter how far we go, the universal class is still as elusive as it is when seen from $\sf{S}$. That is because for every process for obtaining larger sets such as for example the powerset operation in $\sf{ZFC}$, this process can't reach $V$ and thus, from reflection, there is an initial segment of $V$ that can't be reached via said process. 

To see why this is important, let's dedicate a few lines to the intuition behind the notions of limitness, regularity and inaccessibility in a manner strongly influenced by \cite{Infinity_in_mind}. To see why limit and strongly limit cardinals are worth mentioning, note that they are "limit" not only in a sense of being a supremum of an ordinal sequence, they also show that a certain way of obtaining bigger objects from smaller ones limited in terms of possibilities. $\aleph_\lambda$ is a limit cardinal iff there is no $\alpha$ such that $\aleph_{\alpha+1}=\aleph_\lambda$. Strongly limit cardinals point to the limits of the powerset operation. It has been too obvious so far, so let's look at the regular cardinals in this manner. Regular cardinals are those that cannot be\footnote{Assuming $\emph{Choice}$.}, expressed as a supremum of smaller amount of smaller objects\footnote{Just like $\omega$ can not be expressed as a supremum of a finite set consisting solely of finite numbers.}. More precisely, $\kappa$ is regular if there is no way to define it as u union of less than $\kappa$ ordinals, all smaller than $\kappa$. So unless there already is a set of size $\kappa$, \emph{Replacement} is useless in determining whether $\kappa$ is really a set. Note that assuming \emph{Choice}, successor cardinals are always regular, so most\footnote{All provable to exist in $\sf{ZFC}$} limit cardinals are singular cardinals. So if one is traversing the class of all cardinals upwards, successor steps are still sets thanks to the powerset axiom while singular limits cardinal are not proper classes because they are images of smaller sets via \emph{Replacement}. Regular cardinals are, in a way, limits of how far can we get by taking limits of increasing sequences of ordinals obtained via $\emph{Replacement}$.

That all being said, it is easy to see that no cardinals in $\sf{ZFC}$ are both strongly limit and regular because there is no way to ensure they are sets and not proper classes in $\sf{ZFC}$. The only exception to this rule is $\aleph_0$ which needs \emph{Infinity} to exist. % nase otazka je: proc omega a ne jine kardinaly?
It should now be obvious why the fact that $\kappa$ is inaccessible implies that $\kappa\ =\ \aleph_\kappa$.\footnote{This doesn't work backwards, the least fixed point of the $\aleph$ function is the limit of $\{\aleph_0,\ \aleph_{\aleph_0},\ \aleph_{\aleph_{\aleph_0}},\ \ldots \}$, it is singular since the sequence has countably many elements.}

We will also examine the connection between reflection principles and (regular) fixed points of ordinal functions in a manner proposed by Lévy in \cite{Levy60a}. We will also see that, like Lévy has proposed in the same paper, there is a meaningful way to extend the relation between $\sf{S}$ and $\sf{ZFC}$ into a hierarchy of stronger axiomatic set theories. 
% Those are the three lines of thinking that we will find are in fact different facets of the same gem, especially in the section devoted to Inaccessible and Mahlo cardinals.
% viz Shapiro, Stewart. 1987. “Principles of Reflection and Second-order Logic”. Journal of Philosophical Logic 16 (3). Springer: 309–33. http://www.jstor.org/stable/30227043.
% Reflections on Replacement and Reflection: The axioms in a~structuralist setting (Geoffrey Hellman)
%TODO neco o tom, ze kdyz je reflexe formule, da se sama reflektovat?
% The above should make a clear picture of why $\emph{Infinity}$ is a specific case of $\emph{Reflection}$.
%TODO proc je Refl zaroven zobecneny replacement?

% TODO ze "uplne totalni" reflexe se zacykli a rozbije? 

\subsection{Regular Fixed-Point Axioms}
% This small chapter is dedicated to 

Lévy's article mentions various schemata that are not instances of reflection themselves. We will mention them because they are equivalent to $N_0$ and because they are fixed-point theorems, which we will find useful later in this thesis.

\begin{definition}{(Function)}
We say that a first-order formula $\varphi(x, y, u_1, \ldots, u_n)$ with no free variable besides $x, y, u_1, \ldots, u_n$ is a function iff
\begin{equation}
\forall x, y, z (\varphi(x, y, u_1, \ldots, u_n)\ \&\ \varphi(x, z, u_1, \ldots, u_n) \then y=z)
\end{equation}
We will also write functions in the form of "$f(x) = y$". This is defined for given $\varphi(x, y, u_1, \ldots, u_n)$ and given terms $t_1, \ldots, t_n$ as follows
\begin{equation}
f(x)=y \iff \varphi(x, y, t_1, \ldots, t_n)
\end{equation}
\end{definition}

$Ord$ denotes the class of all ordinal numbers.
\begin{definition}{(Strictly increasing function)}\label{def:increasing_function}\\
A function $f: Ord \then Ord$ is said to be \emph{strictly increasing} iff
\begin{equation}
\forall \alpha, \beta \in Ord (\alpha < \beta \then f(\alpha) < f(\beta)).
\end{equation}
\end{definition}

\begin{definition}{(Continuous function)}\label{def:continuous_function}\\
A function $f: Ord \then Ord$ is said to be \emph{continuous} iff
\begin{equation}
\alpha\mbox{ is limit} \then (f(\alpha) = lim_{\beta < \alpha} f(\beta)).
\end{equation}
\end{definition}
Alternatively, a function $f: Ord \then Ord$ is continuous iff for limit $\lambda$, $f(\lambda) = \bigcup_{\alpha < \lambda} f(\alpha)$.
% nebo taky $f(\lambda) = sup {f(\alpha) : \alpha < \lambda}$.

\begin{definition}{(Normal function)}\label{def:normal_function}\\
A function $f: Ord \then Ord$ is said to be \emph{normal} if it is \emph{strictly increasing} and \emph{continuous}.
\end{definition}

\begin{definition}{(Normal function on a set)} Let $\alpha, \delta$ be ordinals. A function $f: \delta \then \alpha$ is called a \emph{normal function on $\alpha$} iff all of the following hold:
\bce[(i)]
\item $f$ is strictly increasing on $\alpha$\footnote{$x \mbox{ is limit} \then (f(x)) = \bigcup_{y < x} f(y)$}
\item $f$ is continuous on $\alpha$
\item the $rng(f)\ =\ \{y: \exists x(f(x)\ =\ y)\}$ is unbounded in $\alpha$.
 \ece
\end{definition} % TODO potrebujeme ji vubec? nema byt (rostouci na alfa), (spojita na alda), ...?

\begin{definition}{Fixed point}\\
We say $\alpha$ is a fixed point of ordinal function $f$ when $\alpha=f(\alpha)$.
\end{definition}

Lévy (\cite{Levy60a}) proposes those axioms as equivalent to one on his reflection principles.
\begin{definition}
$M \iff$ "Every normal function defined for all ordinals has at least one inaccessible number in its range."
\end{definition}
We will rewrite $M$ as a formula to make it clear that it is an axiom scheme and the same can be done with $M'$ as well as $M''$.

Let $\varphi(x, y, u_1, \ldots, u_n)$ be a first-order formula with no free variables besides $x, y, u_1, \ldots, u_n$. The following is equivalent to $M$.
\begin{equation}
\begin{gathered}
\varphi\mbox{ is a normal function}\ \&\ \forall x (x \in Ord \then \exists y(\varphi(x, y, u_1, \ldots, u_n))) \then
\then \exists y (\exists x \varphi(x, y, u_1, \ldots, u_n)\ \&\ cf(y) = y\ \&\ Lim(\kappa))
\end{gathered}
\end{equation}

\begin{definition}
$M' \iff $ "Every normal function defined for all ordinals has at least one fixed point which is inaccessible."
\end{definition}

\begin{definition}
$M'' \iff $ "Every normal function defined for all ordinals has arbitrarily great fixed points which are inaccessible."
\end{definition}


The following axiom is proposed by Drake in \cite{DrakeBook}. 
\begin{definition}{$F$}
Every normal function for all ordinals has a regular fixed point.
\end{definition}

\begin{theorem}
\begin{equation}
F \iff M \iff M' \iff M''
\end{equation}
\end{theorem}

\begin{proof}
One can find the proof of $M \iff M' \iff M''$ in \cite{Levy60a}, \emph{Theorem 1}. %, we will only prove that Drake's $F$ is equivalent too.

TODO podle Levyho

%Since $M'$ makes sure that for arbitrary $f :\ On \then On$ there is an ordinal $\alpha$ such that $In(\alpha)\ \&\ f(\alpha) = \alpha$ that already satisfies $F$.

%For the opposite direction, suppose we have an ordinal function $f$ that has a regular fixed point. We want to prove that $f$ has an inaccessible fixed point. Let's construct a function $f'$ on all ordinals that enumerates the strongly limit ordinals in the range of $f$. $F$ implies that $f'$ also has a regular fixed point, which is therefore an inaccessible ordinal. Let $\alpha$ be the least such ordinal. We have yet to prove that $\alpha$ is also a fixed point of $f$. There is some $\beta$ such that $f(\beta) = \alpha = f'(\alpha)$. Because $f$ is normal, $\beta \leq f(\beta)$\footnote{This can be easily proven: suppose it's not the case and there's the least $\gamma$ such that $f(\gamma) < \gamma$. Because $f$ is increasing, $f(f(\gamma)) < f(\gamma)$, which contradicts the fact that $\gamma$ was supposed to be the least such ordinal.}, so $\beta \leq \alpha$. Since $\alpha$ is regular

%Now let's take the function $f: \alpha \then \aleph_\alpha$, it also has a regular fixed point as implied by $F$. Since for successor ordinals it holds that $\alpha < \aleph_\alpha$, all fixed points are limit ordinals and therefore there is an inaccessible cardinal $\kappa$. For any normal $f$ defined on all ordinals, $f(\kappa)$ is also an inaccessible cardinal.\footnote{Proc?}. We have proven $F \then M$.

% WIKI: Any weakly inaccessible cardinal is also a fixed point of the aleph function.[6] This can be shown in ZFC as follows. Suppose \kappa = \aleph_\lambda is a weakly inaccessible cardinal. If \lambda were a successor ordinal, then \aleph_\lambda would be a successor cardinal and hence not weakly inaccessible. If \lambda were a limit ordinal less than  \kappa , then its cofinality (and thus the cofinality of \aleph_\lambda) would be less than \kappa  and so \kappa  would not be regular and thus not weakly inaccessible. Thus \lambda \geq \kappa  and consequently \lambda = \kappa  which makes it a fixed point.
\end{proof}

\subsection{A Model-Theoretic Intermezzo}
This is a small notational intermezzo. Reflection theorems asdasd Tarski Berkley model-theoretic methods in set theory.

This notation is used for example in \cite{KanamoriBook}.

TODO def $\langle V_\kappa, \in, R  \rangle \models$ asdf

TODO $S \then ZM \then ZM' \then ZM''$, neco jako mahlovy kardinaly, presunout do dane kapitoly


% -----------------------------------------------
\subsection{Reflecting Second-order Formulas}

% TODO proc? sem patri ta "kontradikce"!

To see that there is a~way to transcend $\sf{ZFC}$, let us briefly show how a~model of $\sf{ZFC}$ can be obtained in $\sf{ZFC}_2 + \mbox{"second-order reflection"}$\footnote{$\sf{ZFC}_2$ is an axiomatization of $\sf{ZFC}$ in second-order formulas, to be more rigorously established later.}. This will be more closely examined in section \ref{section:inaccessibility}.

We know that $\sf{ZFC}$ can not be finitely axiomatized in first-order formulas, however if Replacement and Comprehension schemes can be substituted by second-order formulas, $\sf{ZFC}$ becomes $\sf{ZFC}_2$, which is finitely axiomatizable in second-order logic. Therefore if we take second-order reflection into consideration, we can obtain a~set $M$ that is a~model of $\sf{ZFC}_2$. For now, we have left out the details of how exactly is first-order reflection generalised into stronger statements and how second-order axiomatization of $\sf{ZFC}$ looks like as we will examine those problems closely in the following pages. 

% We also need to rigorously define $\sf{ZFC}_2$, the second-order axiomatization of $\sf{ZFC}$ we have already used in the previous section. Let's take advantage of second-order variables and replace Replacement and Specification schemes with a~single Replacement and a Specification axiom respectively. 

Lower-case letters represent first-order variables and upper-case $P$ represents a~second-order variable. \cite{Shapiro87}
\begin{definition}{(\emph{Replacement}\textsubscript{2})}\label{def:replacement_2}\\
\begin{equation}
\begin{gathered}
\forall P (\forall x, y, z (P(x, y) \& P(x, z) \then y\ =\ z) \then \\
\then (\forall x \exists y \forall z (z \in y \iff \exists w \in x (P(w, z)))))
\end{gathered}
\end{equation}
We will denote this axiom \emph{Replacement}\textsubscript{2}.
\end{definition}

\begin{definition}{(\emph{Specification}\textsubscript{2})}\\
\begin{equation}
\forall P \forall x \exists y \forall z \, ( z \in y \iff ( z \in x \& P(z, x)))
\end{equation}
\end{definition}

\begin{definition}{(\sf{ZFC}\textsubscript{2})}\\
Let $\sf{ZFC\textsubscript{2}}$ be a~theory with all axioms identical with the axioms of $\sf{ZFC}$ with the exception of \emph{Replacement} and \emph{Specification} schemes, which are replaced with \emph{Replacement}\textsubscript{2} and \emph{Specification}\textsubscript{2} respectively.
\end{definition}

%The purpose of this chapter is to try to answer these questions, as well as examine the relation of said reflection axioms to large cardinals.

%We will now define reflection for second-order formulas.

%\begin{definition}{Second-order reflection}\label{def:reflection}\\
%TODO
%\end{definition}

% TODO see Hanf-Scott [kanamori:61]?

% TODO full reflection, partial reflection

%TODO diskuse, co vede ve vysledku k nepopsatelnym kardinalum

%\subsection{Rest of the article}
%TODO prejmenovat sekci nebo presunout do patricnych velkych kardinalu (to spis)

%TODO jak znacim usporadane dvojice?
% TODO porovnani Mahlovy a~Lévyho konstrukce, viz ref\{mahlovy kardinaly\}

% TODO asi doplnit jak to souvisi se soucasnou definici slabe Mahlovych kardinalu pres stacionarni mnoziny?

%TODO Pak i lambda konstrukce, zobecnene

%TODO asi citace? presunout do patricne sekce pro reflexi vyssich radu?

%TODO vice-mene Plagiat -- prepsat a~vysvetlit -- Shapiro reflexi zobecnuje ve vice variantach -- diskutovat?

%TODO zdroj?

%\begin{definition}\label{def:reflection_2}
%Let $\varphi(R)$ be a~$\Pi^n_m$-formula which contains only one free variable $R$ which is second-order. Given $R \sub V_\kappa$, we say that $\varphi(R)$ reflects in $V_\kappa$ if there is some $\alpha<\kappa$ such that:
%\begin{equation}
%\mbox{If }(V_\kappa,\in, R)\models \varphi(R), \mbox{ then }(V_\alpha,\in, R\cap V_\alpha) \models \varphi(R\cap V_\alpha).
%\end{equation}
%\end{definition}


\subsection{Inaccessibility}\label{section:inaccessibility}
%The inaccessible cardinal is the smallest of large cardinals\footnote{citation needed.}
\begin{definition}{(limit cardinal)}\label{def:limit}
$\kappa$ is a~\emph{limit cardinal} iff it is $\aleph_\alpha$ for some limit ordinal $\alpha$.
\end{definition}

\begin{definition}{(strong limit cardinal)}\label{def:strong_limit}
$\kappa$ is a~\emph{strong limit cardinal} iff it is a limit cardinal and for every $\lambda < \kappa$, $2^\lambda < \kappa$
\end{definition}

The two above definition become equivalent when we assume $GCH$.

\begin{definition}{(weak inaccessibility)}\label{def:weakly_inaccessible}
An uncountable cardinal $\kappa$ is \emph{weakly inaccessible} iff it is \emph{regular} and \emph{limit}.
\end{definition}
\begin{definition}(inaccessibility)\label{def:inaccessible}
An uncountable cardinal $\kappa$ is \emph{inaccessible} (written $In(\alpha)$) iff it is \emph{regular} and \emph{strongly limit}.
\end{definition}

\

We will now show that the above notion is equivalent to the definition Lévy uses in \cite{Levy60a}, which is, in more contemporary notation, the following:
\begin{theorem}\label{theorem:inaccessible_models_zfc}
The following are equivalent:
\bce
\item $\kappa$ in inaccessible
\item $\langle V_\kappa, \in \rangle \models \sf{ZFC}$
\ece
\end{theorem}

\begin{proof}
Let's first prove that if $\kappa$ is inaccessible, it is a~model of $\sf{ZFC}$. We will do that by verifying the axioms of $\sf{ZFC}$ just like Kanamori does it in in \cite[1.2]{KanamoriBook} and Drake in \cite[Chapter 4]{DrakeBook}. 
\bce[(i)]
\item \emph{Extensionality}:\\
(see \ref{def:extensionality})
\begin{equation}
V_\kappa \models \forall x, y(\forall z (z \in x \iff z \in y) \then x = y) 
\end{equation}
We need to prove that, given two sets that are equal in $V$, they are equal in $V_\kappa$, in other words, that the \emph{Extensionality} formula is reflected, that is
\begin{equation}
V_\kappa \models \forall x, y \in V_\kappa (\forall z \in V_\kappa (z \in x \iff z \in y) \then x = y) 
\end{equation}
But that comes from transitivity. If $x$ and $y$ are in $V_\kappa$ their members are also in $V_\kappa$.

\

\item \emph{Foundation}:\\
(see \ref{def:foundation})
\begin{equation}
V_\kappa \models \forall x (\exists z (z \in x) \then \exists z (z \in x\ \&\ \forall u \neg (u \in z\ \&\ u \in x)))
\end{equation}
The argument for \emph{Foundation} is almost identical to the one for \emph{Extensionality}. For any set $x \in V_\kappa$, transitivity of $V_\kappa$ makes sure that every element of $x$ is also an element of $V_\kappa$ and the same holds for the elements of elements of $x$ et cetera. So statements about those elements are absolute between any transitive structures. $V$ and $V_\kappa$ are both transitive therefore \emph{Foundation} holds and so does its relativisation to $V_\kappa$, $Foundation^{V_\kappa}$.

\

\item \emph{Powerset}:\\
(see \ref{def:powerset})
\begin{equation}
V_\kappa \models \forall x \exists y \forall z (z \subseteq x \then z \in y).
\end{equation}
If we take $x$, an element of $V_\kappa$, $\power{x}$ has to be an element of $V_\kappa$ to, because it is transitive and a strong limit cardinal.

\

\item \emph{Pairing}:\\
(see \ref{def:pairing})
\begin{equation}
V_\kappa \models \forall x, y \exists z (x \in z \land y \in z).
\end{equation}
\emph{Pairing} holds from similar argument like above: let $x$ and $y$ be elements of $V_\kappa$, so there are ordinals $\alpha, \beta < \kappa$ such that $x \in V_\alpha$, $y \in V_\beta$. Without any loss of generality, suppose $\alpha < \beta$, threfore $V_\alpha \subset V_\beta$ which, from transitivity of the cumulative hierarchy, means that $x \in V_\beta$, then $\{x, y\} \in V_{\beta+1}$ which is still in $V_\kappa$ because it is a strong limit cardinal.

\

\item \emph{Union}:\\
(see \ref{def:union})
\begin{equation}
V_\kappa \models \forall x \,\exists y \, \forall z\, \forall w ((w \in z \land z \in x) \then w \in y).
\end{equation}
We want to see that for every $x \in V_\kappa$, this is equivalent to 
\begin{equation}
V_\kappa \models \forall x \in V_\kappa,\exists y \in V_\kappa\, \forall z \in V_\kappa\, \forall w \in V_\kappa ((w \in z \land z \in x) \then w \in y).
\end{equation}
Since $V_\kappa$ is transitive, if $x \in V_\kappa$, all of its elements as well as their elements are in $V_\kappa$. To see that they also form a set themselves we only need to remember that $V_\kappa$ is limit and therefore if $\alpha$ is the least ordinal such that $x \in V_\alpha$, $\bigcup x \in V_{\alpha+1}$.

\

\item \emph{Replacement, Infinity}:\\
(see \ref{def:replacement}, \ref{def:infinity})\\
We know that those hold from \ref{levy_equivalence_contemporary}.
\ece
\

We will now show that if a~set is a~model of $\sf{ZFC}$, it is in fact an inaccessible cardinal. So let $V_\kappa$ be a~model of $\sf{ZFC}$ which means that it is closed under the powerset operation, in other words:
\begin{equation}
\forall \lambda (\lambda < \kappa \then 2^{\lambda} < \kappa)
\end{equation}
which is exactly the definition of strong limitness. $\kappa$ is regular from the following argument by contradiction:\\
Let us suppose for a~moment that $\kappa$ is singular. Therefore there is an ordinal $\alpha < \kappa$ and a~function $F:\ \alpha \then \kappa$ such that the range of $F$ in unbounded in $\kappa$, in other words, $F[\alpha] \subseteq V_\kappa$ and $sup(F[\alpha])\ =\ \kappa$. In order to achieve the desired contradiction, we need to see that it is the case that $F[\alpha] \in V_\kappa$. Let $\varphi(x, y)$ be the following first-order formula: % TODO vyhodit sup, pouzivat radis $\bigcup$
\begin{equation}
F(x)\ =\ y
\end{equation}
Then there is an instance of  Axiom Schema of Replacement that states the following:
\begin{equation}
\begin{gathered}
(\forall x, y, z(\varphi(x, y) \& \varphi(x, z) \then y\ =\ z)) \then \\
\then (\forall x \exists y \forall z (z \in y \iff \exists w (\varphi(w, z))))
\end{gathered}
\end{equation}
Which in turn means that there is a~set $y = F[\alpha]$ and $y \in V_\kappa$, which is the contradiction with $sup(y) = \kappa$ we are looking for.
\end{proof}
% TODO vyhodit sup, pouzivat radis $\bigcup$

\

The same holds for $\sf{ZFC}_2$, the proof is very similar.
\begin{theorem}\label{theorem:inaccessible_models_zfc_2}
\begin{equation}
V_\kappa \models \sf{ZFC}_2 \iff \mbox{$\kappa$ is inaccessible}
\end{equation}
\end{theorem}
\begin{proof}
$\kappa$ is a~strong limit cardinal because from $\sf{ZFC}_2$ and \emph{Powerset} we know that for every $\lambda\ <\ \kappa$, we know that $2^{\lambda} < \kappa$.

$\kappa$ is also regular, because otherwise there would be an ordinal $\alpha$ and a~function $F:\ \alpha \then \kappa$ with a~range unbounded in $\kappa$. 
$\emph{Replacement}^2$ gives us a~set $y\ =\ F[\alpha]$, so $y\ \in\ V_\kappa$, which contradicts the fact that $sup(y)\ =\ \kappa$. It can not be the case that $\kappa \in V_\kappa$.
% TODO vyhodit sup, pouzivat radis $\bigcup$
\

The other direction is exactly like the first part of above theorem \ref{theorem:inaccessible_models_zfc}.
\end{proof}

\

This is how the existence of an inaccessible cardinal is established in \cite{Levy60a}. 
\begin{definition}{$N$}\\
\begin{equation}
\exists u (In(\alpha)\ \&\ \forall x_1, \ldots, x_n (x_1, \ldots, x_n \in u \then (\varphi \iff \varphi^u)))
\end{equation}
\end{definition}
It is interesting to see that the above schema yields the first inaccessible cardinal if we take for $\varphi$ the conjunction of all axioms of $\sf{ZF_2}$.

\

To see that inaccessible cardinal can be also obtained by a fixed-point axiom (or a scheme if were in first-order logic), see the following theorem by Lévy, we won't repeat the proof here, it is available in \cite[Theorem 3]{Levy60a}, 
\begin{theorem}\label{theorem:levy_m_iff_n}
\begin{equation}
M \iff N
\end{equation}
\end{theorem}

%The above makes clear that while the existence of inaccessibles is unprovable in $\sf{ZFC}$ or $\sf{ZFC}_2$, we now know that at least the smallest inaccessible cardinal exists in $\sf{ZF+(Refl^2)}$ because there is a~set that models $\sf{ZFC}$. But how exactly can we reach there with second-order reflection?

% TODO

We have transcended $\sf{ZFC}$, but that is just a~start. Naturally, we could go on and consider the next inaccessible cardinal, which is inaccessible with respect to the theory $\sf{ZFC} + \exists \kappa (\kappa \models \sf{ZFC})$. But let's try to find a faster way up, informally at first. 

Since we can find an inaccessible set larger than any chosen set $M_0$, it is clear that there are arbitrarily large inaccessible cardinals in $V$, they are "unbounded"\footnote{The notion is formaly defined for sets, but the meaning should be obvious.} in $V$. If $V$ were a cardinal, we could say that there are $V$ inaccesible cardinals less than $V$, but this statement of course makes no sense in set theory as is because $V$ is not a set. But being more careful, we could find a property that can be formalized in second-order logic and reflect it to an initial segment of $V$. That would allow us to construct large cardinals more efficiently than by adding inaccessibles one by one. The property we are looking for ought to look like something like this:
\begin{equation}
\begin{gathered}
\kappa \mbox{ is an inaccessible cardinal and}\\
\mbox{there are }\kappa\mbox{ inaccessible cardinals }\mu\ <\ \kappa
\end{gathered}
\end{equation}
This is in fact a fixed-point type of statement. We shall call those cardinals hyper-inaccessible. Now consider the following definition.

\

\begin{definition}{$0$-inaccessible cardinal}\\
A cardinal $\kappa$ is $0$-inaccessible if it is inaccessible.
\end{definition}
We can define \emph{$\alpha$-weakly-inaccessible} cardinals analogously with the only difference that those are limit, not strongly limit.
\

\begin{definition}{$\alpha$-hyper-inaccessible cardinal}\label{def:alpha_inaccessible}\\
For any ordinal $\alpha$, $\kappa$ is called $\alpha$-inaccessible, if $\kappa$ is inaccessible and for each $\beta$ < $\alpha$, the set of $\beta$-inaccessible cardinals less than $\kappa$ is unbounded in $\kappa$.
\end{definition}

\

Because $\kappa$ is inaccessible and therefore regular, the number of $\beta$-inaccessibles below $\kappa$ is equal to $\kappa$. We have therefore successfully formalized the above vague notion of hyper-inaccessible cardinal into a hierarchy of $\alpha$-inaccessibles.

\

Let's now consider iterating this process over again. Since, informally, $V$ would be $\alpha$-inaccessible for any $\alpha$, this property of the universal class could possibly be reflected to an initial segment, the smallest of those will be the first hyper-inaccessible cardinal. Such $\kappa$ is larger than any $\alpha$-inaccessible since from regularity of $\kappa$, for given $\alpha\ <\ \kappa$, $\kappa$ is $\kappa$-th $\alpha$-hyper-inaccessible cardinal. It is in fact "inaccessible" via $\alpha$-inaccessibility.

\

\begin{definition}{Hyper-inaccessible cardinal}\\
$\kappa$ is called the hyper-inaccessible, also $0$-hyper-inaccessible, cardinal if it is $\alpha$-inaccessible for every $\alpha\ <\ \kappa$.
\end{definition}

\

\begin{definition}{$\alpha$-hyper-inaccessible cardinal}\\
For any ordinal $\alpha$, $\kappa$ is called $\alpha$-hyper-inaccessible cardinal if for each ordinal $\beta\ <\ \alpha$, the set of $\beta$-hyper-inaccessible cardinals less the $\kappa$ is inbounded in $\kappa$.
\end{definition}

\

Obviously we could go on and iterate it ad libitum, but the nomenclature would be increasingly confusing. A smarter way to accomplish the same goal is carried out in the following section.

% dukaz viz Kan book 6.1, 6.2 ?

{\color{red}
\begin{comment}

% !!!!!!!!!!!!!!!!!!!!!!!!!!!!!!!!!!!!!!!!!!!!!!!!!

TODO tohle znamena, ze prvoradovou formuli nerozlisime $V$ od (prvniho) nedosazitelneho $\kappa$

\begin{theorem}\label{th:refl_inaccessible}[Lévy] The following are equivalent:
\bce[(i)]
\item $\kappa$ is inaccessible.
\item For every $R \sub V_\kappa$ and every first-order formula $\varphi(R)$, $\varphi(R)$ reflects in $V_\kappa$.
\item For every $R \sub V_\kappa$, the set $C = \set{\alpha<\kappa}{\langle V_\alpha,\in,R \cap V_\alpha\rangle \el \langle V_\kappa,\in,R \rangle}$ is closed unbounded.
\ece
\end{theorem}
\begin{proof}
Let's start with (i) $\then$ (iii) in a~way similar to \cite{KanamoriBook}.\newline
The set $\set{\alpha<\kappa}{\langle V_\alpha,\in,R \cap V_\alpha \rangle \el \langle V_\kappa,\in,R\rangle}$ is clearly closed, it remains to show that it is also unbounded.
To do so, let $\alpha<\kappa$ be arbitrary. Define $\alpha_n < \kappa$ for $n\in\omega$ by recursion as follows:\newline
Set $\alpha_0=\alpha$. Given $\alpha_n < \kappa$ define $\alpha_{n+1}$ to be the least $\beta \geq \alpha_n$ such as 
whenever $y_1,\ldots,y_k \in V_{\alpha_n}$ and
$\langle V_{\kappa}, \in, R \rangle \models \exists v_0 \varphi [v_0, y_1, \ldots, y_k ]$
for some formula $\varphi$, there is an $x \in V_{\beta}$ such that $\langle V_{\kappa}, \in, R\rangle \models \varphi [x, y_1, \ldots, y_k]$.
\newline
Since $\kappa$ is inaccessible, $|V_{\alpha_n}| < \kappa$ and so $\alpha_{n+1} < \kappa$.\newline
Finally, set $\alpha = sup({\alpha_n | n \in \omega})$.  % TODO vyhodit sup, pouzivat radis $\bigcup$
Then $\langle$
 $V_ \alpha, \in, R  \cap V_\alpha \rangle \prec \langle V_{\kappa}, \in, R\rangle$ by the usual (Tarski) criterion for elementary substructure.
 \newline\newline
 The next part, proving $(iii) \then (ii)$, should be elementary since $C$ is closed unbounded, which means that it contains at least countably many elements but we need only one such $\alpha$ to satisfy (\ref{def:reflection_2}).
 \newline
 Finally, we shall prove that $(ii) \then (i)$. Since it obviously holds that $\kappa > \omega$, we have yet to prove that $\kappa$ is regular and a~strong limit. Let's argue by contradiction that it is regular. 
 If it wasn't, there would be a~$\beta < \kappa$ and a~function $F: \beta \implies \kappa$ with range unbounded in $\kappa$. Set $R = \{\beta\} \cup F$. By hypothesis there is an $\alpha < \kappa$ such that $\langle V_\alpha, \in, R \cap V_\alpha \rangle \prec \langle V_\kappa, \in, R \rangle$. Since $\beta$ is the single ordinal in R, $\beta \in V_\alpha$ by elementarity. This yields the desired contradiction since the domain if $F \cap V_\alpha$ cannot be all of $\beta$.
 \newline\newline
 Next, let's see whether $\kappa$ is indeed a~strong limit, again by contradiction. If not, there would be a~$\lambda < \kappa$ such that $2^\lambda \geq \kappa$. Let $G: \power{\lambda} \implies \kappa$ be surjective and set $R = \{\lambda + 1\} \cup G$. By hypothesis, there is an $\alpha < \kappa$ such that $\langle V_\alpha, \in, R \cap V_\alpha \rangle \prec \langle V_\kappa, \in, R \rangle$. $\lambda + 1 \in V_\alpha$ and so $\power{\lambda} \in V_\alpha$, but this is again a~contradiction.
\end{proof}
\
\end{comment}
}
% todo vymenit < za prohnute
% todo vymenit P za potencni P

% todo:
%
% supercompact
%
% =====================================================================================================================================

\subsection{Mahlo Cardinals}

% TODO ze je to trochu diagonalni argument

While the previous chapter introduced us to a notion of inaccessibility and the possibility of iterating it ad libitum in new theories, there is an even faster way to travel upwards in the cumulative hierarchy, that was proposed by Paul Mahlo in his papers (see \cite{Mahlo11}, \cite{Mahlo12} and \cite{Mahlo13}) at the very beginning of the 20th century, and which can be easily reformulated using $\emph(Reflection)$. To see how Lévy's initial statement of reflection was influenced by Mahlo's work, refer to section \ref{sec:Levy1960}. The aim of the following paragraphs is to give an intuitive explanation of the idea behind Mahlo's hierarchy of cardinals, all claims made here ought to be stated formally later in the very same chapter.

At the very end of section \ref{section:inaccessibility}, we have tried to establish the notion of hyper-inaccessibility and iterate it to yield even larger large cardinals. In order to avoid too bulky cardinal names, let's try a different route and establish those cardinals directly via reflection.

\

% TODO WATT
The following two definitions come from \cite{Infinity_in_mind} and while they are rather informal, we will find them very helpful for understanding the Mahlo cardinals.
\begin{definition}{(Fixed-point property)}\\
For any first-order formula $\psi(x, u_1, \ldots, u_n)$ with no free variables other than $x, u_1, \ldots, u_n$, which is any property of ordinals, we say that a property $\varphi$ is a $\emph{fixed-point property}$ if $\varphi$ has the form
\begin{equation}
\begin{gathered}
\mbox{$x$ is an inaccessible cardinal and }\\
\mbox{there are $x$ ordinals less than $x$ that have the property $\psi(x, u_1, \ldots, u_n)$.}
\end{gathered}
\end{equation}
\end{definition}

\

% TODO WATT?
\begin{definition}{(Fixed-point reflection)}\\
If $\varphi$ is a fixed-point property that holds for $V$, it also holds for some $V_\alpha$, an initial segment of $V$.
\end{definition}

Obviously those are in no way rigorous definitions because we have no idea what $\psi(x, u_1, \ldots, u_n)$ looks like. Let's try to restate the same idea in a useful way. But first, let's show that the formal counterpart of the idea of containing "enough" ordinals with a property is the notion of stationary set. 

\begin{definition}{(Supremum)}\\
Given $x$ a set of ordinals, the supremum of $x$, denoted sup($x$), is the least upper bound of $x$. % TODO vyhodit sup, pouzivat radis $\bigcup$
\begin{equation}
sup(x)\ =\ \bigcup x
\end{equation}
\end{definition}

\begin{definition}{(Limit point)}\\
Given $x$, a set of ordinals and an ordinal $\alpha$, we say that $\alpha$ is a \emph{limit point} of $x$ if $sup(x \cap \alpha)\ =\ \alpha$
\end{definition}
% funfuje to, protoze alpha je ordinal, takze kdyz je alpha nejvyssi a je to zaroven mnozina vsech mensich ordinalu, je to presne supremum. kdyz tam neni, ale A obsahuje vsechny mensi ordinaly, je to nejmensi horni zavora, tedy alpha

\begin{definition}{(Set Unbounded in $\alpha$)}
Let $\alpha$ be an ordinal. We say that $x \subset \alpha$ is \emph{unbounded in $\alpha$} iff
\begin{equation}
\forall \beta \in Ord (\beta < \alpha \then \exists \gamma (\gamma \in x (\beta \leq \gamma < \alpha)))
\end{equation} % TODO WTF?
\end{definition}

\begin{comment}

\begin{definition}{(Closed set)}
We say that set $x$ is closed if it contains all of its limit points.\\
\\
Alternatively, subset $x$ of $\alpha$ is closed in $\alpha$ iff for every limit ordinal $\beta < \alpha$, if $C \cap \beta$
is unbounded in $\beta$ then $\beta \in C$.
% TODO WATT
\end{definition}

\begin{definition}{(Club set)}\\
For a regular uncountable cardinal $\kappa$, a set $x\ \subset\ \kappa$ is a \emph{closed unbounded} subset, abbreviated as a \emph{club set}, iff $x$ is both closed and unbounded in $\kappa$.
% which means that for every $\beta$ < $\kappa$ there is a $\beta'\ \in \alpha$ such that $\beta\ <\ \beta'\ <\ \kappa$.
\end{definition}
% TODO mozna kratka uvaha proc to musi byt regularni?

\begin{definition}{(Stationary set)}\\
For a regular uncountable cardinal $\kappa$, a set $A\ \subset\ \kappa$ is stationary iff it intersects every club subset of $\kappa$.
\end{definition}

\begin{theorem}{}\label{club_intersection} 
The intersection of fewer than $\kappa$\footnote{$\kappa$ is again a regular uncountable cardinal and it will always be when we will be talking about club sets.} club subsets of $\kappa$ is a club set.
\end{theorem}
For proof, see \cite[Theorem 8.3]{JechBook}

\begin{definition}{Weakly Mahlo Cardinal}\label{def:weakly_mahlo}\\
$\kappa$ is \emph{weakly Mahlo} $\iff$ it is a~regular limit ordinal and the set of all regular ordinals less then $\kappa$ is stationary in $\kappa$
\end{definition}

\begin{definition}{Mahlo Cardinal}\label{def:mahlo_cardinal}\\
$\kappa$ is a \emph{Mahlo Cardinal} iff it is an inaccessible cardinal and the set of all inaccessible ordinals less then $\kappa$ is stationary in $\kappa$.
\end{definition}
It is interesting to note, that weakly-Mahlo cardinals are fixed points of $\alpha$-weakly inaccessible cardinals, so if $\kappa$ is weakly mahlo,  .. viz Kanamori Proposition 1.1

Analogously, 
\begin{definition}{$\alpha$-Mahlo Cardinal}\label{def:alpha_mahlo_cardinal}\\
$\kappa$ is a \emph{$\alpha$-Mahlo Cardinal} iff it is an $\alpha$-inaccessible cardinal and the set of all $\alpha$-inaccessible ordinals less then $\kappa$ is stationary in $\kappa$.
\end{definition}

\end{comment}

\

In other words, $\kappa$ is a mahlo cardinal if it is inaccessible and every club set in $\kappa$ contains an inaccessible cardinal. This is exactly the notion of fixed-point reflection we were trying to show earlier.
% viz http://euclid.colorado.edu/~monkd/m6730/gradsets12.pdf
%Thus a~Mahlo cardinal $\kappa$ is not only inaccessible, but also has $\kappa$ inaccessibles below it.

\

\cite{DrakeBook}
\begin{definition}{}\label{def:mahlo_equivalent}
The folllowing definitions are equivalent:
\bce[(i)]
\item $\kappa$ is Mahlo
\item $\kappa$ is weakly Mahlo and strong limit
% \item $\kappa$ is regular and the stationary sets below $\kappa$ form a~stationary subset of $\kappa$. wut?
\item The set $\{\lambda < \kappa : \lambda\mbox{ is inaccessible}\}$ is stationary in $\kappa$.
\item Every normal function on $\kappa$ has an inaccessible fixed point.
\ece
\end{definition}

\begin{proof}
$\bold{(i) \iff \bold{(ii)}}$
Let $\kappa_1$ be a mahlo cardinal and let $\kappa_2$ be a strong limit weakly Mahlo cardinal. We know from the definitions that the set $\{\lambda\ <\ \kappa:\ \lambda\mbox{ is inaccessible}\}$ is stationary in both $\kappa_1$ and $\kappa_2$, the only difference being that $\kappa_1$ is a strongly limit cardinal, but $\kappa_2$ would be limit from weak Mahloness, wasn't it for the fact that it is also strong limit. This eliminates the only difference between them and therefore $\kappa_1$ is also strong limit weakly Mahlo cardinal and $\kappa_2$ is Mahlo.

\

$\bold{(i) \then (iii)}$
We know that $\kappa$ is uncountable, regular, strong limit and that the set $S\ =\ \{\lambda\ <\ \kappa: \lambda\mbox{ is regular}\}$ is stationary in $\kappa$. 
We want to prove that $S'\ =\ \{\lambda\ <\ \kappa: \lambda\mbox{ is inaccessible}\}$ is thus also stationary in $\kappa$.

Since stationary set intersects every club set in $\kappa$, let $C$ be any such set. Let $D\ =\ \{ \lambda\ <\ \kappa:\ \lambda\mbox{ is strong limit}\}$. 
$D$ is a club set because TODO.
Since intersection of less than $\kappa$ club sets is a club set, $C \cap D \neq \emptyset$. 

TODO proc $\lambda\ =\ S \cap C \cap D$ je inaccessible?

%$\bold{(iii) \then (i)}$
%This is obvious since every inaccessible cardinal is regular, therefore every set of inaccessibles, that is presumed to be stationary in $\kappa$, is at the same time a set of regular cardinals.

$\bold{(iii) \then (iv)}$

TODO jak to dela Lévy? 

$\bold{(iv) \then (i)}$

TODO jak to dela Lévy? 

range kazde normalni funkce je club v On. (nevadi ze On je trida?)

co treba lemma ze pevne body tvori taky club set

mozna by stacilo ze jsou unbounded, tedy kazda normalni funkce ma libovolne velke pevne body.
\end{proof}

\

TODO obdoba pro $\alpha$-Mahlo kardinaly?

{\color{red}
\begin{comment}
% comment

% kanamori p. 80 6.2 (b)
\begin{theorem}\label{th:refl_mahlo}
$\kappa$ is Mahlo $\iff$ for any $R \subset V_\kappa$ there is an inaccessible cardinal $\alpha < \kappa$ such that $\langle V_\alpha, \in, R \cap V_\alpha \rangle \prec \langle V_\kappa, \in, R \rangle$.
\end{theorem}

\begin{proof}
Start with the proof of (\ref{th:refl_inaccessible}) and add the following:\\
$\kappa$ is Mahlo by the following contradiction. If not, there would be a~$C$ closed unbounded in $\kappa$ containing no inaccessible cardinals. By the hypothesis there is in inaccessible $\alpha < \kappa$ such that $\langle V_\alpha, \in, C \cap V_\alpha \rangle \prec \langle V_\kappa, \in, C \rangle$. By elementarity $C \cap \alpha$ is unbounded in $\alpha$. But then, $\alpha \in C$, which is the contradiction we need.
\end{proof}

\

\end{comment}
}

TODO $\kappa$ is hyper-Mahlo iff $\kappa$ is inaccessible and the set $\{\lambda < \kappa : \lambda\mbox{ is Mahlo}\}$ is stationary in $\kappa$. to je to samy jako $\alpha$-Mahlo, ne?

% TODO viz https://en.wikipedia.org/wiki/Mahlo_cardinal#Mahlo_cardinals_and_reflection_principles

% Note that Mahlo cardinals were first described in 1911, almost 50 years before Lévy's reflection, which was heavily inspired by them.

% " We also state the appropriate generalization for greatly Mahlo cardinals." % viz http://arxiv.org/abs/math/9204218

%TODO veta na zaver, shrnuti

%sjednotil \then a~\implies
% =====================================================================================================================================
% \newpage
\subsection{Indescribality} % Indescribality and Weakly Compact Cardinals

$\alpha$-Mahlo are the extreme of regular fixed-point axioms, they are about as high as we can get via normal functions and stationary sets.

Let's try a different strategy. Remember how we said that (Regular, Limit and) various Large cardinals are in a way all determined by being unreachable by a specific process of creating bigger cardinals from already available ones? 
%TODO neco jako reflecting reflection (uz u Mahlovejch teda)
TODO indescribable -- reflecting indescribability -- we can't reach $V$ by a $\Sigma_1^1$ formula, so there's some initial segment $V_\alpha$ that is also unreachable (we say indescribable) by the means of a ... formula


Let's recall complete reflection theorem first, consider the following:
\begin{equation}
\mbox{For every sentence $\varphi$, there is a limit ordinal $\alpha$ such that }\varphi^V_\alpha \iff \varphi
\end{equation}
We may also require that $\alpha < \beta$, where $\beta$ is an arbitrary ordinal given.

\

%So far, we have considered individual formulas and examined what $V_\alpha$ is the first that 

% TODO kanamori to dela s parametrem, asi to ma smysl
% TODO ja zas vyse psal varphi vzdycky s x1...xn, ale TADY JE TO SENTENCE! !!!
For the exact definition of $\Pi^m_n$ and $\Sigma^m_n$ see \ref{def:analytical_hierarchy}

\begin{definition}{($\Pi^m_n$-indescribable cardinal)}\label{def:pi_mn_indescribable}
We say that $\kappa$ is $\Pi^m_n$-indescribable iff for any $\Pi^m_n$ sentence $\varphi$ such that $V_\kappa \models \varphi$ there is an $\alpha < \kappa$ such that $V_\alpha \models \varphi$
\end{definition}
\begin{definition}{($\Sigma^m_n$-indescribable cardinal)}\label{def:sigma_mn_indescribable}
We say that $\kappa$ is $\Sigma^m_n$-indescribable iff for any $\Sigma^m_n$ sentence $\varphi$ such that $V_\kappa \models \varphi$ there is an $\alpha < \kappa$ such that $V_\alpha \models \varphi$
\end{definition}

\begin{lemma}
Let $\kappa$ be a cardinal, the following holds for any $n \in \omega$. $\kappa$ is $\Pi^1_n$-indescribable iff $\kappa$ is $\Sigma^1_n+1$-indescribable
\end{lemma}

\begin{proof}
The forward direction is obvious, we can always add a spare quantifier over a type 2 variable to turn a $\Pi^1_n$ formula $\varphi$ into a $\exists P \varphi$ which is thus a $\Sigma^1_n+1$ formula.\footnote{Note that unlike in previous sections, $\varphi$ is now a sentence so we don't have to worry whether $P$ is free in $\varphi$.}

To prove the opposite direction, suppose that $V_\kappa \models \exists X \varphi(X)$ where $X$ is a type 2 variable and $\varphi$ is a $\Pi^1_n$ formula with one free variable of type 2. This means that there is a set $S \subseteq V_\kappa$ that is a witness of $\exists X \varphi(X)$, in other words, $\varphi(S)$ holds. We can replace every occurence of $X$ in $\varphi$ by a new predicate symbol $S$, this allows us to say that $\kappa$ is $\Pi^1_n$-indescribable (with respect to $\langle V_\kappa, \in, R, S \rangle$).
\footnote{A different yet interesting approach is taken by Tate in \ref{Tait_constructingcardinals}. He states that for $n\geq 0$, a formula of order $\leq n$ is called a $\Pi^n_0$ and a $\Sigma^n_0$ formula. Then a $\Pi^n_{m+1}$ is a formula of form $\forall Y \psi(Y)$ where $\psi$ is a $\Sigma^n_m$ formula and $Y$ is a variable of type $n$. Finally, a $\Sigma^n_{m+1}$ is the negation of a $\Pi^n_m$ formula. So the above holds ad definitio.}
\end{proof}

The above lemma tells us that we as long as we stay in the realm of type 1 and type 2 variables, we only need to classify indescribable cardinals with respect to $\Pi^1_n$-indescribability.

\begin{theorem}
Let $\kappa$ be an ordinal. The following are equivalent.
\bce[(i)]
\item $\kappa$ is inaccessible
\item $\kappa$ is $\Pi^1_0$-indescribable.
\ece
\end{theorem}

Note that $\Pi^1_0$ formulas are those that contain zero unbound quantifiers over type-2 variables, they are in fact first-order formulas. We have already shown in \ref{theorem:inaccessible_models_zfc} that there is no way to reach an inaccesible cardinal via first-order formulas in $\sf{ZFC}$. We will now prove it again in for formal clarity.

\begin{proof}
TODO asi pridat alternativni definici nedosazitelnosti podle kan. 6.2?
\end{proof}

TODO nejaka veta ze kdyz jsou $\Pi^1_0$-indescribable, jsou i $\Pi^m_n$-indescribable pro $m \leq 1, n \leq 0$? Drake? Obracene! $\Pi^m_n$-indescribable jsou zaroven $\Pi^a_b$-indescribable pro $a < m, b < n$.

The above theorem provides an easy way to show that every following large cardinal is also in inaccessible cardinal\footnote{That is because $\Pi^1_0$ formulas are included $\Pi^m_n$ formulas for $m \leq 1, n \leq 0$.}.


\begin{definition}{(Extension property)}
We say that a cardinal $\kappa$ has the \emph{extension property} iff for any $R \subseteq V_\kappa$ there is a~transitive set 
$X \neq V_\kappa$ and an $S \subseteq X$ such that 
$\langle V_\kappa, \in, R \rangle \prec \langle X, \in, S \rangle$
\end{definition}

%TODO co to znamena?

\begin{definition}{(Weakly compact cardinal)}\label{def:weakly_compact_extension}\\
We say that a cardinal $\kappa$ is \emph{weakly compact} iff it has the extension property.
\end{definition}

%\begin{definition}{(Weakly compact cardinal)}\label{def:weakly_compact_indescribable}
%We say that a cardinal $\kappa$ is \emph{weakly compact} iff it is $\Pi^1_1$-indescribable.
%\end{definition}

The above definitions are equivalent

\begin{theorem}
the following are equivalent:\\
\bce[(i)]
\item $\kappa$ is \emph{Weakly compact}.
\item $\kappa$ is $\Pi^1_1$-indescribable.
\ece
\end{theorem}

For a proof, see \cite{KanamoriBook}[Theorem 6.4]


TODO asi nekde bude meritelny kardinal

TODO viz Drake, Ch.9 par. 3 -- tam se rika ze kdyz $\kappa$ je meritelny kardinal, pak je $\kappa$ $\Pi^2_1$-nepopsatelny kardinal

{\color{red}
\begin{comment}
% balcar - stepanek strana 314, veta 5.10.
\begin{theorem}\label{th:refl_weakly_compact}
Let $\kappa$ be a~weakly compact cardinal. Then for every stationary set $S \subset \kappa$ there is an uncountable regular cardinal $\lambda < \kappa$ such that the set $S \cap \lambda$ is stationary in $\lambda$.
\end{theorem}
\begin{proof}
TODO % WATT??
\end{proof}

\

\end{comment}
}

% \newpage
% =====================================================================================================================================
\subsection{Bernays–G{\"o}del Set Theory}

\


TODO Jech str. 70 \cite{JechBook}


% G{\"o}del–Bernays set theory, also known as Von Neumann–Bernays–G{\"o}del set theory is an axiomatic set theory that  explicitly talks about proper classes as well as sets, which allows it to be finitely axiomatizable, albeit our version stated below contains one schema. It is a~conservative extension of Zermalo–Fraenkel set theory. Using forcing, one can prove equiconsistency of BGC and ZFC. 

\
 
 TODO popis
% Bernays–G{\"o}del set theory contains two types of objects: proper classes and sets. The notion of set, usually denoted by a~lower case letter, is identical to set in ZF, whereas proper classes are usually denoted by upper case letters. The difference between the two is in a~fact, that 
% proper classes are not members of other classes, sets, on the other hand, have to be members of classes.
\begin{definition}(G{\"o}del–Bernay set theory)
\bce[(i)]
\item \emph{extensionality for sets}
\begin{equation}
\forall a \forall b [\forall x(x \in a~\iff x \in b) \then a = b]
\end{equation}
\item \emph{pairing for sets}
\begin{equation}
\forall x \forall y \exists z \forall w [w \in z \iff (w = x \lor w = y)]
\end{equation}
\item \emph{union for sets}
\begin{equation}
\forall a \exists b \forall c [c \in b \iff \exists d ( c \in d \land d \in a)]
\end{equation}
\item \emph{powers for sets}
\begin{equation}
\forall a \exists p \forall b [b \in p \iff (c \in b \then c \in a)]
\end{equation}
\item \emph{infinity for sets}
\begin{equation}
\mbox{There is an inductive set.}
\end{equation}
\item \emph{Extensionality for classes}
\begin{equation}
\forall x (x \in A \iff x \in B) \then A = B
\end{equation}

\item \emph{Foundation for classes}
\begin{equation}
\mbox{Each non-empty class is disjoint from each of its elements.}
\end{equation}

\item \emph{Limitation of size for sets}
\begin{equation}
\mbox{For any class C a~set x such that x=C exists iff}\newline
\end{equation}
\begin{equation}
\mbox{there is no bijection between C and the class V of all sets}
\end{equation}
\item \emph{Comprehension schema for classes}
\begin{equation}
\mbox{For an arbitrary formula }\varphi\mbox{ with no quantifiers over classes, there is a~class $A$~such that }\forall x (x \in A \iff \varphi(x))
\end{equation}
\ece
\end{definition}
%\newpage
The first five axioms are identical to axioms in ZF. \newline
Comprehension schema tells us that proper classes are basically first-order predicates.
{\color{red}
\begin{comment}

% opsano !!!!!
%Since {\sf GB}, G{\"o}del-Bernays' first-order theory with two sorts of variables (sets and classes), is finitely axiomatizable, there is no analogue of L{\'e}vy's theorem \ref{th:refl} provable in {\sf GB}. However if we go above the consistency strength of {\sf GB}, we can derive the existence of an inaccessible from such reflection (with a~second-order parameter).
...
 TODO Plagiat -- prepsat a vysvetlit
 
\begin{definition}\label{def:reflBG}
We say that $\varphi(R)$ with a~class parameter $R$ reflects if there is $\alpha$ such that
\begin{equation}
\varphi(R) \then (V_\alpha,V_{\alpha+1})\models \varphi(R\cap V_\alpha).
\end{equation} 
\end{definition}

%Note that since $\varphi$ may contain class variables, we need to specify the intended range of class variables in $V_\alpha$. As in the previous section, where the parameter $R$ ranged over entire $V_{\alpha+1}$, we postulate that the intended range of the class variables in Definition \ref{def:reflGB} is equal to $V_{\alpha+1}$.

\begin{theorem}\label{th:refl01}
There is a~second-order sentence $\varphi$ which is provable in {\sf GB} such that if $\varphi$ reflects at $\alpha$, i.e. if
\begin{equation}
\varphi \then (V_\alpha,V_{\alpha+1}) \models \varphi,
\end{equation}
then $\alpha$ is an inaccessible cardinal.
\end{theorem}

\begin{proof}
Take $\varphi$ to say ``there is no function from $\gamma \in \mx{ORD}$ cofinal in $\mx{ORD}$ and for every $\gamma \in \mx{ORD}$, $2^\gamma \in \mx{ORD}$''. Clearly, if $\varphi$ reflects at some $\alpha$, then $\alpha$ is inaccessible (here we use that the second-order variable range over $\power{V_\alpha} = V_{\alpha+1}$).
\end{proof}

\

As a~corollary we obtain:

\begin{Cor}\label{cor:refl01}
Second-order reflection in {\sf GB} implies the existence of an inaccessible cardinal.
\end{Cor}
% / opsano, upravit !!!

% \newpage
\subsection{Morse–Kelley Set Theory}
Axioms not 
\bce[(i)]
\item \emph{Extensionality}
\begin{equation}
\forall X \forall Y (\forall z ( z \in X \iff z \in Y) \then X = Y).
\end{equation}
\item \emph{Pairing}
\begin{equation}
asdfg
\end{equation}
\item \emph{Foundation For Classes}
\begin{equation}
asdf
%\forall A ( A \neq \emptyset \then \exists b ( b \in A \& \forall c ( c \in b \then c \nin A ))).
\end{equation}
\item \emph{Class Comprehension}
\begin{equation}
\forall W_1, \ldots, W_n \exists Y \forall x (x \in Y \iff (\phi (x, W_1, \ldots, W_n) \& set(x))).
\end{equation}
Where $set(x)$ is monadic predicate stating that class $x$ is a~set.
\item \emph{Limitation Of Size For Classes}
\begin{equation}
asdf
\end{equation}
\item \emph{Pairing}
\begin{equation}
asdf
\end{equation}
\item \emph{Pairing}
\begin{equation}
asdf
\end{equation}
\ece

\end{comment}
TODO
}

%\newpage
% =====================================================================================================================================

\subsection{The Constructible Universe} % Reflection and the constructible universe

% TODO reflektovat muzeme jenom kardinaly konzistentni s V=L, proc????

% TODO Plagiat -- prepsat a~vysvetlit

The constructible universe, denoted $L$, is a cumulative hierarchy of sets, presented by Kurt Gödel in his 1938 paper \emph{The Consistency of the Axiom of Choice and of the Generalised Continuum Hypothesis}. For a technical description, see below. Assertion of their equality, $V=L$, is called the \emph{axiom of constructibility}. The axiom implies GCH and therefore also AC and contradicts the existence of some of the large cardinals, our goal is to decide whether those introduced earlier are among them.

On order to formally establish this class, we need to formalize the notion of definability first. 
\begin{definition}
We say that a set $X$ is \emph{definable} over a model $\langle M, \in \rangle$ if there is a first-order formula $\varphi$ together with parameters $u_1, \ldots, u_n \in M$ such that
\begin{equation}
X = \{x: x \in M\ \&\ \langle M, \in \rangle \models \varphi(x, u_1, \ldots, u_n)\}
\end{equation}
\end{definition}

\begin{definition}{(Sets definable in $M$)}\\
The following is a set of all definable subsets of a given set $M$, denoted Def($M$).
\begin{equation}
\begin{gathered}
Def($M$) = \{\{y : x \in M \land \langle M, \in \rangle \models \varphi(y, u_1, \ldots, i_n) \} |\\
\mbox{ $\varphi$ is a~first-order formula, }u_1, \ldots, u_n \in M \}
\end{gathered}
\end{equation}
\end{definition}

% Note that fle?

Now we can recursively build $L$.
\begin{definition}{(The Constructible universe)}\label{def:constructible_universe}
\bce[(i)]
\item
\begin{equation}
L_0 := \emptyset
\end{equation}

\item
\begin{equation}
L_{\alpha+1} := Def(L_{\alpha})
\end{equation}
\item
\begin{equation}
L_{\lambda} = \bigcup_{\alpha < \lambda} L_{\alpha}\mbox{ If }\lambda\mbox{ is a~limit ordinal }
\end{equation}
\item
\begin{equation}
L = \bigcup_{\alpha\in Ord} L_{\alpha}
\end{equation}
\ece
\end{definition}

Note that while $L$ bears very close resemblance to $V$, the difference is, that in every successor step of constructing $V$, we take every subset of $V_\alpha$ to be $V_{\alpha+1}$, whereas $L_{\alpha+1}$ consists only of definable subsets of $L_\alpha$. Also note that $L$ is transitive.

In order to 

TODO:
\begin{lemma}\label{lemma:ord_in_l}
$Ord \in L$
\end{lemma}

\begin{lemma}\label{lemma:l_wo}
$L$ is well-ordered.\\
TODO !!
\end{lemma}

\begin{theorem}
Let $L$ be as in \ref{def:constructible_universe}.
\begin{equation}
L \models \sf{ZFC}
\end{equation}
\end{theorem}

\begin{proof}
TODO !!! (strucne) vit \cite{JechBook}[Theorem 13.3]
\bce[(i)]
\item \emph{Extensionality} (see \ref{def:extensionality}):\\
\emph{Extensionality} holds in $L$ because $\Delta_0$ formulas are absolute in transitive classes by \ref{lemma:delta_0_absolute}, \emph{Extensionality} is $\Delta_0$ and $L$ is transitive.

\item \emph{Foundation} (see \ref{def:foundation})\\
Take a non-empty set $X$. Let $x \in X$ be a set such that $X \cap x = \emptyset$. $x$ is therefore defined by the formula $\varphi(x, y) = (x \cap y = \emptyset)$, so $x \in L$. $\varphi$ is $\Delta_0$ and therefore holds in $L$ by \ref{lemma:delta_0_absolute}.

\item \emph{Pairing} (see \ref{def:pairing})\\
Since \emph{Pairin} is also $\Delta_0$, it holds in $L$ by the same argument as \emph{Extensionality} does by \ref{lemma:delta_0_absolute}.

\item \emph{Union} (see \ref{def:union})\\
\emph{Union} is also $\Delta_0$, see \emph{Extensionality} and \ref{lemma:delta_0_absolute}.

\item \emph{Power Set} (see \ref{def:powerset})\\
\emph{Power Set} also holds by \ref{lemma:delta_0_absolute}.

\item \emph{Infinity} (see \ref{def:infinity})\\
$\omega \in L$ by \ref{lemma:ord_in_l} % neni to trochu kruh?

\item \emph{Specification} (see \ref{def:specification})\\
.

\item \emph{Replacement} (see \ref{def:replacement})\\
.
\item \emph{Choice} (see \ref{def:replacement})\\
.

\ece
\end{proof}

\begin{definition}{Constructibility}\\
$L = V$
\end{definition}

The following are a few interesting results that we won't prove but refer interested reader to appropriate resources instead.

\begin{definition}{(GCH)}\\
The following is called the Generalised Continuum Hypothesis, abbreviated as \emph{GCH}. It is an independent statement in $\sf{ZFC}$.
\begin{equation}
\mbox{\emph{GCH} iff }\aleph_{\alpha+1}=2^{\aleph_\alpha} \mbox{ for every ordinal }\alpha
\end{equation}
\end{definition}

\begin{theorem}
\begin{equation}
(L = V) \then GCH
\end{equation}
\end{theorem}

This is proven in cite\{neco\} Gödel? Jech? Kunnen?


% -------------------

TODO L a velke kardinaly


TODO def Con!
\begin{theorem}
\begin{equation}
Con(L + \exists \kappa (\kappa \mbox{" is an Inaccessible Cardinal"}))  
\end{equation}
\end{theorem}

\begin{theorem}
\begin{equation}
Con(L + \exists \kappa (\kappa \mbox{" is a Mahlo Cardinal"}))  
\end{equation}
\end{theorem}


\begin{theorem}
\begin{equation}
Con(L + \exists \kappa (\kappa \mbox{" is a Weakly Inaccessible Cardinal Cardinal"}))  
\end{equation}
\end{theorem}

\begin{theorem}
\begin{equation}
Con(L + \exists \kappa (\kappa \mbox{" is a Measurable Cardinal"}))  
\end{equation}
\end{theorem}

TODO co velky pismena ve jmenech kardinalu?

{\color{red}
\begin{comment}

TODO Plagiat -- prepsat a~vysvetlit
\begin{Fact}
The reflection -- constructed as explained in the previous paragraph (!!! preformulovat !!!) -- with second-order parameters for higher-order formulas (even of transfinite type) does not yield transcendence over $L$.
\end{Fact}

\end{comment}
}

TODO zduvodneni

\

TODO kratka diskuse jestli refl implikuje transcendenci na L - polemika, nazor - V=L a~slaba kompaktnost a~dalsi

TODO neco jako ze meritelny kardinal je nepopsatelny vys nez je hierarchie vsech tvrzeni o L?
\

\subsection{Measurable Cardinal}

TODO refaktorizovat fle:
\begin{definition}{(Ultrafilter)}\\
Given a set $X$, we say $U \subset \power{X}$ is an \emph{ultrafilter} iff all of the following hold:
\bce[(i)]
\item $\emptyset \not\in U$
\item $\forall a, b (\subset X\ \&\ a \subset b\ \&\ a \in U \then b \in U)$
\item $\forall a, b \in U (a \cap b) \in U$
\item $\forall a (a \subset X \then (a \in U \lor (X \setminus a) \in U))$
\ece
\end{definition}

\begin{definition}{($\kappa$-complete ultrafilter)}\\
We say that an ultrafilter $U$ is $\kappa$-complete iff
\end{definition}

\begin{definition}{(non-principal ultrafilter)}\\
TODO
\end{definition}

\begin{definition}{(Measurable Cardinal)}\\
Let $\kappa$ be a caridnal. We say is a \emph{measurable cardinal} iff it is an uncountable cardinal with a $\kappa$-complete, non-principal ultrafilter.
\end{definition}

\begin{theorem}
Let $\kappa$ be a cardinal. $\kappa$ is a measurable cardinal $iff$ it is a $\Pi^2_1$-indescribable cardinal.
\end{theorem}

TODO !!!


TODO proc?

TODO je 
%\newpage
% =====================================================================================================================================
%\section{Higher-order reflection} % preformulovat
%TODO rict ze to je zobecneni a~nejaky dalsi uvodni veci

%\subsection{Sharp}
%TODO je tohle higher-order vec?

%\subsection{Welek: Global Reflection Principles}
%TODO ma to vubec cenu?

\newpage
\section{Conclusion}
TODO na konec

\newpage
\bibliographystyle{plain}
\bibliography{bc_biblio}

\end{document}