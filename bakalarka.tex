\documentclass[12pt,a4paper]{article}

% TODOS:
% spravne uvozovky, symetricke!
% definice splnovani do uvodu!!
% \ref nebo (\ref)?
% (i) emph misto boldu
% projet jeste definice
% projet dukaz axiomu ZF v 2
% pozor na ruzne definice delta_0, zkontroluj jestli to pouzivas jako Jech (jestli to rikame o te fli nebo tim myslime ze je ekvivalentni s nejakou takovou)
% jak se znaci dvojice? $\langle x, y \rangle$!!
% kofinalita je spatne!
% TODO (i), (ii) KEIN BOLD
% \emph{Infinity} uppercase first!
% supremum! PODLE JECHA
% napsat nekam ze V je vzdycky sjednoceni kumulativni hierarchie a ne proste univerzalni trida!!
% Zemnily se citace, zkontrolovat ze nepisu "Jech [Jech]"
% TODO Reclection_1 vs. Reflection --- spravne je \emph{First-order reflection}!!!
% v definici funkce posunout footnote o o kus nahoru
% todo def "axiomatic set theory" jako (RS?) mnozinu formulí v jazyce teorie mnozit, tedy \{\in, =\}
% zmenil se replacement, v Levyho stare verzi opraveno, zkontrolovat v restatementu
% porad si nejsem jistej dukazem Levyho repl. zkontrolovat podle Levyho

% MAHLO:
% viz "Interestingly enough, Gaifman [67] showed that in a concrete sense a weakly Mahlo cardinal is the least upper bound of diagonalizing limit processes from below." (kanamori)

\usepackage{mathrsfs}
\usepackage{amssymb}
\usepackage{amsmath}
\usepackage{amsfonts}
\usepackage{longtable}
\usepackage{paralist}
\usepackage{lineno}
\usepackage{verbatim}
\linenumbers

% \usepackage{mathrsfs}
\usepackage{amssymb}
\usepackage{amsmath}
\usepackage{amsfonts}
\usepackage{longtable}
\usepackage{paralist}
\usepackage{lineno}
\usepackage{verbatim}
\linenumbers

% \usepackage{mathrsfs}
\usepackage{amssymb}
\usepackage{amsmath}
\usepackage{amsfonts}
\usepackage{longtable}
\usepackage{paralist}
\usepackage{lineno}
\usepackage{verbatim}
\linenumbers

% \include{00_headers.tex}
\usepackage{color} %pro barevné odkazy, příp. nadpisy
\definecolor{odkazy}{rgb}{0.21,0.27,0.53} %tmavì modrá
\definecolor{nadpisy}{rgb}{0.5812,0.0665,0.0659} %cihlová
%
% Parametry prevodu do pdf
\providecommand{\hypersetup}[1]{}%
\hypersetup{%
unicode,% ? Pravdepodobne bezvyznamne
pdfauthor={Mikuláš Mrva},
pdftitle={Reflection principles and large cardinals},
pdfsubject={Reflection principles and large cardinals},
pdfkeywords={set theory, large cardinals, reflection principle, ZFC, Azriel Lévy},
pdffitwindow=false,% Inicialni umisteni textu v okne Readeru
bookmarksopen=true,% Panel zalozek inicialne zobrazen
% Je-li tohle nastaveno jinak, nektere odkazy nekdy nefunguji
hypertexnames=false,
plainpages=false,
%pdfpagelabels,
%
breaklinks=true,% Radkovy lom smi prijit do klikatelneho odkazu
linkcolor=odkazy,% Graficka podoba odkazu
citecolor=odkazy,% ...
colorlinks=true,% ...
pdfhighlight=/O% ... (vzhled odkazu pri stisknuti)
}%
% Inputenc je asi zbytecne.
% Option 'split' ovlivnuje deleni slov obsahujicich v sobe rozdelovnik
\usepackage[utf8x]{inputenc} % UTF-8 ?
%\usepackage[czech]{babel} %dnes už je však hotová integrace èeštiny do babelu
%\usepackage[split]{czech} %dnes už je však hotová integrace èeštiny do babelu
%
%\usepackage{logdp} %užiteèné drobnosti
%\usepackage{amsthm} %lepšší práce s větami
%\usepackage{amsmath} %nová prostøedí pro matematiku a vylepšení tìch stávajících
%\usepackage{latexsym,amsfonts,amssymb} % nová písmenka
\usepackage{fancyhdr} % zápatí a záhlaví
%\usepackage[nottoc]{tocbibind} % přidá do obsahu položky Literatura a Rejstřík
\usepackage{csquotes}
\pagestyle{plain}
%pøedbìžné nastavení hlavièky (balík fancyhdr)
%\headheight 13.6pt %možná ji bude tøeba zvednout, fancyhdr si pak stìžuje: \headheight
% too small, make it at least Xpt
\headheight 14.5pt %možná ji bude tøeba zvednout, fancyhdr si pak stìžuje: \headheight too \fancyhead{}
\fancyhead[R]{\leftmark}
\fancyfoot{}
\fancyfoot[C]{\thepage}


\newtheorem{theorem}{Theorem}[section]
\newtheorem{Claim}[theorem]{Claim}
\newtheorem{definition}[theorem]{Definition}
\newtheorem{Cor}[theorem]{Corollary}
\newtheorem{Fact}[theorem]{Fact}
\newtheorem{lemma}[theorem]{Lemma}
\newtheorem{sublemma}[theorem]{Sublemma}
\newtheorem{ex}[theorem]{Example}
\newtheorem{remark}[theorem]{Remark}
\newtheorem{obs}[theorem]{Observation}
\newtheorem{que}[theorem]{Question}
\newtheorem{conjecture}[theorem]{Conjecture}

\renewcommand{\theequation}{\thesection.\arabic{equation}}

\newenvironment{proof}
{\noindent \textit{Proof.}}
{\hspace*{\fill} $\Box$}

\newcommand{\toch}{\fbox{\small {\bf ??}}}
\newcommand{\bt}[1]{{\underset{\widetilde{}}{#1}}}
\newcommand{\trcl}[1]{\ensuremath{\mathrm{trcl}(\{#1\})}}
\newcommand{\cf}[1]{\ensuremath{\mathrm{cf}(#1)}}
\newcommand{\cl}[1]{\ensuremath{\mathrm{cl}}(#1)}
\newcommand{\ord}[1]{\ensuremath{\mathrm{ORD}}(#1)}
\newcommand{\dom}[1]{\ensuremath{\mathrm{dom}}(#1)}
\newcommand{\rng}[1]{\ensuremath{\mathrm{rng}}(#1)}
\newcommand{\power}[1]{\ensuremath{\mathscr{P}} (#1)}
\newcommand{\set}[2]{\ensuremath{\{#1 \,|\, #2 \}}}
\newcommand{\seq}[2]{\ensuremath{\langle #1 \,|\, #2 \rangle}}
\newcommand{\singl}[1]{\ensuremath{\{#1\}}}
\newcommand{\pair}[2]{\ensuremath{\{ #1, #2 \}}}
\newcommand{\restr}[2]{\ensuremath{#1 \! \upharpoonright \! #2}}
\renewcommand{\iff}{\leftrightarrow}
\newcommand{\Iff}{\Leftrightarrow}
\newcommand{\el}{\prec}
\newcommand{\iso}{\cong}
\newcommand{\sub}{\subseteq}
\newcommand{\super}{\supseteq}
\newcommand{\la}{\langle}
\newcommand{\ra}{\rangle}
\newcommand{\embed}{\rightarrow}
\newcommand{\mc}{\mathcal}
\newcommand{\supr}[1]{\mathrm{sup}\,#1}
\newcommand{\then}{\rightarrow}
\newcommand{\conc}{^{\smallfrown}}
\newcommand{\bb}{\mathbb}
\newcommand{\supp}[1]{\mathrm{supp}(#1)}
\newcommand{\beq}{\begin{equation}}
\newcommand{\eeq}{\end{equation}}
\newcommand{\brm}{\begin{remark}\begin{rm}}
\newcommand{\erm}{\end{rm}\end{remark}}
\newcommand{\mx}{\mathrm}
\newcommand{\bce}{\begin{compactenum}}
\newcommand{\ece}{\end{compactenum}}
\newcommand{\op}[2]{\la #1, #2 \ra}
\newcommand{\treq}{\trianglelefteq}
\newcommand{\et}{\mathrel{\&}}

\newcommand\defeq{\mathrel{\overset{\makebox[0pt]{\mbox{\normalfont\tiny\sffamily def}}}{=}}}


\begin{document}
%titulní stránka
\begin{titlepage}
%\fontsize{16.16pt}{25pt}\selectfont
\Large
\begin{center}
Univerzita Karlova v~Praze, Filozofick{\/á} fakulta\\
Katedra logiky

\vspace{8.5em}
\textsc{Mikluáš Mrva}\\[1.4em]
{REFLECTION PRINCIPLES AND LARGE CARDINALS}\\
Bakalářská práce\\[6.8em]
Vedoucí práce: Mgr. Radek Honzík, Ph.D.\\[6.8em]
2015
\end{center}
\end{titlepage}\





\vspace{\fill}
\noindent 
Prohlašuji, že jsem bakalářkou práci vypracoval samostatně a~že jsem uvedl všechny použité prameny a~literaturu.

\bigskip
\noindent V~Praze 14.~dubna 2015\\[3em]
\hspace*{\fill}Mikuláš Mrva\hspace*{3em}
\clearpage

\begin{abstract}
\noindent Práce zkoumá vztah tzv. principů reflexe a velkých kardinálů. Lévy ukázal, že v ZFC platí tzv. věta o reflexi~a dokonce, že věta o reflexi je ekvivalentní schématu nahrazení a~axiomu nekonečna nad teorií ZFC bez axiomu nekonečna a~schématu nahrazení. Tedy lze na větu o~reflexi pohlížet jako na svého druhu axiom nekonečna. Práce zkoumá do jaké míry a~jakým způsobem lze větu o reflexi zobecnit a~jaký to má vliv na existenci tzv. velkých kardinálů. Práce definuje nedosažitelné, Mahlovy a nepopsatelné kardinály a ukáže, jak je lze zavést pomocí reflexe. Přirozenou limitou kardinálů získaných reflexí jsou kardinály nekonzistentní s L. Práce nabídne intuitivní zdůvodněn, proč tomu tak je.

\end{abstract}
\bigskip
\renewcommand{\abstractname}{Abstract}
\begin{abstract}
\noindent This thesis aims to examine relations between so called "Reflection Principles" and Large cardinals. Lévy has shown that Reflection Theorem is a sound theorem of ZFC and it is equivalent to Replacement Scheme and the Axiom of Infinity. From this point of view, Reflection theorem can be seen a~specific version of an Axiom of Infinity. This paper aims to examine the Reflection Principle and its generalisations with respect to existence of Large Cardinals. This thesis will establish Inaccessible, Mahlo and Indescribable cardinals and their definition via reflection. A natural limit of Large Cardinals obtained via reflection are cardinals inconsistent with L. The thesis will offer an intuitive explanation of why this is the case.
\end{abstract}
\clearpage

\tableofcontents
\clearpage

% podekovani firme co vyrabi club mate -- Loscher gmbh?
\pagestyle{fancy} %detailní definice chování záhlaví
\renewcommand{\sectionmark}[1]{\markboth{\slshape\thesection.\ #1}{}}


\usepackage{color} %pro barevné odkazy, příp. nadpisy
\definecolor{odkazy}{rgb}{0.21,0.27,0.53} %tmavì modrá
\definecolor{nadpisy}{rgb}{0.5812,0.0665,0.0659} %cihlová
%
% Parametry prevodu do pdf
\providecommand{\hypersetup}[1]{}%
\hypersetup{%
unicode,% ? Pravdepodobne bezvyznamne
pdfauthor={Mikuláš Mrva},
pdftitle={Reflection principles and large cardinals},
pdfsubject={Reflection principles and large cardinals},
pdfkeywords={set theory, large cardinals, reflection principle, ZFC, Azriel Lévy},
pdffitwindow=false,% Inicialni umisteni textu v okne Readeru
bookmarksopen=true,% Panel zalozek inicialne zobrazen
% Je-li tohle nastaveno jinak, nektere odkazy nekdy nefunguji
hypertexnames=false,
plainpages=false,
%pdfpagelabels,
%
breaklinks=true,% Radkovy lom smi prijit do klikatelneho odkazu
linkcolor=odkazy,% Graficka podoba odkazu
citecolor=odkazy,% ...
colorlinks=true,% ...
pdfhighlight=/O% ... (vzhled odkazu pri stisknuti)
}%
% Inputenc je asi zbytecne.
% Option 'split' ovlivnuje deleni slov obsahujicich v sobe rozdelovnik
\usepackage[utf8x]{inputenc} % UTF-8 ?
%\usepackage[czech]{babel} %dnes už je však hotová integrace èeštiny do babelu
%\usepackage[split]{czech} %dnes už je však hotová integrace èeštiny do babelu
%
%\usepackage{logdp} %užiteèné drobnosti
%\usepackage{amsthm} %lepšší práce s větami
%\usepackage{amsmath} %nová prostøedí pro matematiku a vylepšení tìch stávajících
%\usepackage{latexsym,amsfonts,amssymb} % nová písmenka
\usepackage{fancyhdr} % zápatí a záhlaví
%\usepackage[nottoc]{tocbibind} % přidá do obsahu položky Literatura a Rejstřík
\usepackage{csquotes}
\pagestyle{plain}
%pøedbìžné nastavení hlavièky (balík fancyhdr)
%\headheight 13.6pt %možná ji bude tøeba zvednout, fancyhdr si pak stìžuje: \headheight
% too small, make it at least Xpt
\headheight 14.5pt %možná ji bude tøeba zvednout, fancyhdr si pak stìžuje: \headheight too \fancyhead{}
\fancyhead[R]{\leftmark}
\fancyfoot{}
\fancyfoot[C]{\thepage}


\newtheorem{theorem}{Theorem}[section]
\newtheorem{Claim}[theorem]{Claim}
\newtheorem{definition}[theorem]{Definition}
\newtheorem{Cor}[theorem]{Corollary}
\newtheorem{Fact}[theorem]{Fact}
\newtheorem{lemma}[theorem]{Lemma}
\newtheorem{sublemma}[theorem]{Sublemma}
\newtheorem{ex}[theorem]{Example}
\newtheorem{remark}[theorem]{Remark}
\newtheorem{obs}[theorem]{Observation}
\newtheorem{que}[theorem]{Question}
\newtheorem{conjecture}[theorem]{Conjecture}

\renewcommand{\theequation}{\thesection.\arabic{equation}}

\newenvironment{proof}
{\noindent \textit{Proof.}}
{\hspace*{\fill} $\Box$}

\newcommand{\toch}{\fbox{\small {\bf ??}}}
\newcommand{\bt}[1]{{\underset{\widetilde{}}{#1}}}
\newcommand{\trcl}[1]{\ensuremath{\mathrm{trcl}(\{#1\})}}
\newcommand{\cf}[1]{\ensuremath{\mathrm{cf}(#1)}}
\newcommand{\cl}[1]{\ensuremath{\mathrm{cl}}(#1)}
\newcommand{\ord}[1]{\ensuremath{\mathrm{ORD}}(#1)}
\newcommand{\dom}[1]{\ensuremath{\mathrm{dom}}(#1)}
\newcommand{\rng}[1]{\ensuremath{\mathrm{rng}}(#1)}
\newcommand{\power}[1]{\ensuremath{\mathscr{P}} (#1)}
\newcommand{\set}[2]{\ensuremath{\{#1 \,|\, #2 \}}}
\newcommand{\seq}[2]{\ensuremath{\langle #1 \,|\, #2 \rangle}}
\newcommand{\singl}[1]{\ensuremath{\{#1\}}}
\newcommand{\pair}[2]{\ensuremath{\{ #1, #2 \}}}
\newcommand{\restr}[2]{\ensuremath{#1 \! \upharpoonright \! #2}}
\renewcommand{\iff}{\leftrightarrow}
\newcommand{\Iff}{\Leftrightarrow}
\newcommand{\el}{\prec}
\newcommand{\iso}{\cong}
\newcommand{\sub}{\subseteq}
\newcommand{\super}{\supseteq}
\newcommand{\la}{\langle}
\newcommand{\ra}{\rangle}
\newcommand{\embed}{\rightarrow}
\newcommand{\mc}{\mathcal}
\newcommand{\supr}[1]{\mathrm{sup}\,#1}
\newcommand{\then}{\rightarrow}
\newcommand{\conc}{^{\smallfrown}}
\newcommand{\bb}{\mathbb}
\newcommand{\supp}[1]{\mathrm{supp}(#1)}
\newcommand{\beq}{\begin{equation}}
\newcommand{\eeq}{\end{equation}}
\newcommand{\brm}{\begin{remark}\begin{rm}}
\newcommand{\erm}{\end{rm}\end{remark}}
\newcommand{\mx}{\mathrm}
\newcommand{\bce}{\begin{compactenum}}
\newcommand{\ece}{\end{compactenum}}
\newcommand{\op}[2]{\la #1, #2 \ra}
\newcommand{\treq}{\trianglelefteq}
\newcommand{\et}{\mathrel{\&}}

\newcommand\defeq{\mathrel{\overset{\makebox[0pt]{\mbox{\normalfont\tiny\sffamily def}}}{=}}}


\begin{document}
%titulní stránka
\begin{titlepage}
%\fontsize{16.16pt}{25pt}\selectfont
\Large
\begin{center}
Univerzita Karlova v~Praze, Filozofick{\/á} fakulta\\
Katedra logiky

\vspace{8.5em}
\textsc{Mikluáš Mrva}\\[1.4em]
{REFLECTION PRINCIPLES AND LARGE CARDINALS}\\
Bakalářská práce\\[6.8em]
Vedoucí práce: Mgr. Radek Honzík, Ph.D.\\[6.8em]
2015
\end{center}
\end{titlepage}\





\vspace{\fill}
\noindent 
Prohlašuji, že jsem bakalářkou práci vypracoval samostatně a~že jsem uvedl všechny použité prameny a~literaturu.

\bigskip
\noindent V~Praze 14.~dubna 2015\\[3em]
\hspace*{\fill}Mikuláš Mrva\hspace*{3em}
\clearpage

\begin{abstract}
\noindent Práce zkoumá vztah tzv. principů reflexe a velkých kardinálů. Lévy ukázal, že v ZFC platí tzv. věta o reflexi~a dokonce, že věta o reflexi je ekvivalentní schématu nahrazení a~axiomu nekonečna nad teorií ZFC bez axiomu nekonečna a~schématu nahrazení. Tedy lze na větu o~reflexi pohlížet jako na svého druhu axiom nekonečna. Práce zkoumá do jaké míry a~jakým způsobem lze větu o reflexi zobecnit a~jaký to má vliv na existenci tzv. velkých kardinálů. Práce definuje nedosažitelné, Mahlovy a nepopsatelné kardinály a ukáže, jak je lze zavést pomocí reflexe. Přirozenou limitou kardinálů získaných reflexí jsou kardinály nekonzistentní s L. Práce nabídne intuitivní zdůvodněn, proč tomu tak je.

\end{abstract}
\bigskip
\renewcommand{\abstractname}{Abstract}
\begin{abstract}
\noindent This thesis aims to examine relations between so called "Reflection Principles" and Large cardinals. Lévy has shown that Reflection Theorem is a sound theorem of ZFC and it is equivalent to Replacement Scheme and the Axiom of Infinity. From this point of view, Reflection theorem can be seen a~specific version of an Axiom of Infinity. This paper aims to examine the Reflection Principle and its generalisations with respect to existence of Large Cardinals. This thesis will establish Inaccessible, Mahlo and Indescribable cardinals and their definition via reflection. A natural limit of Large Cardinals obtained via reflection are cardinals inconsistent with L. The thesis will offer an intuitive explanation of why this is the case.
\end{abstract}
\clearpage

\tableofcontents
\clearpage

% podekovani firme co vyrabi club mate -- Loscher gmbh?
\pagestyle{fancy} %detailní definice chování záhlaví
\renewcommand{\sectionmark}[1]{\markboth{\slshape\thesection.\ #1}{}}


\usepackage{color} %pro barevné odkazy, příp. nadpisy
\definecolor{odkazy}{rgb}{0.21,0.27,0.53} %tmavì modrá
\definecolor{nadpisy}{rgb}{0.5812,0.0665,0.0659} %cihlová
%
% Parametry prevodu do pdf
\providecommand{\hypersetup}[1]{}%
\hypersetup{%
unicode,% ? Pravdepodobne bezvyznamne
pdfauthor={Mikuláš Mrva},
pdftitle={Reflection principles and large cardinals},
pdfsubject={Reflection principles and large cardinals},
pdfkeywords={set theory, large cardinals, reflection principle, ZFC, Azriel Lévy},
pdffitwindow=false,% Inicialni umisteni textu v okne Readeru
bookmarksopen=true,% Panel zalozek inicialne zobrazen
% Je-li tohle nastaveno jinak, nektere odkazy nekdy nefunguji
hypertexnames=false,
plainpages=false,
%pdfpagelabels,
%
breaklinks=true,% Radkovy lom smi prijit do klikatelneho odkazu
linkcolor=odkazy,% Graficka podoba odkazu
citecolor=odkazy,% ...
colorlinks=true,% ...
pdfhighlight=/O% ... (vzhled odkazu pri stisknuti)
}%
% Inputenc je asi zbytecne.
% Option 'split' ovlivnuje deleni slov obsahujicich v sobe rozdelovnik
\usepackage[utf8x]{inputenc} % UTF-8 ?
%\usepackage[czech]{babel} %dnes už je však hotová integrace èeštiny do babelu
%\usepackage[split]{czech} %dnes už je však hotová integrace èeštiny do babelu
%
%\usepackage{logdp} %užiteèné drobnosti
%\usepackage{amsthm} %lepšší práce s větami
%\usepackage{amsmath} %nová prostøedí pro matematiku a vylepšení tìch stávajících
%\usepackage{latexsym,amsfonts,amssymb} % nová písmenka
\usepackage{fancyhdr} % zápatí a záhlaví
%\usepackage[nottoc]{tocbibind} % přidá do obsahu položky Literatura a Rejstřík
\usepackage{csquotes}
\pagestyle{plain}
%pøedbìžné nastavení hlavièky (balík fancyhdr)
%\headheight 13.6pt %možná ji bude tøeba zvednout, fancyhdr si pak stìžuje: \headheight
% too small, make it at least Xpt
\headheight 14.5pt %možná ji bude tøeba zvednout, fancyhdr si pak stìžuje: \headheight too \fancyhead{}
\fancyhead[R]{\leftmark}
\fancyfoot{}
\fancyfoot[C]{\thepage}


\newtheorem{theorem}{Theorem}[section]
\newtheorem{Claim}[theorem]{Claim}
\newtheorem{definition}[theorem]{Definition}
\newtheorem{Cor}[theorem]{Corollary}
\newtheorem{Fact}[theorem]{Fact}
\newtheorem{lemma}[theorem]{Lemma}
\newtheorem{sublemma}[theorem]{Sublemma}
\newtheorem{ex}[theorem]{Example}
\newtheorem{remark}[theorem]{Remark}
\newtheorem{obs}[theorem]{Observation}
\newtheorem{que}[theorem]{Question}
\newtheorem{conjecture}[theorem]{Conjecture}

\renewcommand{\theequation}{\thesection.\arabic{equation}}

\newenvironment{proof}
{\noindent \textit{Proof.}}
{\hspace*{\fill} $\Box$}

\newcommand{\toch}{\fbox{\small {\bf ??}}}
\newcommand{\bt}[1]{{\underset{\widetilde{}}{#1}}}
\newcommand{\trcl}[1]{\ensuremath{\mathrm{trcl}(\{#1\})}}
\newcommand{\cf}[1]{\ensuremath{\mathrm{cf}(#1)}}
\newcommand{\cl}[1]{\ensuremath{\mathrm{cl}}(#1)}
\newcommand{\ord}[1]{\ensuremath{\mathrm{ORD}}(#1)}
\newcommand{\dom}[1]{\ensuremath{\mathrm{dom}}(#1)}
\newcommand{\rng}[1]{\ensuremath{\mathrm{rng}}(#1)}
\newcommand{\power}[1]{\ensuremath{\mathscr{P}} (#1)}
\newcommand{\set}[2]{\ensuremath{\{#1 \,|\, #2 \}}}
\newcommand{\seq}[2]{\ensuremath{\langle #1 \,|\, #2 \rangle}}
\newcommand{\singl}[1]{\ensuremath{\{#1\}}}
\newcommand{\pair}[2]{\ensuremath{\{ #1, #2 \}}}
\newcommand{\restr}[2]{\ensuremath{#1 \! \upharpoonright \! #2}}
\renewcommand{\iff}{\leftrightarrow}
\newcommand{\Iff}{\Leftrightarrow}
\newcommand{\el}{\prec}
\newcommand{\iso}{\cong}
\newcommand{\sub}{\subseteq}
\newcommand{\super}{\supseteq}
\newcommand{\la}{\langle}
\newcommand{\ra}{\rangle}
\newcommand{\embed}{\rightarrow}
\newcommand{\mc}{\mathcal}
\newcommand{\supr}[1]{\mathrm{sup}\,#1}
\newcommand{\then}{\rightarrow}
\newcommand{\conc}{^{\smallfrown}}
\newcommand{\bb}{\mathbb}
\newcommand{\supp}[1]{\mathrm{supp}(#1)}
\newcommand{\beq}{\begin{equation}}
\newcommand{\eeq}{\end{equation}}
\newcommand{\brm}{\begin{remark}\begin{rm}}
\newcommand{\erm}{\end{rm}\end{remark}}
\newcommand{\mx}{\mathrm}
\newcommand{\bce}{\begin{compactenum}}
\newcommand{\ece}{\end{compactenum}}
\newcommand{\op}[2]{\la #1, #2 \ra}
\newcommand{\treq}{\trianglelefteq}
\newcommand{\et}{\mathrel{\&}}

\newcommand\defeq{\mathrel{\overset{\makebox[0pt]{\mbox{\normalfont\tiny\sffamily def}}}{=}}}


\begin{document}
%titulní stránka
\begin{titlepage}
%\fontsize{16.16pt}{25pt}\selectfont
\Large
\begin{center}
Univerzita Karlova v~Praze, Filozofick{\/á} fakulta\\
Katedra logiky

\vspace{8.5em}
\textsc{Mikluáš Mrva}\\[1.4em]
{REFLECTION PRINCIPLES AND LARGE CARDINALS}\\
Bakalářská práce\\[6.8em]
Vedoucí práce: Mgr. Radek Honzík, Ph.D.\\[6.8em]
2015
\end{center}
\end{titlepage}\





\vspace{\fill}
\noindent 
Prohlašuji, že jsem bakalářkou práci vypracoval samostatně a~že jsem uvedl všechny použité prameny a~literaturu.

\bigskip
\noindent V~Praze 14.~dubna 2015\\[3em]
\hspace*{\fill}Mikuláš Mrva\hspace*{3em}
\clearpage

\begin{abstract}
\noindent Práce zkoumá vztah tzv. principů reflexe a velkých kardinálů. Lévy ukázal, že v ZFC platí tzv. věta o reflexi~a dokonce, že věta o reflexi je ekvivalentní schématu nahrazení a~axiomu nekonečna nad teorií ZFC bez axiomu nekonečna a~schématu nahrazení. Tedy lze na větu o~reflexi pohlížet jako na svého druhu axiom nekonečna. Práce zkoumá do jaké míry a~jakým způsobem lze větu o reflexi zobecnit a~jaký to má vliv na existenci tzv. velkých kardinálů. Práce definuje nedosažitelné, Mahlovy a nepopsatelné kardinály a ukáže, jak je lze zavést pomocí reflexe. Přirozenou limitou kardinálů získaných reflexí jsou kardinály nekonzistentní s L. Práce nabídne intuitivní zdůvodněn, proč tomu tak je.

\end{abstract}
\bigskip
\renewcommand{\abstractname}{Abstract}
\begin{abstract}
\noindent This thesis aims to examine relations between so called "Reflection Principles" and Large cardinals. Lévy has shown that Reflection Theorem is a sound theorem of ZFC and it is equivalent to Replacement Scheme and the Axiom of Infinity. From this point of view, Reflection theorem can be seen a~specific version of an Axiom of Infinity. This paper aims to examine the Reflection Principle and its generalisations with respect to existence of Large Cardinals. This thesis will establish Inaccessible, Mahlo and Indescribable cardinals and their definition via reflection. A natural limit of Large Cardinals obtained via reflection are cardinals inconsistent with L. The thesis will offer an intuitive explanation of why this is the case.
\end{abstract}
\clearpage

\tableofcontents
\clearpage

% podekovani firme co vyrabi club mate -- Loscher gmbh?
\pagestyle{fancy} %detailní definice chování záhlaví
\renewcommand{\sectionmark}[1]{\markboth{\slshape\thesection.\ #1}{}}


\section{Introduction}\label{sec:introduction}
The central point of this thesis is the so called \emph{reflection principle}, which could be informally expressed like this:
\begin{displayquote}
\emph{For every property that holds in the universe of all sets, there is a set in which this property holds.}
\end{displayquote}

Clearly, this formulation is rather vague and we should be extremely cautious when dealing with the word ``property''. 
One problem that immediately comes to mind is that ``being the set of all sets'' must not be considered a property in this sense, otherwise we run into the well–known paradox of Russell.
This is a well–known problem that exemplifies the fact that reflection is a phenomenon that is closely connected to the very foundations of mathematics.
This is also emphasised by the fact that the very first explicit use of reflection in a mathematical proof can be found in Gödel's paper \emph{The Consistency of the Axiom of Choice and of the Generalised Continuum Hypothesis with the Axioms of Set Theory}\footnote{See \cite{Godel1940consistency}.}
that deals with the consistency of the \emph{generalised continuum hypothesis}, which is a question that played an important part in the development of set theory in the 20\textsuperscript{th} century.
Furthermore, Lévy's article \emph{Axiom Schemata of Strong Infinity in Axiomatic Set Theory}, that is a cornerstone of this thesis is concerned primarily with the so called \emph{strong axioms (or axiom schemata) of infinity}, which are axioms or axiom schemata that imply the existence of the set of all natural numbers. This assertion is called the \emph{Axiom of Infinity}\footnote{For a rigorous definition, see definition \bref{def:infinity} later in this section.}, but they also imply the existence of larger sets whose existence can not be proved in the current theory\footnote{For the purposes of this thesis, unless stated otherwise, this will be the \emph{Zermelo–Fraenkel set theory}, that is formally established in definition \bref{def:zfc}.}.

As we will show in chapter 2, reflection is closely related to the \emph{Axiom Schema of Replacement}, which was the subject of philosophical debates because it wasn't included in the original axiomatic set theory proposed by Zermelo and unlike other axioms in the \emph{Zermelo–Fraenkel set theory}, its presence is not justified from the iterative conception of a set, but rather from its usefulness. Unlike \emph{Replacement Schema}, reflection is not so easily questioned from a platonist\footnote{According to \emph{Stanford Encyclopedia of Philosophy}, ``mathematical platonism is the metaphysical view that there are abstract mathematical objects whose existence is independent of us and our language, thought, and practices. Just as electrons and planets exist independently of us, so do numbers and sets. And just as statements about electrons and planets are made true or false by the objects with which they are concerned and these objects' perfectly objective properties, so are statements about numbers and sets. Mathematical truths are therefore discovered, not invented.''} point of view, but it may be formulated in two different was. The two following informal interpretations of reflections are based on \cite{HellmanInfinite}. Their purpose is to illustrate the difference between a platonist and a structuralist\footnote{According to wikipedia, ``Structuralism is a theory in the philosophy of mathematics that holds that mathematical theories describe structures of mathematical objects. Mathematical objects are exhaustively defined by their place in such structures. Consequently, structuralism maintains that mathematical objects do not possess any intrinsic properties but are defined by their external relations in a system.''} approach towards reflection.
\begin{displayquote}
``The true situation (in the universe of sets) is reflected in arbitrarily high level of the cumulative hierarchy.''
\end{displayquote}
\begin{displayquote}
``We're interested in structures so large that certain attempts to describe them fail to distinguish them from various proper initial segments–hence small fragments–of them.''
\end{displayquote}
There is no point in dedicating more space to the philosophy of mathematics as it is outside the scope of this thesis, it is only worth noting that the author usually thinks of reflection in the latter sense which may be reflected in the way this thesis is written.

After introducing the elementary theoretical tools required for this task in the rest of this chapter, in chapter 2, we will review the \emph{Reflection Theorem} that originally formulated by Richard Montague in 1961\footnote{Note that Lévy's paper was published in 1960, a year before Montague's, but Lévy refers to Montague and not vice versa. While this may seem confusing, it is because Montague gave a lecture on this topic at a conference at the Cornell University in 1957. It is also interesting that Lévy's article refers for Montague's reflection to a publication by Montague and Solomon Feferman called \emph{The method of arithmetization and some of its applications} which was never finished. This is explained by Solomon Feferman in \cite{Feferman2008}.} and extended by Azriel Lévy in his aforementioned article and then restate it in a way that is more in line with today's set theory. 
This part of the thesis deals with the fact that when the term ``property'' is restricted to first–order formulas in the language of set theory, it does not behave like a axiom of strong infinity, but it is equivalent to the \emph{Axiom of Infinity} and \emph{Replacement Schema}, which is one of the key set–forming principles in the \emph{Zermelo–Fraenkel set theory}.

It is in chapter 3 where will examine some large cardinal properties and in a manner similar to Lévy's article, we will introduce axiom schemata that come from reflection and lead towards \emph{inaccessible} and \emph{Mahlo cardinals}. We will briefly argue that Mahlo's operation exhausts large cardinals reachable via reflection from below and introduce indescribable cardinals, which are also based on reflection, but lead us into higher–order logic. We will introduce \emph{weakly inaccessible cardinals} and show that they are also based on reflection and examine their relation to the cardinals presented earlier. Finally, we will examine Gödel's constructible universe and see whether the large cardinals we have introduced are compatible with the \emph{Axiom of Constructibility}, an assertion that every set is definable.

%\begin{displayquote}
%``The Universe of sets cannot be uniquely characterised (i. e. distinguished from all its initial elements) by any internal structural property of the membership relation in it, which is expressible in any logic of finite of transfinite type, including infinitary logics of any cardinal order.''
%\end{displayquote}
%\rightline{{\rm --- Kurt Gödel \cite{GodelWang}}}

\subsection{Notation and Terminology}
\subsubsection{The Language of Set Theory}
This text assumes the knowledge of basic terminology and some results from first–order predicate logic, see any entry–level like \cite{hamiltonBook}. For this reason, we won't introduce the notions of \emph{language}, \emph{function symbol}, \emph{predicate}, \emph{term}, \emph{model} and \emph{interpretation} that are used in definition \bref{def:satisfaction}.


All proofs are based on \cite{JechBook} unless explicitly stated otherwise. Notable amount of inspiration is also drawn from \cite{KanamoriBook} and \cite{DrakeBook}.

We will now shortly review the basic notions that allow us to define the \emph{Zermelo–Fraenkel} set theory.

% theory with the Axiom of Choice ($\sf{ZFC}$), a first–order set theory in the language $\mathscr{L} = \{=, \in\}$, which will be sometimes referred to as \emph{the language of set theory}. In Chapter 3, the Axiom of Choice is assumed. When tlaking about higher–order logic, we will usually denote type 1 variables, which are elements of the domain of discourse, by lowercase letters, mostly $u, v, w, x, y, z, p_1, p_2, p_3,  \ldots$ while type 2 variables, which represent $n$–ary relations of the domain of discourse for any natural number $n$, are usually denoted by uppercase letters $A, B, C, X, Y, Z$. Note that there are exception to convention rules as $f$ usually denotes a function, which is in fact a type 2 variable. On the other hand, $M$ stands for a set.


When we talk about a \emph{class}, we have the notion of a definable class in mind. 
If $\varphi(x, p_1, \ldots, p_n)$ is a formula in the language of set theory, we call 
\begin{equation}
A = \{x : \varphi(x, p_1, \ldots, p_n)\}
\end{equation}
a class of all sets satisfying $\varphi(x, p_1, \ldots, p_n)$ in a sense that 
\begin{equation}
x \in A \iff \varphi(x, p_1, \ldots, p_n)
\end{equation}
for some $p_1, \ldots, p_n$. Given classes $A$, $B$, one can easily define the elementary set operations such as $A \cap B$, $A \cup B$, $A \setminus B$, $\bigcup A$, see the first chapter of \cite{JechBook} for details.
Axioms are the tools by which we can decide whether a particular class is ``small enough'' to be considered a set\footnote{``Small enough'' means that it doesn't lead to a paradox similar to the famous Russell's paradox.}. A class that fails to be considered a set is called a \emph{proper class}.

We will often write something like ``$M$ is a limit ordinal'', it should always be clear that this can be rewritten as a formula that was introduced earlier. Tuples are notated as $\la a, b \ra$.

\


\subsubsection{The Axioms}

\begin{definition}{(The Existence of a Set)}\label{def:existence_of_a_set}
\begin{equation}
\exists x (x = x)
\end{equation}
\end{definition}
% The above axiom is usually omitted because it can be deduced from the axiom of \emph{Infinity} (see below), but since we will be using set theories that omit \emph{Infinity}, this will be useful.

\begin{definition}{(Axiom of Extensionality)}\label{def:extensionality}
\begin{equation}
\forall x, y (x = y \iff \forall z (z \in x \iff z \in y)) %ok
\end{equation}
\end{definition}

\begin{definition}{(Axiom Schema of Specification)}\label{def:specification}\\
The following yields an axiom for every first–order formula $\varphi(x, p_1, \ldots, p_n)$ with no free variables other than $x, p_1, \ldots, p_n$.
\begin{equation}
\forall x, p_1, \ldots, p_n \exists y \forall z ( z \in y \iff z \in x \et \varphi(z, p_1, \ldots, p_n)) %ok
\end{equation}
\end{definition}

We will now provide two definitions that are not axioms, but will be helpful in establishing the next axioms in a more comprehensible way.
\begin{definition}{($x \subseteq y$, $x \subset y$)}\label{def:subset}
\begin{equation}
x \subseteq y \iff (\forall z \in x) z \in y
\end{equation}
\begin{equation}
x \subset y \iff x \subseteq y \et x \neq y
\end{equation}
We read $x \subseteq y$ as \emph{x is a subset of y} and $x \subset y$ as \emph{x is a proper subset of y}.
\end{definition}

\begin{definition}{(Empty Set)}\label{def:emptyset}
%Let $\varphi = \neg(x = x)$, $y$ is an arbitrary set, we there exists at least one set $y$ from \bref{def:existence_of_a_set} or \emph{Infinity})
%\begin{equation}
%\emptyset \defeq \{x : x \in y\ \et \varphi(x)\}
%\end{equation}
%We know that $\emptyset$ is a set from \emph{specification} and it is the same set for every $y$ given from \emph{extensionality}.
%co radsi ``for an arbitrary $x$''
For an arbitrary set~$x$, the empty set, represented by the symbol ``$\emptyset$'', is the set defined by the following formula:
\beq
(\forall y \in x)(y \in \emptyset \iff \neg(y = y))
\eeq
Clearly $\emptyset$ is a set due to \emph{Specification Schema}, there is only one such set due to the \emph{Axiom of Extensionality}, no matter which $x$ is chosen. 
\end{definition}

\begin{definition}{(Axiom of Pairing)}\label{def:pairing}
\begin{equation}
\forall x, y \exists z \forall q (q \in z \iff q = x \lor q = y) % ok
\end{equation}
\end{definition}

\begin{definition}{(Axiom of Union)}\label{def:union}
\begin{equation}
\forall x \exists y \forall z (z \in y \iff \exists q( z \in q \et q \in x)) % ok
\end{equation}
\end{definition}

%\begin{definition}{(Set Intersection)}\\
%\beq
%x \cap y = \{ z: z \in x \et z \in y \}
%\eeq
%\end{definition}
%
%\begin{definition}{(Set Union)}\\
%\beq
%x \cup y = \{ z: z \in x \lor z \in y \}
%\eeq
%\end{definition}

\begin{definition}{(Axiom of Foundation)}\label{def:foundation}
\begin{equation}
\forall x (x \neq \emptyset \then (\exists y \in x) (x \cap y = \emptyset)) %ok
\end{equation}
\end{definition}

\begin{definition}{(Axiom of Power Set)}\label{def:powerset}
\begin{equation}
\forall x \exists y \forall z (z \in y \iff z \subseteq x) %ok
\end{equation}
\end{definition}

\begin{definition}{(Axiom of Infinity)}\label{def:infinity} %?
\begin{equation}
\exists x (\emptyset \in x \et (\forall y \in x)(y\cup\{y\} \in x))\label{eq:axiom_of_infinity}
\end{equation}
The least set satisfying \eref{eq:axiom_of_infinity} is denoted $\omega$.
\end{definition}

\begin{definition}{(Function)}\label{def:function}\\
Given an arbitrary first–order formula $\varphi(x, y, p_1, \ldots, p_n)$, we say that $\varphi$ is a~function iff
\begin{equation}\label{def:function_formula}
\forall x, y, z, p_1, \ldots, p_n (\varphi(x, y, p_1, \ldots, p_n) \et \varphi(x, z, p_1, \ldots, p_n) \then y = z)
\end{equation}
\end{definition}
When $\varphi(x, y)$ is a~function, we also write the following:
\begin{equation}
\varphi(x, y) \mbox{ iff } f(x) = y
\end{equation}
Alternatively, $f = \{\langle x, y \rangle : \varphi(x, y)\}$ is a~class.

Let us introduce a few more definitions that will make the two remaining axioms more comprehensible.
\begin{definition}{(Power Set Function)}\\
Given a set~$x$, the \emph{power set of $x$}, denoted $\power{x}$ and satisfying the definition \bref{def:powerset}\, is defined as follows:
\begin{equation}
\power{x} = \{y: y \subseteq x\}
\end{equation}
\end{definition}

% TODO hezci formulace
\begin{definition}{(Domain of a Function)}\label{def:dom}\\
Let $f$ be a function. We call the \emph{domain of $f$} the class of all sets for which $f$ is defined. We use ``$Dom(f)$'' to refer to this set.
%We read the following as ``$Dom(f)$ is the domain of $f$''.
\begin{equation}
\forall x (x \in Dom(f) \iff \exists y (f(x) = y))
\end{equation}
\end{definition}
We say ``$f$ is a function on $A$'', $A$ being a class, if $A = dom(f)$.

\begin{definition}{(Range of a Function)}\label{def:rng}\\
Let $f$ be a function. We call the \emph{range of $f$} the set of all sets that are images of other sets via $f$. We use ``$Rng(f)$'' to refer to this set. %We read the following as ``$Rng(f)$ is the range of $f$''.
\begin{equation}
\forall x (x \in Rng(f) \iff \exists y (f(y) = x))
\end{equation}
\end{definition}
We say that $f$ is \emph{a function into $A$}, $A$ being a class, iff $rng(f) \subseteq A$.
We say that $f$ is \emph{a function onto $A$} iff $rng(f) = A$. %, in other words,
%\begin{equation}
%(\forall y \in A)(\exists x \in dom(f))(f(x) = y)
%\end{equation}
We say a function $f$ is a \emph{one to one function}, iff
\begin{equation}
(\forall x_1, x_2 \in dom(f))(f(x_1) = f(x_2) \then x_1 = x_2)
\end{equation}
We say that $f$ is a bijection iff it is a one to one function that is onto.

Note that \emph{Dom(f)} and \emph{Rng(f)} are not definitions in a strict sense, they are in fact definition schemas that yield definitions for every function $f$ given. Also note that they can be easily modified for $\varphi$ instead of $f$, with the only difference being the fact that it is then defined only for those $\varphi$s that are functions, which must be taken into account. This is worth noting as we will use the notions of \emph{function} and \emph{formula} interchangably.

\begin{definition}{(Function Defined For All Ordinals)}\label{def:function_dfao}\\
We say a function $f$ is \emph{defined for all ordinals}, this is sometimes written $f: Ord \then A$ for any class $A$, if $Dom(f) = Ord$.\
Alternatively,
\begin{equation}
(\forall \alpha \in Ord)(\exists y \in A)(f(\alpha) = y))
\end{equation}
\end{definition}

\begin{definition}{(Axiom Schema of Replacement)}\label{def:replacement}\\
The following is an axiom for every first–order formula $\varphi(x, p_1, \ldots, p_n)$ with no free variables other than $x, p_1, \ldots, p_n$.
\begin{equation}
``\varphi\mbox{ is a function}''\then \forall x \exists y \forall z (z \in y \iff (\exists q \in x)(\varphi(x, y, p_1, \ldots, p_n)))
\end{equation}
\end{definition}

\begin{definition}{(Choice function)}\label{def:choice_function}\\
We say that a function $f$ is a \emph{choice function on $x$} iff
\beq
dom(f) = x \setminus \{\emptyset\}) \et (\forall y \in dom(f))(f(y) \in y)
\eeq
\end{definition}


\begin{definition}{(Axiom of Choice)}\label{def:choice}\\
For every set~$x$~there is a function $f$ that is a choice function on $x$.
\end{definition}
One might be unsettled by the fact that this definition quantifies over functions, which are generally classes, but in this particular case, since $dom(f) = x$ and~$x$~is a set, $f$ is also a set due to \emph{Replacement}\footnote{If the underlying theory includes of implies \emph{Replacement}.}.

% We will refer to the axioms by their name, written in italic type, e.g. \emph{Foundation} refers to the Axiom of Foundation. Now we need to define the set theories to be used in the article. 

\begin{definition}{$(\sf{S})$}\label{def:s}\\ %ok
We call $\sf{S}$ an axiomatic theory in the language $\mathscr{L} = \{=, \in\}$ with exactly the following axioms:
\bce[(i)]
\item \emph{Existence of a Set} (see definition \bref{def:existence_of_a_set})
\item \emph{Axiom of Extensionality} (see definition \bref{def:extensionality})
\item \emph{Axiom of Specification} (see definition \bref{def:specification})
\item \emph{Axiom of Foundation} (see definition \bref{def:foundation})
\item \emph{Axiom of Pairing} (see definition \bref{def:pairing})
\item \emph{Axiom of Union} (see definition \bref{def:union})
\item \emph{Axiom of Power Set} (see definition \bref{def:powerset})
\ece
\end{definition}

\begin{definition}{$(\sf{ZF})$}\label{def:zf}\\ %ok
We call $\sf{ZF}$ an axiomatic theory in the language $\mathscr{L} = \{=, \in\}$ that contains all the axioms of $\sf{S}$ in addition to the following:
\bce[(i)]
\item \emph{Axiom of Replacement} schema (see definition \bref{def:replacement})
\item \emph{Axiom of Infinity} (see definition \bref{def:infinity})
\ece
\emph{Existence of a Set} is usually left out because it is a consequence of the \emph{Axiom of Infinity}.
\end{definition}

\begin{definition}{$(\sf{ZFC})$}\label{def:zfc}\\ % ok
$\sf{ZFC}$ is an axiomatic theory in the language $\mathscr{L} = \{=, \in\}$ that contains all the axioms of $\sf{ZF}$ plus \emph{Choice}, see definition \bref{def:choice}).
\end{definition}

\subsubsection{The Transitive Universe}
\begin{definition}{(Transitive Class)}\label{def:transitivity}\\ % ok
We say a class $A$ is \emph{transitive} iff
\begin{equation}
(\forall x \in A)(x \subseteq A)\mbox{.}
\end{equation}
\end{definition}

\begin{definition}{(Well–Ordered Class)}\label{def:well_ordering} % ok
A class $A$ is said to be \emph{well–ordered by $\in$} iff the following hold:
\bce[(i)]
\item $(\forall x \in A)(x \not\in x)$ (Antireflexivity)
\item $(\forall x, y, z \in A)(x \in y \et y \in z \then x \in z)$ (Transitivity)
\item $(\forall x \subseteq A)(x \neq \emptyset \then (\exists y \in x)(\forall z \in x)(z = y \lor z \in y)))$ (Existence of the least element)
\ece
\end{definition}

\begin{definition}{(Ordinal Number)}\label{def:ordinal}\\ % ok
A set~$x$~is said to be an \emph{ordinal number} if it is \emph{transitive} and \emph{well–ordered by $\in$}. 
\end{definition}
For the sake of brevity, we usually just say ``$x$~is an \emph{ordinal}''. 
Note that ``$x$~is an ordinal'' is a well–defined formula in the language of set theory, 
since transitivity is defined in definition \bref{def:transitivity} via a first–order formula and well–ordering\footnote{See definition \bref{def:well_ordering}.} is in fact a conjunction of four first–order formulas.
Ordinals will be usually denoted by lower case greek letters, starting from the beginning of the alphabet: $\alpha, \beta, \gamma, \ldots$.
Given two different ordinals $\alpha, \beta$, we will write $\alpha < \beta$ for $\alpha \in \beta$, see Lemma 2.11 in \cite{JechBook} for technical details.

\begin{definition}{(Non–Zero Ordinal)}\\ % ok 
We say an ordinal $\alpha$ is \emph{non–zero} iff $\alpha \neq \emptyset$.
\end{definition}

\begin{definition}{(Successor Ordinal)}\label{def:successor_ordinal}\\ % ok 
Consider the following function defined for all ordinals. Let $\beta$ be an arbitrary ordinal. We call $S$ the \emph{successor function}.
\begin{equation}
S(\beta) = \beta \cup \{\beta\}
\end{equation}
An ordinal $\alpha$ is called a \emph{successor ordinal} iff there is an ordinal $\beta$, such that $\alpha = S(\beta)$. We also write $\alpha = \beta+1$.
\end{definition}

\begin{definition}{(Limit Ordinal)}\label{def:limit_ordinal}\\ % ok
A non–zero ordinal $\alpha$ is called a \emph{limit ordinal} iff it is not a successor ordinal.
\end{definition}

\begin{definition}{(Ord)}\label{def:ord}\\  % ok
\emph{The class of all ordinal numbers}, which we will denote ``$Ord$''\footnote{Some authors use ``$On$'' instead of ``$Ord$'', we will stick to the notation used in \cite{JechBook}.} is the proper class defined as follows.
\begin{equation}
x \in Ord \iff x\mbox{ is an ordinal}
\end{equation}
\end{definition}

%The following construction will be often referred to as the \emph{Von Neumann's Hierarchy}, sometimes also the \emph{Von Neumann's Universe}. 
%, the former referring more to the construction with the individual levels in mind, the latter referring more to the class $V$, but they can be interchanged with no confusion caused.

\begin{definition}{(Von Neumann's Hierarchy)}\label{def:von_neumann}\\ % ok 
The \emph{Von Neumann's hierarchy} is a collection of sets indexed by the elements of $Ord$, defined recursively in the following way:
\bce[(i)]
\item 
\begin{equation}
V_0 = \emptyset\mbox{,}
\end{equation}
\item 
\begin{equation}
V_{\alpha+1} = \power{V_\alpha}\mbox{ for any ordinal $\alpha$,}
\end{equation}
\item
\begin{equation} 
V_\lambda = \bigcup_{\beta < \lambda} V_\beta \mbox{ for a limit ordinal $\lambda$,}
\end{equation}
\item
\beq
V = \bigcup_{\alpha \in Ord} V_\alpha\mbox{.}
\eeq
\ece
We will also refer to the \emph{Von Neumann's hierarchy} as \emph{Von Neumann's universe} or the \emph{cumulative hierarchy}.
This definition is only correct in a theory that contains or implies \emph{Replacement Schema}. 
Even though $V$ is sometimes also used for the universal class that contains all sets, in this thesis, it will always mean the $V$ defined above.
\end{definition}

\begin{definition}{(Rank)}\label{def:rank}\\ % ok
Given a set~$x$, we say that the rank of~$x$~(written as $rank(x)$) is the least ordinal $\alpha$ such that $x \in V_{\alpha+1}$
\end{definition}
Due to \emph{Axiom of Regularity}, every set has a rank.\footnote{See chapter 6 of \cite{JechBook} for details.} 
The Von Neumann's hierarchy defined above can also be defined by the fact that every $V_\alpha$ is a set of all set with rank less than $\alpha$.

\begin{definition}{(Order–type)}\label{def:order_type}\\ % TODO check
Given an arbitrary well–ordered set~$x$, we say that an ordinal $\alpha$ is the \emph{order–type} of $x$ iff~$x$~and $\alpha$ are isomorphic.
\end{definition}

\subsubsection{Cardinal Numbers}

% todo def one.to–one mapping?
\begin{definition}{(Cardinality)}\\
Given a set~$x$, let the cardinality of~$x$, written $|x|$, be defined as the smallest ordinal number such that there is a one to one mapping from~$x$~onto $\alpha$.
\end{definition}

\begin{definition}{(Aleph function)}\label{def:aleph}\\
Let $\omega$ be the least set satisfying the \emph{Axiom of Infinity}.
We will recursively define the function $\aleph$ for all ordinals.
\bce[(i)]
\item $\aleph_0 = \omega$,
\item $\aleph_{\alpha+1}$ is the least cardinal larger than $\aleph_\alpha$\footnote{``The least cardinal larger than $\aleph_\alpha$'' is sometimes notated as $\aleph_\alpha^{+}$.},
\item $\aleph_\lambda = \bigcup_{\beta < \lambda}\aleph_\beta$ for a limit ordinal $\lambda$.
\ece
If $\kappa = \aleph_\alpha$ and $\alpha$ is a successor ordinal, we call $\kappa$ a \emph{successor cardinal}. If $\alpha$ is a limit ordinal, we call $\kappa$ a \emph{limit cardinal}.
\end{definition} % mam def. succ ordinaly?

\begin{definition}{(Cardinal number)}\label{def:cardinal}\\
\bce[(i)]
\item A set~$x$~is called a \emph{finite cardinal} iff $x \in \omega$.
\item A set is called an \emph{infinite cardinal} iff there is an ordinal $\alpha$ such that $\aleph_\alpha = x$.
\item A set is called a \emph{cardinal} iff it is either a \emph{finite cardinal} or an \emph{infinite cardinal}.
\ece
\end{definition}
We say $\kappa$ is an uncountable cardinal iff it is an infinite ordinal and $\aleph_0 < \kappa$.
Infinite cardinals will be notated by lowercase greek letters from the middle of the alphabet, e.g. $\kappa, \mu, \nu, \ldots$ with the possible exception of $\lambda$, which is next to $\kappa$ in the greek alphabet, but is also sometimes used to denote limit ordinals.

For formal details as well as why every set can be well–ordered assuming the \emph{Axiom of Choice}, and therefore has a cardinality, see \cite{JechBook}. % proc je to tady?

\begin{definition}{(Sequence)}\label{def:sequence}\\
We say that a function $\varphi(x, y)$ is a \emph{sequence} iff there is an ordinal $\alpha$ such that $dom(\varphi) = \alpha$. In other words, a function is called a sequence if it is defined exactly for every ordinal from below some $\alpha$. We then say it is an $\alpha$–sequence. We usually write $\langle \beta_i : i \in \alpha \rangle$ or $\langle \beta_0, \beta_1, \ldots \rangle$ when referring to a sequence, for every $i \in dom(\varphi)$, $\beta_i$ then denotes the respective elements of $rng(\varphi)$.
\end{definition}

\begin{definition}{(Cofinal Subset)}\label{def:cofinal_subset}\\
Given a class $A$ of ordinals, we say that $B \subseteq A$ is \emph{cofinal in $A$} iff
\beq
(\forall x \in A)(\exists y \in B)(x \in y)\mbox{.}
\eeq
\end{definition}

\begin{definition}{(Cofinality of a Limit Ordinal)}\label{def:cofinality}\\ % a co https://math.berkeley.edu/~jhicks/links/SOTS/cskipper112613.pdf? 
Let $\lambda$ be a limit ordinal. 
We say that the \emph{cofinality} of $\lambda$ is $\kappa$ iff $\kappa$ is the least ordinal, such that there is a cofinal $\kappa$–sequence $\langle \beta_\xi : \xi < \kappa \rangle$ satisfying
\begin{equation}
sup(\{\beta_\xi: \xi < \kappa\}) = \lambda\mbox{.}
\end{equation}
We write $cf(\lambda) = \kappa$.
\end{definition}
Note that $cf(\alpha)$ is alway a cardinal\footnote{If $cf(\alpha)$ is not a cardinal, so $|cf(\alpha)| < cf(\alpha)$, then there is a mapping from $|cf(\alpha)|$ onto $cf(\alpha)$. But then the range of this mapping is a cofinal subset of $cf(\alpha)$ that is strictly smaller than $cf(\alpha)$.}.

\begin{definition}{(Regular Cardinal)}\label{def:regular_cardinal}\\
We say an infinite cardinal $\kappa$ is \emph{regular} iff $cf(\kappa) = \kappa$. % TODO chyba podle honzika, watt?
\end{definition}

\begin{definition}{(Strong Limit Cardinal)}\label{def:strong_limit_cardinal}\\
We say that an ordinal $\kappa$ is a \emph{strong limit cardinal} if it is a \emph{limit cardinal} and 
\begin{equation}
(\forall \alpha \in \kappa)(|\power{\alpha}| \in \kappa)\mbox{.}
\end{equation}
\end{definition}

\begin{definition}{(Generalised Continuum Hypothesis)}\label{def:gch}\\
\begin{equation}
(\forall \alpha \in Ord)(\aleph_{\alpha+1}=|\power{\aleph_\alpha}|)
\end{equation}
\end{definition}
If \emph{GCH} holds (for example in Gödel's $L$, see chapter 3), the notions of limit cardinal and strong limit cardinal are equivalent.

\

\subsubsection{Relativisation and Absoluteness} % TODO ujasni si co se deje s parametrama pri relativizaci. co na to Jech?
\begin{definition}{(Relativization)}\label{def:relativization}\\
Let $M$ be a class, $R \subseteq M\times M$ and let $\varphi(p_1, \ldots, p_n)$ be a first–order formula with no free variables besides $p_1, \ldots, p_n$. 
The \emph{relativization of $\varphi$ to $M$ and $R$} is the formula, written as $\varphi^{M, R}$, defined in the following inductive manner:
\bce[(i)]
\item $(x \in y)^{M,R} \iff R(x, y)$,
\item $(x = y)^{M,R} \iff x = y$,
\item $(\neg \varphi)^{M,R} \iff \neg \varphi^{M,R}$,
\item $(\varphi \et \psi)^{M,R} \iff \varphi^{M,R} \et \psi^{M,R}$,
\item $(\varphi \lor \psi)^{M,R} \iff \varphi^{M,R} \lor \psi^{M,R}$,
\item $(\varphi \then \psi)^{M,R} \iff \varphi^{M,R} \then \psi^{M,R}$,
\item $(\exists x \varphi(x))^{M,R} \iff (\exists x \in M) \varphi^{M,R}(x)$,
\item $(\forall x \varphi(x))^{M,R} \iff (\forall x \in M) \varphi^{M,R}(x)$.
\ece

\end{definition}
When $R=\in\cap(M \times M)$, we usually write $\varphi^M$ instead of $\varphi^{M, R}$. When we talk about $\varphi^M(p_1, \ldots, p_n)$, it is understood that $p_1, \ldots, p_n \in M$.

% TODO definice splnovani!

\begin{definition}{(Satisfaction in a Structure)}\label{def:satisfaction}\\
Let $M$ be a set and $R$ a binary relation on $M$. Let $Terms$ be the set of all terms, let $e: Terms \then M$ be any evaluation function. Let $\varphi$  be a first–order formula in the language of set theory.

% neobratne, preformulovat, je to
We say that $\varphi$ \emph{holds in $\langle M, R \rangle$ under the evaluation $e$}, we write $\langle M, R \rangle~\models~\varphi[e]$, iff any of the following hold:
\bce[(i)]
\item $\varphi$ is the formula ``$s = t$'', $s, t$ are terms, both $e(s)$ and $e(t)$ are defined, and $e(s) = e(t)$.
\item $\varphi$ is the formula ``$s \in t$'', $s, t$ are terms, both $e(s)$ and $e(t)$ are defined, and the pair $\la e(s), e(t) \ra$ is in $R$.
\item $\varphi$ is the formula ``$\neg\psi$'' and not $\langle M, R \rangle~\models~\psi[e]$
\item $\varphi$ is the formula ``$\psi_1 \et \psi_2$'' and both $\langle M, R \rangle~\models~\psi_1[e]$ and $\langle M, R \rangle~\models~\psi_2[e]$.
\item $\varphi$ is the formula ``$\psi_1 \lor \psi_2$'' and either $\langle M, R \rangle~\models~\psi_1[e]$ or $\langle M, R \rangle~\models~\psi_2[e]$.
\item $\varphi$ is the formula ``$\psi_1 \then \psi_2$'' and either not $\langle M, R \rangle~\models~\psi_1[e]$ or $\langle M, R \rangle~\models~\psi_2[e]$.
\item $\varphi$ is the formula ``$\psi_1 \then \psi_2$'' and either not $\langle M, R \rangle~\models~\psi_1[e]$ or $\langle M, R \rangle~\models~\psi_2[e]$.
\item $\varphi$ is the formula ``$\forall x_1 \psi$'' and $\langle M, R \rangle~\models~\psi[e']$ for every $e'$ that differs from $e$ only in the value of $x_1$.
\item $\varphi$ is the formula ``$\forall x_1 \psi$'' and $\langle M, R \rangle~\models~\psi[e']$ for every $e'$ that differs from $e$ only in the value of $x_1$.
\item $\varphi$ is the formula ``$\exists x_1 \psi$'' and $\langle M, R \rangle~\models~\psi[e']$ for some $e'$ that differs from $e$ only in the value of $x_1$.
\ece
% https://en.wikibooks.org/wiki/Formal_Logic/Predicate_Logic/Satisfactio
% https://en.wikipedia.org/wiki/T-schema
If $\varphi$ is a sentence, we also write $\langle M, R \rangle~\models~\varphi$. If $\varphi$ is not a sentence, 
the universal closure of $\varphi$ is assumed to be used instead of $\varphi$ if no evaluation is explicitly metioned.
\end{definition}
Note that we say that $M$ is a set. 

We will use $\langle M, R \rangle \models \varphi(p_1, \ldots, p_n)$ and $\varphi^M(p_1, \ldots, p_n)$ interchangably.

\begin{definition}{(Absoluteness)}\\
Given a transitive class $M$, we say a formula $\varphi$ is \emph{absolute in $M$} if for all $p_1, \ldots, p_n \in M$
\begin{equation}
\varphi^M(p_1, \ldots, p_n) \iff \varphi(p_1, \ldots, p_n)
\end{equation}
\end{definition}

\begin{definition}{(Hierarchy of First–Order Formulas)}\\
\bce[(I)]
\item A first–order formula $\varphi$ is $\Delta_0$ iff it is logically equivalent to a first–order formula $\varphi'$ satisfying any of the following:
\bce[(i)]
\item $\varphi'$ contains no quantifiers
\item $y$ is a set, $\psi$ is a $\Delta_0$–formula, and $\varphi'$ is either $(\exists x \in y)\psi(y)$ or $(\forall x \in y)\psi(y)$.
\item $\psi_1, \psi_2$ are $\Delta_0$–formulas and $\varphi'$ is any of the following: $\psi_1 \lor \psi_2$, $\psi_1 \et \psi_2$, $\psi_1 \then \psi_2$, $\neg \psi_2$, 
\ece
\item If a formula is $\Delta_0$ it is also $\Sigma_0$ and $\Pi_0$
\item A formula $\varphi$ is $\Pi_n+1$ if it is logically equivalent to a formula $\varphi'$ such that $\varphi' = \forall x \psi$ where $\psi$ is a $\Sigma_n$–formula for any $n < \omega$.
\item A formula $\varphi$ is $\Sigma_n+1$ if it is logically equivalent to a formula $\varphi'$ such that $\varphi' = \forall x \psi$ where $\psi$ is a $\Pi_n$–formula for any $n < \omega$.
\ece
\end{definition} % TODO kolem ty parovaci fce je neco blbe :(
% Note that we can use the pairing function so that for $\forall p_1, \ldots, p_n \psi(p_1, \ldots, p_n)$, there is a logically equivalent formula of the form $\forall x \psi'(x)$.

\begin{lemma}{($\Delta_0$ absoluteness)}\label{lemma:delta_0_absoluteness}\\
Let $\varphi$ be a $\Delta_0$–formula, then $\varphi$ is absolute in any transitive class $M$.
\end{lemma}

\begin{proof}
This will be proved by induction over the complexity of a given $\Delta_0$–formula $\varphi$. Let $M$ be an arbitrary transitive class. 

As $M$ is transitive, atomic formulas are always absolute by the definition of relativisation, see definition \bref{def:relativization}.
Suppose that $\Delta_0$–formulas $\psi_1$ and $\psi_2$ are absolute in $M$. 
Then from relativization, $(\psi_1 \et \psi_2)^M \iff \psi_1^M \et \psi_2^M$, which is equivalent to $\psi_1 \et \psi_2$ from the induction hypothesis. 
The same holds for $\lor, \then$ and $\neg$.

Suppose that a $\Delta_0$–formula $\psi$ is absolute in $M$. Let $y$ be a set and let $\varphi = (\exists x \in y) \psi(x)$. 
From relativization, $(\exists x \psi(x))^M \iff (\exists x \in M) \psi^M(x)$. 
Since the induction hypothesis makes it clear that $\psi^M \iff \psi$, we get 
\beq
((\exists x \in y) \psi(x))^M \iff (\exists x \in y\cap M) \psi(x)^M \iff (\exists x \in y\cap M) \psi(x)\mbox{,}
\eeq
which is equivalent to $\varphi^M \iff \varphi$. Note that from transitivity of $M$, is $x \in M$ and $x \in y$, it is the case that $x \in y \cap M$. % urcite z tranzitivity?
The same argument applies to $\varphi = (\forall x \in y) \psi(x)$.
\end{proof}

% todo co Devlin –– p.27 –– downward a upward absoluteness? % nepouzivame
\begin{lemma}{(Downward Absoluteness)}\label{lemma:downward_absoluteness}\\
Let $\varphi$ be a $\Pi_1$–formula and $M$ a transitive class. Then the following holds:
\begin{equation}
(\forall p_1, \ldots, p_n \in M)(\varphi(p_1, \ldots, p_n) \then \varphi(p_1, \ldots, p_n)^M)
\end{equation}
\end{lemma}
\begin{proof}
Since $\varphi(p_1, \ldots, p_n)$ is $\Pi_1$, there is a $\Delta_0$–formula $\psi(p_1, \ldots, p_n, x)$ such that $\varphi = \forall x \psi(p_1, \ldots, p_n, x)$. 
From relativization and lemma \bref{lemma:delta_0_absoluteness},
\beq
\varphi^M(p_1, \ldots, p_n) \iff (\forall x \in M)\psi(p_1, \ldots, p_n, x)\mbox{.}
\eeq

Assume that for $p_1, \ldots, p_n \in M$ fixed, that $\forall x \psi(p_1, \ldots, p_n, x)$ holds, but $(\forall x \in M)\psi(p_1, \ldots, p_n, x)$ does not. 
Therefore $\exists x \neg \psi(p_1, \ldots, p_n, x)$, which contradicts $\forall x \psi(p_1, \ldots, p_n, x)$.
\end{proof}


\begin{lemma}{(Upward Absoluteness)}\label{lemma:upward_absoluteness}\\
Let $\varphi$ be a $\Sigma_1$–formula and $M$ a transitive class. Then the following holds:
\begin{equation}
(\forall p_1, \ldots, p_n \in M)(\varphi^M(p_1, \ldots, p_n) \then \varphi(p_1, \ldots, p_n))
\end{equation}
\end{lemma}
\begin{proof}
Since $\varphi(p_1, \ldots, p_n)$ is $\Sigma_1$, there is a $\Delta_0$–formula $\psi(p_1, \ldots, p_n, x)$ such that $\varphi = \exists x \psi(p_1, \ldots, p_n, x)$. 
From relativization and lemma \bref{lemma:delta_0_absoluteness},
\beq
\varphi^M(p_1, \ldots, p_n) \iff (\exists x \in M)\psi(p_1, \ldots, p_n, x)\mbox{.}
\eeq

Assume that for $p_1, \ldots, p_n \in M$ fixed, that $(\exists x \in M)\psi(p_1, \ldots, p_n, x)$ holds, but $\exists x \psi(p_1, \ldots, p_n, x)$ does not. This is an obvious contradiction.
\end{proof}


\subsubsection{More Functions}

\begin{definition}{(Strictly Increasing Function)}\label{def:increasing_function}\\
A function $f: Ord \then Ord$ is said to be \emph{strictly increasing} iff
\begin{equation}
(\forall \alpha, \beta \in Ord) (\alpha < \beta \then f(\alpha) < f(\beta)).
\end{equation}
\end{definition}

\begin{definition}{(Continuous Function)}\label{def:continuous_function}\\
A function $f: Ord \then Ord$ is said to be \emph{continuous} iff
\begin{equation}
\mbox{``}\lambda\mbox{ is limit'' } \then f(\lambda) = \bigcup_{\alpha < \lambda} f(\alpha).
\end{equation}
\end{definition}

\begin{definition}{(Normal Function)}\label{def:normal_function}\\
A function $f: Ord \then Ord$ is said to be \emph{normal} iff it is \emph{strictly increasing} and \emph{continuous}.
\end{definition}

\begin{definition}{(Fixed Point)}\label{def:fixed_point}\\
We say~$x$~is a fixed point of a function $f$ iff $x=f(x)$.
\end{definition}

\begin{definition}{(Unbounded Class)}\label{def:unbounded_class}\\
We say a class $A$ of ordinals is unbounded iff
\begin{equation}
\forall x (\exists y \in A) (x < y)\mbox{.}
\end{equation}
\end{definition}

\begin{definition}{(Limit Point)}\label{def:limit_point}\\
Given a class $x \subseteq Ord$, we say that $\alpha \neq \emptyset$ is a limit point of~$x$~iff 
\begin{equation}
\alpha = \bigcup(x \cap \alpha)
\end{equation}
\end{definition}

\begin{definition}{(Closed Class)}\label{def:closed_class}\\
We say a class $A \subseteq Ord$ is closed iff it contains all its limit points.
\end{definition}

\begin{definition}{(Club set)}\label{def:club_set}\\
For a regular uncountable cardinal $\kappa$, a set $x \subset \kappa$ is a \emph{closed unbounded} subset, abbreviated as a \emph{club set}, iff~$x$~is both closed and unbounded in $\kappa$.
\end{definition}

\begin{definition}{(Stationary set)}\label{def:stationary_set}\\
For a regular uncountable cardinal $\kappa$, we say a set $A \subset \kappa$ is stationary in $\kappa$ iff it intersects every club subset of $\kappa$.
\end{definition}

\subsubsection{Structure, Substructure and Embedding}

Structures will be denoted $\langle M, \in, R \rangle$ where $M$ is a domain, $\in$ stands for the standard membership relation, it is assumed to be restricted to the domain\footnote{To be totally explicit, we should write $\langle M, \in \cap M \times M, R \rangle$.}, $R \subseteq M$ is an unary relation on the domain. 

\begin{definition}{(Elementary Embedding)}\label{def:elementary_embedding}\\
Given the structures $\langle M_0, \in, R \rangle$, $\langle M_1, \in, R \rangle$ and a one–to–one function $j: M_0 \then M_1$, we say $j$ is an \emph{elementary embedding} of $M_0$ into $M_1$, we write $j: M_0 \prec M_1$, when the following holds for every formula $\varphi(p_1, \ldots, p_n)$ and every $p_1, \ldots, p_n \in M_0$:
\begin{equation}
\langle M_0, \in, R \rangle \models \varphi(p_1, \ldots, p_n) \iff \langle M_1, \in, R \rangle  \models \varphi(j(p_1), \ldots, j(p_n))
\end{equation}
\end{definition}


\begin{definition}{(Elementary Substructure)}\label{def:elementary_substructure}\\
Given the structures $\langle M_0, \in, R \rangle$, $\langle M_1, \in, R \rangle$ and a one–to–one function $j: M_0 \then M_1$ such that $j: M_0 \prec M_1$, we say that $M_0$ is an \emph{elementary substructure} of $M_1$, denoted as $M_0 \prec M_1$, iff $j$ is an identity on $M_0$. In other words
\begin{equation}
\langle M_0, \in, R \rangle \models \varphi(p_1, \ldots, p_n) \iff \langle M_1, \in, R \rangle  \models \varphi(p_1, \ldots, p_n)
\end{equation}
for $p_1, \ldots, p_n \in M_0$
\end{definition}


%\newpage

\section{Levy's First-Order Reflection}\label{sec:first_order}

\subsection{Lévy's Original Paper}\label{sec:levy1960}
This section is based on Lévy's paper \emph{Axiom Schemata of Strong Infinity in Axiomatic Set Theory}, \cite{Levy60a}. It presents Lévy's general reflection principle and its equivalence to \emph{Replacement} and \emph{Infinity} under $\sf{S}$\footnote{See definition (\ref{def:s}).}.

First, we should point out that set theory has changed over the last 66 years and show a few notable, albeit only formal, differences.
%When reading Lévy's article, one should bear in mind that while the author often speaks about a~model of $\sf{ZF}$, usually denoted $u$, it doesn't necessarily mean that there is a set $u$ that is a model of $\sf{ZF}$\footnote{This is indeed impossible to prove in $\sf{ZF}$ due to Gödel's Incompleteness.}, we are nowadays used to using the notion of universal class $V$ in similar sense, even though independently from a particular axiomatic set theory. %We will review the exact meaning of the notion of a standard complete model in a moment.
%The theory $\sf{ZF}$ is practically identical to the theory we have established in (\ref{def:zf}), the differences are only formal.
One might be confused by the fact that Lévy treats the \emph{Subsets} axiom, which we call \emph{Specification}, as a single axiom rather than a schema. He even takes the conjunction of all axioms of $\sf{ZF}$ and treats it like a formula. This is possible because the underlying logic calculus is different. Lévy works with set theories formulated in the \emph{non-simple applied first order functional calculus}, see beginning of \emph{Chapter IV} in \cite{church1996introduction} for details. For now, we only need to know that the calculus contains a substitution rule for functional variables. This way, \emph{Subsets} is de facto a schema even though it sometimes treated as a single formula\footnote{This way, the conjunction of all axioms is then in fact an axiom schema.} but the logic is still first-order since one can't quantify over functional variables. We will use the usual first-order axiomatization of \sf{ZFC} as seen on \cite{JechBook}.
% todo koukni do churche jak se to s tim ma, p.219
It should also be noted that the logical connectives look different. The now usual symbol for an universal quantifier does not appear, $\forall x \varphi (x)$ would be written as $(x) \varphi (x)$. The symbol for negation is ``$\sim$'', implication is written as ``$\supset$'' and equivalence is ``$\equiv$''. We will use standard notation with ``$\neg$'', ``$\then$'' and ``$\iff$'' respectively when presenting Lévy's results.

%The following definitions are not used in contemporary set theory, but they illustrate 1960's set theory mind-set and they are used heavily in Lévy's text, so we will include and explain them for clarity. 
%Generally in this chapter, $\sf{Q}$ stands for an arbitrary axiomatic set theory. % used for general definitions, $u$ is usually a model of $\sf{Q}$, counterpart of today's the universal class $V$.

\

This subsection uses $\sf{ZF}$ instead of the usual $\sf{ZFC}$ as the underlying theory. % neni to zbytecny?

\begin{definition}{(Standard Complete Model of a Set Theory)}\label{def:scm_q}\\
Let $\sf{Q}$ be an arbitrary axiomatic set theory.
We say that $u$ is a standard complete model of $\sf{Q}$ iff
\bce[(i)]
\item $(\forall \sigma \in \sf{Q})(\langle u, \in \rangle \models \sigma)$
\item $\forall y (y \in u \then y \subset u)$ ($u$ is transitive)
% \item $\forall e \langle x, y \rangle \in e \iff (y \in u \et x \in y)$ % co je $e$? omg je to jinak, je to model vsech axiomu pro vsechny $e$!!! relatiizace do u, e?
\ece 
We write $Scm^{\sf{Q}}(u)$.
\end{definition}

\begin{definition}{(Cardinals Inaccessible With Respect to $\sf{Q}$)}\label{def:levy_inaccessible_q}\\
Let $\sf{Q}$ be an arbitrary axiomatic set theory. We say that a cardinal $\kappa$ is inaccessible with respect to theory $\sf{Q}$ iff
\begin{equation}
Scm^{\sf{Q}}(V_\kappa)
\end{equation}
We write $In^{\sf{Q}}(\kappa)$.\footnote{To be able to define $V_\kappa$, we need to work in a logic that contains the \emph{Replacement Schema} or any of it's equivalents. It should be noted that we don't work in an arbitrary theory \sf{Q}, but in \sf{ZF}, which contains the \emph{Replacement Schema}. $Scm^{\sf{Q}}(V_\kappa)$ in fact says ``\sf{ZF} thinks that $V_\kappa$ is a transitive model of \sf{Q}''.}
\end{definition}

\begin{definition}{(Inaccessible Cardinal With Respect to $\sf{ZF}$)}\label{def:levy_inaccessible}\\
When a cardinal $\kappa$ is inaccessible with respect to $\sf{ZF}$, we only say that it is inaccessible. We write $In(\kappa)$.
\begin{equation}
In(\kappa) \iff In^{\sf{ZF}}(\kappa)
\end{equation}
\end{definition}
The above definition of inaccessibles is used because it doesn't require \emph{Choice}.

For the definition of relativization, see (\ref{def:relativization}). The notation used by Lévy is ``$Rel(u, \varphi)$'', we will stick to ``$\varphi^{u}$''.
\begin{definition}{($N$)}\label{def:levy_axiom_n}\\
The following is an axiom schema of complete reflection over $\sf{ZF}$, denoted $N$. For every first-order formula $\varphi$ in the language of set theory with no free variables except for $p_1, \ldots , p_n$, the following is an instance of schema $N$.
\begin{equation}
\exists u (Scm^{\sf{ZF}}(u) \et (\forall p_1, \ldots, p_n \in u)(\varphi \iff \varphi^{u}))
\end{equation}
%where $\varphi$ is a~formula which contains no free variables except for $p_1, \ldots , p_n$.
\end{definition}

\begin{definition}{(N')}\label{def:levy_axiom_n'}\\
For any first-order formulas $\varphi_1, \ldots, \varphi_m$ in the language of set theory with no free variables except for $p_1, \ldots , p_n$, the following is an instance of schema $N'$.
\begin{equation}
\exists u (z \in u \et Scm^{\sf{ZF}}(u) \et (\forall p_1, \ldots, p_n \in u)(\varphi_1 \iff \varphi_1^{u}) \et \ldots \et \varphi_m \iff \varphi_m^{u}))
\end{equation}
\end{definition}

\begin{definition}{(N')}\label{def:levy_axiom_n'}\\
For any first-order formulas $\varphi_1, \ldots, \varphi_m$ in the language of set theory with no free variables except for $p_1, \ldots , p_n$, the following is an instance of schema $N'$.
\begin{equation}
\exists u (Scm^{\sf{ZF}}(u) \et (\forall p_1, \ldots, p_n \in u)(\varphi_1 \iff \varphi_1^{u}) \et \ldots \et \varphi_m \iff \varphi_m^{u}))
\end{equation}
\end{definition}

Let $\sf{S}$ be an axiomatic set theory defined in (\ref{def:s}). 

This is \emph{Theorem 2} in \cite{Levy1960a}
\begin{lemma}{($N \iff N'' \iff N'$)}\label{lemma:n_iff_n'}\\
The schemas $N$, $N'$ and $N''$ are equivalent under \sf{S}.
\end{lemma}

\begin{proof}
We will execute this proof in the theory \sf{ZF}, but the reader should note that we are neither using \emph{Replacement} nor \emph{Infinity}, 
so for schemas similar to $N$, $N'$, $N''$ but with ``$Scm^{\sf{S}}(u)$'' instead of ``$Scm^{\sf{ZF}}(u)$'', the proof works equally well.

Clearly, $N' \then N'' \then N$. 

Now, assuming $N$ and given the formulas $\varphi_1, \ldots, \varphi_n$, we will prove $N''$. Consider the following formula:
\beq
\psi = \bigvee\limits_{i=1}^t t = i \et \varphi_i
\eeq
We will take advantage of the fact that natural numbers are defined by atomic formulas and therefore absolute in transitive structures. 
From $N$, we get such $u$ that $Scm^{\sf{ZF}}(u) \et (\forall p_1, \ldots, p_n \in u)(\bigvee\limits_{i=1}^t t = i \et \varphi_i \iff \bigvee\limits_{i=1}^t t = i \et \varphi_i^{u})$.
This already satisfies $N''$.

In order to prove $N'$ from $N''$, let's add two more formulas. Given $p_1, \ldots, p_n$, we denote
\beq
\varphi_{m+1} = \exists u (z \in u \et Scm^{\sf{ZF}}(u) \et (\forall p_1, \ldots, p_n \in u)(\bigvee\limits_{i=1}^m \varphi_i = \varphi_i^u))
\eeq
\beq
\varphi_{m+2}  = \forall z \varphi_{m+1}
\eeq
So, by $N''$, we have a set $u$ that satisfies $Scm^{\sf{ZF}}(u)$ as well as the following:
\beq
(\forall p_1, \ldots, p_n \in u)(\varphi_i \iff \varphi_i^u) \mbox{ for }1 \leq i \leq m \label{eq:levy_th_2_eq_3}
\eeq
\beq
z \in u \then \varphi_{m+1} \iff \varphi_{m+1}^u\label{eq:levy_th_2_eq_4}
\eeq
\beq
\varphi_{m+2} \iff \varphi_{m+2}^u\label{eq:levy_th_2_eq_5}
\eeq
By $Scm^{\sf{ZF}}(u)$ and (\ref{eq:levy_th_2_eq_3}), we get $(\forall z \in u) \varphi_{m+1}$, % watt? nemel bych reflektovat i $\varphi_{m+1}$ v eq:levy_th_2_eq_3 ?
so together with (\ref{eq:levy_th_2_eq_4}), we get $(\forall z \in u) \varphi_{m+1}^u$, exactly $\varphi_{m+2}^u$, so by (\ref{eq:levy_th_2_eq_5}) we get $\varphi_{m+2}$. 
But $\varphi_{m+2}$ is exactly the instance of $N'$ we were looking for.
\end{proof}

\begin{definition}{($N_0$)}\label{def:levy_axiom_n0}\\
Axiom schema $N_0$ is similar to $N$ defined above, but with $\sf{S}$ instead of $\sf{ZF}$. For every $\varphi$, a first-order fomula in the language of set theory with no free variables except $p_1, \ldots , p_n$, the following is an instance of $N_0$.
\begin{equation}
\exists u (Scm^{\sf{S}}(u) \et (\forall p_1, \ldots , p_n \in u)(\varphi \iff \varphi^{u}))
\end{equation}
%where $\varphi$ is a~formula which contains no free variables except for $p_1, \ldots , p_n$.
\end{definition}

We will now show that in $\sf{S}$, $N_0$ implies both \emph{Replacement} and \emph{Infinity}.

\

Let $N_0$ be defined as in (\ref{def:levy_axiom_n0}), for \emph{Infinity} see (\ref{def:infinity}).
\begin{theorem}\label{theorem:n0_implies_infinity}\
In $\sf{S}$, the axiom schema $N_0$ implies \emph{Infinity}.
\end{theorem}

\begin{proof} % Levy to bere pres V_\lambda \models 
Let $\varphi = \forall x \exists y (y = x \cup \{x\})$. 
This clearly holds in $\sf{S}$ because given a set $x$, there is a set $y = x \cup \{x\}$ obtained via \emph{Pairing} and \emph{Union}. %\emph{Powerset} and \emph{Specification}.
From $N_0$, there is a set $u$ such that $\varphi^{u}$ holds. % TODO upresnit
 This $u$ satisfies the conditions required by \emph{Infinity}.
\end{proof}

Lévy proves this theorem in a different way. He argues that for an arbitrary formula $\varphi$, $N_0$ gives us $\exists u Scm^{\sf{S}}(u)$ and this $u$ already satisfies \emph{Infinity}. 
To do this, we would need to prove lemma (\ref{lemma:scm_s_is_limit}) earlier on, we will do that later in this chapter.
%We would need to prove (\ref{lemma:scm_s_is_limit}), which will happen later in this chapter, but we don't know that yet. % mozna taky trochu preformulovat

\

Let $\sf{S}$ be a set theory defined in (\ref{def:s}), $N_0$ a schema defined in (\ref{def:levy_axiom_n0}) and \emph{Replacement} a schema defined in (\ref{def:replacement}).
\begin{theorem} 
%In $\sf{S}$ with $N_0$ implies \emph{Replacement}.\\
%TODO jedno nebo druhe % nebo treti
In $\sf{S}$, the axiom schema $N_0$ implies \emph{Replacement}.
%\beq
%\sf{S}, N_0 \vdash \mbox{\emph{Replacement}}
%\sf{S} \vdash N_0 \then \mbox{\emph{Replacement}}
%\eeq % nebo S \vdash N_0 \then Replacement % (veta o dedukci?) nebo \sf{S} + N_0 \vdash Repl
\end{theorem}
% todo zkontrolovat ========================================================================================== % TODO asi cely blbe? ...
\begin{proof}
Let $\varphi(x, y, p_1, \ldots, p_n)$ be a~formula with no free variables except $x, y, p_1, \ldots, p_n$.
%Out goal is to prove
%\beq
%``\varphi\mbox{ is a function}''\then \forall x \exists y \forall z (z \in y \iff (\exists q \in x)(\varphi(x, y, p_1, \ldots, p_n)))
%\eeq
Let a set $x$ be given and let $\chi$ be an instance of the \emph{Replacement} schema for the $\varphi$ given.
We want to verify in $\sf{S}$ that given a formula $\varphi$, the instance of $N_0$ for $\varphi$ implies $\chi$.
\begin{equation}
\begin{gathered}
\chi = \forall x', y', z(\varphi(x', y', p_1, \ldots, p_n) \et \varphi(x', z, p_1, \ldots, p_n) \then y' = z') \\
\then \exists y \forall z (z \in y \iff (\exists q \in x)(\varphi(x, y, p_1, \ldots, p_n)))
\end{gathered}
\end{equation}

Since it can be shown that $N_0$ is equivalent to $N'_0$ similar to $N'$ in lemma (\ref{lemma:n_iff_n'}), 
there is a set $u$ such that $Scm^{\sf{S}}(u)$, $x \in u$ and all of the following hold:
\bce[(i)]
\item $\varphi \iff \varphi^{u}$
\item $\exists y \varphi \iff (\exists y \varphi)^{u}$
\item $\chi \iff \chi^{u}$
% \item $\forall x, p_1, \ldots, p_n \chi \iff (\forall x, p_1, \ldots, p_n \chi)^{u}$.
\ece
%The above formulas are instances of the $N_0$ schema for $\varphi$, $\exists y \varphi$, $\chi$ and the universal closure of $\chi$ respectively.
%By $N_0$, there exists a set $u$ where all four formulas hold.\footnote{Despite the fact that $N_0$ is defined for one formula, we have just used it for four at once. To make this formally possible, we can either prove that $N_0$ is equivalent to a more general version for any finite number of formulas or we can reflect their conjunction and argue that if $\langle u, \in \rangle \models \varphi_1 \et \ldots \et \varphi_n$, then $(\langle u, \in \rangle \models \varphi_1), \ldots, (\langle u, \in \rangle \models \varphi_n)$.}
From relativization, $(\exists y \varphi)^{u}$ is equivalent to $(\exists y \in u) \varphi^{u}$, together with (i) and (ii), we get
\begin{equation}
(\exists y \in u)\varphi \iff \exists y \varphi
\end{equation}

If $\varphi$ is a~function, %then for every $x \in u$, which is also $x \subset u$ since $u$ is transitive from $Scm^{\sf{S}}(u)$,
it maps the elements of $x$, which are also elements of $u$ due to transitivity of $u$, to elements $u$. From \emph{Specification}, we can find $y$, a~set of all images of the elements of $x$ via $\varphi$. That means the following is an instance of \emph{Replacement} for $\varphi$ and $x$ and it holds in $u$, so we get $\chi^{u}$, which means that $\chi$ holds by (iii).
% \beq
% (\forall p_1, \ldots, p_n \in u) \chi\mbox{.}
% \eeq
% By (iii) this is equivalent to 
% \beq
% (\forall p_1, \ldots, p_n \in u) \chi^{u}\mbox{.}
% \eeq 
% which is exactly 
% \beq
% (\forall p_1, \ldots, p_n \chi)^{u}\mbox{.}
% \eeq
% From (iv), $\forall x, p_1, \ldots, p_n \chi$ holds. 
\end{proof} %  POZOR na X!

% ======================================================= % zkontrolovano sem!
What we have just proven is only a single theorem from Lévy's aforementioned article, we will introduce other interesting results, dealing with inaccessible and Mahlo cardinals, later in their appropriate context in chapter 3. % porad trochu kostrbaty?

% =====================================================================================================================================

\subsection{Contemporary Restatement}
We will now introduce and prove a theorem that is called Lévy's Reflection in contemporary set theory. The only difference is that while Lévy originally reflects a formula $\varphi$ from $V$ to a set $u$ which is a \emph{standard complete model of $\sf{S}$}, we say that there is a $V_\lambda$ for a limit $\lambda$ that reflects $\varphi$. Those two conditions are equivalent due to lemma (\ref{lemma:scm_s_is_limit}).

%\begin{definition}{(Reflection\textsubscript{1})}\label{def:reflection_1}\\ % co spis Levy's reflection principle?
%Let $\varphi(p_1, \ldots, p_n)$ be a first-order formula in the language of set theory. Than the following holds for any such $\varphi$.
%\begin{equation}
%\forall M_0 \exists M (M_0 \subseteq M \et (\varphi^M(p_1, \ldots, p_n) \iff \varphi(p_1, \ldots, p_n)))
%\end{equation}
%\end{definition}

% Note that this is a restatement of both Lévy's $N$ and $N_0$ from the previous chapter\footnote{see (\ref{def:levy_axiom_n}) and (\ref{def:levy_axiom_n0}}). We prefer to call it \emph{First-order reflection} so it complies with how other axioms and schemata are named. \footnote{We will not use the name $N_0$, because it might be confusing to work $N_0$ and $M_0$ where $M_0$ is a set and $N_0$ is an axiom schema.} 
% Note that the subscript 1 refers to the fact that $\varphi(p_1, \ldots, p_n)$ is a first-order formula, and since we're using the word ``reflection'' in less strict meaning throughout this thesis, distinguishing between the two just by using italic font face for the schema might cause confusion.

% We will now prove the equivalence of \emph{First-order reflection} with \emph{Replacement} and \emph{Infinity} in $\sf{S}$ in two parts. First, we will show that \emph{First-order reflection} is a theorem of $\sf{ZFC}$, then we shall show that the second implication, which proves \emph{Infinity} and \emph{Replacement} from \emph{First-order reflection}, also holds.

% The following lemma is usually done in two parts, the first being for one formula, the other for $n$ formulas. We will only state and prove the more general version.

\begin{lemma}\label{lemma:reflection_lemma}\
Let $\varphi_1, \ldots, \varphi_n$ be first-order formulas in the language of set theory, all with $m$ free variables
\footnote{For formulas with a different number of free variables, take for $m$ the highest number of parameters among those formulas. Add spare parameters to every formula that has less than $m$ parameters in a way that preserves the last parameter, which we will denote $x$.
E.g. let $\varphi'_i$ be the a~formula with $k$ parameters, $k < m$. Let us set $\varphi_i(p_1, \ldots, p_{m-1}, x) = \varphi'_i(p_1, \ldots, p_{k-1}, x)$, notice that the parameters $p_k, \ldots, p_{m-1}$ are not used.}.
\bce[(i)]
\item For each set $M_0$ there is such set $M$ that $M_0 \subset M$ and the following holds for every $i$, $1 \leq i \leq n$:
\begin{equation}\label{equation:refl_lemma_i}
\exists x \varphi_i(p_1, \ldots, p_{m-1}, x) \then (\exists x \in M) \varphi_i(p_1, \ldots, p_{m-1}, x)
\end{equation}
for every $p_1, \ldots, p_{m-1} \in M$.

\item Furthermore, there is a limit ordinal $\lambda$ such that $M_0 \subset V_\lambda$ and the following holds for each $i$, $1 \leq i \leq n$:
\begin{equation}\label{equation:refl_lemma_ii}
\exists x \varphi_i(p_1, \ldots, p_{m-1}, x) \then (\exists x \in V_\lambda) \varphi_i(p_1, \ldots, p_{m-1}, x)
\end{equation}
for every $p_1, \ldots, p_{m-1} \in M$.

\item Assuming \emph{Choice}, there is $M$, $M_0 \subset M$ such that (\ref{equation:refl_lemma_i}) holds for every $M,\ i \leq n$ and $|M| \leq |M_0| \cdot \aleph_0$.
\ece
\end{lemma}

\begin{proof}
We will simultaneously prove statements (i) and (ii), denoting $M^T$ the transitive set required by part (ii).
Steps in the construction of $M^T$ that are not explicitly included are equivalent to steps for $M$.

Let us first define an operation $H_i(p_1, \ldots, p_{m-1})$ that yields the set of $x$'s with minimal rank\footnote{Rank is defined in (\ref{def:rank})} satisfying $\varphi_i(p_1, \ldots, p_{m-1}, x)$ for $p_1, \ldots, p_{m-1}$ and for every $i$, $1 \leq i \leq n$.

\begin{equation}
H_i(p_1, \ldots, p_n) = \{x \in C_i: (\forall z \in C)(rank(x) \leq rank(z))\}
\end{equation}
for each $1 \leq i \leq n$, where
\begin{equation}
C_i = \{x: \varphi_i(p_1, \ldots, p_{m-1}, x)\} \mbox{ for $1 \leq i \leq n$}
\end{equation}

\

Next, let's construct $M$ from given $M_0$ by induction. 
\begin{equation}
M_{i+1} = M_i \cup \bigcup_{j=0}^{n} \bigcup \{H_j(p_1, \ldots, p_{m-1}): p_1, \ldots, p_{m-1} \in M_i\}
\end{equation}
In other words, in each step we include into the construction the elements satisfying $\varphi(p_1, \ldots, p_{m-1}, x)$ for $p_1, \ldots, p_{m-1}$ from the previous step.
For statement (ii), this is the only part that differs from (i). To end up with a transitive $M$, we need to extend every step to it's transitive closure transitive closure of $M_{i+1}$ from (i). In other words, let $\gamma$ be the smallest ordinal such that 
\begin{equation}
(M^T_i \cup \bigcup_{j=0}^{n} \{\bigcup\{H_j(p_1, \ldots, p_{m-1}): p_1, \ldots, p_{m-1} \in M_i\}\}) \subset V_\gamma
\end{equation}
Then the incremental step is
\begin{equation}
M^T_{i+1} = V_\gamma
\end{equation}
and the final $M$ is obtained by joining the previous steps.
\begin{equation}
M = \bigcup_{i=0}^{\infty} M_i, \mbox{  }M^T = \bigcup_{i=0}^{\infty} M^T_i = V_\lambda\mbox{ for some limit }\lambda\mbox{.}
\end{equation}

\

We have yet to finish part (iii).
Let's try to construct a~set $M'$ that satisfies the same conditions like $M$ but is kept as small as possible. Assuming the Axiom of Choice, we can modify the construction so that the cardinality of $M'$ is at most $|M_0| \cdot \aleph_0$. Note that the size of $M$ in the previous construction is determined by the size of $M_0$ and, most importantly, by the size of $H_i(p_1, \ldots, p_{m-1})$ for every $i$, $1 \leq i \leq n$ in individual iterations of the construction. Since (i) only ensures the existence of an $x$ that satisfies $\varphi_i(p_1, \ldots, p_{m-1}, x)$ for any $i$, $1 \leq i \leq n$, we only need to add one $x$ for every set of parameters but $H_i(u_1, \dots, u_{m-1})$ can be arbitrarily large. Let $F$ be a~choice function on $\power{M'}$. Also let $h_i(p_1, \ldots, p_{m-1}) = F(H_i(p_1, \ldots, p_{m-1}))$ for $i$, where $1 \leq i \leq n$, which means that $h$ is a~function that outputs an $x$ that satisfies $\varphi_i(p_1, \ldots, p_{m-1}, x)$ for $i$ such that $1 \leq i \leq n$ and has minimal rank among all such sets. The induction step needs to be redefined to
\begin{equation}
M'_{i+1} = M'_i \cup \bigcup_{j=0}^n \{ h_j(p_1, \ldots, p_{m-1}): p_1, \ldots, p_{m-1} \in M'_i \}
\end{equation}
This way, the amount of elements added to $M'_{i+1}$ in each step of the construction is the same as the amount of $m$-tuples of parameters that yielded elements not included in $M'_i$. It is easy to see that if $M_0$ is finite, $M'$ is countable because it was constructed as a countable union of sets that are themselves at most countable. If $M_0$ is countable or larger, the cardinality of $M'$ is equal to the cardinality of $M_0$.\footnote{It can not be smaller because $|M'_{i+1}| \geq |M'_i|$ for every $i$. It may not be significantly larger because the maximum of elements added is the number of $n$-tuples in $M'_i$, which is of the same cardinality as $M'_i$.}
Therefore $|M'| \leq |M_0| \cdot \aleph_0$
\end{proof}

\begin{theorem}{(Lévy's first-order reflection theorem)}\label{theorem:first_order_reflection}\\
Let $\varphi(p_1, \ldots, p_n)$ be a~first-order formula.
\bce[(i)]
\item For every set $M_0$ there exists a set $M$ such that $M_0 \subset M$ and the following holds:
\begin{equation}
\varphi^M(p_1, \ldots, p_n) \iff \varphi(p_1, \ldots, p_n)\label{equation:levy_theorem_i}
\end{equation}
for every $p_1, \ldots, p_n \in M$.

\item For every set $M_0$ there is a~transitive set $M$, $M_0 \subset M$ such that the following holds:
\begin{equation}
\varphi^M(p_1, \ldots, p_n) \iff \varphi(p_1, \ldots, p_n)
\end{equation}
for every $p_1, \ldots, p_n \in M$.

\item For every set $M_0$ there is a limit ordinal $\lambda$ such that $M_0 \subset V_{\lambda}$ and the following holds:
\begin{equation}
\varphi^{V_{\lambda}}(p_1, \ldots, p_n) \iff \varphi(p_1, \ldots, p_n)
\end{equation}
for every $p_1, \ldots, p_n \in M$.

\item Assuming \emph{Choice}, for every set $M_0$ there is $M$ such that $M_0 \subset M$ and $|M| \leq |M_0| \cdot \aleph_0$ and the following holds:
\begin{equation}
\varphi^M(p_1, \ldots, p_n) \iff \varphi(p_1, \ldots, p_n)
\end{equation}
for every $p_1, \ldots, p_n \in M$.
\ece
\end{theorem}

\begin{proof}
% CO ty parametry? \varphi(p_1, \ldots, p_n) nebo jen \varphi ?
%Before we start, note that the following holds for any set $M$ if $\varphi$ is an atomic formula(x1 \in x2 or x1 = x2), as a direct consequence of relativisation to $M, \in$\footnote{See (\ref{def:relativization}). Also note that this only holds for relativization to $M, \in$, not $M, E$ for arbitrary $E$.}. 
%\begin{equation}
%\varphi \iff \varphi^M
%\end{equation}
Let's now prove (i) for given $\varphi$ via induction by complexity. We can safely assume that $\varphi$ contains no quantifiers besides ``$\exists$'' and no logical connectives other than ``$\neg$'' and ``$\et$''.
Let $\varphi_1, \ldots, \varphi_n$ be all subformulas of $\varphi$. Then there is a set $M$, obtained by the means of lemma (\ref{lemma:reflection_lemma}), for all of the formulas $\varphi_1, \ldots, \varphi_n$. 

Let's first consider atomic formulas in the form of either $x_1 = x_2$ or $x_1 \in x_2$. % preformulovat trochu
It is clear from relativisation\footnote{See (\ref{def:relativization}). This only holds for relativization to $M, \in \cap M \times M$, not $M, R$ for an arbitrary $R$.} that (\ref{equation:levy_theorem_i}) holds for both cases, $(x_1 = x_2)^M \iff (x_1 = x_2)$ and $(x_1 \in x_2)^M \iff (x_1 \in x_2)$.

\

We now want to verify the inductive step. First, take $\varphi = \neg \varphi'$. From relativization, we get $(\neg \varphi')^M \iff \neg (\varphi'^M)$.
Because the induction hypothesis tells us that $\varphi'^M \iff \varphi'$, the following holds:
\begin{equation}
(\neg \varphi')^{M} \iff \neg (\varphi'^M) \iff \neg \varphi'
\end{equation}

The same holds for $\varphi = \varphi_1 \et \varphi_2$. From the induction hypothesis, we know that $\varphi_1^M \iff \varphi_1$ and $\varphi_2^M \iff \varphi_2$, which together with relativization for formulas in the form of $\varphi_1 \et \varphi_2$ gives us
\begin{equation}
(\varphi_1 \et \varphi_2)^M \iff \varphi_1^M \et \varphi_2^M \iff \varphi_1 \et \varphi_2
\end{equation}

\
% kde jsem tu najednou vzal parametry? TODO
Let's now examine the case when $\varphi = \exists x \varphi'(p_1, \ldots, p_n, x)$. The induction hypothesis tells us that $\varphi'^M(p_1, \ldots, p_n, x) \iff \varphi'(p_1, \ldots, p_n, x)$,
so, together with above lemma (\ref{lemma:reflection_lemma}), the following holds:
\begin{equation}
\begin{gathered}
\varphi(p_1, \ldots, p_n, x) \\
\iff \exists x \varphi'(p_1, \ldots, p_n, x) \\
\iff (\exists x \in M) \varphi'(p_1, \ldots, p_n, x) \\
\iff (\exists x \in M) \varphi'^M (p_1, \ldots, p_n, x) \\
\iff (\exists x \varphi'(p_1, \ldots, p_n, x))^M \\
\iff \varphi^M(p_1, \ldots, p_n, x)
\end{gathered}
\end{equation}
Which is what we wanted to prove for part (i). %\ref{equation:levy_theorem_i} holds for all subformulas $\varphi_1, \ldots, \varphi_n$ of a given formula $\varphi$.

\

%So far we have proven part $\bold{(i)}$ of this theorem for one formula $\varphi$. 
We now need to verify that the same holds for any finite number of formulas $\varphi_1, \ldots, \varphi_n$. 
This has in fact been already done since lemma (\ref{lemma:reflection_lemma}) gives us a set $M$ for any finite amount of formulas and given $M_0$. We can therefore find a set $M$ for the union of all of their subformulas. When we obtain such $M$, it should be clear that it also reflects every formula in $\varphi_1, \ldots, \varphi_n$.

\

Since $V_\lambda$ is a~transitive set, by proving $\bold{(iii)}$ we also satisfy $\bold{(ii)}$. To do so, we only need to look at part $\bold{(ii)}$ of lemma (\ref{lemma:reflection_lemma}). All of the above proof also holds for $M = V_lambda$. 

To finish part $\bold{(iv)}$, we take $M$ of size $\leq |M_0| \cdot \aleph_0$, which exists due to part $\bold{(iii)}$ of lemma (\ref{lemma:reflection_lemma}), the rest being identical.
\end{proof}

% TODO existuje jich dokonce club set!
% viz http://ozark.hendrix.edu/~yorgey/settheory/15-reflection-principle.pdf


\

Let $\sf{S}$ be a set theory defined in (\ref{def:s}), for $\sf{ZFC}$ see definition (\ref{def:zfc}).

% ACHTUNG
% Sm je definovany jako konjunkce vsech relativizovanych axiomu, takze to je kruh...
% presunout do contemporary restatement?
% viz Levy str. (224)
% todo s/u/V_lambda
% DRAKE!!! ch.3, ch.4 dukaz v_alfa models ZFC pro limitni alfa bez nekonecna etc
% Drake dokazuje reflexi v ch.3 par.6.3

The two following lemmas are based on \cite{DrakeBook}[Chapter 3, Theorem 1.2].
\begin{lemma}\label{lemma:extensionality_in_transitive} % Drake ch.3 Theorem 1.2
If $M$ is a transitive set, then $\langle M, \in \rangle \models \mbox{\emph{Extensionality}}$.
\end{lemma}

\begin{proof}
Given a transitive set $M$, we want to show that the following holds.
\beq
\langle M, \in \rangle \models \forall x, y (x = y \iff \forall z (z \in x \iff z \in y))
\eeq % TODO pozor na definici splnovani!
% From satisfaction, we get that for every $x$, $y$, the following holds % neni v definici splnovani, ze volny musej platit vsechny?
Given arbitrary $x, y \in M$, we want to prove that $\langle M, \in \rangle \models (x = y \iff \forall z (z \in x \iff z \in y))$.
This is equivalent to % ref na definici pravdy
$\langle M, \in \rangle \models x = y \mbox{ iff } \langle M, \in \rangle \models \forall z(z \in x \iff z \in y)$, 
which is the same as $x = y \mbox{ iff } \langle M, \in \rangle \models \forall z(z \in x \iff z \in y)$.

So all elements of $x$ are also elements of $y$ in $M$, and vice versa. Because $M$ is transitive, all elements of $x$ and $y$ are in $M$, so $\langle M, \in \rangle \models \forall z(z \in x \iff z \in y)$ holds iff $x$ and $y$ contain the same elements and are therefore equal.
\end{proof}

\begin{lemma}\label{lemma:foundation_in_transitive}
If $M$ is a transitive set, then $\langle M, \in \rangle \models \mbox{\emph{Foundation}}$.
\end{lemma}

\begin{proof}
We want to prove the following:
\beq
\langle M, \in \rangle \models \forall x (x \neq \emptyset \then (\exists y \in x) (x \cap y = \emptyset))
\eeq

Given an arbitrary non-empty $x \in M$ let's show that $\langle M, \in \rangle \models (\exists y \in x) (x \cap y = \emptyset)$.

Because $M$ is transitive, every element of $x$ is an element of $M$. Take for $y$ the element of $x$ with the lowest rank\footnote{Rank is defined in (\ref{def:rank}).}. It should be clear that there is no $z \in y$ such that $z \in x$, because then $rank(z) < rank(y)$, which would be a contradiction.
\end{proof}

Let $\sf{S}$ be a set theory as defined in (\ref{def:s}). 
\begin{lemma}\label{lemma:scm_s_is_limit}\
The following holds for every $\lambda$.
\begin{equation}
\mbox{``$\lambda$ is a limit ordinal''} \then V_\lambda \models \sf{S} % proc ne?
\end{equation}
\end{lemma}

\begin{proof}
Given an arbitrary limit ordinal $\lambda$, we will verify the axioms of \sf{S} one by one.
\bce[(i)]
\item \emph{The existence of a set} comes from the fact that $V_\lambda$ is a non-empty set because limit ordinal is non-zero by definition.
% Jech 12.10, 12.11

\item \emph{Extensionality} holds from (\ref{lemma:extensionality_in_transitive}).

\item \emph{Foundation} holds from (\ref{lemma:foundation_in_transitive}).

\item \emph{Union}:\\ % TODO kontrola % separatni lemma, ze plati v kazdem V_\alpha?
% (see (\ref{def:union}))
%\begin{equation}
%\forall x \exists y \forall z (z \in y \iff \exists q( z \in q \et q \in x))
%\end{equation}
Given any $x \in V_\lambda$, we want verify that $y = \bigcup x$ is also in $V_\lambda$. Note that $y = \bigcup x$ is a $\Delta_0$-formula.
\beq
y = \bigcup x \iff (\forall z \in y)(\exists q \in x) z \in q \et (\forall z \in x)(\forall q \in z) q \in y
\eeq
So by lemma (\ref{lemma:delta_0_absoluteness})
\beq
% y = \bigcup x \iff (y = \bigcup x)^{V_\lambda}
y = \bigcup x \iff V_\lambda \models y = \bigcup x
\eeq
% asi ok ^

\item \emph{Pairing}: \\ % TODO kontrola
% (see (\ref{def:pairing}))
%\beq
%\forall x, y \exists z \forall q (q \in z \iff q = x \lor q = y)
%\eeq
Given two sets $x, y \in V_\lambda$, we want to show that $z = \{x, y\}$ is also an element of $V_\lambda$.
\beq
z = \{x, y\} \iff x \in z \et y \in z \et (\forall q \in z)(q = x \lor q = y)
\eeq
So $z = \{x, y\}$ is a $\Delta_0$-formula, and thus by lemma (\ref{lemma:delta_0_absoluteness}) it holds that
\beq
%z = \{x, y\} \iff (z = \{x, y\})^{V_\lambda}
z = \{x, y\} \iff V_\lambda \models z = \{x, y\}
\eeq
% asi ok ^


\item \emph{Powerset}: \\
%\begin{equation}
%\forall x \exists y \forall z (z \subseteq x \iff z \in y)
%\end{equation}
Given any $x \in V_\lambda$, we want to make sure that $\power{x} \in V_\lambda$. Let $\varphi(y)$ denote the formula $y \in \power{x} \iff y \subset x$.
according to definition of subset (\ref{def:subset}), $y \subset x$ is $\Delta_0$, so for any given $x, y \in V_\lambda$, $y = \power{x} \iff V_\lambda \models y = \power{x}$.
Because $\lambda$ is limit and $rank(\power{x}) = rank(x)+1$, if $\power{x} \in V_\lambda$ for every $x \in V_\lambda$.
% cajk

\item \emph{Specification}: \\ %zkontrolovat, podle Draka, pripadne Jech 12.11
Given a first-order formula $\varphi$, we want to show the following:
\beq
V_\lambda \models \forall x, p_1, \ldots, p_n, \exists y \forall z (z \in y \iff z \in x \et \varphi(z, p_1, \ldots, p_n))
\eeq
Given any $x$ along with parameters $p_1, \ldots, p_n$ in $V_\lambda$, we set
\beq
y~=~\{z~\in~x~:~\varphi^{V_\lambda}(z, p_1, \ldots, p_n)\}
\eeq
From transitivity of $V_\lambda$ and the fact that $y \subset x$ and $x \in V_\lambda$, we know that $y \in V_\lambda$, 
so $V_\lambda \models \forall z (z \in y \iff z \in x \et \varphi(z, p_1, \dots, p_n))$.
\ece
\end{proof}

% The lemma we've just proven shows how important are the axioms of \emph{Infinity} and \emph{Replacement} for \sf{ZFC}. If we only add \emph{Infinity}, we don't get above $V_{\omega+\omega}$\footnote{TODO citation needed!}, so replacements if needed to 

%TODO citace! http://ozark.hendrix.edu/~yorgey/settheory/13-SI.pdf
% http://ozark.hendrix.edu/~yorgey/settheory/12-relative-consistency-2.pdf 
% -- jako konzistence ZF-reg -- ukazeme ze V jakozto sjednoveni V_alpha je model?

% TODO v kanamorim? co dukaz?
%\begin{definition}{(First-Order Reflection Schema)}\\ % wtf, pro jedu nebo kolik formuli?
%Let $\varphi_1, \ldots, \varphi_n$ be first-order formulas in the language of set theory.
%For each set $M_0$ there is such set $M$ that $M_0 \subset M$ and the following holds for every $i$, $1 \leq i \leq n$:
%\begin{equation}\label{equation:refl_lemma_i}
%\exists x \varphi_i(p_1, \ldots, p_{m-1}, x) \then (\exists x \in M) \varphi_i(p_1, \ldots, p_{m-1}, x)
%\end{equation}
%for every $p_1, \ldots, p_{m-1} \in M$.
%\end{definition}
% elegantnejsi by bylo formulovat to pro jednu formuli a pak ukazat, ze se to totez.

\begin{definition}{(First-Order Reflection Schema)}\label{def:first_order_reflection}\\ % Jednu
%Let $\varphi$ be a first-order formula in the language of set theory.
% and a set $M_0$ there is such set $M$ that $M_0 \subseteq M$ and the following holds for every $p_1, \ldots, p_n \in M$:
For every first-order formula $\varphi$, the following is an axiom:
\begin{equation}
\forall M_0 \exists M (M_0 \subseteq M \et (\varphi(p_1, \ldots, p_n) \iff \varphi(p_1, \ldots, p_n)^M)) % drive \langle M, \in \rangle \models \varphi, ale to vypada divne
\end{equation}
We will refer to this axiom schema as \emph{First-order reflection}.
\end{definition}

% TODO staci vzit uzaver obou fli a udelat konjunkci

% pozor na to, ze jsme dokazali zdanlive silnejsi tvrzeni pro n formuli a ted predpokladame jenom jednu.
% zkontroluj ze to tak je a napis ze silnejsi vysledek a slabsi predpoklad jsou cool, jsou ve skutecnostni ekvivalentni

Let \emph{Infinity} and \emph{Replacement} be as defined in (\ref{def:infinity}) and (\ref{def:replacement}) respectively.

\begin{theorem}\label{theorem:levy_equivalence_contemporary}
\emph{First-order reflection} is equivalent to \emph{Infinity} $ \et $ \emph{Replacement} under $\sf{S}$.
\end{theorem}
% lemma:model_of_s?
\begin{proof}
Since (\ref{theorem:first_order_reflection}) already gives us one side of the implication, we are only interested in showing the converse which we shall do in two parts:

$\bold{\emph{First-order reflection} \then \emph{Infinity}}$
This is done exactly like (\ref{theorem:n0_implies_infinity}). We pick for $\varphi$ the formula $(\forall y \in x)(y \cup \{y\} \in x)$, $M_0 = \{\emptyset\}$. From (\ref{def:first_order_reflection}), there is a set $M$ that satisfies $\varphi$, so there is an inductive set. We have picked $M_0$ so that $\emptyset \in M$ obviously holds and $M$ is the witness for 
\beq
\exists x(\emptyset \in x \et (\forall y \in x)(y \cup \{y\} \in x))
\eeq
which is exactly (\ref{def:infinity}).

\

$\bold{\emph{First-order reflection} \then \emph{Replacement}}$

%Given a~formula $\varphi(x, y, p_1, \ldots, p_n)$, we can suppose that if it holds for given $x, y, p_1, \ldots, p_n$, it is reflected in a set $M$ \footnote{Which means that for $x, y, p_1, \ldots, p_n \in M$, $\varphi^M(x, y, p_1, \ldots, p_n) \iff \varphi(x, y, p_1, \ldots, p_n)$.}
%What we want to obtain is the following:
%\begin{equation}
%\begin{gathered}
%\forall x, y, z (\varphi(x, y, p_1, \ldots, p_n) \et \varphi(x, z, p_1, \ldots, p_n) \then y = z) \then\\
%\then \forall X \exists Y \forall y\ (y \in Y \iff \exists x (\varphi(x, y, p_1, \ldots, p_n) \et x \in X ))
%\end{gathered}
%\end{equation}

%TODO OMG FIX! Drake nebo jech nebo Kanamori!
Let's first point out that while \emph{First-order reflection} gives us a set for one formula, we can generalize it to hold for any finite number of formulas. We will show how is it done for two formulas, which is what we will use in this proof. Given two first-order formulas $\varphi, \psi$, we can suppose that there are formulas $\varphi'$ and $\psi'$ that are equivalent to $\varphi$ and $\psi$ respectively, but their free variables are different \footnote{This is plausible since we can for example substitute all free variables in $\varphi'$ for $x_0, x_2, x_4, \ldots$ and use $x_1, x_3, x_5, \ldots$ for free variables in $\psi'$, the resulting formulas will be equivalent.}. Let $\xi = \varphi \et \psi$, given any $M_0$, we can find a $M$ such that $\xi \iff \xi^M$. It is easy to see that from relativisation, the following holds:
\beq
\varphi \et \psi \iff \varphi' \et \psi' \iff \xi \iff \xi^M \iff (\varphi' \et \psi')^M \iff \varphi'^M \et \psi'^M \iff \varphi^M \et \psi^M
\eeq

Now given a function $\varphi(x, y)$, we know from \emph{First-order reflection} that for every $M_0$, there is a set $M$ such that $M_0 \subseteq M$ and both
\beq
(\forall x,y \in M)(\varphi(x, y) \iff \varphi^M(x, y))
\eeq 
and
\beq
(\forall x, y \in M)(\exists y \varphi(x, y) \iff (\exists y \varphi(x, y))^M)
\eeq 
hold, the latter being equivalent to 
\beq
(\forall x, y \in M)(\exists y \varphi(x, y) \iff (\exists y \in M) \varphi^M(x, y))
\eeq
Therefore 
\beq
(\forall x, y \in M)(\exists y \varphi(x, y) \iff (\exists y \in M) \varphi(x, y))
\eeq
holds too.
That means that we have a set $M$ such that for every $x \in M$, if $\varphi$ is defined for $x$, $(\exists y \in M) \varphi(x, y)$. 

To show that \emph{Replacement} holds for this particular $\varphi$, we need to verify that given a set $M_0$, $M'_0 = \{ y : (\exists x \in M_0) \varphi(x, y)\}$ is also a set. But since $M_0 \subseteq M$ and because given any $x \in M$, there is $y \in M$ satisfying $\varphi(x, y)$, the following is a set due to \emph{Specification}:
\beq
M'_0 = \{ y : (\exists x \in M_0) \varphi(x, y)\} = \{ y \in M : (\exists x \in M_0) \varphi(x, y)\}
\eeq
% tranzitivita? V_\lambda?

%We also know that $x, y \in M$, in other words for every $X$, $Y = \{y : \varphi(x, y, p_1, \ldots, p_n)\}$ and we know that $X \subset M$ and $Y \subset M$, which, together with the specification schema implies that $Y$, the image of $X$ over $\varphi$, is a~set.
\end{proof}

\

% Stronger reflection: 
% a: any number of formulas
% b: there is a club set of $M$s.
% asi do dalsi kapitoly

% co s tema $\emph{Reflection}$ / reflection / Reflection ??
We have shown that $\emph{Reflection}$ for first-order formulas, \emph{First-order reflection} is a~theorem of $\sf{ZFC}$.%, which means that it won't yield us any large cardinals. 
We have also shown that it can be used instead of the \emph{Infinity} and \emph{Replacement} scheme, but $\sf{ZFC}\ +\ \emph{First-order reflection}$ is a~conservative extension of $\sf{ZF}$. Besides being a~starting point for more general and powerful statements, it can be used to show that $\sf{ZF}$ is not finitely axiomatizable. This follows from the fact that \emph{Reflection} gives a~model to any consistent finite set of formulas. % or their conjunction? % nedokazal jsem verzi pro n formuli!
So if $\varphi_1, \ldots, \varphi_n$ would be the axioms of $\sf{ZFC}$, $\emph{Reflection}$ would prove that every model of $\sf{ZFC}$ contains a smaller model of $\sf{ZFC}$, which would in turn contradict the Second Gödel's Theorem\footnote{See chapter \ref{section:inaccessibility} for further details.}.

It is also worthwhile to note that, in a~way, Reflection is dual to compactness. 
% ref http://www.helsinki.fi/sls2015/materials/Fontanella%20Scandinavian%20Summer%20School.pdf
Compactness says that given a set of sentences, if every finite subset yields a~model, so does the whole set. Reflection, on the other hand, says that while the whole set has no model in the underlying theory, every finite subset has a model.

Furthemore, $\emph{Reflection}$ can be used in ways similar to upward Löwenheim–Skolem theorem.
Since Reflection extends any set $M_0$ into a~model of given formulas $\varphi_1, \ldots, \varphi_n$, we can choose the lower bound of the size of $M$ by appropriately choosing $M_0$.

In the next section, we will try to generalize \emph{Reflection} in a~way that transcends $\sf{ZF}$ and yields some large cardinals.
\newpage
% =============================================================

% \section{Reflection And Large Cardinals}

\begin{comment} % ======================================================================== //
% TODO aspon par slov !!!


% TODO druhoradova reflexe - odkaz na ORP_final (Koellner?) Co shapiro?
% TODO Tate?
% TODO Pro vyssi rady odkaz na Welche, \cite{Welch12globalreflection}

In this chapter we aim to examine stronger reflection properties in order to reach cardinals unavailable in $\sf{ZFC}$. Like we said in the first chapter, 
the variety of reflection principles comes from the fact that there are many way to formalise ``properties of the universal class''. It is not always obvious what properties hold for $V$ because, as Tarski
has shown, there is no way to formalise satisfaction for proper classes. We have shown that reflecting properties as first–order formulas doesn't allow us to leave $\sf{ZFC}$. We will broaden the class of admissible properties to be reflected and see whether there is a~natural limit in the height or width on the reflected universe and also see that no matter how far we go, the universal class is still as elusive as it is when seen from $\sf{S}$. That is because for every process for obtaining larger sets such as for example the power set operation in $\sf{ZFC}$, this process can't reach $V$ and thus, from reflection, there is an initial segment of $V$ that can't be reached via said process.

To see why this is important, let's dedicate a few lines to the intuition behind the notions of limitness, regularity and inaccessibility in a manner strongly influenced by \cite{Infinity_in_mind}. To see why limit and strongly limit cardinals are worth mentioning, note that they are ``limit'' not only in a sense of being a supremum of an ordinal sequence, they also show that a certain way of obtaining larger sets from smaller ones is limited. We will see that all of the alternatives offered in this thesis are in a sense limited. 
$\aleph_\lambda$ is a limit cardinal if there is no $\alpha$ such that $\aleph_{\alpha+1}=\aleph_\lambda$. Strongly limit cardinals point to the limits of the power set operation. It has been too obvious so far, so let's look at the regular cardinals in this manner. Regular cardinals are those that cannot be\footnote{Assuming the $\emph{Axiom of Choice}$.}, expressed as a supremum of smaller amount of smaller objects\footnote{Just like $\omega$ can not be expressed as a supremum of a finite set consisting solely of finite numbers.}. More precisely, $\kappa$ is regular if there is no way to define it as a union of less than $\kappa$ ordinals, all smaller than $\kappa$. So unless there already is a set of size $\kappa$, \emph{Replacement} is useless in determining whether $\kappa$ is really a set. Note that assuming the \emph{Axiom of Choice}, successor cardinals are always regular, so most\footnote{All provable to exist in $\sf{ZFC}$.} limit cardinals are singular cardinals. So if one is traversing the class of all cardinals upwards, successor steps are still sets thanks to the power set axiom while singular limit cardinals are not proper classes because they are suprema of images of smaller sets via \emph{Replacement}. Regular cardinals are, in a way, limits of how far can we get by taking limits of increasing sequences of ordinals obtained via $\emph{Replacement}$. 

In order to reach an inaccessible cardinal of size $\kappa$, one has to pass at least $\kappa$ limit ordinals. Them, to get to a Mahlo cardinal of size $\kappa$, one has to move past $\kappa$ inaccessible cardinals. This concept is then iterable for hyper–Mahlo cardinals, as we will see later in this section.

% That all being said, it is easy to see that no cardinals in $\sf{ZFC}$ are both strongly limit and regular because there is no way to ensure they are sets and not proper classes in $\sf{ZFC}$. The only exception to this rule is $\aleph_0$ which needs \emph{Infinity} to exist. % nase otazka je: proc omega a ne jine kardinaly?
% It should now be obvious why the fact that $\kappa$ is inaccessible implies that $\kappa = aleph_\kappa$.\footnote{This doesn't work backwards, the least fixed point of the $\aleph$ function is the limit of $\{\aleph_0,\ \aleph_{\aleph_0},\ \aleph_{\aleph_{\aleph_0}},\ \ldots \}$, it is singular since the sequence has countably many elements.}

% We will first examine the connection between reflection principles and (regular) fixed points of ordinal functions in a manner proposed by Lévy in \cite{Levy60a}. %We will also see that, like Lévy has proposed in the same paper, there is a meaningful way to extend the relation between $\sf{S}$ and $\sf{ZFC}$ into a hierarchy of stronger axiomatic set theories. 
% Those are the three lines of thinking that we will find are in fact different facets of the same gem, especially in the section devoted to Inaccessible and Mahlo cardinals.
% viz Shapiro, Stewart. 1987. “Principles of Reflection and Second–order Logic”. Journal of Philosophical Logic 16 (3). Springer: 309–33. http://www.jstor.org/stable/30227043.
% Reflections on \emph{Replacement} and Reflection: The axioms in a~structuralist setting (Geoffrey Hellman)
%TODO neco o tom, ze kdyz je reflexe formule, da se sama reflektovat?
% The above should make a clear picture of why $\emph{Infinity}$ is a specific case of $\emph{Reflection}$.
%TODO proc je Refl zaroven zobecneny replacement?

% TODO ze ``uplne totalni'' reflexe se zacykli a rozbije? nebo ne?


\end{comment} % ======================================================================== //

\subsection{Regular Fixed–Point Axioms}\label{sec:regular_fixed_points}
Lévy's article mentions various schemata that are not instances of reflection per se, but deal with fixed points of normal ordinal functions. 
After proving a helpful lemma, we will introduce them and show that they are equivalent to \emph{First–Order Reflection Schema}\footnote{For the definition, see definition \bref{def:first_order_reflection_schema}.}.

% 
%
% This small chapter is dedicated to ?
%

\begin{lemma}{(Fixed–Point Lemma for Normal Functions)}\label{lemma:normal_fixed_point}\\
Let $f$ be a normal function defined for all ordinals\footnote{For the definition of normal function, see definition \bref{def:normal_function}.}. Then all of the following hold:
\bce[(i)]
\item $\forall \lambda(\mbox{``$\lambda$ is a limit ordinal''} \then \mbox{``f($\lambda$) is a limit ordinal''})$
\item $\forall \alpha (\alpha \leq f(\alpha))$
\item $\forall \alpha \exists \beta (\alpha < \beta \et f(\beta) = \beta)$
\item The fixed points of $f$ form a closed unbounded class.\footnote{See definition \bref{def:closed_class} for the definition of a closed class, definition \bref{def:unbounded_class} for the definition of unboundedness.}
\ece
\end{lemma}

\begin{proof}
Let $f$ be a normal function defined for all ordinals.
\bce[(i)]
\item
Suppose $\lambda$ is a limit ordinal. 
For an arbitrary ordinal $\alpha < \lambda$, the fact that $f$ is strictly increasing means that $f(\alpha) < f(\lambda)$ and for any ordinal $\beta$, 
satisfying $\alpha < \beta < \lambda$, $f(\alpha) < f(\beta) < f(\lambda)$. 
We know that there is such $\beta$ from limitness of $\lambda$.
Because $f$ is continuous and $\lambda$ is limit, $f(\lambda) = \bigcup_{\gamma < \lambda} f(\gamma)$.% and since $\beta < \lambda$, $f(\beta) < f(\lambda)$. 
Therefore $\lambda$ is limit, so is $f(\lambda)$.
%So we have found $f(\beta)$ such that $f(\alpha) < f(\beta) < f(\lambda)$, therefore $f(\lambda)$ is a limit ordinal.\\

\item This step will be proved using the transfinite induction.
Since $f$ is defined for all ordinals, there is an ordinal $\alpha$ such that $f(\emptyset) = \alpha$ and because $\emptyset$ is the least ordinal, (ii) holds for $\emptyset$.

Suppose (ii) holds for some $\beta$ from the induction hypothesis. It then holds for $\beta+1$ because $f$ is strictly increasing. 

For a limit ordinal $\lambda$, suppose (ii) holds for every $\alpha < \lambda$. (i) implies that $f(\lambda)$ is also limit, 
so there is a strictly increasing $\kappa$–sequence $\langle \alpha_0, \alpha_1, \ldots \rangle$ for some $\kappa$ such that $\lambda = \bigcup_{i<\kappa} \alpha_i$. Because $f$ is strictly increasing, the $\kappa$–sequence $\la f(\alpha_0), f(\alpha_1), \ldots \ra$ is also strictly increasing, in then holds from the induction hypothesis that $\alpha_i \leq f(\alpha_i)$ for each $i \leq \kappa$. 
Thus, $\lambda \leq f(\lambda)$.

\item For an arbitrary $\alpha$, let there be an $\omega$–sequence $\langle \alpha_0, \alpha_1, \ldots \rangle$, 
such that $\alpha_0 = \alpha$ and $\alpha_{i+1} = f(\alpha_i)$ for each $i < \omega$.
This sequence is stricly increasing because so is $f$. 
Now, there's a limit ordinal $\beta = \bigcup_{i < \omega} \alpha_i$, we want to show that this is a fixed point of $f$. 
Because $f$~is continuous,
\beq
f(\beta) = f(\bigcup_{i < \omega} \alpha_i) = \bigcup_{i < \omega} f(\alpha)\mbox{.}
\eeq 
We have defined the above sequence so that 
\beq
f(\beta) = \bigcup_{i < \omega} f(\alpha) = \bigcup_{i < \omega} \alpha_{i+1}\mbox{,}
\eeq 
which means we are done, since 
\beq
\bigcup_{i < \omega} \alpha_{i+1} = \bigcup_{i < \omega} \alpha_{i}  = \beta\mbox{.}
\eeq

% Todo http://math.stackexchange.com/questions/1865519/when-is-the-union-of-a-set-of-ordinals-a-limit-ordinal/1865527#1865527
\item The class of fixed points of $f$ is obviously unbounded because in (iii), we start with an arbitrary ordinal.
It remains to show that it is closed, this is based on \cite{DrakeBook}, \emph{chapter 4}. 
Let $Y$ be a non–empty set of fixed points of $f$ such that $\bigcup Y \not\in Y$. Since $f$ is defined on ordinals, $Y$ is a set of ordinals, so $\bigcup Y$ is an ordinal.
$\bigcup Y$ is a limit ordinal. 
If it were a successor ordinal, suppose that $\alpha+1 = \bigcup Y$, then $\alpha \in \bigcup Y$, which would mean that there is some~$x$~such that $\alpha \in x \in Y$. 
But the least such~$x$~is $\alpha+1$, so $\bigcup Y~\in~Y$.
%We will show that $\bigcup Y$ is a limit ordinal because $Y$ doesn't have a maximal element.

Note that $\alpha < \bigcup Y \mbox{ iff } \exists \xi \in Y (\alpha < \xi)$. Since $f$ is defined for all ordinals and $\bigcup Y$ is a limit ordinal, $f(\bigcup Y) = \bigcup_{\alpha \in Y} f(\alpha)$, but because $Y$ is a set of fixed points of $f$,
\beq
f(\bigcup Y) = \bigcup_{\alpha \in Y} f(\alpha) = \bigcup Y\mbox{,}
\eeq
so $\bigcup Y$ is a limit point of $Y$.
\ece
\end{proof}

\begin{lemma}\label{lemma:normal_enumerates_club}\
Let $\alpha$ be a limit ordinal. Then the following hold:
%\bce[(i)] % TODO pozor na club / ``closed unbounded''
%\item 
If $C$ is a club subset of $\alpha$, then there is an ordinal $\beta$ and a normal function $f: \beta \then \alpha$ such that $rng(f) = C$. We say that $f$ \emph{enumrates} $C$.
%\item If $\beta$ is an ordinal and $f$ is a normal function such that $f: \beta \then \alpha$ and $rng(f)$ is unbounded in $\alpha$, then $rng(f)$ is a closed unbouded set in $\alpha$.
%\ece
\end{lemma}

This proof in inspired by \cite{MonkGradsets09}.

\begin{proof}
%\bce[(i)]
%\item 
Let $\beta$ be the order–type\footnote{See definition \bref{def:order_type}.} of $C$ and let $f$ be the isomorphism from $\beta$ onto $C$. Since $C \subseteq \alpha$, $f$ is an increasing function from $\beta$ into $\alpha$. To show that $f$ is continuous, let $\gamma$ be a limit ordinal below $\beta$, let $\epsilon = \bigcup_{\delta<\gamma} f(\delta)$. We want to verify that $f(\gamma) = \epsilon$. Since $\epsilon$ is a limit ordinal, we only need to show that $C \cap \epsilon$ is inbounded in $\epsilon$.

Take $\zeta < \epsilon$. Then there is a $\delta < \gamma$ such that $\zeta < f(\delta)$. 
Since $\gamma$ is limit, $\delta + 1 < \gamma$ and also $f(\delta + 1) < f(\gamma)$, we know that $f(\delta) \in C \cap \epsilon$. 
But that means that $C \cap \epsilon$ is unbounded in $\epsilon$, so $\epsilon \in C$. We have also shown that $\epsilon$ is closed unbounded in the image of $\gamma$ over $f$.
Therefore, $f(\gamma) = \epsilon = \bigcup_{\delta < \gamma} f(\delta)$, so $f$ is normal.
% \ece
\end{proof}

It should be clear that while this lemma works with club subsets of an ordinal, we can formulate analogous statement for club classes, which then yields a normal function defined for all ordinals, with the only exception that there is no such $\beta$ is an the beginning of the above proof because $f$ is then a function from $Ord$ to $Ord$ and proper classes have no order–type.

% TODO lemma ze limity tvori club?
% http://math.stackexchange.com/questions/109292/the-set-of-limit-points-of-an-unbounded-set-of-ordinals-is-closed-unbounded
% Lévy proposes in \cite{Levy60a} those axioms as equivalent to \emph{Reflection\textsubscript{1}}.

\begin{definition}{(\emph{Axiom Schema $M$\textsubscript{1}})}\label{def:levy_m}\\
``Every normal function defined for all ordinals has at least one inaccessible number in its range.''
\end{definition}
Lévy uses ``$M$'' to refer to this axiom but since we also use ``$M$'' for sets and models, for example in definition \bref{theorem:first_order_reflection}, we will call the above axiom ``\emph{Axiom Schema $M$\textsubscript{1}}'' to avoid confusion.

%Now we will express \emph{Axiom $M$\textsubscript{1}} as a formula to make it clear that it is an axiom scheme and the same can be done with \emph{Axiom $M$\textsubscript{2}} as well as \emph{Axiom Schema $M$} introduced immediately afterwards. Since it is an axiom schema and we will later dive into second–order logic, we may also want to refer to \emph{Axiom $M$\textsubscript{2}} as opposed \emph{Axiom $M$\textsubscript{1}}, the former being a single second–order sentence obtained by the obvious modification of \emph{Axiom $M$\textsubscript{1}}.\footnote{Second–order set theory will be introduced in the next subsection.}

In order to be able to meaningfully work with this schema, we must clarify what it actually states. 
Because we are working in first–order logic, and a \emph{normal function defined for all ordinals} is a proper class, we can not quantify over functions that are not sets. 
Instead, we will think of \emph{Axiom Schema $M$\textsubscript{1}} as schema that, given a formula $\varphi$, states % By ``every normal function defined for all ordinals has at least one inaccessible number in its range'', we mean the axiom schema that yields 
``If $\varphi$ is a normal function defined for all ordinals, then $\varphi$ has at least one inaccessible number in its range''%
\footnote{More formally, let $\varphi(x, y, p_1, \ldots, p_n)$ be a first–order formula with no free variables besides $x, y, p_1, \ldots, p_n$. The following is equivalent to \emph{Axiom $M$\textsubscript{1}}.
\begin{equation}
\begin{gathered}
\mbox{``$\varphi$ is a normal function''} \et \forall x (x \in Ord \then \exists y(\varphi(x, y, p_1, \ldots, p_n))) \then\\
\then \exists y (\exists x \varphi(x, y, p_1, \ldots, p_n) \et cf(y) = y \et (\forall x \in \kappa)(\exists y \in \kappa)(x > y))
\end{gathered}
\end{equation}}.
We will approach the following two axiom schemata in a similar manner.

\begin{definition}{(Axiom Schema $M$\textsubscript{2})}\\
``Every normal function defined for all ordinals has at least one fixed point which is inaccessible.''
\end{definition}

\begin{definition}{(Axiom Schema $M$\textsubscript{3})}\\
``Every normal function defined for all ordinals has arbitrarily great fixed points which are inaccessible.''
\end{definition}

Similar axiom is proposed in \cite{DrakeBook}.
\begin{definition}{(Axiom Schema $F$)}\label{def:axiom_f}\\
``Every normal function has a regular fixed point.''
\end{definition}

\begin{lemma}\label{lemma:limit_fixed_normal_function}
Let $f$ be a normal function defined for all ordinals.
\bce[(i)]
\item There is a is normal function $g_1$ defined for all ordinals that enumerates the class $\{\alpha : f(\alpha) = \alpha \}$.
\item There is a is normal function $g_2$ defined for all ordinals that enumerates the class $\{ \lambda : \mbox{``$f(\lambda)$ is a strong limit cardinal.''}\}$. 
\ece
\end{lemma}

\begin{proof}
We know that (ii) holds from lemma \bref{lemma:normal_fixed_point} and lemma \bref{lemma:normal_enumerates_club}.

Clearly, there is no largest strong limit ordinal $\nu$, because the limit of\\
$\la \nu, \power{\nu}, \power{\power{\nu}}, \ldots \ra$ is again a limit ordinal. % pozor na strong limit
The class of strong limit ordinals is closed because a limit of strong limit ordinals of is always a strong limit ordinal.
Let $h$ be a function enumerating limit ordinals that exists from lemma \bref{lemma:normal_enumerates_club}.
Then $g_1(\alpha) = f(h(\alpha))$ for every ordinal $\alpha$ is normal and defined for all ordinals.
\end{proof}

The following is \emph{Theorem 1} in \cite{Levy60a}, the parts dealing with \emph{Axiom Schema $F$} come from \cite{DrakeBook}.

\begin{theorem}
The following are all equivalent:
\bce[(i)]
\item \emph{Axiom Schema $M$\textsubscript{1}},
\item \emph{Axiom Schema $M$\textsubscript{2}},
\item \emph{Axiom Schema $M$\textsubscript{3}},
\item \emph{Axiom Schema $F$}.
\ece
\end{theorem}

\begin{proof}
It is clear that \emph{Axiom Schema $M$\textsubscript{3}} is a stronger version of \emph{Axiom Schema $M$\textsubscript{2}}, which is in turn a stronger version of both \emph{Axiom Schema $M$\textsubscript{1}} and \emph{Axiom Schema $F$\textsubscript{1}}. 

We will now prove that \emph{Axiom Schema $F$} $\then$ \emph{Axiom Schema $M$\textsubscript{2}}. 
Lemma \bref{lemma:limit_fixed_normal_function} tells us that given a normal function $f$ defined for all ordinals, 
there is a normal function $g_1$ defined for all ordinals that enumerates the fixed points of $f$. 
There is also a function $g_2$ that enumerates the strong limit ordinals in $rng(f)$.
By \emph{Axiom Schema $F$}, $g_2$ has a regular fixed point $\kappa$, which is also a strong limit ordinal, so 
\beq
f(\kappa) = g_2(\kappa) = \kappa \mbox{ and $\kappa$ is inaccessible.} 
\eeq
So every normal function defined for all ordinals has a regular fixed point.

We have yet to show that \emph{Axiom Schema $M$\textsubscript{1}} $\then$ \emph{Axiom Schema $M$\textsubscript{3}}. Again by lemma \bref{lemma:limit_fixed_normal_function}, there is a normal function $g$ defined for all ordinals that enumerates the fixed points of $f$. Let $h_\alpha(\beta) = g(\alpha+\beta)$ for any given ordinal $\alpha$, then $h_\alpha$~is a~normal function defined for all ordinals.
Then, given an arbitrary $\alpha$, from \emph{Axiom Schema $M$\textsubscript{1}}, there is a $\beta$ such that $\gamma = h_\alpha(\beta)$ is inaccessible. 
Because $\gamma = g(\alpha+\beta)$, thus $f(\gamma) = \gamma$. 
Since $\alpha \leq f'(\alpha)$ for any ordinal $\alpha$ and any normal function $f'$, we know that $\alpha \leq \alpha + \gamma \leq \gamma$, so $\gamma$ is inaccessible and arbitrarily large, depending on the choice of $\alpha$.
\end{proof}

To see how those schemata relate to reflection, let's introduce a stronger version of \emph{First–Order Reflection Schema}\footnote{See definition \bref{def:first_order_reflection_schema}.} from the previous chapter. 
But in order to do this, we must establish the inaccessible cardinal first.

% zkontorluj jeslti jsme to dokazali
% pak si rekneme, ze jsme to dokazali abychom videli ze existence nedosazitelneho kardinalu neni dokazatelna v ZFC
\subsection{Inaccessible Cardinal}\label{sec:inaccessible}
\begin{definition}
An uncountable cardinal $\kappa$ is \emph{inaccessible} iff it is \emph{regular} and \emph{strongly limit}. We write $In(\kappa)$ to say that $\kappa$ is an inaccessible cardinal.
\end{definition}

An uncountable cardinal that is regular and limit is called a \emph{weakly inaccessible cardinal}, we will only use the (strongly) inaccessible cardinal, but most of the results are similar for weakly inaccessibles, including higher types of ordinals that will be presented later in this chapter.

\begin{theorem}\label{theorem:inaccessible_models_zfc}
Let $\kappa$ be an inaccessible cardinal.
\beq
\langle V_\kappa, \in \rangle~\models~\sf{ZFC}
\eeq
\end{theorem}

We will prove this theorem in a way similar to \cite{KanamoriBook}.

\begin{proof}
Most of this is already done in lemma \bref{lemma:scm_s_is_limit}, we only need to verify that \emph{Replacement} and \emph{Infinity} axioms hold in $V_\kappa$.

\emph{Infinity} holds because $\kappa$ is uncountable, so $\omega \in V_\kappa$.

To verify \emph{Replacement}, let~$x$~be an element of $V_\kappa$ and $f$ a function from~$x$~to $V_\kappa$. Let $y = \{z \in V_\kappa : (\exists q \in x) f(q) = z \}$, so $y \subset V_\kappa$, it remains to show that $y \in V_\kappa$. Because $f$ is a function, we know that $|y| \leq |x| \leq \kappa$. But since $\kappa$ is regular, $\{rank(z) : z \in y\} \subseteq \alpha$ for some $\alpha < \kappa$, and so $x \in V_{\alpha+1} \in V_\kappa$. Therefore $y \in V_\kappa$.
\end{proof}

\begin{definition}{(Inaccessible Reflection Schema)}\label{def:inaccessible_reflection}\\
For every first–order formula $\varphi$, the following is an axiom:
\beq
\forall M_0 \exists \kappa (M_0 \subseteq V_\kappa \et In(\kappa) \et (\varphi(p_1, \ldots, p_n) \iff \varphi(p_1, \ldots, p_n)^{V_\kappa}))
\eeq
We will refer to this axiom schema as \emph{Inaccessible Reflection Schema}. Note that $M$ is a set, even though we often use upper–case letters for classes. 
This is due to fact that ``$M$'' is used in the same meaning in theorem \bref{theorem:first_order_reflection}.
\end{definition}

We have added the requirement that $\alpha$ is inaccessible, which trivially means that there is an inaccessible cardinal. By taking appropriate $M_0$, it can be shown that in a theory that includes the \emph{Inaccessible Reflection Schema}, there is a closed unbounded class of inaccessible cardinals. Since we know that for an inaccessible $\kappa$, $V_\kappa$ is a model of \sf{ZFC}, \emph{Inaccessible Reflection Schema} is equivalent to
\beq
\forall M_0 \exists \kappa (M_0 \subseteq V_\kappa \et \langle V_\kappa, \in \rangle~\models~\sf{ZFC} \et (\varphi(p_1, \ldots, p_n) \iff \varphi(p_1, \ldots, p_n)^{V_\kappa}))
\eeq
because we have proved in the last section that for an inaccessible $\kappa$,
\beq
\langle V_\kappa, \in \rangle~\models~\sf{ZFC}\mbox{.}
\eeq

\begin{theorem}
\emph{Inaccessible Reflection Schema} is equivalent to \emph{Axiom schema $F$}.
\end{theorem}

This is \emph{Theorem 4.1} in chapter 4 of \cite{DrakeBook}, also equivalent to \emph{Theorerem 3} in \cite{Levy60a}.

\begin{proof} 
Let's start by showing that \emph{Inaccessible Reflection Schema} implies \emph{Axiom schema $F$}. 
It should be clear from previous results that we can reflect two formulas to a single set, for example by taking the conjunction of universal closures of the formulas.

Given a normal function $f$ defined for all ordinals, we want to show that it has a regular fixed point. 
%Let $\varphi_1$ be the formula $f(\gamma) = \delta$ and let $\varphi_2$ be $\forall \gamma \exists \delta f(\gamma) = \delta$. 
For any ordinal $\alpha$, there is an ordinal $\kappa$ such that 
\beq
\alpha < \kappa \et In(\kappa) \et (\forall \gamma, \delta \in V_\kappa)(f(\gamma) = \delta \iff (f(\gamma) = \delta)^{V_\kappa})
\eeq
and
\beq
\alpha < \kappa \et In(\kappa) \et \forall \gamma \exists \delta (f(\gamma) = \delta) \iff (\forall \gamma \exists \delta f(\gamma) = \delta)^{V_\kappa}\mbox{.}
\eeq
Since $V_\kappa$ is the set of all sets of rank less than $\kappa$ and since every ordinal is the rank of itself, there is an inaccessible ordinal $\kappa$ such that
\beq
(\forall \gamma < \kappa)(\exists \delta < \kappa)(f^{V_\kappa} (\gamma) = \delta)\label{eq:reflected_function}\mbox{.}
\eeq
We also know that $f(\gamma) = \delta$ iff $(f(\gamma) = \delta)^{V_\kappa}$. 
Now since $\kappa$ is a limit ordinal and $f$ is continuous we get
\beq
f(\kappa) = \bigcup_{\gamma < \kappa} f^{V_\kappa}(\gamma) = \bigcup_{\gamma < \kappa} f(\gamma)\mbox{.}
\eeq
From \eref{eq:reflected_function} and the fact that $f$ is increasing, we know that $\kappa \leq \bigcup_{\gamma < \kappa} f(\gamma) \leq \kappa$. Therefore $\kappa$ is an inaccessible fixed point of $f$.

For the opposite direction, it suffices to show that since there is an inaccessible cardinal due to \emph{Axiom schema $F$}, given a first–order formula $\varphi$, there is an arbitrarily large inaccessible cardinal $\kappa$ for which 
\beq
%\varphi \iff \langle V_\kappa, \in \rangle~\models~\varphi\mbox{.}\label{eq:ch3_f_iff_m_1}
\varphi \iff \varphi^{V_\kappa}\mbox{.}\label{eq:ch3_f_iff_m_1}
\eeq
Note that the arbitrary size of $\kappa$ means given an arbitrary ordinal $\alpha$, there is a $\kappa$ satisfying $\alpha \in \kappa$ and \eref{eq:ch3_f_iff_m_1}.
In the previous chapter, in theorem \bref{theorem:first_order_reflection}, we have shown that we can easily obtain a limit ordinal satisfying \eref{eq:ch3_f_iff_m_1}. Note that since for any set $M_0$, there is such $\alpha$ that $M_0 \subseteq V_\alpha$, there is a closed unbounded class of sets satisfying \eref{eq:ch3_f_iff_m_1}, which are levels in the cumulative hierarchy, so there is a club class of $\kappa$s satisfying \eref{eq:ch3_f_iff_m_1}.

Let $f$ be a normal function defined for all ordinals that enumerates this club class, there is such $f$ by lemma \bref{lemma:normal_enumerates_club}. 
Let $g$ be the function that enumerates strong limit ordinals in $rng(f)$, there is one by lemma \bref{lemma:limit_fixed_normal_function}. 
Then $g$ has a regular fixed point $\kappa$, which is also a regular fixed point of $f$, so \eref{eq:ch3_f_iff_m_1} holds for $\kappa$.
\end{proof}

\begin{definition}{(\sf{ZMC})}\\
We will call $\sf{ZMC}$ an axiomatic set theory that contains all axioms and schemas of $\sf{ZFC}$ together with \emph{Axiom Schema $M$\textsubscript{1}}.
\end{definition}
We have decided to call it $\sf{ZMC}$, because Lévy uses $\sf{ZM}$, derived from $\sf{ZF}$, which is more intuitive, but we also need the axiom of choice, thus, $\sf{ZMC}$.

As a sidenote, we should note that \sf{ZMC} is extension of \sf{ZFC}, which is in turn an extension of \sf{S}. 
This way, reflection can be seen as a natural continuation of the \emph{Axiom of Infinity} and \emph{Replacement Schema}. % TODO Levy to hodne resi.

\subsection{Mahlo Cardinals}
We have shown that \sf{ZMC} contains arbitrarily large inaccessible cardinals. To return to reflection–style argument, is there a set that satisfies this property? To be able to properly answer this question, we have to formulate the notion of ``containing arbitrarily large cardinals'' more carefully. While we have previously used club sets, this is not an option in this case because inaccessibles don't form a club class in \sf{ZMC}\footnote{Note that cofinality of the limit of the first $\omega$ inaccessibles is $\omega$, which makes is singular.}. % we could try to formulate stronger versions of \emph{Axiom Schame $M$\textsubscript{1}}. 

%  Let's shortly review what \emph{Axiom Schema $M$\textsubscript{1}} says in order to formulate even stronger version.  % ??
We have shown earlier in this chapter that there is a simple relation between normal functions defined for all ordinals and closed unbounded classes.
We will now use a similar approach utilising normal functions.
By saying that for a class of ordinals $C$, a normal function $f$ has at least one element of $C$ in its range, we say that $C$ is stationary. 
Or, as Drake writes in \cite{DrakeBook} when dealing with the class of inaccessible cardinals, and a cardinal $\kappa$, in which inaccessibles are stationary:
\begin{displayquote}
`` The class of inaccessible cardinals is so rich that there are members $\kappa$ of the class such that no normal function on $\kappa$ can avoid this class; however we climb though $\kappa$, provided we are continuous at limits (so that we are enumerating a closed subset of $\kappa$), we shall eventually have to hit an inaccessible.''
\end{displayquote}

\begin{definition}{(Mahlo Cardinal)}\label{def:mahlo_cardinal}\\
We say that $\kappa$ is a \emph{Mahlo Cardinal} iff it is an inaccessible cardinal and the set $\{\lambda < \kappa : \lambda \mbox{ is inaccessible}\}$ is stationary in $\kappa$.
\end{definition}

Alternatively, $\kappa$ is Mahlo iff $\langle V_\kappa, \in \rangle~\models~\sf{ZMC}$ as shown above, this is also sometimes written as \emph{Ord is Mahlo}. There are also \emph{weakly Mahlo cardinals}, that are defined via weakly inaccessible cardinals below them, Mahlo cardinals are then also called \emph{strongly Mahlo} to highlight the difference, but we will only use the term \emph{Mahlo cardinal}.

Mahlo cardinals are related to reflection principles in an interesting way. Note that given a formula $\varphi$, \emph{First–Order Reflection Schema} gives us a club set of ordinals $\alpha$ such that $V_\alpha$ reflects $\varphi$, all below the first inaccessible cardinal. We have then used a different reflection schema to obtain arbitrarily high inaccessible cardinals $\kappa$ such that $V_\kappa$ refpects $\varphi$. Now we have a cardinal in which this reflection schema holds, so we are in fact reflecting reflection. Beware that this is done rather informally, because \emph{Axiom Schema $M$\textsubscript{1}} is a countable set of axioms, which can not be reflected via the schemas introduced so far. One way to deal with this would be to extend reflection for second– and possibly higher–order formulas, but we would have to be very careful with the notion of satisfaction. % TODO citace Tait, Welch
For now, let us explore where can stationary sets take us because as we have shown, their connection to reflection is quite clear.

What would happen if we strengthened \emph{Axiom Schema $M$\textsubscript{1}} to say that every normal function has a Mahlo cardinal in its range?

\begin{definition}{(hyper–Mahlo cardinal)}\label{def:hyper_mahlo_cardinal}\\
We say that $\kappa$ is a \emph{hyper–Mahlo cardinal} iff it is inaccessible and the set 
\beq
\{\lambda < \kappa : \lambda \mbox{ is Mahlo}\}
\eeq
is stationary in $\kappa$.
\end{definition}

\begin{definition}{(hyper–hyper–Mahlo cardinal)}\label{def:hyper_hyper_mahlo_cardinal}\\
We say that $\kappa$ is a \emph{hyper–hyper–Mahlo cardinal} iff it is inaccessible and the set 
\beq
\{\lambda < \kappa : \lambda \mbox{ is hyper–Mahlo}\}
\eeq
is stationary in $\kappa$.
\end{definition}

It is clear that one can continue in this direction, but the nomenclature gets increasingly confusing even if we rewrite them as \emph{hyper\textsuperscript{$\alpha$}–Mahlo cardinals} instead of repeating the prefix.
To see there is a more elegant way to reach those cardinals, we will now establish an operation that elegantly exhausts all such cardinals.

\begin{definition}{(Mahlo Operation)}\label{def:mahlo_operation}\\
Let $A$ be a class of ordinals. Let
\beq
H(A) = \{\alpha \in A: A \cap \alpha \mbox{ is stationary in }\alpha\}\mbox{.}
\eeq
We call $H$ the \emph{Mahlo's operation}.
\end{definition}

If we pick for $A$ the class of all inaccessible cardinals, $H(A)$ is the class of Mahlo cardinals.
It is easy to see that is $A$ is the class of all $\alpha$–Mahlo cardinals, $H(A)$ is the class of $\alpha+1$–Mahlo cardinals, $H(H(A))$ is the class of $\alpha+2$–Mahlo cardinals and so on.

\begin{definition}{(Iterated Mahlo Operation)}\label{def:iterated_mahlo_operation}\\
Let $A$ be a class of ordinals. We shall extend the Mahlo operation in the following way:
\bce[(i)]
\item $H^0(A) = A$,
\item $H^{\alpha+1}(A) = H(H^{\alpha}(A))$,
\item $H^{\lambda}(A) = \bigcap_{\alpha < \lambda} H^{\alpha}(X)$ for limit $\lambda$.
\ece
\end{definition}

Clearly if $A$ is the class of inaccessibles, $H^{\alpha}(A)$ is the class of $\alpha$–Mahlo cardinals. To get to hyper–Mahlo cardinals, we can diagonalise the operation.

\begin{definition}{(Diagonal Mahlo Operation)}\label{def:diagonal_mahlo_operation}\\
Let $A$ be a class of ordinals. Then the \emph{diagonal Mahlo operation} is defined as follows:
\beq
H^{\Delta}(A) = \{\alpha \in Ord: \forall \beta < \alpha (\alpha \in H^{\beta}(X))\}\mbox{.}
\eeq
\end{definition}

We can further diagonalise the diagonal version and continue this process ad libitum in order to reach all large cardinals accessible \emph{from below}. 
To see what is meant by \emph{from below}, note that the approach that led us to the \emph{Mahlo operation} was taking a property, for example regularity, that is already available in our current theory, e.g. \sf{ZFC}, and making an assertion of the height of the universe such that there are ``enough'' other ordinals holding this property in a sense that a normal function defined on ordinals inevitably has at least one such ordinal in its range.

\subsection{Indescribable Cardinals}

Indescribability is another approach towards large cardinals that is based on reflection. 
We will briefly introduce the basic definitions and show that it yield large cardinals, but most of them are not reachable from below in a sense established at the end of previous subsection.

Most of the results presented in this subchapter come from \cite{KanamoriBook}.

Since this chapter uses higher–order logic, we need to introduce the hierarchy of formulas first.

\begin{definition}{(Higher–Order Variables)}\label{def:higher_order_variables}\\
Let $M$ be a structure and $D$ its domain. In first–order logic, variables range over individuals, that is, over elements of $D$. We shall call those \emph{type~1 variables} for the purposes of higher–order logic. Type~2 variables then range over collections, that is, the elements of $\power{D}$. Generally, type $n$ variables are defined for any $n \in \omega$ such that they range over $\mathscr{P}^{n-1}(D)$.
\end{definition}
We will use lowercase latin letters for type~1 variables for backward compatibility with first–order logic, type~2 variables will be represented by uppercase letters, mostly $P, X, Y, Z$, higher–order variables won't be needed in this thesis. If we wanted to define satisfaction for second–order formulas in a model $\la V_\alpha, \in \ra$ that we have often used in this thesis, type~2 variables would be interpreted to range over a set is isomorphic to $V_{\alpha+1}$\footnote{It might be useful to keep a separate version instead of using $V_{\alpha+1}$ so that we can distinguish between sets and classes that turn out to have the same extension. See \cite{Koellner2009ORP} for details.}.

\begin{definition}{(Full Prenex Normal Form)}\label{def:pnf}\\
We say a formula is in the \emph{prenex normal form} if it is written as a block of quantifiers followed by a quantifier–free part.\\
We say a formula is in the \emph{full prenex normal form} if it is written in \emph{prenex normal form} and if there are type $n+1$ quantifiers, they are written before type $n$ quantifiers.
\end{definition}
It is an elementary that every formula is equivalent to a formula in the full prenex normal form.

% ===================================================== TODO check ===============================================
\begin{definition}{(Hierarchy of Formulas)}\label{def:analytical_hierarchy}\\
Let $\varphi$ be a formula in the prenex formal form.
\bce[(i)]
\item We say $\varphi$ is a $\Delta^0_0$–formula if it contains only bounded quantifiers.
\item We say $\varphi$ is a $\Sigma^0_0$–formula or a $\Pi^0_0$–formula if it is a $\Delta^0_0$–formula.
\item We say $\varphi$ is a $\Pi^{m+1}_0$–formula if it is a $\Pi^m_n$– or $\Sigma^m_n$–formula for any $n \in \omega$ or if it is a $\Pi^m_n$– or $\Sigma^m_n$–formula with additional free variables of type $m+1$.
\item We say $\varphi$ is a $\Sigma^m_0$–formula if it is a $\Pi^m_0$–formula.
\item We say $\varphi$ is a $\Sigma^m_n+1$–formula if it is of a form $\exists P_1, \ldots, P_i \psi$ for any non–zero $i$, where $\psi$ is a $\Pi^m_n$–formula and $P_1, \ldots, P_i$ are type $m+1$ variables.
\item We say $\varphi$ is a $\Pi^m_n+1$–formula if it is of a form $\forall P_1, \ldots, P_i \psi$ for any non–zero $i$, where $\psi$ is a $\Sigma^m_n$–formula and $P_1, \ldots, P_i$ are type $m+1$ variables.
\ece
\end{definition}


\begin{definition}{(Describability)}\label{def:describability}\\
We say an ordinal $\alpha$ is described by a sentence $\varphi$ in the language $\mathscr{L}$ with relation symbols $P_1, \ldots, P_n$ given iff
\begin{equation}
\langle V_\alpha, \in, P_1, \ldots, P_n \rangle~\models~\varphi
\end{equation}
but for every $\beta < \alpha$
\begin{equation}
\langle V_\beta, \in, P_1 \cap V_\beta, \ldots, P_n \cap V_\beta \rangle \not\models \varphi\mbox{.}
\end{equation}
\end{definition}

For the definition of a $\Pi^m_n$–formula and a $\Sigma^m_n$–formula, see definition \bref{def:analytical_hierarchy}.

\begin{definition}{($\Pi^m_n$–Indescribable Cardinal)}\label{def:pi_mn_indescribable}\\
We say that $\kappa$ is $\Pi^m_n$–indescribable iff it is not described by any $\Pi^m_n$–formula.
\end{definition}
\begin{definition}{($\Sigma^m_n$–Indescribable Cardinal)}\label{def:sigma_mn_indescribable}\\
We say that $\kappa$ is $\Sigma^m_n$–indescribable iff it is not described by any $\Sigma^m_n$–formula.
\end{definition}

To see that this notion is based in reflection, let us recall the opening quote of this thesis by Gödel which says \emph{``The Universe of sets cannot be uniquely characterised (i.~e.~distinguished from all its initial elements) by any internal structural property of the membership relation on it.''}. A cardinal $\kappa$ is $\Pi^m_n$–indescribable\footnote{This holds for $\Sigma^m_n$–formulas alike.} iff every $\Pi^m_n$–formula fails to describe $V_\kappa$ and describes an initial segment instead.
In a sense, $V_\kappa$ reflects the ``property''\footnote{In this case, we are not using the word to refer to a definable class, but on a meta level to refer to a property expressible in the natural language, hence the quotation marks.} of indescribability of the universal class with respect to certain classes of formulas.

% Since we are interested inaccessing cardinals from below, we will limit ourselves to $\Pi^1_n$–formulas, with the exception of the measurable cardinal, that is included for context.

\begin{lemma}
Let $\kappa$ be a cardinal, then the following holds for any $n \in \omega$. $\kappa$ is $\Pi^1_n$–indescribable iff $\kappa$ is $\Sigma^1_{n+1}$–indescribable.
\end{lemma}

\begin{proof}
The forward direction is obvious, we can always add a spare quantifier over a type~2 variable to turn a $\Pi^1_n$ formula $\varphi$ into a $\exists P \varphi$ which is then a $\Sigma^1_{n+1}$–formula.\footnote{Note that unlike in previous sections, it is worth noting that $\varphi$ is now a sentence so we don't have to worry whether $P$ is free in $\varphi$.}

To prove the opposite direction, suppose that $\langle V_\kappa, \in \rangle~\models~\exists X \varphi(X)$ where $X$ is a type~2 variable and $\varphi$ is a $\Pi^1_n$–formula with one free variable of type~2. 
This means that there is a set $S \subseteq V_\kappa$ that is a witness of $\exists X \varphi(X)$, in other words, $\varphi[S]$ holds. 
We can replace every occurence of $X$ in $\varphi$ by a new predicate symbol $S$, this allows us to say that $\kappa$ is $\Pi^1_n$–indescribable (with respect to $\langle V_\kappa, \in, R, S \rangle$).\footnote{A different yet interesting approach is taken by Tate in \cite{Tait_constructingcardinals}. He states that for $n\geq 0$, a formula of order $\leq n$ is called a $\Pi^n_0$ and a $\Sigma^n_0$ formula. Then a $\Pi^n_{m+1}$ is a formula of form $\forall Y \psi(Y)$ where $\psi$ is a $\Sigma^n_m$ formula and $Y$ is a variable of type $n$. Finally, a $\Sigma^n_{m+1}$ is the negation of a $\Pi^n_m$ formula. So the above holds ad definitio.}
\end{proof}

The above lemma makes it clear that, without the loss of generality, we can suppose that all formulas with no higher than type~2 variables are $\Pi^1_n$–formulas.

\begin{lemma}\label{lemma:inaccessible_clubset}
If $\kappa$ is an inaccessible cardinal and given $R \subseteq V_\kappa$, then the following is a club set in $\kappa$:
\begin{equation}
\{\alpha \in \kappa : \langle V_\alpha, \in, R \cap V_\alpha \rangle \prec \langle V_\kappa, \in, R \rangle \}\label{eq:inacc_lemma_set}\mbox{.}
\end{equation}
\end{lemma}

\begin{proof}
To see that \eref{eq:inacc_lemma_set} is closed, let us recall that a $A \subseteq \kappa$ is closed iff for every ordinal $\alpha$ such that $\emptyset < \alpha < \kappa$, it holds that if $A \cap \alpha$ is unbounded in $\alpha$ then $\alpha \in A$. Since $\kappa$ is an inaccessible cardinal, thus strong limit, it is closed under limits of sequences of ordinals smaller than $\kappa$.
%TODO neco s $V_\kappa$, ze je tranzitivni a tak jso vsechny $V_\alpha$ pro $\alpha<\kappa$ $V_\alpha \in V_\kappa$
In order to verify that it is unbounded, we will use a recursively defined $\kappa$–sequence $\la \alpha_0, \alpha_1, \ldots \ra$
to build $\la V_\alpha, \in, R \cap V_\alpha \ra$, an elementary substructure of $\la V_\kappa, \in, R \ra$ such that $\alpha > \alpha_0$ for an arbitrary ordinal $\alpha_0 < \kappa$.
% that is built above an arbitrary $V_{\alpha_0}$, $\alpha_0 <\kappa$.
Let us fix one such $\alpha_0$. Given $\alpha_n$, $\alpha_{n+1}$ is defined as the least $\beta$, $\alpha_n \leq \beta$ that satisfies 
the following for any formula $\varphi$ for $p_1, \ldots, p_m \in V_{\alpha_{n}}, m \in \omega$:
\begin{equation}
\begin{gathered}
\mbox{If }\langle V_\kappa, \in, R \rangle~\models~\exists x \varphi(p_1, \ldots, p_n)\mbox{,}\\
\mbox{then }\exists x \in V_\beta \mbox{ such that }\langle V_\kappa, \in, R \rangle~\models~\varphi(x, p_1, \ldots, p_n)\mbox{.}
\end{gathered}
\end{equation}

Let $\alpha = \bigcup_{n < \omega} \alpha_n$. Then 
\beq
\langle V_\alpha, \in, R \cap V_\alpha \rangle \prec \langle V_\kappa, \in, R \rangle\mbox{,}
\eeq
in other words, for any $\varphi$ with given arbitrary parameters $p_1, \ldots, p_n \in V_\alpha$, it holds that
\beq
\langle V_\alpha, \in, R \cap V_\alpha \rangle~\models~\varphi(p_1, \ldots, p_n) \iff \langle V_\kappa, \in, R \rangle~\models~\varphi(p_1, \ldots, p_n)\mbox{.}
\eeq
Which should be clear from the construction of $\alpha$.
\end{proof}

\begin{theorem}
Let $\kappa$ be an ordinal. The following are equivalent.
\bce[(i)]
\item $\kappa$ is inaccessible.
\item $\kappa$ is $\Pi^1_0$–indescribable.
\ece
\end{theorem}

Note that $\Pi^1_0$ formulas are those that contain zero unbound quantifiers over type–2 variables, they are in fact first–order formulas, but with additional type~2 free variables allowed.

\begin{proof}
$\Pi^1_0$–sentences contain type~2 variables, but only type~1 quantifiers. We want to prove that $\kappa$ is an inaccessible cardinal iff whenever a formula tries to describe $\kappa$ in the sense of definition \bref{def:describability}, the formula fails to do so and describes a initial segment thereof instead.
We have already shown in theorem \bref{theorem:inaccessible_models_zfc} that there is no way to climb the cumulative hierarchy to the height of an inaccesible cardinal via first–order formulas in $\sf{ZFC}$. We will now prove that adding unqantified type~2 variables does not make it possible, note that all of the axiom schemata used in the previous chapter can be rewritten to use a type~2 variable instead of a given function.

For (i)$\then$(ii), suppose that $\kappa$ is inaccessible.

Then there is, by lemma \bref{lemma:inaccessible_clubset} a club set of ordinals $\alpha$ such that $V_\alpha$ is an elementary substructure of $V_\kappa$. 
For $\kappa$ to be $\Pi^1_0$–indescribable, we need to make sure that given an arbitrary $\Pi^1_0$–formula $\varphi$ satisfied in the structure $\langle V_\kappa, \in, R \rangle$, there is an ordinal $\alpha < \kappa$, such that $\langle V_\alpha, \in, R \cap V_\alpha \rangle~\models~\varphi$. But this follows from the definition of elementary substructure.

For (ii)$\then$(i), suppose $\kappa$ is not inaccessible, so it is either singular, or there is a cardinal $\nu < \kappa$ such that $\kappa \leq \power{\nu}$ or $\kappa=\omega$. 


%For the successor case, there is some $\nu$ so that $\nu+1=\kappa$. 
%Let us take $R = \{\nu\}$ and $\varphi = \exists x \psi(x)$ such that
%\begin{eqaution}
Suppose $\kappa$ is singular. Then there is a cardinal $\nu~<~\kappa$ and a function $f:~\nu~\then~\kappa$ such that $rng(f)$ is cofinal in $\kappa$. Since $f \subseteq V_\kappa$, we can add $f$ as a relation to the language. We can do the same with $\{\nu\}$. That means $\langle~V_\kappa,~\in,~P_1,~P_2\ra$ with $P_1~=~f, P_2~=~\{\nu\}$ is a structure.
Let 
\beq
\varphi = (P_1 \neq \emptyset \et rng(P_1) = P_2)\footnote{$rng(x)=y$ is a first–order formula, see definition \bref{def:rng}.}\mbox{.}
\eeq
Since for every $\alpha~<~\nu$, $P_1 \cap V_\alpha = \emptyset$, $\varphi$ is false and therefore describes $\kappa$. That contradicts the fact that $\kappa$ was supposed to be $\Pi^1_0$–indescribable, but $\varphi$ is a first–order formula.

Suppose there is a cardinal $\nu$ satisfying $\kappa \leq \power{\nu}$. Let there be a function $f: \power{\nu} \then \kappa$ that is onto. Then, like in the previous paragraph, we can obtain a structure $\langle V_\kappa, \in, P_1, P_2 \rangle$, where $P_1 = f$ like before, but this time $P_2 = \power{\nu}$. Again, 
\beq
\varphi = (P_1 \neq \emptyset \et rng(P_1) = P_2)
\eeq
describes $\kappa$.

Finally, suppose $\kappa = \omega$, then the first-order sentence $\varphi = \forall x \exists y (x \in y)$ describes $\kappa$, which is a contradiction.
\end{proof}

Generally, it should be clear that it a cardinal $\kappa$ is $\Pi^m_n$–indescribable, it is also $\Pi^{m'}_{n'}$–indescribable for every $m'<m, n'<n$. By the same line of thought, if a cardinal $\kappa$ satisfies the property implied by $\Pi^m_n$–indescribability, it satisfies all properties implied by $\Pi^{m'}_{n'}$–indescribability for $m'<m, n'<n$. For example, if $\kappa$ is $\Pi^m_n$–indescribable for $m \geq 1$ then it is also an inaccessible cardinal.

% TODO pozorovani ze kdyz je $\kappa$ $\Pi$

\begin{theorem}\
If a cardinal $\kappa$ is $\Pi^1_1$–indescribable, then it is a Mahlo cardinal.
\end{theorem}

\begin{proof}
Assuming that $\kappa$ is $\Pi^1_1$–indescribable, we want to prove that every club set of in $\kappa$ contains an inaccessible cardinal. 

Consider the following $\Pi^1_1$–sentence $\varphi$:
\beq
\begin{gathered}
\varphi = \forall P (``\mbox{P is a function}''\then \forall x \exists y \forall z (z \in y \iff (\exists q \in x)(P(x, y, p_1, \ldots, p_n))))\\
\et \forall x \exists y \forall z (z \in y \iff z \subseteq x)
\end{gathered}
\eeq
%\begin{gathered}\label{eq:inac}
% \forall P (\mbox{``$P$ is a function''} \et \exists x(x = dom(P) \lor \power{x} = dom(P)) \then\\
%\then \exists y(y = rng(P)))
%\end{gathered}
%\end{equation}
where $P$ is a type~2 variable and the rest are type~1 variables,  ``$P$ is a function'' is a first–order formula defined in definition \bref{def:function}. 
As has been shown earlier in this chapter, given a cardinal $\mu$, the following holds if and only if $\mu$ is inaccessible:
\begin{equation}
\langle V_\mu, \in \rangle~\models~\varphi\mbox{.}
\end{equation}

Now fix an arbitrary $C \subset \kappa$, a club set in $\kappa$. We want to show that it contains an inaccessible cardinal. 
Since $C$ is a subset of $\kappa$ and therefore a subset of $V_\kappa$, we can use the structure $\langle V_\kappa, \in, C \rangle$ instead of $\langle V_\kappa, \in \rangle$. 
Then the following holds:
\begin{equation}
\langle V_\kappa, \in, C \rangle~\models~\varphi \et \mbox{``$C$ is unbounded''.}\footnote{``$C$ is unbounded'' is a first–order formula, see definition \bref{def:unbounded_class}.}
\end{equation}
Note that this holds because $\kappa$ is $\Pi^1_1$–indescribable, and therefore also $\Pi^1_0$–indescribable.
So $\kappa$ is itself inaccessible and therefore $\langle V_\kappa, \in, C \rangle~\models~\varphi$.

Since  $\kappa$ is $\Pi^1_1$–indescribable and $\varphi \et \mbox{``$C$ is unbounded''}$ is equivalent to a $\Pi^1_1$–formula, there must be an ordinal $\alpha$ that satisfies
\begin{equation}
\langle V_\alpha, \in, C \cap V_\alpha \rangle~\models~\varphi \et \mbox{``$C$ is unbounded'',}
\end{equation}
which implies that $\alpha$ is inaccessible; it is regular because it reflects \emph{Replacement} and it is limit because if $\alpha$ were a successor ordinal, it couldn't contain an unbounded class of ordinals.

We only need to verify that $\alpha \in C$, which is clear from the fact that $C$ is a club set in $\kappa$ and it is unbounded in $\alpha$.
\end{proof}

There is an even stronger large cardinal property implied by $\Pi_1^1$–indescribability that is based on reflection.

\begin{definition}{(Extension Property)}\label{def:extension_property}\\
We say a cardinal $\kappa$ has the \emph{extension property} iff for all $U \subset V_\kappa$ there exists a transitive set $X$ such that $\kappa \in X$, and a set $S \subset X$, such that $(V_\kappa, \in, U)$ is an elementary substructure of $(X, \in, S)$.
\end{definition}

\begin{definition}{(Weakly Compact Cardinal)}\label{def:weakly_compact_cardinal}\\
We say that a cardinal $\kappa$ is \emph{weakly compact} iff it has the extension property.
\end{definition}

\begin{theorem}\
A cardinal $\kappa$ is $\Pi_1^1$–indescribable iff it is weakly compact.
\end{theorem}
For the proof, see \cite{KanamoriBook}.
Note that the extension property is also very similar to reflection

We will now introduce the measurable cardinal, which is not based on reflection from below in our sense, but illustrates the fact that indescribability leads to cardinals that contradict \emph{Axiom of Constructibility}, that will be introduced right after the measurable cardinal.

\begin{definition}{(Ultrafilter)}\\
Given a set $x$, we say $U \subset \power{x}$ is an \emph{ultrafilter} over $x$ iff all of the following hold:
\bce[(i)]
\item $\emptyset \not\in U$,
\item $\forall y, z (y \subset x \et z \subset x \et y \subset z \et y \in U \then z \in U)$,
\item $(\forall y, z \in U)(y \cap z) \in U$,
\item $\forall y (y \subset x \then (y \in U \lor (x \setminus y) \in U))$.
\ece
\end{definition}

\begin{definition}{($\kappa$–Complete Ultrafilter)}\\
We say that an ultrafilter $U$ is $\kappa$–complete iff it is closed under intersection of $\kappa$–many elements. More precisely,
\beq
(\forall \gamma < \kappa)(\{a_\alpha : \alpha < \gamma \} \subseteq U \then \bigcup_{\alpha < \gamma} a_\alpha \in U)\mbox{.}
\eeq
\end{definition}

\begin{definition}{(Measurable Cardinal)}\\
We say that a cardinal $\kappa$ is a \emph{measurable cardinal} iff there is a $\kappa$–complete ultrafilter over $\kappa$.
\end{definition}

\begin{theorem}
Let $\kappa$ be a cardinal. If $\kappa$ is a measurable cardinal then the following hold:
\bce[(i)]
\item $\kappa$ is $\Pi^2_1$–indescribable.
\item Given $U$, a normal ultrafilter over $\kappa$, a relation $R \subseteq V_\kappa$ and a $\Pi^2_1$–formula $\varphi$ such that $\langle V_\kappa, \in, R \rangle \models \varphi$, then
\beq
\{ \alpha < \kappa : \langle V_\alpha, \in, R \cap V_\alpha \rangle \models \varphi \} \in U\mbox{.}
\eeq
\ece
\end{theorem}
For a proof, see \emph{Proposition 6.5} in \cite{KanamoriBook}.

%\begin{theorem}
%If $\kappa$ is a measurable cardinal and $U$ is a normal ultrafilter over $\kappa$, the following holds:
%\begin{equation}
%\{ \alpha < \kappa: \mbox{"$\alpha$ is totally indescribable"}\} \in U\mbox{.}
%\end{equation}
%\end{theorem}
%For a proof, see \emph{Proposition 6.6} in \cite{KanamoriBook}.

\subsection{The Constructible Universe}

The constructible universe, denoted $L$, is a cumulative hierarchy of sets, presented by Kurt Gödel in his paper \cite{Godel1940consistency}.
Assertion of its equality to the \emph{Von Neumann's hierarchy}, $V=L$, is called the \emph{Axiom of Constructibility}. 
The axiom implies $GCH$ and $AC$ and contradicts the existence of some\ large cardinals, our goal is to decide whether those introduced earlier are among them.

On order to formally establish this class, we need to formalise the notion of definability first. 
\begin{definition}{(Definability)}\label{def:definability}\\ % musi ta fle byt prvoradova?
We say that a set $X$ is \emph{definable} over a model $\langle M, \in \rangle$ if there is a formula $\varphi$ together with parameters $p_1, \ldots, p_n \in M$ such that
\begin{equation}
X = \{x: x \in M \et \langle M, \in \rangle~\models~\varphi(x, p_1, \ldots, p_n)\}
\end{equation}
\end{definition}

\begin{definition}{(The Set of Definable Subsets)}\label{def:definable_powerset}\\
The following is a set of all definable subsets of a given set $M$, denoted Def($M$).
\begin{equation}
\begin{gathered}
Def(M) = \{\{y : x \in M \et \langle M, \in \rangle~\models~\varphi(y, u_1, \ldots, i_n) \} :\\
\mbox{ $\varphi$ is a~first–order formula, }p_1, \ldots, p_n \in M \}
\end{gathered}
\end{equation}
\end{definition}

We will use $Def(M)$ in the following construction in the way the power set operation is used when constructing the usual Von Neumann's hierarchy of sets\footnote{For that reason, some authors use $\mathscr{P}^{*} (M)$ instead of $Def(M)$, see section 11 of \cite{PinterBook} for one such example.}.

% Now we can recursively build $L$.
\begin{definition}{(The Constructible Universe)}\label{def:constructible_universe}\\
The \emph{constructible universe} is a collection of sets similar to the Von Neumann's hierarchy but consisting only of definable sets.
\bce[(i)]
\item
\beq
L_0 = \emptyset\mbox{,}
\eeq

\item
\beq
L_{\alpha+1} = Def(L_{\alpha})\mbox{ for any ordinal $\alpha$,}
\eeq
\item
\beq
L_{\lambda} = \bigcup_{\alpha < \lambda} L_{\alpha}\mbox{ For a~limit ordinal }\lambda\mbox{,}
\eeq
\item
\beq
L = \bigcup_{\alpha\in Ord} L_{\alpha}\mbox{.}
\eeq
\ece
\end{definition}

Note that while $L$ bears very close resemblance to $V$, the difference is, that in every successor step of constructing $V$, we take every subset of $V_\alpha$ to be $V_{\alpha+1}$, whereas $L_{\alpha+1}$ consists only of definable subsets of $L_\alpha$. Also note that $L$ is transitive.

%In order to 
\begin{theorem}
Let $L$ be as in definition \bref{def:constructible_universe}.
\begin{equation}
\la L, \in \ra \mbox{ is a model of \sf{ZFC}}
\end{equation}
\end{theorem}
For details, refer to Theorem 13.3 in \cite{JechBook}.

\begin{definition}{(Constructibility)}\\
The axiom of constructibility states that every set is constructible. It is usually denoted as $L = V$.
\end{definition}

Without providing a proof, we will introduce two important results established by Gödel in \cite{Godel1940consistency}. 
% \sf{ZF} stands for Zermelo–Fraenkel set theory as introduced in definition \bref{def:zf}. % wtf

\begin{theorem}{(Constructibility $\then$ Choice)}
\begin{equation}
\sf{ZF} \proves \mbox{\emph{Constructibility}} \then \mbox{\emph{Axiom of Choice}} 
\end{equation}
\end{theorem}

The $GCH$ refers to the \emph{Generalised Continuum Hypothesis}, see definition \bref{def:gch}.
\begin{theorem}{(Constructibility $\then$ Generalised Continuum Hypothesis)}\label{theorem:l_then_gch}
\begin{equation}
\sf{ZF} \proves \mbox{\emph{Constructibility}} \then \mbox{\emph{GCH}} 
\end{equation}
\end{theorem}
It is worth mentioning that Gödel's proof of \emph{Construcibility} $\then$ \emph{GCH} featured the first formal use of a reflection principle. 
For the actual proofs, see for example \cite{Kunen_independence},

Since \emph{GCH} implies that $\kappa$ is a limit cardinal iff $\kappa$ is a strong limit cardinal for every $\kappa$, the distinctions between inaccessible and weakly inaccessible cardinals as well as between Mahlo and weakly Mahlo cardinals vanish.

% =============================================================================================

\begin{theorem}{(Inaccessibility in $L$)}\label{theorem:inaccessible_in_l}\\
Let $\kappa$ be an inaccessible cardinal. Then $In(\kappa)^L$.
\end{theorem}
\begin{proof}
We want to show that the following are all true for an inaccessible cardinal $\kappa$:
\bce[(i)] 
\item $\mbox{``$\kappa$ is a cardinal''}^L$,
\item $(\omega < \kappa)^L$,
\item $\mbox{``$\kappa$ is regular''}^L$,
\item $\mbox{``$\kappa$ is limit''}^L$.\footnote{While inaccessible cardinals are strong limit cardinals, since \emph{GCH} holds in $L$, $\mbox{``$\kappa$ is limit''}^L$,
implies $\mbox{``$\kappa$ is strong limit''}^L$.}
\ece

Suppose $\mbox{``$\kappa$ is not a cardinal''}^L$ holds, then there is a cardinal $\mu$, $\mu < \kappa$ and a function $f:\mu\then\kappa$, $f \in L$, such that $\mbox{``$f:\mu\then\kappa$ is onto''}^L$. But since ``$f$ is onto'' is a $\Delta_0$ formula and $\Delta_0$ formulas are are absolute in transitive structures\footnote{See lemma \bref{lemma:delta_0_absoluteness}.} and $L$ is a transitive class, $\mbox{``$f$ is onto''}^L \iff \mbox{``$f$ is onto''}$, this contradicts the fact that $\kappa$ is a cardinal.
$(\omega < \kappa)^L$ holds because $\omega \in \kappa$ and because ordinals remain ordinals in $L$, so $(\omega \in \kappa)^L$.

In order to see that $\mbox{``$\kappa$ is regular''}^L$, we can repeat the argument by contradiction used to show that $\kappa$ is a cardinal in $L$. If $\kappa$ was singular, there is a $\mu < \kappa$ together with a function $f: \mu \then \kappa$ that is onto, but since ``$f$ is onto'' implies $\mbox{``$f$ is onto''}^L$, we have reached a contradiction with the fact that $\kappa$ is regular, but singular in $L$.

It now suffices to show that $\mbox{``$\kappa$ is a limit cardinal''}^L$. That means, that for any given $\lambda<\kappa$, we need to find an ordinal $\mu$ such that $\lambda < \mu < \kappa$ that is also a cardinal in $L$. But since cardinals remain cardinals in $L$ by an argument with surjective functions just like above, it holds.\end{proof}

\begin{theorem}{(Mahloness in $L$)}\label{theorem:mahlo_in_l}\\
Let $\kappa$ be a Mahlo cardinal. Then $\mbox{``$\kappa$ is Mahlo''}^L$.
\end{theorem}
% http://math.stackexchange.com/questions/1791631/reference-mahlo-cardinals-remain-mahlo-in-l/1792486#1792486
% TODO citace webu?

\begin{proof}
Let $\kappa$ be a Mahlo cardinal. From the definition of Mahloness in definition \bref{def:mahlo_cardinal}, it should be clear that we want prove that $\kappa$ is inaccessible in $L$ and 
\begin{equation}
\mbox{``The set }\{\alpha : \alpha \in \kappa \et \mbox{'$\alpha$ is inaccessible'}\}\mbox{ is stationary in $\kappa$''}^L
\end{equation}

Since we have shown that an inaccessible cardinals remain inaccessible in $L$ in the previous theorem, $In(\kappa)^L$ holds.

Now consider the two following sets:
\beq
S = \{\alpha \in \kappa : In(\alpha)\}\label{eq:mahloness_l_eq_1}
\eeq
\beq
T = \{\alpha \in \kappa : In(\alpha)^L\}\label{eq:mahloness_l_eq_2}
\eeq
Since inaccessible cardinals are inaccessible in $L$ from theorem \bref{theorem:inaccessible_in_l}, $S \subseteq T$.
So if $T$ is stationary in $\kappa$, we are done. Suppose for contradiction that it is not the case. 
Therefore there is a $C \subset \kappa$ satisfying $\mbox{``$C$ is a club set in $\kappa$''}^L$, but it is the case that $T \cap C = \emptyset$.
But because $\mbox{``$C$ is a club set in $\kappa$''}$ is equivalent to a $\Delta_0$ formula, 
\beq
\mbox{``$C$ is a club set in $\kappa$''}^M \iff \mbox{``$C$ is a club set in $\kappa$'',}
\eeq
ergo $C$ is a club set in $\kappa$. But since it has o intersection with $T$, it can't have an intersection with a subset thereof, which contradicts the fact that $S$ is stationary in $\kappa$.

$\kappa$ remains Mahlo in $L$.
\end{proof}

It should be clear that the above process can be iterated over again. Since Mahlo cardinals are absolute in $L$, the same argument using stationary sets can be carried out for hyper–Mahlo cardinals and so on. It is clear that since a regular and an inaccessible cardinal in consistent with \emph{Constructibility}, so should be the higher properties acquired from assuring the existence of regular, inaccessible and Mahlo fixed points of normal functions.


\begin{theorem}
If there is a measurable cardinal, then $V \neq L$.
\end{theorem}
This is proved in \cite{scott_measurable_constructible} or \cite{KanamoriBook}.
% Measurable cardinals yield inconsistency with the \emph{Axiom of Constructibility}, which was shown by Dana S. Scott in his article \cite{scott_measurable_constructible}. 

\subsection{Conclusion}
To have an intuitive idea of why apart from measurability, every large cardinal property we have established is absolute in $L$, let us stress that measurability is the only one that does not deal with the height of the cumulative hierarchy of sets.
The assertion of the existence of an inaccessible cardinal can be informally rephrased as
``The universe of all sets is so big in terms of height, there are ordinals unreachable by the power set operation''\footnote{This approach is embodied in the definition of \sf{Q}–inaccessibility used by Lévy, see definition \bref{def:levy_inaccessible_q}, that can be understood as ``given a set theory with some means of traversing the cumulative hierarchy upwards, a cardinal is inaccessible with respect to \sf{Q} if it can't be reached by those means alone''.}. 
Gödel's Constructible universe deals only with the width of the universe, which is kept as small as possible, so there is no way it can be inconsistent with assertions that deal with height and have no implications in terms of width. Similarly, the Mahlo operation only deals with ordinals, therefore it's not surprising that it has no implications on width of the universe alone. 
This is not the case with measurability. 
Measurability is such a strong statement that even though it seems to explicitly speak of height only, 
the existence of a measurable cardinal implies the existence of non–constructible subsets of $\omega$\footnote{See \cite{DrakeBook}, p. 196.}.

% TODO ze meritelny resi sirku a ne vysku, ale nase karidnaly mluvi o vysce WATT


%\newpage
\section{Conclusion}
\begin{comment}
After establishing an intuitive concept of reflection, we have reviewed Lévy's original proof of the equivalence of Replacement Schema and the Axiom of Infinity with his first-order reflection principle, we have then reformulated and proved the same result in contemporary terms. We have also shown that the same results can be obtained via axiom schemas stating the existence of regular fixed points on normal ordinal functions. After examining the concept of regular fixed points and seeing how it relates to stationary sets and Mahlo cardinals, we have introduced to notion of indescribable cardinal to see that Inaccessible, Mahlo and even Hyper-inaccessible cardinals are still significantly smaller than measurable cardinals, therefore concluding that this application of the reflection principle does not lead to transcendence over $L$.
\end{comment}
\newpage
\bibliographystyle{apalike}
\bibliography{bc_biblio}

\end{document}