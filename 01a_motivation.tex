The central point of this thesis is the so called \emph{reflection principle}, which could be informally expressed like this:
\begin{displayquote}
\emph{For every property that holds in the universe of all sets, there is a set in which this property holds.}
\end{displayquote}

Clearly, this formulation is rather vague and we should be extremely cautious when dealing with the word ``property''. 
One problem that immediately comes to mind is that ``being the set of all sets'' must not be considered a property in this sense, otherwise we run into the well–known paradox of Russell.
This is a well–known problem that exemplifies the fact that reflection is a phenomenon that is closely connected to the very foundations of mathematics.
This is also emphasised by the fact that the very first explicit use of reflection in a mathematical proof can be found in Gödel's paper \emph{The Consistency of the Axiom of Choice and of the Generalised Continuum Hypothesis with the Axioms of Set Theory}\footnote{See \cite{Godel1940consistency}.}
that deals with the consistency of the \emph{generalised continuum hypothesis}, which is a question that played an important part in the development of set theory in the 20\textsuperscript{th} century.
Furthermore, Lévy's article \emph{Axiom Schemata of Strong Infinity in Axiomatic Set Theory}, that is a cornerstone of this thesis is concerned primarily with the so called \emph{strong axioms (or axiom schemata) of infinity}, which are axioms or axiom schemata that imply the existence of the set of all natural numbers. This assertion is called the \emph{Axiom of Infinity}\footnote{For a rigorous definition, see definition \bref{def:infinity} later in this section.}, but they also imply the existence of larger sets whose existence can not be proved in the current theory\footnote{For the purposes of this thesis, unless stated otherwise, this will be the \emph{Zermelo–Fraenkel set theory}, that is formally established in definition \bref{def:zfc}.}.

As we will show in chapter 2, reflection is closely related to the \emph{Axiom Schema of Replacement}, which was the subject of philosophical debates because it wasn't included in the original axiomatic set theory proposed by Zermelo and unlike other axioms in the \emph{Zermelo–Fraenkel set theory}, its presence is not justified from the iterative conception of a set, but rather from its usefulness. Unlike \emph{Replacement Schema}, reflection is not so easily questioned from a platonist\footnote{According to \emph{Stanford Encyclopedia of Philosophy}, ``mathematical platonism is the metaphysical view that there are abstract mathematical objects whose existence is independent of us and our language, thought, and practices. Just as electrons and planets exist independently of us, so do numbers and sets. And just as statements about electrons and planets are made true or false by the objects with which they are concerned and these objects' perfectly objective properties, so are statements about numbers and sets. Mathematical truths are therefore discovered, not invented.''} point of view, but it may be formulated in two different was. The two following informal interpretations of reflections are based on \cite{HellmanInfinite}. Their purpose is to illustrate the difference between a platonist and a structuralist\footnote{According to wikipedia, ``Structuralism is a theory in the philosophy of mathematics that holds that mathematical theories describe structures of mathematical objects. Mathematical objects are exhaustively defined by their place in such structures. Consequently, structuralism maintains that mathematical objects do not possess any intrinsic properties but are defined by their external relations in a system.''} approach towards reflection.
\begin{displayquote}
``The true situation (in the universe of sets) is reflected in arbitrarily high level of the cumulative hierarchy.''
\end{displayquote}
\begin{displayquote}
``We're interested in structures so large that certain attempts to describe them fail to distinguish them from various proper initial segments–hence small fragments–of them.''
\end{displayquote}
There is no point in dedicating more space to the philosophy of mathematics as it is outside the scope of this thesis, it is only worth noting that the author usually thinks of reflection in the latter sense which may be reflected in the way this thesis is written.

After introducing the elementary theoretical tools required for this task in the rest of this chapter, in chapter 2, we will review the \emph{Reflection Theorem} that originally formulated by Richard Montague in 1961\footnote{Note that Lévy's paper was published in 1960, a year before Montague's, but Lévy refers to Montague and not vice versa. While this may seem confusing, it is because Montague gave a lecture on this topic at a conference at the Cornell University in 1957. It is also interesting that Lévy's article refers for Montague's reflection to a publication by Montague and Solomon Feferman called \emph{The method of arithmetization and some of its applications} which was never finished. This is explained by Solomon Feferman in \cite{Feferman2008}.} and extended by Azriel Lévy in his aforementioned article and then restate it in a way that is more in line with today's set theory. 
This part of the thesis deals with the fact that when the term ``property'' is restricted to first–order formulas in the language of set theory, it does not behave like a axiom of strong infinity, but it is equivalent to the \emph{Axiom of Infinity} and \emph{Replacement Schema}, which is one of the key set–forming principles in the \emph{Zermelo–Fraenkel set theory}.

It is in chapter 3 where will examine some large cardinal properties and in a manner similar to Lévy's article, we will introduce axiom schemata that come from reflection and lead towards \emph{inaccessible} and \emph{Mahlo cardinals}. We will briefly argue that Mahlo's operation exhausts large cardinals reachable via reflection from below and introduce indescribable cardinals, which are also based on reflection, but lead us into higher–order logic. We will introduce \emph{weakly inaccessible cardinals} and show that they are also based on reflection and examine their relation to the cardinals presented earlier. Finally, we will examine Gödel's constructible universe and see whether the large cardinals we have introduced are compatible with the \emph{Axiom of Constructibility}, an assertion that every set is definable.

%\begin{displayquote}
%``The Universe of sets cannot be uniquely characterised (i. e. distinguished from all its initial elements) by any internal structural property of the membership relation in it, which is expressible in any logic of finite of transfinite type, including infinitary logics of any cardinal order.''
%\end{displayquote}
%\rightline{{\rm --- Kurt Gödel \cite{GodelWang}}}
