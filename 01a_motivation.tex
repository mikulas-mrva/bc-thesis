\subsection{Motivation and Origin}
\begin{displayquote}
``The Universe of sets cannot be uniquely characterised (i. e. distinguished from all its initial elements) by any internal structural property of the membership relation in it, which is expressible in any logic of finite of transfinite type, including infinitary logics of any cardinal order.''
\end{displayquote}
\rightline{{\rm --- Kurt Gödel \cite{GodelWang}}}
\begin{comment}
To understand why need reflection in the first place, let's think about infinity for a moment. In the intuitive sense, infinity is an upper limit of all numbers. 
But for centuries, this was merely a philosophical concept of limitlessness. % zdroj?
Probably the best-known classic problems involving infinity are the famous paradoxes of Zeno. % zdroj?
In response to those, Aristotle introduced the distinction between an actual and a potential infinity\footnote{See Aristotle’s Physics, Book III}. % presun do citaci
By potential infinity we understand that concept of a process does in unbounded in a sense that it could continue for an arbitrary amount of time, but is also never complete. % prepsat, asi citace z aristotela?
Actual infinity, is, on the other hand, the concept of infinity contained in a bounded space, just like the number of fractions between 0 and 1. Aristotle argued, that the potential infinity is (in today's words) well defined, as opposed to the actual infinity, which is a vague and incoherent concept. He didn't think it's possible for infinite amount of entities to inhabit a bounded place in space or time, rejecting Zeno's thought experiments as a whole. But it's not our aim to get into much detail.  % radsi citace?
The aspect of infinity that is relevant to our interests is the human inability to directly experience limitlessness in contrast to how easily can one talk about infinity and limitlessness in the natural language. 
The short trip into history hopefully served as an example of the fact that certain statements can easily be considered either meaningful or meaningless.  % co?
But while infinity of any kind can't be experienced directly through senses, much effort has been made by philosophers to find a way to meaningfully talk about infinite. 
To see how this leads to reflection, see what Aquinas wrote in his Summa Theologica \footnote{Part I, Question 7, Article 3, Reply to Objection 1}:
\begin{displayquote}
A geometrician does not need to assume a~line actually infinite, but takes some actually finite line, from which he subtracts whatever he finds necessary; which line he calls infinite.
\end{displayquote}
He seems to acknowledge, that infinity can not be reached directly, but for practical purposes it is enough to take a limited part of the whole. One can that act as if it was the whole because the part has all the properties needed at the moment. This, as we shall see in a moment, is in fact an instance of reflection, even though the term itself was introduced centuries later.
% dava to vubec smysl?

To illustrate this elusiveness of infinity, let us remember the early days of set theory. 
When Cantor proved that there are at least two distinct infinite quantities, this effectively turned what previously was an abstract, unreachable absolute, into a mathematical object, a set. 
But just as one infinity was seemingly tamed, about 10 years later, Russell's paradox uncovered the fact that there is another absolute, the paradoxical collection of all sets. 
% tohle asi neni uplne pravda krome ZFC:
Mathematicians have decided to focus on axiomatic set theories so that the paradoxical collection was kept out of sets, being considered a class instead \footnote{When we use the words "class" and "property" in this section, "property" refers to statement in natural or formal language that can be meaningfully stated for sets, the notion of class then refers to the collection of all sets holding that particular property. For all practical purposes, the two are synonyms. They will be later properly redefined for use in formal context.}
This is where reflection comes in again. 

The original idea behind reflection principles probably comes from what could be informally called \textquote{universality of the universe}.
Trying express the universe as a~set $\{x  |  x = x\}$, we either decide to make such statement on a meta-level, or directly in a theory that formalizes the concept of class\footnote{Like the Bernays–Gödel set theory, for example.}. 
But since it is practical to consider sets formed by a property, we must carefully formalize the notion of property of that we stay within the formal framework of a given theory. % to asi neni pravda, viz comprehension a repalcement, vlastnr 
Reflection can be seen as reverting this approach. Even thought we have, in a sense, included the infinity into the set theory in the form of $\omega$, the set of all natural numbers, as well as the hierarchy of larger infinite sets constructed from $\omega$, there is still an unreachable absolute.
Since we have weakened the notion of property so that it only yields sets, there is obviously no way to directly describe the whole universe, every attempt to do so inevitably fails.

% precti si neco poradne o platonismu
If one was was to hold a platonistic view on the philosophy of mathematics, assuming that the sets themselves objectively exist, reflection, the fact that every description of the universe collapses to a bounded object within the universe, can be percieved as the imperfection of formal systems. % nekdo to rikal, NAJDI TO! Drake?
Similarly, while Gödel's second theorem implies that no formal system\footnote{To be more precise, no formal system satisfying specific properties, see Gödel's results for details.} proves everything, one might argue that some the independed statements objectively hold, but the system is not strong enough to verify the fact. 
Speaking of Gödel, it is worth noting that reflection made its first in set-theoretical appearance in Gödel's proof of the Generalised Continuum Hypothesis in the constructible universe L, but it was around even earlier as a~concept. 
Gödel himself regarded it as very close to Russell's reducibility axiom (an earlier equivalent of the axiom schema of separation proposed by Zermelo). %
Richard Montague then studied reflection properties as a~tool for verifying that Replacement is not finitely axiomatizable. a~few years later Lévy proved in \cite{Levy60a} the equivalence of reflection with Axiom of infinity together with Replacement in proof we shall examine closely in chaper 2.
% konec korektury
From this point of view, we can argue that $\omega$ was established as an object satisfying a property attributed informally to the universe of all sets. That is, the property of "being the collection of all natural numbers". But since there was no was to reach it from below, it had to be explicitly brought into existence\footnote{Existence as in "the theory knows that $\omega$ exists".} by the axiom of infinity. 

\

The purpose of the previous paragraphs were formulated from a naive platonistic view, which makes it easy to talk about the universe of all sets, even though the formulations are very informal. Now it should be made clear, that one does not need to informally talk about the universe of all sets or any other ideal objects as if they inhabited an ideal world. We will look at the theory from a structuralist point of view, inspired by Hilbert, Shapiro and Geoffrey Hellman. This allows us to dismiss questions about objective existence of objects beyond mathematical theories. Instead, we can consider objects  meaningful if and only if there is a consistent\footnote{Thanks to Gödel, we only need to care whether it's consistent relatively to the axiomatic set theory of Zermelo and Fraenkel.} formal system which admits the existence of such objects. Starting with $\omega$, a set of all finite ordinals\footnote{See the next section for the exact definition. Until then, finite ordinals are synonymous to natural numbers.}, which is then extended to the Von Neumann's hierarchy $V$, we will examine axioms that can consistently be added to the Zermelo–Fraenkel theory so that the hierarchy is extended to a larger model that contains the previous one as an initial segment. To see why this is also reflection, one should bear in mind that we will create models that reflect certain properties.

Later we will see that large cardinals discussed in this paper are natural extension of this process. We can informally obtain the inaccessible cardinal by arguing the the axioms of the Zermelo–Fraenkel set theory hold in the universe and establishing an object, let's call it $\kappa_I$ for now, that satisfies this property. But then this either leads to a stronger set theory exntended in order to be able to talk about $\kappa_I$. But this process iterates for the new set theory to yield Mahlo cardinals, but it is also clear that even by arbitrary iterations of this principle, the universe still can't be reached. On the final pages of this thesis, we will argue, that while this iterative process seems to lead to a very large object, it is in fact not strong enough to be inconsistent with Gödel's L, which was designed to be the minimal model of set the Zermelo–Fraenkel set theory in terms of width\footnote{The model minimal in terms if height of the universe is the inaccessible cardinal.}. Present day set theory is able to consider many large cardinals far above the hierarchy introduced here. Even though those are beyond the scope of this work, we will briefly mention the measurable cardinal later on.

\end{comment}