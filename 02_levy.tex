\section{Levy's First-Order Reflection}\label{sec:first_order}

\subsection{Lévy's Original Paper}\label{sec:levy1960}
This section is based on Lévy's paper \emph{Axiom Schemata of Strong Infinity in Axiomatic Set Theory} \cite{Levy60a} % hnusna citace!
from 1960. It presents Lévy's general reflection principle and its equivalence to \emph{Replacement} and \emph{Infinity} under $\sf{S}$\footnote{See definition (\ref{def:s}).}.

First, we should point out that set theory has changed over the last 66 years, we will now point out a few notable differences. When reading Lévy's article, one should bear in mind that while the author often speaks about a~model of $\sf{ZF}$, usually denoted $u$, it doesn't necessarily mean that there is a set $u$ that is a model of $\sf{ZF}$\footnote{This is indeed impossible to prove in $\sf{ZF}$ due to Gödel's Incompleteness.}, nowadays we are used to using the notion of universal class $V$ in similar sense, albeit independently of a particular axiomatic theory. We will review the exact meaning of the notion of a standard complete model in a moment.
The theory $\sf{ZF}$ is almost identical to the theory we have established in (\ref{def:zf}).
One might be confused by the fact that Lévy treats the \emph{Subsets} axiom, which is in fact \emph{Specification} as a single axiom rather than a schema. He even takes the conjunction of all axioms of $\sf{ZF}$ and treats it like a formula.  This is possible because the underlying logic calculus is different. Lévy works with set theories formulated in the \emph{non-simple applied first order functional calculus}, see Chapter IV in \cite{church1996introduction} for details. For now, we only need to know that the calculus contains a substitution rule for functional variables. This way, \emph{Subsets} is de facto a schema even though it sometimes treated as a single formula.
% todo koukni do churche jak se to s tim ma
It should also be noted that the logical conectives look difference. The usual symbol for universal quantifier does not appear, $\forall x \varphi (x)$ would be written as $(x) \varphi (x)$. The symbol for negation is "$\sim$", implication is written as "$\supset$" and equivalence is "$\equiv$". We will use standard notation with "$\neg$", "$\then$" and "$\iff$" respectively when presenting Lévy's results.

%The following definitions are not used in contemporary set theory, but they illustrate 1960's set theory mind-set and they are used heavily in Lévy's text, so we will include and explain them for clarity. 
%Generally in this chapter, $\sf{Q}$ stands for an arbitrary axiomatic set theory. % used for general definitions, $u$ is usually a model of $\sf{Q}$, counterpart of today's the universal class $V$.

\

This subsection uses $\sf{ZF}$ instead of the usual $\sf{ZFC}$ as the underlying theory. % neni to zbytecny?

%For the following definition, let us add that Lévy didn't consider the axiom of subsets a schema, it was formulated as $\forall x \exists y \forall z (z \in y \iff z \in x \et p(z))$. A substitution rule included in the logic so that the $p(z)$ function in the axiom can be substituted for any formula. 

\begin{definition}{(Standard Complete Model of a Set Theory)}\label{def:scm_q}\\
Let $\sf{Q}$ be an arbitrary set theory given. % and let $e$ be a binary relation on u. 
We say that that $u$ is a standard complete model of $\sf{Q}$, which is usually written as $Scm^{\sf{Q}}(u)$, iff
\bce[(i)]
\item $(\forall \sigma \in \sf{Q})(u \models \sigma)$
\item $\forall y (y \in u \then y \subset u)$
% \item $\forall e \langle x, y \rangle \in e \iff (y \in u \et x \in y)$ % co je $e$? omg je to jinak, je to model vsech axiomu pro vsechny $e$!!! relatiizace do u, e?
\ece 
\end{definition}

\begin{definition}{(Inaccessible Cardinal With Respect to $\sf{Q}$)}\label{def:levy_inaccessible_q}\\
Let $\sf{Q}$ be an arbitrary axiomatic first-order set theory. We say that a cardinal $\kappa$ is inaccessible with respect to $\sf{Q}$, we write $In^{\sf{Q}}(\kappa)$.
\begin{equation}
In^{\sf{Q}}(\kappa) \defeq Scm^{\sf{Q}}(V_\kappa).
\end{equation}
\end{definition}

\begin{definition}{(Inaccessible Cardinal With Respect to $\sf{ZF}$)}\label{def:levy_inaccessible}\\
When a cardinal $\kappa$ is inaccessible with respect to $\sf{ZF}$, we only say that it is inaccessible. We write $In(\kappa)$.
\begin{equation}
In(\kappa) \defeq In^{\sf{ZF}}(\kappa)
\end{equation}
\end{definition}
The above definition of inaccessibles is used because it doesn't require \emph{Choice}.

For the definition of relativization, see (\ref{def:relativization}). The notation used by Lévy is "$Rel(u, \varphi)$", we will stick to "$\varphi^{u}$".
\begin{definition}{($N$)}\label{def:levy_axiom_n}\\
The following is an axiom schema of complete reflection over $\sf{ZF}$, denoted as $N$:
\begin{equation}
\exists u (Scm^{\sf{ZF}}(u) \et \forall x_1, \ldots , x_n (x_1, \ldots , x_n \in u \then \varphi \iff \varphi^{u}))
\end{equation}
where $\varphi$ is a~formula which contains no free variables except for $x_1, \ldots , x_n$.
\end{definition}

\begin{definition}{($N_0$)}\label{def:levy_axiom_n0}\\
The following is almost identical to axiom schema $N$, but with $\sf{S}$ instead of $\sf{ZF}$. We will call it $N_0$:
\begin{equation}
\exists u (Scm^{\sf{S}}(u) \et \forall x_1, \ldots , x_n (x_1, \ldots , x_n \in u \then \varphi \iff \varphi^{u}))
\end{equation}
where $\varphi$ is a~formula which contains no free variables except for $x_1, \ldots , x_n$.
\end{definition}

Let $\sf{S}$ be an axiomatic set theory defined in (\ref{def:s}). We will now show that in $\sf{S}$, $N_0$ implies both \emph{Replacement} and \emph{Infinity}.

\

Let $N_0$ be defined as in (\ref{def:levy_axiom_n0}), for \emph{Infinity} see (\ref{def:infinity}).
\begin{theorem}\
In $\sf{S}$, the schema $N_0$ implies \emph{Infinity}.
\end{theorem}

\begin{proof}
Let $\varphi = \forall x \exists y (y = x \cup \{x\})$. This clearly holds in $\sf{S}$ because given any set $x$, we can always obtain the set $x \cup \{x\}$ via \emph{Powerset} and \emph{Specification}.
From $N_0$, there then exists a set $u$ such that $\varphi^{u}$ holds. This $u$ satisfies the conditions required by \emph{Infinity}, so we're done.
\end{proof}

\begin{theorem} % todo zkontrolovat ==========================================================================================
In $\sf{S}$, the schema $N_0$ implies \emph{Replacement}.\\
TODO jedno nebo druhe % nebo treti
\beq
\sf{S} + N_0 \vdash \mbox{\emph{Replacement}}
\eeq % nebo S \proves N_0 \then Replacement % (veta o dedukci?)
\end{theorem}

\begin{proof}
Let $\varphi(x, y, p_1, \ldots, p_n)$ be a~formula with no free variables except $x, y, p_1, \ldots, p_n$ for an arbitrary natural number $n$.

\begin{equation}
\begin{gathered}
\chi = \forall x, y, z(\varphi(x, y, p_1, \ldots, p_n) \et \varphi(x, z, p_1, \ldots, p_n) \then y = z) \\
\then \forall x \exists y \forall z (z \in y \iff \exists q (q \in x \et \varphi(q, z, p_1, \ldots, p_n)))
\end{gathered}
\end{equation}
Let $\chi$ be an instance of \emph{Replacement} schema for the above $\varphi$. Let the following formulas be instances of the $N_0$ schema for formulas  $\varphi$, $\exists y \varphi$, $\chi$ and $\forall x, p_1, \ldots, p_n \chi$ respectively:
\bce[(i)]
\item $x, y, p_1, \ldots, p_n \in u \then (\varphi \iff \varphi^{u}) $
\item $x, p_1, \ldots, p_n \in u \then (\exists y \varphi \iff (\exists y \varphi)^{u})$
\item $x, p_1, \ldots, p_n \in u \then (\chi \iff \chi^{u})$
\item $\forall x, p_1, \ldots, p_n (\chi \iff (\forall x, p_1, \ldots, p_n \chi)^{u})$
\ece

From relativization, we also know that $(\exists y \varphi)^{u}$ is equivalent to $(\exists y \in u) \varphi^{u}$.
Therefore $(ii)$ is equivalent to
\begin{equation}
x, p_1, \ldots, p_n \in u \then (\exists y \in u) \varphi^{u}. 
\end{equation}

If $\varphi$ is a~function\footnote{See definition (\ref{def:function})}, then for every $x \in u$, which is also $x \subset u$ by the transitivity of $Scm^{\sf{S}}(u)$,
it maps elements of $x$ onto $u$. From the axiom scheme of comprehension\footnote{Lévy uses its equivalent, axiom of subsets}, we can find $y$, a~set of all images of elements of $x$.
That gives us $x, p_1, \ldots, p_n \in u \then \chi$. By $\bold{(iii)}$ we get $x, p_1, \ldots, p_n \in u \then \chi^{u}$, the universal closure of this formula is $(\forall x, p_1, \ldots, p_n \chi)^{u}$, 
which together with $\bold{(iv)}$ yields $\forall x, p_1, \ldots, p_n \chi$. Via universal instantiation, we end up with $\chi$. We have inferred replacement for a given arbitrary formula. 
\end{proof}

What we have just proven is just a single theorem from the above mentioned article by Lévy, we will introduce other interesting propositions, mostly related to the existence of large cardinals, later in their appropriate context in chapter 3.

% =====================================================================================================================================

\subsection{Contemporary Restatement}
We will now a theorem that is referred to as Lévy's Reflection in contemporary set theory. The only difference is that while Lévy reflects $\varphi$ from $V$ to a set $u$ which is a \emph{standard complete model of $\sf{S}$}, we say that there is a $V_\alpha$ for a limit $\alpha$ that reflects $\varphi$. Those two conditions are equivalent due to lemma (\ref{lemma:scm_s_is_limit}).

%\begin{definition}{(Reflection\textsubscript{1})}\label{def:reflection_1}\\  % co spis Levy's reflection principle?
%Let $\varphi(p_1, \ldots, p_n)$ be a first-order formula in the language of set theory. Than the following holds for any such $\varphi$.
%\begin{equation}
%\forall M_0 \exists M (M_0 \subseteq M \et (\varphi^M(p_1, \ldots, p_n) \iff \varphi(p_1, \ldots, p_n)))
%\end{equation}
%\end{definition}

% Note that this is a restatement of both Lévy's $N$ and $N_0$ from the previous chapter\footnote{see (\ref{def:levy_axiom_n}) and (\ref{def:levy_axiom_n0}}). We prefer to call it \emph{Reflection\textsubscript{1}} so it complies with how other axioms and schemata are named. \footnote{We will not use the name $N_0$, because it might be confusing to work $N_0$ and $M_0$ where $M_0$ is a set and $N_0$ is an axiom schema.} 
% Note that the subscript 1 refers to the fact that $\varphi(p_1, \ldots, p_n)$ is a first-order formula, and since we're using the word "reflection" in less strict meaning throughout this thesis, distinguishing between the two just by using italic font face for the schema might cause confusion.

% We will now prove the equivalence of \emph{Reflection\textsubscript{1}} with \emph{Replacement} and \emph{Infinity} in $\sf{S}$ in two parts. First, we will show that \emph{Reflection\textsubscript{1}} is a theorem of $\sf{ZFC}$, then we shall show that the second implication, which proves \emph{Infinity} and \emph{Replacement} from \emph{Reflection\textsubscript{1}}, also holds.

% The following lemma is usually done in two parts, the first being for one formula, the other for $n$ formulas. We will only state and prove the more general version.

\begin{lemma}\label{lemma:reflection_lemma}\
Let $\varphi_1, \ldots, \varphi_n$ be first-order formulas in the language of set theory, all with $m$ free variables\footnote{For formulas with a different number of free variables, take for $m$ the highest number of parameters among those formulas. Add spare parameters to every formula that has less than $m$ parameters in a way that preserves the last parameter, which we will denote $x$. E.g. let $\varphi'_i$ be the a~formula with $k$ parameters, $k < m$. Let us set $\varphi_i(p_1, \ldots, p_{m-1}, x) \defeq \varphi'_i(p_1, \ldots, p_{k-1}, x)$, notice that the parameters $p_k, \ldots, p_{m-1}$ are not used.}.
\bce[(i)]
\item For each set $M_0$ there is such set $M$ that $M_0 \subset M$ and the following holds for every $i$, $1 \leq i \leq n$:
\begin{equation}\label{equation:refl_lemma_i}
\exists x \varphi_i(p_1, \ldots, p_{m-1}, x) \then (\exists x \in M) \varphi_i(p_1, \ldots, p_{m-1}, x)
\end{equation}
for every $p_1, \ldots, p_{m-1} \in M$.

\item Furthermore, there is an ordinal $\alpha$ such that $M_0 \subset V_\alpha$ and the following holds for each $i$, $1 \leq i \leq n$:
\begin{equation}\label{equation:refl_lemma_ii}
\exists x \varphi_i(p_1, \ldots, p_{m-1}, x) \then (\exists x \in V_\alpha) \varphi_i(p_1, \ldots, p_{m-1}, x)
\end{equation}
for every $p_1, \ldots, p_{m-1} \in M$.

\item Assuming \emph{Choice}, there is $M$, $M_0 \subset M$ such that (\ref{equation:refl_lemma_i}) holds for every $M,\ i \leq n$ and $|M| \leq |M_0| \cdot \aleph_0$.
\ece
\end{lemma}

\begin{proof}
We will simultaneously prove statements \emph{(i)} and \emph{(ii)}, denoting $M^T$ the transitive set required by part \emph{(ii)}.
Steps in the construction of $M^T$ that are not explicitly included are equivalent to steps for $M$.

Let us first define an operation $H_i(p_1, \ldots, p_{m-1})$ that yields the set of $x$'s with minimal rank\footnote{Rank is defined in (\ref{def:rank})} satisfying $\varphi_i(p_1, \ldots, p_{m-1}, x)$ for $p_1, \ldots, p_{m-1}$ and for every $i$, $1 \leq i \leq n$.

\begin{equation}
H_i(p_1, \ldots, p_n) = \{x \in C_i: (\forall z \in C)(rank(x) \leq rank(z))\}
\end{equation}
for each $1 \leq i \leq n$, where
\begin{equation}
C_i = \{x: \varphi_i(p_1, \ldots, p_{m-1}, x)\} \mbox{ for $1 \leq i \leq n$}
\end{equation}

\

Next, let's construct $M$ from given $M_0$ by induction. 
\begin{equation}
M_{i+1} = M_i \cup \bigcup_{j=0}^{n} \bigcup \{H_j(p_1, \ldots, p_{m-1}): p_1, \ldots, p_{m-1} \in M_i\}
\end{equation}
In other words, in each step we include into the construction the elements satisfying $\varphi(p_1, \ldots, p_{m-1}, x)$ for $p_1, \ldots, p_{m-1}$ from the previous step.
For statement \emph{(ii)}, this is the only part that differs from \emph{(i)}. To end up with a transitive $M$, we need to extend every step to it's transitive closure transitive closure of $M_{i+1}$ from \emph{(i)}. In other words, let $\gamma$ be the smallest ordinal such that 
\begin{equation}
(M^T_i \cup \bigcup_{j=0}^{n} \{\bigcup\{H_j(p_1, \ldots, p_{m-1}): p_1, \ldots, p_{m-1} \in M_i\}\}) \subset V_\gamma
\end{equation}
Then the incremental step is
\begin{equation}
M^T_{i+1} = V_\gamma
\end{equation}
and the final $M$ is obtained by joining the previous steps.
\begin{equation}
M = \bigcup_{i=0}^{\infty} M_i, \mbox{  }M^T = \bigcup_{i=0}^{\infty} M^T_i = V_\alpha
\end{equation}

\

We have yet to finish part \emph{(iii)}.
Let's try to construct a~set $M'$ that satisfies the same conditions like $M$ but is kept as small as possible. Assuming the Axiom of Choice, we can modify the process so that the cardinality of $M'$ is at most $|M_0| \cdot \aleph_0$. Note that the size of $M$ in the previous construction is determined by the size of $M_0$ and, most importantly, by the size of $H_i(p_1, \ldots, p_{m-1})$ for every $i$, $1 \leq i \leq n$ in individual iterations of the construction. Since \emph{(i)} only ensures the existence of an $x$ that satisfies $\varphi_i(p_1, \ldots, p_{m-1}, x)$ for any $i$, $1 \leq i \leq n$, we only need to add one $x$ for every set of parameters but $H_i(u_1, \dots, u_{m-1})$ can be arbitrarily large. Let $F$ be a~choice function on $\power{M'}$. Also let $h_i(p_1, \ldots, p_{m-1}) = F(H_i(p_1, \ldots, p_{m-1}))$ for $i$, where $1 \leq i \leq n$, which means that $h$ is a~function that outputs an $x$ that satisfies $\varphi_i(p_1, \ldots, p_{m-1}, x)$ for $i$ such that $1 \leq i \leq n$ and has minimal rank among all such sets. The induction step needs to be redefined to
\begin{equation}
M'_{i+1} = M'_i \cup \bigcup_{j=0}^n \{ H_j(p_1, \ldots, p_{m-1}): p_1, \ldots, p_{m-1} \in M'_i \}
\end{equation}
This way, the amount of elements added to $M'_{i+1}$ in each step of the construction is the same as the amount of $m$-tuples of parameters that yielded elements not included in $M'_i$. It is easy to see that if $M_0$ is finite, $M'$ is countable because it was constructed as a countable union of at most countable sets. If $M_0$ is countable or larger, the cardinality of $M'$ is equal to the cardinality of $M_0$.\footnote{It can not be smaller because $|M'_{i+1}|  \geq |M'_i|$ for every $i$. It may not be significantly larger because the maximum of elements added is the number of $n$-tuples in $M'_i$, which is of the same cardinality is $M'_i$.}
Therefore $|M'| \leq |M_0| \cdot \aleph_0$
\end{proof}

\begin{theorem}{(Lévy's first-order reflection theorem)}\label{theorem:first_order_reflection}\\
Let $\varphi(p_1, \ldots, p_n)$ be a~first-order formula.
\bce[(i)]
\item For every set $M_0$ there exists $M$ such that $M_0 \subset M$ and the following holds:
\begin{equation}
\varphi^M(p_1, \ldots, p_n) \iff \varphi(p_1, \ldots, p_n)\label{equation:levy_theorem_i}
\end{equation}
for every $p_1, \ldots, p_n \in M$.

\item For every set $M_0$  there is a~transitive set $M$, $M_0 \subset M$ such that the following holds:
\begin{equation}
\varphi^M(p_1, \ldots, p_n) \iff \varphi(p_1, \ldots, p_n)
\end{equation}
for every $p_1, \ldots, p_n \in M$.

\item For every set $M_0$ there is $\alpha$ such that $M_0 \subset V_{\alpha}$ and the following holds:
\begin{equation}
\varphi^{V_{\alpha}}(p_1, \ldots, p_n) \iff \varphi(p_1, \ldots, p_n)
\end{equation}
for every $p_1, \ldots, p_n \in M$.

\item Assuming \emph{Choice}, for every set $M_0$ there is $M$ such that $M_0 \subset M$ and $|M| \leq |M_0| \cdot \aleph_0$ and the following holds:
\begin{equation}
\varphi^M(p_1, \ldots, p_n) \iff \varphi(p_1, \ldots, p_n)
\end{equation}
for every $p_1, \ldots, p_n \in M$.
\ece
\end{theorem}

\begin{proof}
% CO ty parametry? \varphi(p_1, \ldots, p_n) nebo jen \varphi ?
%Before we start, note that the following holds for any set $M$ if $\varphi$ is an atomic formula(x1 ∈ x2 or x1 = x2), as a direct consequence of relativisation to $M, \in$\footnote{See \ref{def:relativization}. Also note that this only holds for relativization to $M, \in$, not $M, E$ for arbitrary $E$.}. 
%\begin{equation}
%\varphi \iff \varphi^M
%\end{equation}

Let's now prove \emph{(i)} for given $\varphi$ via induction by complexity. We can safely assume that $\varphi$ contains no quantifiers besides "$\exists$" and no logical connectives other than "$\neg$" and "$\et$".
Let $\varphi_1, \ldots, \varphi_n$ be all subformulas of $\varphi$. Then there is a set $M$, obtained by the means of lemma \ref{lemma:reflection_lemma}, for all of the formulas $\varphi_1, \ldots, \varphi_n$. 

Let's first consider atomic formulas in the form of either $x_1 = x_2$ or $x_1 \in x_2$. % preformulovat trochu
It is clear from relativisation\footnote{See \ref{def:relativization}. This only holds for relativization to $M, \in$, not $M, E$ for an arbitrary relation $E$.} that \ref{equation:levy_theorem_i} holds for both cases.

\bce[(i)]
\item $(x_1 = x_2)^M \iff (x_1 = x_2)$
\item $(x_1 \in x_2)^M \iff (x_1 \in x_2)$
\ece

\

We now want to verify the inductive step. First, take $\varphi = \neg \varphi'$. From the relativization, we get
\begin{equation}
(\neg \varphi')^M \iff \neg (\varphi'^M)
\end{equation}
Because the induction hypothesis tell us that $\varphi'^M \iff \varphi'$, the following holds:
\begin{equation}
(\neg \varphi')^{M} \iff \neg (\varphi'^M) \iff \neg \varphi'
\end{equation}

The same holds for $\varphi = \varphi_1 \et \varphi_2$. From the induction hypothesis, we know that $\varphi_1^M \iff \varphi_1$ and $\varphi_2^M \iff \varphi_2$, which together with relativization for formulas in the form of $\varphi_1 \et \varphi_2$ gives us
\begin{equation}
(\varphi_1 \et \varphi_2)^M \iff \varphi_1^M \et \varphi_2^M \iff \varphi_1 \et \varphi_2
\end{equation}

\
% kde jsem tu najednou vzal parametry? TODO
Let's now examine the case when, from the induction hypethesis, $M$ reflects $\varphi'(p_1, \ldots, p_n, x)$ and we are interested in $\varphi = \exists x \varphi'(p_1, \ldots, p_n, x)$.
The induction hypothesis tells us that 
\begin{equation}
\varphi'^M(p_1, \ldots, p_n, x) \iff \varphi'(p_1, \ldots, p_n, x)
\end{equation}
so, together with above lemma \ref{lemma:reflection_lemma}, the following holds:
\begin{equation}
\begin{gathered}
\varphi(p_1, \ldots, p_n, x) \\
\iff \exists x \varphi'(p_1, \ldots, p_n, x) \\
\iff (\exists x \in M) \varphi'(p_1, \ldots, p_n, x) \\
\iff (\exists x \in M) \varphi'^M (p_1, \ldots, p_n, x) \\
\iff (\exists x \varphi'(p_1, \ldots, p_n, x))^M \\
\iff \varphi^M(p_1, \ldots, p_n, x)
\end{gathered}
\end{equation}
Which is what we wanted to prove for part (i). %\ref{equation:levy_theorem_i} holds for all subformulas $\varphi_1, \ldots, \varphi_n$ of a given formula $\varphi$.

\

%So far we have proven part $\bold{(i)}$ of this theorem for one formula $\varphi$. 
We now need to verify that the same holds for any finite number of formulas $\varphi_1, \ldots, \varphi_n$. 
This has in fact been already done since lemma \ref{lemma:reflection_lemma} gives us a set $M$ for any (finite) amount of formulas and given $M_0$. We can therefore find a set $M$ for the union of all of their subformulas. When we obtain such $M$, it should be clear that it also reflects every formula in $\varphi_1, \ldots, \varphi_n$.

\

Since $V_\alpha$ is a~transitive set, by proving $\bold{(iii)}$ we also satisfy $\bold{(ii)}$. To do so, we only need to look at part $\bold{(ii)}$ of lemma \ref{lemma:reflection_lemma}. All of the above proof also holds for $M = V_\alpha$. 

To finish part $\bold{(iv)}$, we take $M$ of size $\leq |M_0| \cdot \aleph_0$, which exists due to part $\bold{(iii)}$ of lemma \ref{lemma:reflection_lemma}, the rest being identical.
\end{proof}

% TODO existuje jich dokonce club set!
% viz http://ozark.hendrix.edu/~yorgey/settheory/15-reflection-principle.pdf


\

Let $\sf{S}$ be a set theory defined in \ref{def:s}, for $\sf{ZFC}$ see \ref{def:zfc}.

% ACHTUNG
% Sm je definovany jako konjunkce vsech relativizovanych axiomu, takze to je kruh...
% presunout do contemporary restatement?
% viz Levy str. (224)
% todo s/u/V_lambda
% DRAKE!!! ch.3, ch.4 dukaz v_alfa models ZFC pro limitni alfa bez nekonecna etc
% Drake dokazuje reflexi v ch.3 par.6.3

The two following lemmas are based on \cite{DrakeBook}[Chapter 3, Theorem 1.2].
\begin{lemma}\label{lemma:extensionality_in_transitive} % Drake ch.3 Theorem 1.2
Iff $M$ is a transitive set, then $M \models \mbox{\emph{Extensionality}}$.
\end{lemma}

\begin{proof}
Given a transitive set $M$, we want to show that the following holds.
\beq
M \models \forall x, y (x = y \iff \forall z (z \in x \iff z \in y))
\eeq % TODO pozor na definici splnovani!
% From satisfaction, we get that for every $x$, $y$, the following holds % neni v definici splnovani, ze volny musej platit vsechny?
Given arbitrary sets $x, y \in M$, we want to prove
\beq
M \models (x = y \iff \forall z (z \in x \iff z \in y))
\eeq
This is equivalent to % ref na definici pravdy
\beq
M \models x = y \mbox{ iff } M \models \forall z(z \in x \iff z \in y)
\eeq
Which is the same as 
\beq
x = y \mbox{ iff } M \models \forall z(z \in x \iff z \in y)
\eeq
So all elements of $x$ are also elements of $y$ in $M$, and vice versa.  Because $M$ is transitive, all elements of $x$ and $y$ are in $M$, so $M \models \forall z(z \in x \iff z \in y)$ holds iff $x$ and $y$ contain the same elements and are therefore equal.
\end{proof}

\begin{lemma}\label{lemma:foundation_in_transitive}
If $M$ is a transitive set, then $M \models \mbox{\emph{Foundation}}$.
\end{lemma}

\begin{proof}
We want to prove
\beq
M \models \forall x (x \neq \emptyset \then (\exists y \in x) (x \cap y = \emptyset))
\eeq

Given an arbitrary $x \in M$ such that $x \neq \emptyset$, we want to prove
\beq
M \models (\exists y \in x) (x \cap y = \emptyset)
\eeq

Because $M$ is transitive, every element of $x$ is an element of $M$. Take for $y$ the element of $x$ with the lowest rank. It should be clear that there is no $z \in y$ such that $z \in x$, because then $rank(z) < rank(y)$, which is a contradiction.
TODO is rank defined? % TODO !!!!!!!
\end{proof}

Let $\sf{S}$ be a set theory as defined in (\ref{def:s}). %We will first prove a lemma to show what's mentioned as obvious in \cite{Levy60a} and that is a fact, that any set $u$ such that $Scm^{\sf{S}}(u)$ is a limit ordinal.
\begin{lemma}\label{lemma:scm_s_is_limit}\
The following holds for every $\lambda$.
\begin{equation}
\mbox{"$\lambda$ is a limit ordinal"} \then V_\lambda \models \sf{S} % proc ne?
\end{equation}
\end{lemma}

\begin{proof}
Given an arbitrary limit ordinal $\lambda$, we will verify the axioms of \sf{S} one by one.
\bce[(i)]
\item \emph{The existence of a set} comes from the fact that $V_\lambda$ is a non-empty set because limit ordinal is non-zero.
% Jech 12.10, 12.11

% Drake ch.3, par.1.2 ! "extensionalita a foundation plati v kazde tranzitivni mnozine" -- mozna jako lemma? odkaz na draka?
\item \emph{Extensionality} holds from \ref{lemma:extensionality_in_transitive}

\item \emph{Foundation} holds from \ref{lemma:foundation_in_transitive}

% ==================================== POTUD zkontrolovano ======================================================

\item \emph{Union}:\\ % TODO kontrola % separatni lemma, ze plati v kazdem V_\alpha?
(see (\ref{def:union}))
\begin{equation}
\forall x \exists y \forall z (z \in y \iff \exists q( z \in q \et q \in x))
\end{equation}
Given any $x \in V_\lambda$, we want verify that $y = \bigcup x$ is also in $V_\lambda$. Note that $y = \bigcup x$ is a $\Delta_0$-formula.
\beq
y = \bigcup x \iff (\forall z \in y)(\exists q \in x) z \in q \et (\forall z \in x)(\forall q \in z) q \in y
\eeq
So by lemma (\ref{lemma:delta_0_absoluteness})
\beq
y = \bigcup x \iff (y = \bigcup x)^{V_\lambda}
\eeq
% asi ok ^

\item \emph{Pairing}:\\ % TODO kontrola
(see (\ref{def:pairing}))
\beq
\forall x, y \exists z \forall q (q \in z \iff q = x \lor q = y)
\eeq
Given two sets $x, y \in V_\lambda$, we want to show that $z$, defined as $z = \{x, y\}$, is also an element of $V_\lambda$.
\beq
z = \{x, y\} \iff x \in z \et y \in z \et (\forall q \in z)(q = x \lor q = y)
\eeq
So $z = \{x, y\}$ is a $\Delta_0$-formula, and thus by lemma (\ref{lemma:delta_0_absoluteness}) it holds that
\beq
z = \{x, y\} \iff (z = \{x, y\})^{V_\lambda}
\eeq
% asi ok ^


\item \emph{Powerset}: \\ (see (\ref{def:powerset})) % TODO kontrola
\begin{equation}
\forall x \exists y \forall z (z \subseteq x \iff z \in y)
\end{equation}
Given $x \in u$, we want to make sure that $\power{x} \in u$. Let $\varphi$ denote the formula $y \in \power{x} \iff y \subset x$.
We know that $y \subset x$ is $\Delta_0$ according to definition (\ref{def:subset}). We also know that given $x$, $\varphi$ holds for every $y$ due to the definition of $\power{x}$.
That means that $\varphi \iff \varphi^u$ and therefore we can conclude that $u \models \varphi$.

\item \emph{Specification}: \\ %zkontrolovat, podle Draka
Given a first-order formula $\varphi$, we want to show the following
\beq
V_\lambda \models \forall x, p_1, \ldots, p_n, \exists y \forall z (z \in y \iff z \in x \et \varphi(z, p_1, \ldots, p_n))
\eeq
Given any $x$ along with parameters $p_1, \ldots, p_n$, we set $y = \{z \in y : \varphi^{V_\lambda}(z, p_1, \ldots, p_n) \}$. From transitivity of $V_\lambda$ and the fact that $y \subset x$ and $x \in V_\lambda$, we can conclude that $y \in V_\lambda$, so $V_\lambda \models \forall z (z \in y \iff z \in x \et \varphi(z, p_1, \dots, p_n))$.
\ece
\end{proof}

Let \emph{Infinity} and \emph{Replacement} be as defined in \ref{def:infinity} and \ref{def:replacement} respectively.

%TODO citace! http://ozark.hendrix.edu/~yorgey/settheory/13-SI.pdf
% http://ozark.hendrix.edu/~yorgey/settheory/12-relative-consistency-2.pdf 
%-- jako konzistence ZF-reg -- ukazeme ze V jakozto sjednoveni V_alpha je model?

% TODO v kanamorim? co dukaz?
\begin{definition}{(First-Order Reflection Schema)}\\ % wtf, pro jedu nebo kolik formuli?
Let $\varphi_1, \ldots, \varphi_n$ be first-order formulas in the language of set theory.
For each set $M_0$ there is such set $M$ that $M_0 \subset M$ and the following holds for every $i$, $1 \leq i \leq n$:
\begin{equation}\label{equation:refl_lemma_i}
\exists x \varphi_i(p_1, \ldots, p_{m-1}, x) \then (\exists x \in M) \varphi_i(p_1, \ldots, p_{m-1}, x)
\end{equation}
for every $p_1, \ldots, p_{m-1} \in M$.
\end{definition}
% elegantnejsi by bylo formulovat to pro jednu formuli a pak ukazat, ze se to totez.



\begin{theorem}\label{theorem:levy_equivalence_contemporary}
\emph{Reflection}\textsubscript{1} is equivalent to \emph{Infinity} $ \et $ \emph{Replacement} under $\sf{S}$.
\end{theorem}
% lemma:model_of_s?
\begin{proof}
Since \ref{theorem:first_order_reflection} already gives us one side of the implication, we are only interested in showing the converse which we shall do in two parts:

$\bold{\emph{Reflection\textsubscript{1}} \then \emph{Infinity}}$
From \emph{Reflection\textsubscript{1}}, we know that for any first-order formula $\varphi$ and a set $M_0$, there is a $M$ such that $M_0 \subseteq M$ and $\varphi^M \iff \varphi$. Let's pick \emph{Powerset} for $\varphi$, then by \emph{Reflection\textsubscript{1}} there is a set that satisfies \emph{Powerset}, ergo there is a strong limit cardinal, which in turn satisfies \emph{Infinity}.
% WTF
\

$\bold{\emph{Reflection} \then \emph{Replacement}}$

Given a~formula $\varphi(x, y, p_1, \ldots, p_n)$, we can suppose that it is reflected in any $M$ \footnote{Which means that for $x, y, p_1, \ldots, p_n \in M$, $\varphi^M(x, y, p_1, \ldots, p_n) \iff \varphi(x, y, p_1, \ldots, p_n)$.}
What we want to obtain is the following:
\begin{equation}
\begin{gathered}
\forall x, y, z (\varphi(x, y, p_1, \ldots, p_n) \et \varphi(x, z, p_1, \ldots, p_n) \then y = z) \then\\
\then \forall X \exists Y \forall y\ (y \in Y \iff \exists x (\varphi(x, y, p_1, \ldots, p_n) \et x \in X ))
\end{gathered}
\end{equation}

We do also know that $x, y \in M$, in other words for every $X$, $Y = \{y\ |\ \varphi(x, y, p_1, \ldots, p_n)\}$ and we know that $X \subset M$ and $Y \subset M$, which, together with the specification schema implies that $Y$, the image of $X$ over $\varphi$, is a~set.
\end{proof}

\

We have shown that $\emph{Reflection}$ for first-order formulas, \emph{Reflection\textsubscript{1}} is a~theorem of $\sf{ZF}$, which means that it won't yield us any large cardinals. We have also shown that it can be used instead of the \emph{Infinity} and \emph{Replacement} scheme, but $\sf{ZF}\ +\ \emph{Reflection}_1$ is a~conservative extension of $\sf{ZF}$. Besides being a~starting point for more general and powerful statements, it can be used to show that $\sf{ZF}$ is not finitely axiomatizable. That follows from the fact that \emph{Reflection} gives a~model to any finite number of (consistent) formulas. So if $\varphi_1, \ldots, \varphi_n$ for any finite $n$ would be the axioms of $\sf{ZF}$, $\emph{Reflection}$ would always contain a~model of itself, which would in turn contradict the Second Gödel's Theorem\footnote{See chapter \ref{section:inaccessibility} for further details.}.
Notice that, in a~way, reflection is complementary to compactness. Compactness argues that given a set of sentences, if every finite subset yields a~model, so does the whole set. Reflection, on the other hand, says that while the whole set has no model in the underlying theory, every finite subset does have one.

Also, notice how reflection can be used in ways similar to upward
Löwenheim–Skolem theorem. Since Reflection extends any set $M_0$ into a~model of given formulas $\varphi_1, \ldots, \varphi_n$, we can choose the lower bound of the size of $M$ by appropriately choosing $M_0$.

In the next section, we will try to generalize \emph{Reflection} in a~way that transcends $\sf{ZF}$ and finally yields some large cardinals.
\newpage
% =============================================================
