\section{Introduction}\label{sec:introduction}

%Reflection principle is a kind of a theorem scheme stating the following:
% taky debilni formulace, ale co uz
\subsection{Motivation and Origin}
\begin{displayquote}
The Universe of sets cannot be uniquely characterised (i. e. distinguished from all its initial elements) by any internal structural property of the membership relation in it, which is expressible in any logic of finite of transfinite type, including infinitary logics of any cardinal order.
\end{displayquote}
\rightline{{\rm --- Kurt Gödel \cite{GodelWang}}}

To understand why do need reflection in the first place, let's think about infinity for a moment. In the intuitive sense, infinity is an upper limit of all numbers. But for centuries, this was merely a philosophical concept of limitlessness, the probably best-known classic problems involving infinity are the famous Zeno's paradoxes. In response to those, Aristotle introduced the distinction between actual and potential infinity\footnote{See Aristotle’s Physics, Book III}. By potential infinity we understand that concept of a process does in unbounded in a sense that it could continue for an arbitrary amount of time, but is also never complete. Imagine trying to count all natural numbers. Actual infinity, is, on the other hand, the concept of infinity contained in a bounded space, just like the number of fractions between 0 and 1. This distinction was established by Aristotle who argued, that the potential infinity is (in today's words) well defined, as opposed to the actual infinity, which he considered a vague incoherent concept. He didn't think it's possible for infinite amount of entities to inhabit a bounded place in space or time, rejecting Zeno's thought experiments as a whole. But it's not our aim to get into much detail. 

The aspect of infinity that is relevant to our interests is the human inability to directly experience limitlessness in contrast to how easily can one talk about infinity and limitlessness in the natural language. The short trip into history hopefully served as an example of the fact that certain statements can easily be considered either meaningful or meaningless. 
But while infinity of any kind can't be experienced directly through senses, much effort has been made by philosophers to find a way to meaningfully talk about infinite. 
To see how this leads to reflection, let's think about what Aquinas wrote in his Summa Theologica \footnote{Part I, Question 7, Article 3, Reply to Objection 1}:
\begin{displayquote}
A geometrician does not need to assume a~line actually infinite, but takes some actually finite line, from which he subtracts whatever he finds necessary; which line he calls infinite.
\end{displayquote}
He seems to acknowledge, that infinity can not be reached directly, but for practical purposes it is enough to take a limited part of the whole. One can that act as if it was the whole because the part has all the properties needed at the moment. This, as we shall see in a moment, is in fact an instance of reflection.


To illustrate this elusiveness of infinity, let us remember the early days of set theory. When Cantor proved that there are at least two distinct infinite quantities, this effectively turned what previously was an abstract, unreachable absolute, into a mathematical object, a set. But just as one infinity was seemingly tamed, about 10 years later, Russell's paradox uncovered the fact that there is another absolute, the paradoxical collection of all sets. Mathematicians have decided to focus on axiomatic set theories so that the paradoxical collection was kept out of sets, being considered a class instead \footnote{When we use the words "class" and "property" in this section, "property" refers to statement in natural or formal language that can be meaningfully stated for sets, the notion of class then refers to the collection of all sets holding that particular property. For all practical purposes, the two are synonyms. They will be later properly redefined for use in formal context.}
This is where reflection comes in again. 

The original idea behind reflection principles probably comes from what could be informally called \textquote{universality of the universe}.
If we try to express the universe as a~set $\{x  |  x = x\}$, we either decide to make such statement on a meta-level, or directly in a theory that formalizes the concept of class, like the Bernays–Gödel set theory.


TODO 
Another obstacle of constructing a~set of all sets comes from Georg Cantor, who proved that the set of all subsets of a~set (let $x$ be the set and $\power (x)$ its powerset) is strictly larger that $x$. That would turn every aspiration to finally establish an universal set into a~contradictory infinite regression.\footnote{An intuitive analogy of this \emph{reductio ad infinitum} is the status of $\omega$, which was originally thought to be an unreachable absolute, only to become starting point of Cantor's hierarchy of sets growing beyond all boundaries around the end of the $19^{th}$ century}. We will use $V$ to denote the class of all sets. %Riemann!
%\newpage
From previous thoughts we can easily argue, that it is impossible to construct a~property that holds for $V$ and no set and is neither paradoxical like $\{x  |  x = x\}$ nor trivial. Previous observation can be transposed to a~rather naive formulation of the reflection principle:

TODO

Reflection made its first in set-theoretical appearance in G{\"o}del's proof of GCH in L  (citace Kanamori ? Lévy and set theory), but it was around even earlier as a~concept. G{\"o}del himself regarded it as very close to Russel's reducibility axiom (an earlier equivalent of the axiom schema of Zermelo's separation). Richard Montague then studied reflection properties as a~tool for verifying that Replacement is not finitely axiomatizable (citace?). a~few years later Lévy proved in \cite{Levy60a} the equivalence of reflection with Axiom of infinity together with Replacement in proof we shall examine closely in chaper 2.

\

$\emph{Reflection}$ Any property which holds in $V$ already holds in some initial segment of $V$. 

\

To avoid vagueness of the term "property", we could informally reformulate the above statement into a~schema: 

\

For every first-order formula\footnote{this also works for finite sets of formulas \cite[p.~168]{JechBook}} $\varphi$ holds in $V \iff \varphi$ holds in some initial segment of $V$.

\

Interested reader should note that this is a~theorem scheme rather than a~single theorem. \footnote{If there were a~single theorem stating "for any formula $\varphi$ that holds in $V$ there is an initial segment of $V$ where $\varphi$ also holds", we would obtain the following contradiction with the second G{\"o}del's theorem: In ZFC, any finite group of axioms of ZFC holds in some initial segment of the universe. If we take the largest of those initial segments it is still strictly smaller than the universe and thus we have, via compactness, constructed a~model of ZFC within ZFC. That is, of course a~harsh contradiction. This also leads to an elegant way to prove that ZFC is not finitely axiomatizable.}

%Reflection is, among others, a~tool for constructing models of finitely axiomatizable theories within stronger frameworks of proving impossibility thereof. ?!

%\subsection{A few historical remarks on reflection}\label{sec:History}  % mozna reflection itself?
% co motivace? (jako treba ze kdyz existuje inaccessible cardinal tak je ZFC relativne konzistentni, protoze $V_\kappa$ je jeji model)
% viz kanamori
% jak to bylo s tim Godelem?
% nejdriv Montague ukazal ze ZFC neni konecne axiomatizovatelna "in a~strong sense"
%The first notion of reflection in our sense was formally stated and proved as a~theorem of ZF by Azriel L{\'e}vy in his 1960 article \emph{Axiomatic schemata of strong infinity in axiomatic set theory}\cite{Levy60a}. Only a~year later, this was followed by Richard Montague's \emph{Fraenkel's addition to the axioms of Zermelo}.
 %citace?
 % konecna axiomatizovatelnost? reflexe spolu se 2. Godelovou vetou, ~Montague
 
 
 % motivace ke kardinalum:
 % velke kardinaly vznikly v podstate reflektovanim zajimavych vlastnosti absolutniho nekonecna, mnoho z nich ma i omega
 % popis vyvoje axiomu nekonecna: zadny->nejaky->silnejsi??


\

\subsection{Reflection in Platonism and Structuralism}
TODO cite "reflection in a structuralist setting"

TODO veci o tom, ze reflexe je ok protoze reflektuje veci ktere objektivne plati, protoze plati pro V ...

TODO souvislost s kompaktnosti, hranice formalich systemu nebo alespon ZFC



