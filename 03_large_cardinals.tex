\section{Reflection And Large Cardinals}

In this chapter we aim to examine stronger reflection properties in order to reach cardinals unavailable in $\sf{ZFC}$. Like we said in the first chapter, % [TODO odkaz na uvod]
the variety of reflection principles comes from the fact that there are many way to formalize "properties of the universal class". It is not always obvious what properties hold for $V$ because, (TODO Tarski)
%unlike Lévy's approach, not much attention is paid to what exactly is this $V$, and, more importantly, there are many ways to formalize the notion of property. 
We have shown that reflecting properties as first-order formulas doesn't allow us to leave $\sf{ZFC}$. We will broaden the class of admissible properties to be reflected and see whether there is a~natural limit in the height or width on the reflected universe and also see that no matter how far we go, the universal class is still as elusive as it is when seen from $\sf{S}$. That is because for every process for obtaining larger sets such as for example the powerset operation in $\sf{ZFC}$, this process can't reach $V$ and thus, from reflection, there is an initial segment of $V$ that can't be reached via said process.

To see why this is important, let's dedicate a few lines to the intuition behind the notions of limitness, regularity and inaccessibility in a manner strongly influenced by \cite{Infinity_in_mind}. To see why limit and strongly limit cardinals are worth mentioning, note that they are "limit" not only in a sense of being a supremum of an ordinal sequence, they also show that a certain way of obtaining larger sets from smaller ones is limited. We will see that all of the alternatives offered in this thesis are in a sense limited. 
$\aleph_\lambda$ is a limit cardinal iff there is no $\alpha$ such that $\aleph_{\alpha+1}=\aleph_\lambda$. Strongly limit cardinals point to the limits of the powerset operation. It has been too obvious so far, so let's look at the regular cardinals in this manner. Regular cardinals are those that cannot be\footnote{Assuming $\emph{Choice}$.}, expressed as a supremum of smaller amount of smaller objects\footnote{Just like $\omega$ can not be expressed as a supremum of a finite set consisting solely of finite numbers.}. More precisely, $\kappa$ is regular if there is no way to define it as a union of less than $\kappa$ ordinals, all smaller than $\kappa$. So unless there already is a set of size $\kappa$, \emph{Replacement} is useless in determining whether $\kappa$ is really a set. Note that assuming \emph{Choice}, successor cardinals are always regular, so most\footnote{All provable to exist in $\sf{ZFC}$} limit cardinals are singular cardinals. So if one is traversing the class of all cardinals upwards, successor steps are still sets thanks to the powerset axiom while singular limits cardinal are not proper classes because they are suprema of images of smaller sets via \emph{Replacement}. Regular cardinals are, in a way, limits of how far can we get by taking limits of increasing sequences of ordinals obtained via $\emph{Replacement}$.

TODO prepsat -- regularita a replacement, proc reflexe dava silnejsi veci

That all being said, it is easy to see that no cardinals in $\sf{ZFC}$ are both strongly limit and regular because there is no way to ensure they are sets and not proper classes in $\sf{ZFC}$. The only exception to this rule is $\aleph_0$ which needs \emph{Infinity} to exist. % nase otazka je: proc omega a ne jine kardinaly?
It should now be obvious why the fact that $\kappa$ is inaccessible implies that $\kappa = aleph_\kappa$.\footnote{This doesn't work backwards, the least fixed point of the $\aleph$ function is the limit of $\{\aleph_0,\ \aleph_{\aleph_0},\ \aleph_{\aleph_{\aleph_0}},\ \ldots \}$, it is singular since the sequence has countably many elements.}

We will also examine the connection between reflection principles and (regular) fixed points of ordinal functions in a manner proposed by Lévy in \cite{Levy60a}. We will also see that, like Lévy has proposed in the same paper, there is a meaningful way to extend the relation between $\sf{S}$ and $\sf{ZFC}$ into a hierarchy of stronger axiomatic set theories. 
% Those are the three lines of thinking that we will find are in fact different facets of the same gem, especially in the section devoted to Inaccessible and Mahlo cardinals.
% viz Shapiro, Stewart. 1987. “Principles of Reflection and Second-order Logic”. Journal of Philosophical Logic 16 (3). Springer: 309–33. http://www.jstor.org/stable/30227043.
% Reflections on \emph{Replacement} and Reflection: The axioms in a~structuralist setting (Geoffrey Hellman)
%TODO neco o tom, ze kdyz je reflexe formule, da se sama reflektovat?
% The above should make a clear picture of why $\emph{Infinity}$ is a specific case of $\emph{Reflection}$.
%TODO proc je Refl zaroven zobecneny replacement?

% TODO ze "uplne totalni" reflexe se zacykli a rozbije? nebo ne?

\subsection{Regular Fixed-Point Axioms}
% This small chapter is dedicated to 

Lévy's article mentions various schemata that are not instances of reflection per se. We will mention them because they are equivalent to \emph{Reflection\textsubscript{1}}\footnote{For definition, see \ref{def:reflection_1}}.

% Lévy proposes in \cite{Levy60a} those axioms as equivalent to \emph{Reflection\textsubscript{1}}.
\begin{definition}{(\emph{Axiom $M$\textsubscript{1}})}\label{def:levy_m}\\
"Every normal function defined for all ordinals has at least one inaccessible number in its range."
\end{definition}
Lévy uses "$M$" to refer to this axiom but since we also use "$M$" for sets and models, for example in \ref{def:reflection_1}, we will call the above axiom "\emph{Axiom $M$\textsubscript{1}}" to avoid confusion.

Now we will express \emph{Axiom $M$\textsubscript{1}} to formula to make it clear that it is an axiom scheme and the same can be done with \emph{Axiom $M'$\textsubscript{1}} as well as \emph{Axiom $M''$\textsubscript{1}} introduced immediately afterwards. Since it is an axiom schema and we will later dive into second-order logic, we may also want to refer to \emph{Axiom $M$\textsubscript{2}} as opposed \emph{Axiom $M$\textsubscript{1}}, the former being a single second-order sentence obtained by the obvious modification of \emph{Axiom $M$\textsubscript{1}}.\footnote{Second-order set theory will be introduced in the next subsection.}

Let $\varphi(x, y, p_1, \ldots, p_n)$ be a first-order formula with no free variables besides $x, y, p_1, \ldots, p_n$. The following is equivalent to \emph{Axiom $M$\textsubscript{1}}.
\begin{equation}
\begin{gathered}
\mbox{"$\varphi$ is a normal function"} \et \forall x (x \in Ord \then \exists y(\varphi(x, y, p_1, \ldots, p_n))) \then\\
\then \exists y (\exists x \varphi(x, y, p_1, \ldots, p_n) \et cf(y) = y \et (\forall x \in \kappa)(\exists y \in \kappa)(x > y))
\end{gathered}
\end{equation}\footnote{"$\varphi$ is a normal function" is equivalent to the following first-order formula: }

\begin{definition}{(Axiom $M'$\textsubscript{1})}\\
Every normal function defined for all ordinals has at least one fixed point which is inaccessible.
\end{definition}

\begin{definition}{(Axiom $M''$\textsubscript{1})}\\
"Every normal function defined for all ordinals has arbitrarily great fixed points which are inaccessible."
\end{definition}

The following axiom is proposed by Drake in \cite{DrakeBook}. 
\begin{definition}{(Axiom $F$\textsubscript{1})}\\
Every normal function defined for all ordinals has a regular fixed point.
\end{definition}


\begin{lemma}{(Fixed-point lemma for normal functions)}\label{lemma:normal_fixed_point}\\
Let $f$ be a normal function defined for all ordinals. The all of the following hold
\bce[(i)]
\item $\forall \lambda(\mbox{"$\lambda$ is a limit ordinal"} \then \mbox{"f($\lambda$) is a limit ordinal"})$
\item $\forall \alpha (\alpha \leq f(\alpha))$
\item $\forall \alpha \exists \beta (\alpha < \beta \et f(\beta) = \beta) \mbox{($f$ has arbitrarily large fixed points.)}$
\item The fixed points of $f$ form a closed unbounded class.\footnote{See \ref{def:closed_class} for the definition of closed set, \ref{def:unbounded class}}
\ece
\end{lemma}

\begin{proof}
Let $f$ be a normal function.
\bce[(i)]
\item Proof of $\bold{(i)}$:\\
Suppose $\lambda$ is a limit ordinal. For an arbitrary ordinal $\alpha < \lambda$, the fact that $f$ is strictly increasing means that $f(\alpha) < f(\lambda)$ and for an ordinal $\beta$, $\beta < \alpha$, $f(\alpha) < f(\beta)$. Because $f$ is continuous and $\lambda$ limit, $f(\lambda) = \bigcup_{\alpha < \lambda} f(\alpha)$ and since $\beta < \lambda$, $f(\beta) < f(\lambda)$. So we have found $f(\beta)$ such that $f(\alpha) < f(\beta) < f(\lambda)$, therefore $f(\lambda)$ is a limit ordinal.\\

\item This step will be proven using the transfinite induction.
Since $f$ is defined for all ordinals, there is an ordinal $\alpha$ such that $f(\emptyset) = \alpha$ and because $\emptyset$ is the least ordinal, $\bold{(ii)}$ holds for $\emptyset$.

Suppose $\bold{(ii)}$ holds for some $\beta$ form the induction hypothesis. It the holds for $\beta+1$ because $f$ is strictly increasing. 

For a limit ordinal $\lambda$, suppose $\bold{(ii)}$ holds for every $\alpha < \lambda$. $\bold{(i)}$ implies that $f(\lambda)$ is also limit, 
so there is a strictly increasing $\kappa$-sequence $\langle \alpha_0, \alpha_1, \ldots \rangle$ for some $\kappa$ such that $\lambda = \bigcup_{i<\kappa} \alpha_i$. Because $f$ is stricly increasing, the $\kappa$-sequence $\langle f(\alpha_0), f(\alpha_1), \ldots$ is also strictly increasing, the induction hypothesis implies that $\alpha_i \leq f(\alpha_i)$ for each $i \leq \kappa$. Thus, $\lambda \leq f(\lambda) \rangle$.

\item 
For a given $\alpha$, let there be a $\omega$-sequence $\langle \alpha_0, \alpha_1, \ldots \rangle$, such that $\alpha_0 = \alpha$ and $\alpha_{i+1} = f(\alpha_i)$ for each $i < \omega$.
This sequence is stricly increasing because so is $f$. Now, there's a limit ordinal $\beta = \bigcup_{i < \omega} \alpha_i$, we want to show that this is the fixed point. So  $f(\beta) = f(\bigcup_{i < \omega} \alpha_i) = \bigcup_{i < \omega} f(\alpha)$ because $f$ is continuous. We have defined the above sequence so that $\beta$, $\bigcup_{i < \omega} f(\alpha) = \bigcup_{i < \omega} \alpha_{i+1}$, which means we are done, since $\bigcup_{i < \omega} \alpha_{i+1} = \bigcup_{i < \omega} \alpha_{i}  = \beta$.

\item The class of fixed points of $f$ is obviously unbounded by $\bold{(iii)}$. It remains to show that it is closed. 
TODO def closed?

\ece
\end{proof}

\begin{theorem}
\begin{equation}
\emph{Axiom $M$\textsubscript{1}} \iff \emph{Axiom $M'$\textsubscript{1}} \iff \emph{Axiom $M''$\textsubscript{1}} \iff \emph{Axiom $F$\textsubscript{1}}
\end{equation}
\end{theorem}

This is \emph{Theorem 1} in \cite{Levy60a}.
\begin{proof}
It is clear that \emph{Axiom $M''$\textsubscript{1}} is a stronger version of \emph{Axiom $M'$\textsubscript{1}}, which is in turn a stronger version of both \emph{Axiom $M$\textsubscript{1}} and \emph{Axiom $F$\textsubscript{1}}, so the implication \emph{Axiom $M''$\textsubscript{1}} $\then$ \emph{Axiom $M'$\textsubscript{1}} $\then$ \emph{Axiom $M$\textsubscript{1}} is satisfied and \emph{Axiom $M'$\textsubscript{1}} $\then$ \emph{Axiom $F$\textsubscript{1}} holds too.

We will now make sure that  \emph{Axiom $M$\textsubscript{1}} $\then$  \emph{Axiom $M''$\textsubscript{1}} also holds. 
Let $f$ be a normal function defined for all ordinals. % such that there is $\varphi$, $f(x) = y \iff \varphi$ that satisfies \emph{Axiom $M$\textsubscript{1}}.
Let $g$ be a normal function that counts the fixed points of $f$. Lemma \ref{lemma:normal_fixed_point} implies that there arbitrarily many fixed points of $f$, therefore $g$ is defined for all ordinals. Let there be another family of functions, $h_\alpha(\beta) = g(\alpha+\beta)$, obviously $h_\alpha$ is defined for all ordinals for every $\alpha \in Ord$ because so is $g$. Given an arbitrary ordinal $\gamma$, from \emph{Axiom $M$\textsubscript{1}} we can assume that there is an ordinal $\delta$ such that such that $h_\alpha(\delta) = \kappa$, where $\kappa$ is inaccessible. 
But since $\kappa = g(\alpha+\delta)$, $\kappa$ is a fixed point of $f$. To show that there are arbitrarily many fixed points of $f$, notice that $\gamma$ is arbitrary and $h_\gamma$ is a normal function, so, by lemma \ref{lemma:normal_fixed_point}, $(\forall \alpha \in Ord)(\alpha \leq f(\alpha)$, therefore $\gamma \leq \gamma + \alpha \leq \kappa$, in other words, there is $\kappa$ above an arbitrary ordinal $\gamma$.

Now we need to show that \emph{Axiom $F$\textsubscript{1}} implies any of the remaining axioms.
TODO nevyhodime F?
\end{proof}

\begin{definition}{$\sf{ZMC}$}\\
We will call $\sf{ZMC}$ a set theory that contains all axioms and schemas of $\sf{ZFC}$ together with the schema \emph{Axiom $M$\textsubscript{1}}.
\end{definition}
We have decided to call it $\sf{ZMC}$, because Lévy uses $\sf{ZM}$, derived from $\sf{ZF}$, which is more intuitive, but we also need the axiom of choice, thus, $\sf{ZMC}$.


Let's now prove that in $\sf{ZFC}$, the above \emph{Axiom M} is equivalent to \emph{Reflection\textsubscript{1}} as defined in \ref{def:reflection_1}. This is proven in \cite{Levy60a} as \emph{Theorem 3}. % TODO tohle by bylo fajn dokazat, protoze tam je ten Montague
\begin{theorem}\label{theorem:levy_m_iff_reflection}
\begin{equation}
\sf{ZFC} \models \mbox{\emph{Axiom M}} \iff \mbox{\emph{Reflection\textsubscript{1}}}
\end{equation}
\end{theorem}

% -----------------------------------------------
TODO nedosazitelne kardinaly -- reflektuj presne formule, schemata

\subsection{Inaccessibility}\label{section:inaccessibility}
%The inaccessible cardinal is the smallest of large cardinals\footnote{citation needed.}
\begin{definition}{(limit cardinal)}\label{def:limit}
$\kappa$ is a~\emph{limit cardinal} iff it is $\aleph_\alpha$ for some limit ordinal $\alpha$.
\end{definition}

\begin{definition}{(strong limit cardinal)}\label{def:strong_limit}
$\kappa$ is a~\emph{strong limit cardinal} iff it is a limit cardinal and for every $\lambda < \kappa$, $2^\lambda < \kappa$
\end{definition}

The two above definition become equivalent if we assume $GCH$.

\begin{definition}{(weak inaccessibility)}\label{def:weakly_inaccessible}
An uncountable cardinal $\kappa$ is \emph{weakly inaccessible} iff it is \emph{regular} and \emph{limit}.
\end{definition}
\begin{definition}(inaccessibility)\label{def:inaccessible}
An uncountable cardinal $\kappa$ is \emph{inaccessible} iff it is \emph{regular} and \emph{strongly limit}.
\end{definition}

\

TODO neni tohle cely hotovy v Contemporary restatement??? porovnat ktera je lepsi a sjednotit!!!

We will now show that the above notion is equivalent to the definition Lévy uses in \cite{Levy60a}, which is, in more contemporary notation, the following:
\begin{theorem}\label{theorem:inaccessible_models_zfc}
The following are equivalent:
\bce
\item $\kappa$ in inaccessible
\item $\langle V_\kappa, \in \rangle \models \sf{ZFC}$
\ece
\end{theorem}

\begin{proof}
Let's first prove that if $\kappa$ is inaccessible, it is a~model of $\sf{ZFC}$. We will do that by verifying the axioms of $\sf{ZFC}$ just like Kanamori does it in in \cite[1.2]{KanamoriBook} and Drake in \cite[Chapter 4]{DrakeBook}. 
\bce[(i)]
\item \emph{Extensionality}:\\
(see \ref{def:extensionality})
\begin{equation}
V_\kappa \models \forall x, y(\forall z (z \in x \iff z \in y) \then x = y) 
\end{equation}
We need to prove that, given two sets that are equal in $V$, they are equal in $V_\kappa$, in other words, that the \emph{Extensionality} formula is reflected, that is
\begin{equation}
V_\kappa \models \forall x, y \in V_\kappa (\forall z \in V_\kappa (z \in x \iff z \in y) \then x = y) 
\end{equation}
But that comes from transitivity. If $x$ and $y$ are in $V_\kappa$ their members are also in $V_\kappa$.

\

\item \emph{Foundation}:\\
(see \ref{def:foundation})
\begin{equation}
V_\kappa \models \forall x (\exists z (z \in x) \then \exists z (z \in x \et \forall u \neg (u \in z \et u \in x)))
\end{equation}
The argument for \emph{Foundation} is almost identical to the one for \emph{Extensionality}. For any set $x \in V_\kappa$, transitivity of $V_\kappa$ makes sure that every element of $x$ is also an element of $V_\kappa$ and the same holds for the elements of elements of $x$ et cetera. So statements about those elements are absolute between any transitive structures. $V$ and $V_\kappa$ are both transitive therefore \emph{Foundation} holds and so does its relativisation to $V_\kappa$, $Foundation^{V_\kappa}$.

\

\item \emph{Powerset}:\\
(see \ref{def:powerset})
\begin{equation}
V_\kappa \models \forall x \exists y \forall z (z \subseteq x \then z \in y).
\end{equation}
If we take $x$, an element of $V_\kappa$, $\power{x}$ has to be an element of $V_\kappa$ to, because it is transitive and a strong limit cardinal.

\

\item \emph{Pairing}:\\
(see \ref{def:pairing})
\begin{equation}
V_\kappa \models \forall x, y \exists z (x \in z \land y \in z).
\end{equation}
\emph{Pairing} holds from similar argument like above: let $x$ and $y$ be elements of $V_\kappa$, so there are ordinals $\alpha, \beta < \kappa$ such that $x \in V_\alpha$, $y \in V_\beta$. Without any loss of generality, suppose $\alpha < \beta$, threfore $V_\alpha \subset V_\beta$ which, from transitivity of the cumulative hierarchy, means that $x \in V_\beta$, then $\{x, y\} \in V_{\beta+1}$ which is still in $V_\kappa$ because it is a strong limit cardinal.

\

\item \emph{Union}:\\
(see \ref{def:union})
\begin{equation}
V_\kappa \models \forall x \,\exists y \, \forall z\, \forall w ((w \in z \land z \in x) \then w \in y).
\end{equation}
We want to see that for every $x \in V_\kappa$, this is equivalent to 
\begin{equation}
V_\kappa \models \forall x \in V_\kappa,\exists y \in V_\kappa\, \forall z \in V_\kappa\, \forall w \in V_\kappa ((w \in z \land z \in x) \then w \in y).
\end{equation}
Since $V_\kappa$ is transitive, if $x \in V_\kappa$, all of its elements as well as their elements are in $V_\kappa$. To see that they also form a set themselves we only need to remember that $V_\kappa$ is limit and therefore if $\alpha$ is the least ordinal such that $x \in V_\alpha$, $\bigcup x \in V_{\alpha+1}$.

\

\item \emph{Replacement, Infinity}:\\
(see \ref{def:replacement}, \ref{def:infinity})\\
TODO !!!!

to spis ty pred tim zname z dukazu v S, viz contemporary restatement.

udelat z toho lemma?

co ten replacement?? druha implikace Levyho vety?
\ece
\

We will now show that if a~set is a~model of $\sf{ZFC}$, it is in fact an inaccessible cardinal. So let $V_\kappa$ be a~model of $\sf{ZFC}$ which means that it is closed under the powerset operation, in other words:
\begin{equation}
\forall \lambda (\lambda < \kappa \then 2^{\lambda} < \kappa)
\end{equation}
which is exactly the definition of strong limitness. $\kappa$ is regular from the following argument by contradiction:\\
Let us suppose for a~moment that $\kappa$ is singular. Therefore there is an ordinal $\alpha < \kappa$ and a~function $F:\ \alpha \then \kappa$ such that the range of $F$ in unbounded in $\kappa$, in other words, $F[\alpha] \subseteq V_\kappa$ and $sup(F[\alpha]) = kappa$. In order to achieve the desired contradiction, we need to see that it is the case that $F[\alpha] \in V_\kappa$. Let $\varphi(x, y)$ be the following first-order formula: % TODO vyhodit sup, pouzivat radis $\bigcup$
\begin{equation}
F(x)\ =\ y
\end{equation}
Then there is an instance of \emph{Replacement} that states the following:
\begin{equation}
\begin{gathered}
(\forall x, y, z(\varphi(x, y) \et \varphi(x, z) \then y\ =\ z)) \then \\
\then (\forall x \exists y \forall z (z \in y \iff \exists w (\varphi(w, z))))
\end{gathered}
\end{equation}
Which in turn means that there is a~set $y = F[\alpha]$ and $y \in V_\kappa$, which is the contradiction with $sup(y) = \kappa$ we are looking for.
\end{proof}
TODO vyhodit sup, pouzivat radis $\bigcup$

% \

% The same holds for $\sf{ZFC}_2$, the proof is very similar.
% \begin{theorem}\label{theorem:inaccessible_models_zfc_2}
% \begin{equation}
% V_\kappa \models \sf{ZFC}_2 \iff \mbox{$\kappa$ is inaccessible}
% \end{equation}
% \end{theorem}
% \begin{proof}
% $\kappa$ is a~strong limit cardinal because from $\sf{ZFC}_2$ and \emph{Powerset} we know that for every $\lambda\ <\ \kappa$, we know that $2^{\lambda} < \kappa$.

% $\kappa$ is also regular, because otherwise there would be an ordinal $\alpha$ and a~function $F:\ \alpha \then \kappa$ with a~range unbounded in $\kappa$. 
% $\emph{Replacement}^2$ gives us a~set $y\ =\ F[\alpha]$, so $y\ \in\ V_\kappa$, which contradicts the fact that $sup(y) = kappa$. It can not be the case that $\kappa \in V_\kappa$.
% TODO vyhodit sup, pouzivat radis $\bigcup$
% \

% The other direction is exactly like the first part of above theorem \ref{theorem:inaccessible_models_zfc}.
% \end{proof}

% \

%The above makes clear that while the existence of inaccessibles is unprovable in $\sf{ZFC}$ or $\sf{ZFC}_2$, we now know that at least the smallest inaccessible cardinal exists in $\sf{ZF+(Refl^2)}$ because there is a~set that models $\sf{ZFC}$. But how exactly can we reach there with second-order reflection?

% TODO

We have transcended $\sf{ZFC}$, but that is just a~start. Naturally, we could go on and consider the next inaccessible cardinal, which is inaccessible with respect to the theory $\sf{ZFC} + \exists \kappa (\kappa \models \sf{ZFC})$. But let's try to find a faster way up, informally at first. 

Since we can find an inaccessible set larger than any chosen set $M_0$, it is clear that there are arbitrarily large inaccessible cardinals in $V$, they are "unbounded"\footnote{The notion is formaly defined for sets, but the meaning should be obvious.} in $V$. If $V$ were a cardinal, we could say that there are $V$ inaccesible cardinals less than $V$, but this statement of course makes no sense in set theory as is because $V$ is not a set. But being more careful, we could find a property that can be formalized in second-order logic and reflect it to an initial segment of $V$. That would allow us to construct large cardinals more efficiently than by adding inaccessibles one by one. The property we are looking for ought to look like something like this:
\begin{equation}
\begin{gathered}
\kappa \mbox{ is an inaccessible cardinal and}\\
\mbox{there are }\kappa\mbox{ inaccessible cardinals }\mu\ <\ \kappa
\end{gathered}
\end{equation}
This is in fact a fixed-point type of statement. We shall call those cardinals hyper-inaccessible. Now consider the following definition.

\

\begin{definition}{$0$-inaccessible cardinal}\\
A cardinal $\kappa$ is $0$-inaccessible if it is inaccessible.
\end{definition}
We can define \emph{$\alpha$-weakly-inaccessible} cardinals analogously with the only difference that those are limit, not strongly limit.
\

\begin{definition}{$\alpha$-hyper-inaccessible cardinal}\label{def:alpha_inaccessible}\\
For any ordinal $\alpha$, $\kappa$ is called $\alpha$-inaccessible, if $\kappa$ is inaccessible and for each $\beta$ < $\alpha$, the set of $\beta$-inaccessible cardinals less than $\kappa$ is unbounded in $\kappa$.
\end{definition}

\

Because $\kappa$ is inaccessible and therefore regular, the number of $\beta$-inaccessibles below $\kappa$ is equal to $\kappa$. We have therefore successfully formalized the above vague notion of hyper-inaccessible cardinal into a hierarchy of $\alpha$-inaccessibles.

\

Let's now consider iterating this process over again. Since, informally, $V$ would be $\alpha$-inaccessible for any $\alpha$, this property of the universal class could possibly be reflected to an initial segment, the smallest of those will be the first hyper-inaccessible cardinal. Such $\kappa$ is larger than any $\alpha$-inaccessible since from regularity of $\kappa$, for given $\alpha\ <\ \kappa$, $\kappa$ is $\kappa$-th $\alpha$-hyper-inaccessible cardinal. It is in fact "inaccessible" via $\alpha$-inaccessibility.

\

\begin{definition}{Hyper-inaccessible cardinal}\\
$\kappa$ is called the hyper-inaccessible, also $0$-hyper-inaccessible, cardinal if it is $\alpha$-inaccessible for every $\alpha\ <\ \kappa$.
\end{definition}

\

\begin{definition}{$\alpha$-hyper-inaccessible cardinal}\\
For any ordinal $\alpha$, $\kappa$ is called $\alpha$-hyper-inaccessible cardinal if for each ordinal $\beta\ <\ \alpha$, the set of $\beta$-hyper-inaccessible cardinals less the $\kappa$ is inbounded in $\kappa$.
\end{definition}

\

Obviously we could go on and iterate it ad libitum, yielding $\alpha$-hyper-$\ldots$-hyper-inaccessibles, but the nomenclature would be increasingly confusing. A smarter way to accomplish the same goal is carried out in the following section.

% dukaz viz Kan book 6.1, 6.2 ?

\begin{comment}

% !!!!!!!!!!!!!!!!!!!!!!!!!!!!!!!!!!!!!!!!!!!!!!!!!

TODO tohle znamena, ze prvoradovou formuli nerozlisime $V$ od (prvniho) nedosazitelneho $\kappa$

\begin{theorem}\label{th:refl_inaccessible}[Lévy] The following are equivalent:
\bce[(i)]
\item $\kappa$ is inaccessible.
\item For every $R \sub V_\kappa$ and every first-order formula $\varphi(R)$, $\varphi(R)$ reflects in $V_\kappa$.
\item For every $R \sub V_\kappa$, the set $C = \set{\alpha<\kappa}{\langle V_\alpha,\in,R \cap V_\alpha\rangle \el \langle V_\kappa,\in,R \rangle}$ is closed unbounded.
\ece
\end{theorem}
\begin{proof}
Let's start with $\bold{(i)} \then \bold{(iii)}$ in a~way similar to \cite{KanamoriBook}.\newline
The set $\set{\alpha<\kappa}{\langle V_\alpha,\in,R \cap V_\alpha \rangle \el \langle V_\kappa,\in,R\rangle}$ is clearly closed, it remains to show that it is also unbounded.
To do so, let $\alpha<\kappa$ be arbitrary. Define $\alpha_n < \kappa$ for $n\in\omega$ by recursion as follows:\newline
Set $\alpha_0=\alpha$. Given $\alpha_n < \kappa$ define $\alpha_{n+1}$ to be the least $\beta \geq \alpha_n$ such as 
whenever $y_1,\ldots,y_k \in V_{\alpha_n}$ and
$\langle V_{\kappa}, \in, R \rangle \models \exists v_0 \varphi [v_0, y_1, \ldots, y_k ]$
for some formula $\varphi$, there is an $x \in V_{\beta}$ such that $\langle V_{\kappa}, \in, R\rangle \models \varphi [x, y_1, \ldots, y_k]$.
\newline
Since $\kappa$ is inaccessible, $|V_{\alpha_n}| < \kappa$ and so $\alpha_{n+1} < \kappa$.\newline
Finally, set $\alpha = sup({\alpha_n | n \in \omega})$.  % TODO vyhodit sup, pouzivat radis $\bigcup$
Then $\langle$
 $V_ \alpha, \in, R  \cap V_\alpha \rangle \prec \langle V_{\kappa}, \in, R\rangle$ by the usual (Tarski) criterion for elementary substructure.
 \newline\newline
 The next part, proving $\bold{(iii)} \then \bold{(ii)}$, should be elementary since $C$ is closed unbounded, which means that it contains at least countably many elements but we need only one such $\alpha$ to satisfy (\ref{def:reflection_2}).
 \newline
 Finally, we shall prove that $(ii) \then \bold{(i)}$. Since it obviously holds that $\kappa > \omega$, we have yet to prove that $\kappa$ is regular and a~strong limit. Let's argue by contradiction that it is regular. 
 If it wasn't, there would be a~$\beta < \kappa$ and a~function $F: \beta \implies \kappa$ with range unbounded in $\kappa$. Set $R = \{\beta\} \cup F$. By hypothesis there is an $\alpha < \kappa$ such that $\langle V_\alpha, \in, R \cap V_\alpha \rangle \prec \langle V_\kappa, \in, R \rangle$. Since $\beta$ is the single ordinal in R, $\beta \in V_\alpha$ by elementarity. This yields the desired contradiction since the domain if $F \cap V_\alpha$ cannot be all of $\beta$.
 \newline\newline
 Next, let's see whether $\kappa$ is indeed a~strong limit, again by contradiction. If not, there would be a~$\lambda < \kappa$ such that $2^\lambda \geq \kappa$. Let $G: \power{\lambda} \implies \kappa$ be surjective and set $R = \{\lambda + 1\} \cup G$. By hypothesis, there is an $\alpha < \kappa$ such that $\langle V_\alpha, \in, R \cap V_\alpha \rangle \prec \langle V_\kappa, \in, R \rangle$. $\lambda + 1 \in V_\alpha$ and so $\power{\lambda} \in V_\alpha$, but this is again a~contradiction.
\end{proof}
\
\end{comment}

% todo vymenit < za prohnute
% todo vymenit P za potencni P

% =====================================================================================================================================

\subsection{Mahlo Cardinals}

TODO axiomy?

While the previous chapter introduced us to a notion of inaccessibility and the possibility of iterating it ad libitum in new theories, there is an even faster way to travel upwards in the cumulative hierarchy, that was proposed by Paul Mahlo in his articles (see \cite{Mahlo11}, \cite{Mahlo12} and \cite{Mahlo13}) at the very beginning of the 20th century, and which can be easily reformulated using reflection.

\begin{theorem}\label{club_intersection} 
Let $\kappa$ be a regular uncountable cardinal. The intersection of fewer than $\kappa$ club subsets of $\kappa$ is a club set.
\end{theorem}
For the proof, see \cite[Theorem 8.3]{JechBook}

\begin{definition}{Weakly Mahlo Cardinal}\label{def:weakly_mahlo}\\
$\kappa$ is \emph{weakly Mahlo} $\iff$ it is a~weakly-inaccessible ordinal and the set of all regular ordinals less then $\kappa$ is stationary in $\kappa$
\end{definition}

\begin{definition}{Mahlo Cardinal}\label{def:mahlo_cardinal}\\
$\kappa$ is a \emph{Mahlo Cardinal} iff it is an inaccessible cardinal and the set of all inaccessible ordinals less then $\kappa$ is stationary in $\kappa$.
\end{definition}
% It is interesting to note, that weakly-Mahlo cardinals are fixed points of $\alpha$-weakly inaccessible cardinals, so if $\kappa$ is weakly mahlo,  .. viz Kanamori Proposition 1.1

Analogously, 
\begin{definition}{$\alpha$-Mahlo Cardinal}\label{def:alpha_mahlo_cardinal}\\
$\kappa$ is a \emph{$\alpha$-Mahlo Cardinal} iff it is an $\alpha$-inaccessible cardinal and the set of all $\alpha$-inaccessible ordinals less then $\kappa$ is stationary in $\kappa$.
\end{definition}

In other words, $\kappa$ is a (weakly-)Mahlo cardinal if it is (weakly-)inaccessible and every club set in $\kappa$ contains an (weakly-)inaccessible cardinal. Alternatively, a cardinal is (weakly-)Mahlo if it is (weakly-)inaccesible and there are $\kappa$ (weakly-)inaccessibles below $\kappa$.
% viz http://euclid.colorado.edu/~monkd/m6730/gradsets12.pdf
%Thus a~Mahlo cardinal $\kappa$ is not only inaccessible, but also has $\kappa$ inaccessibles below it.

%\cite{DrakeBook}

In a fashion similar to hyper-inaccessible cardinals, hyper-Mahlo cardinals can be defined as well.

TODO Lévy tady nekde? posloupnost modelu?

TODO co s nima edla Jech?

TODO proc se vys nedostaneme pevnyma bodama?

TODO explicitni reflexe? reflektuji reflexi nedosazitelnosti?

TODO Drake p.121!!

% TODO $\kappa$ is hyper-Mahlo iff $\kappa$ is inaccessible and the set $\{\lambda < \kappa : \lambda\mbox{ is Mahlo}\}$ is stationary in $\kappa$. to je to samy jako $\alpha$-Mahlo, ne?

% TODO viz https://en.wikipedia.org/wiki/Mahlo_cardinal#Mahlo_cardinals_and_reflection_principles

% Note that Mahlo cardinals were first described in 1911, almost 50 years before Lévy's reflection, which was heavily inspired by them.

% " We also state the appropriate generalization for greatly Mahlo cardinals." % viz http://arxiv.org/abs/math/9204218

%TODO veta na zaver, shrnuti

%sjednotil \then a~\implies
% =====================================================================================================================================
% \newpage
\subsection{Second-order Reflection} % Indescribality and Weakly Compact Cardinals 
Let's try a different approach in formalizing reflection. We have seen that reflecting individual first-order formulas doesn't even transcend $\sf{ZFC}$, we have examined what can be done with axiom schemas. The aim of this chapter is to examine second-order formulas as possible axioms. Note that second-order variables (which will be established as type 2 variables later in the text) are subcollections of the universal class, but so are functions and relations. So first-order axiom schemata can also be interpreted as formulas with free second-order variables, which quantify over first-order variables only, we only need to customize the underlying theory accordingly. For example, the satisfaction relation was so far defined for first-order formulas only, but we will deal with that in a moment. Also note that by rewriting \emph{replacement} and \emph{comprehension} to single axioms, $\sf{ZFC}$ becomes finitely axiomatizable, which in turn means that the reflection theorem as stated in section \label{sec:first_order} does not hold for higher-order theories because of Gödel's second incompleteness theorem. We will explore stronger axioms of reflection instead.

TODO nemam nekam napsat ze u vsech velkych karidnalu je to "existujou pokud .. "?

Let us establish a formal background first. We will now introduce higher-order formulas.

\begin{definition}{(Higher-order variables)}\label{def:higher_order_variables}\\
Let $M$ be a structure and $D$ it's domain. In first-order logic, variables range over individuals, that is, over elements of $D$. We shall call those \emph{type 1 variables} for the purposes of higher-order logic. Type 2 variables then range over collections, that is, the elements of $\power{D}$. Generally, type $n$ variables are defined for any $n \in \omega$ such that they range over $\mathscr{P}^{n-1}(D)$.
\end{definition}
We will use lowercase latin letters for type 1 variables for backwards compatibility with first-order logic, type 2 variables will be represented by upper-case letters, mostly $P, X, Y, Z$. If we ever stumble upon type 3 variables in this text, they shall be represented as $\mathscr{X}, \mathscr{Y}, \mathscr{Z}$ or in a similar font.
% todo lepsi slovo nez variables?
% For the following definition, we need variables and quantifiers of higher orders. Let \emph{type 1} variables be usual variables of first-order set theory. % spis logic?

\begin{definition}{(Full prenex normal form)}\label{def:pnf}\\
We say a formula is in the \emph{prenex normal form} if it is written as a block of quantifiers followed by a quantifier-free part.\\
We say a formula is in the \emph{Full prenex normal form} if it is written in \emph{prenex normal form} and if there are type $n+1$ quantifiers, they are written before type $n$ quantifiers.
\end{definition}
It is an elementary that every formula is equivalent to a formula in the prenex normal form.


\begin{definition}{(Hierarchy of formulas)}\label{def:analytical_hierarchy}\\
Let $\varphi$ be a formula in the prenex formal form.
\bce[(i)]
\item We say $\varphi$ is a $\Delta^0_0$-formula if it contains only bounded quantifiers.
\item We say $\varphi$ is a $\Sigma^0_0$-formula or a $\Pi^0_0$-formula if it is a $\Delta^0_0$-formula.
\item We say $\varphi$ is a $\Pi^{m+1}_0$-formula if it is a $\Pi^m_n$- or $\Sigma^m_n$-formula for any $n \in \omega$ or if it is a $\Pi^m_n$- or $\Sigma^m_n$-formula with additional free variables of type $m+1$.
\item We say $\varphi$ is a $\Sigma^m_0$-formula if it is a $\Pi^m_0$-formula.
\item We say $\varphi$ is a $\Sigma^m_n+1$-formula if it is of a form $\exists P_1, \ldots, P_i \psi$ for any non-zero $i$, where $\psi$ is a $\Pi^m_n$-formula and $P_1, \ldots, P_i$ are type $m+1$ variables.
\item We say $\varphi$ is a $\Pi^m_n+1$-formula if it is of a form $\forall P_1, \ldots, P_i \psi$ for any non-zero $i$, where $\psi$ is a $\Sigma^m_n$-formula and $P_1, \ldots, P_i$ are type $m+1$ variables.
\ece
\end{definition}

Now that we have introduced higher types of quantifiers, we will use it to formulate reflection. But first, let's make it clear how relativization works for higher-order quantifiers and type 2 parameters. Let $\alpha, \kappa$ be ordinals such that $\alpha < \kappa$, $R \subseteq V_\kappa$.
\begin{equation}
R^{V_\alpha} \defeq R \cap V_\alpha
\end{equation}
And let $\exists^{m}$ be a quantifier that ranges over type $m$ variables, let $P$ represent a type $m$ variable, let $\varphi$ be a type $m$ formula with the only free variable $P$.
\begin{equation}
(\exists P \varphi(P))^{V_\alpha} \defeq (\exists \power^(m-1){V_\alpha})\varphi^{V_\alpha}(P))
\end{equation}


\begin{definition}{(Reflection)}\label{def:reflection_2}\\
Let $\varphi(R)$ be a $\Pi^n_m$-formula with one free variable of type type 2 denoted $P$. We say $\varphi(R)$ reflects in $V_\kappa$ if for every $R \sub V_\kappa$ there is an ordinal $\alpha<\kappa$ such that the following holds:
\begin{equation}
\begin{gathered}
\mbox{If }(V_\kappa,\in, R)\models \varphi(R),\\
\mbox{ then }(V_\alpha,\in, R\cap V_\alpha) \models \varphi(R\cap V_\alpha).
\end{gathered}
\end{equation}
\end{definition}

This formalization of the notion of reflection allows us to describe Inaccessible and Mahlo cardinals more easily, which we will do in the following section. 

It is important to see, that while we can now reflect $\Pi^m_n$-formulas for arbitrary $m, n \in \omega$, they can only have type 2 free variables. 
This formalization of reflection can not be extended to higher-order parameters as is. This will be briefly reviewed in the next paragraph.

In order to extend reflection as a stated above in \ref{def:reflection_2}, we need to make sure that given the domain of the structure, $V_\kappa$, we know what relativization to $V_\alpha$, $\alpha < \kappa$, means.
Since a type 3 parameters are collections of subcollections of $V_\kappa$ and we can already relativize subcollections of $V_\kappa$, this seems to be a reasonable way to extend relativization to type 3 parameters:
\begin{equation}
\mathscr{R}^{V_\alpha} = \{R^{V_\alpha} : R \in \mathscr{R} \}
\end{equation}
Where $R^{V_\alpha}$ is type 2 relativization, which is $R \cap V_\alpha$.
% ========================================================
\subsection{Indescribality}

Since this section talks about indescribability, this is how an ordinal is described according to Drake \cite[Chapter 9]{DrakeBook}.
\begin{definition}\
We say an ordinal $\alpha$ is described by a formula $\varphi(P_1, \ldots, P_n)$ with type 2 parameters $P_1, \ldots, P_n$ given iff
\begin{equation}
\langle V_\alpha, \in \rangle \models \langle \varphi(P_1, \ldots, P_n)
\end{equation}
but for every $\beta < \alpha$
\begin{equation}
\langle V_\beta, \in \rangle \not\models \varphi(P_1 \cap V_\beta, \ldots, P_n \cap V_\beta)
\end{equation}
\end{definition}

Drake then notes that the same notion can be established for sentences if the corresponding type 2 parameters are added to the language. Since the this approach is used by Kanamori in \cite{KanamoriBook}, we will stick to that too.\footnote{The first definition is included because the author of this thesis finds it more intuitive.}
\begin{definition}{Describability}\label{def:describability}\\
We say an ordinal $\alpha$ is described by a sentence $\varphi$ in the language $\mathscr{L}$ with relation symbols $P_1, \ldots, P_n$ given iff
\begin{equation}
\langle V_\alpha, \in, P_1, \ldots, P_n \rangle \models \varphi
\end{equation}
but for every $\beta < \alpha$
\begin{equation}
\langle V_\beta, \in, P_1 \cap V_\beta, \ldots, P_n \cap V_\beta \rangle \not\models \varphi
\end{equation}
\end{definition}

% $\alpha$-Mahlo cardinals are the extreme of regular fixed-point axioms, they are about as high as we can get via normal functions and stationary sets. % preformulovat

%TODO indescribable -- reflecting indescribability -- we can't reach $V$ by a $\Sigma_1^1$ formula, so there's some initial segment $V_\alpha$ that is also unreachable (we say indescribable) by the means of a ... formula

\begin{definition}{($\Pi^m_n$-indescribable cardinal)}\label{def:pi_mn_indescribable}
We say that $\kappa$ is $\Pi^m_n$-indescribable iff it is not described by any $\Pi^m_n$-formula.
\end{definition}
\begin{definition}{($\Sigma^m_n$-indescribable cardinal)}\label{def:sigma_mn_indescribable}
We say that $\kappa$ is $\Sigma^m_n$-indescribable iff it is not described by any $\Sigma^m_n$-formula.
\end{definition}

To see that this notion is based in reflection, note that for $\Pi^m_n$-formulas\footnote{This holds for $\Sigma^m_n$-formulas alike.}, a cardinal $\kappa$ is $\Pi^m_n$-indescribable iff every $\Pi^m_n$-formula reflects in $\kappa$ in the sense of definition \ref{def:reflection_2}.

\begin{lemma}
Let $\kappa$ be a cardinal, the following holds for any $n \in \omega$. $\kappa$ is $\Pi^1_n$-indescribable iff $\kappa$ is $\Sigma^1_n+1$-indescribable
\end{lemma}

\begin{proof}
The forward direction is obvious, we can always add a spare quantifier over a type 2 variable to turn a $\Pi^1_n$ formula $\varphi$ into a $\exists P \varphi$ which is obviously a $\Sigma^1_n+1$ formula.\footnote{Note that unlike in previous sections, it is worth noting that $\varphi$ is now a sentence so we don't have to worry whether $P$ is free in $\varphi$.}

To prove the opposite direction, suppose that $V_\kappa \models \exists X \varphi(X)$ where $X$ is a type 2 variable and $\varphi$ is a $\Pi^1_n$ formula with one free variable of type 2. This means that there is a set $S \subseteq V_\kappa$ that is a witness of $\exists X \varphi(X)$, in other words, $\varphi(S)$ holds. We can replace every occurence of $X$ in $\varphi$ by a new predicate symbol $S$, this allows us to say that $\kappa$ is $\Pi^1_n$-indescribable (with respect to $\langle V_\kappa, \in, R, S \rangle$).
\footnote{A different yet interesting approach is taken by Tate in \ref{Tait_constructingcardinals}. He states that for $n\geq 0$, a formula of order $\leq n$ is called a $\Pi^n_0$ and a $\Sigma^n_0$ formula. Then a $\Pi^n_{m+1}$ is a formula of form $\forall Y \psi(Y)$ where $\psi$ is a $\Sigma^n_m$ formula and $Y$ is a variable of type $n$. Finally, a $\Sigma^n_{m+1}$ is the negation of a $\Pi^n_m$ formula. So the above holds ad definitio.}
\end{proof}

The above lemma makes it clear that we can suppose that all formulas with no higher than type 2 variables are $\Pi^1_n$-formulas, $n \in \omega$, without the loss of generality.

\begin{lemma}\label{lemma:inaccessible_clubset}\
If $\kappa$ is an inaccessible cardinal and given $R \subseteq V_\kappa$, then the following is a club set in $\kappa$:
\begin{equation}
\{\alpha : \alpha < \kappa \et \langle V_\alpha, \in, R \cap V_\alpha \rangle \prec \langle V_\kappa, \in, R \rangle \}\label{eq:inacc_lemma_set}
\end{equation}
\end{lemma}

\begin{proof}
To see that \ref{eq:inacc_lemma_set} is closed, let us recall that a $A \subseteq \kappa$ is closed iff for every ordinal $\alpha < \lambda$, $\alpha \neq \emptyset$: if $A \cap \alpha$ is unbounded in $\alpha$ then $\alpha \in A$. Since $\kappa$ is an inaccessible cardinal, thus strong limit, it is closed under limits of sequences of ordinals lesser than $\kappa$.  

TODO neco s $V_\kappa$, ze je tranzitivni a tak jso vsechny $V_\alpha$ pro $\alpha<\kappa$ $V_\alpha \in V_\kappa$

We want to verify that it is unbounded, we will use a recursively defined sequence $\alpha_0, \alpha_1, \ldots$
to build an elementary substructure of $\langle V_\kappa, \in, R \rangle$ that is built above an arbitrary $\alpha_0 <\kappa$ .
Let us fix an arbitrary $\alpha_0 < \kappa$. Given $\alpha_n$, $\alpha_n+1$ is defined as the least $\beta$, $\alpha_n \leq \beta$ that satisfies 
the following for any formula $\varphi$, $p_1, \ldots, p_m \in V_{\alpha_{n}}, m \in \omega$:
\begin{equation}
\begin{gathered}
\mbox{If }\langle V_\kappa, \in, R \rangle \models \exists x \varphi(p_1, \ldots, p_n)\mbox{,}\
\mbox{then }\langle V_\kappa, \in, R \rangle \models \varphi(x, p_1, \ldots, p_n)
\end{gathered}
\end{equation}

Let $\alpha = \bigcup_{n < \omega} \alpha_n$. 

Then $\langle V_\alpha, \in, R \cap V_\alpha \rangle \prec \langle V_\kappa, \in, R \rangle$, in other words, for any $\varphi$ with given arbitrary parameters $p_1, \ldots, p_n \in V_\alpha$, it holds that
\begin{equation}
\langle V_\alpha, \in, R \cap V_\alpha \rangle \models \varphi(p_1, \ldots, p_n) \iff \langle V_\kappa, \in, R \rangle \models \varphi(p_1, \ldots, p_n)
\end{equation}
Which should be clear from the construction of $\alpha$
\end{proof}

\begin{theorem}
Let $\kappa$ be an ordinal. The following are equivalent.
\bce[(i)]
\item $\kappa$ is inaccessible
\item $\kappa$ is $\Pi^1_0$-indescribable.
\ece
\end{theorem}

%Note that $\Pi^1_0$ formulas are those that contain zero unbound quantifiers over type-2 variables, they are in fact first-order formulas, only with additional free type 2 variables allowed. For an example of a formula with type 1 quantifiers and a type 2 free variable, the axiom schemas used in previous parts, e.g. \emph{Replacement\textsubscript{1}}.


\begin{proof}
Since $\Pi^1_0$-sentences are first-order sentences, we want to prove that $\kappa$ is an inaccessible cardinal iff whenever a first-order tries to describe $\kappa$ in the sense of definition \ref{def:describability}, the formula fails to do so and describes a initial segment thereof instead.
We have already shown in \ref{theorem:inaccessible_models_zfc} that there is no way to reach an inaccesible cardinal via first-order formulas in $\sf{ZFC}$. We will now prove it again in for formal clarity.

For $\bold{(i) \then (ii)}$, suppose that $\kappa$ is inaccessible.

Then there is, by lemma \ref{lemma:inaccessible_clubset} a club set of ordinals $\alpha$ such that $V_\alpha$ is an elementary substructures of $V_\kappa$. For $\kappa$ to be $\Pi^1_0$inderscribable, we need to make sure that given an arbitrary first-order sentence $\varphi$ satisfied in the structure $\langle V_\kappa, \in, R \rangle$, there is an ordinal $\alpha < \kappa$, such that $\langle V_\alpha, \in, R \cap V_\alpha \rangle \models \varphi$. But this follows from the definition of elementary substructure.

For $\bold{(ii) \then (i)}$, suppose $\kappa$ is not inaccessible, so it is either singular, or there is a cardinal $\nu < \kappa$ such that $\kappa \leq \power{\nu}$ or $\kappa=\omega$. 


%For the successor case, there is some $\nu$ so that $\nu+1=\kappa$. 
%Let's take $R = \{\nu\}$ and $\varphi = \exists x \psi(x)$ such that
%\begin{eqaution}
Suppose $\kappa$ is singular. Then there is a cardinal $\nu < \kappa$ and a function $f: \nu \then \kappa$ such that $rng(f)$ is cofinal in $\kappa$. Since $f \subseteq V_\kappa$, we can add $f$ as a relation to the language. We can do the same with $\{\nu\}$. That means $\langle V_\kappa, \in, P_1, P_1$ with $P_1 = f, P_2 = \{\nu\}$ is a structure, 
let $\varphi = P_1 \neq \emptyset \et rng(P_1) = P_2$\footnote{$rng(x)=y$ is a first-order formula, see \ref{def:rng}.}. Since for every $\alpha < \nu$, $P_1 \cap V_\alpha = \emptyset$, $\varphi$ is false and therefore describes $\kappa$. That contradicts the fact that $\kappa$ was supposed to be $\Pi^1_0$-indescribable, but $\varphi$ is a first-order formula.

Suppose there a cardinal $\nu$ satisfying $\kappa \leq \power{\nu}$. Let there be a function $f: \power{\nu} \then \kappa$ that is onto. Then, like in the previous paragraph, we can obtain a structure $\langle V_\kappa, \in, P_1, P_2 \rangle$, where $P_1 = f$ like before, but this time $P_2 = \power{\nu}$. Again, $\varphi = P_1 \neq \emptyset \et rng(P_1) = P_2$ describes $\kappa$.

Finally, suppose $\kappa = \omega$, then the sentence $\varphi = \forall x \exists y (x \in y)$ describes $\kappa$, there is obviously no $\alpha < \omega$ such that $\langle V_\alpha, \in \rangle \models \varphi$.

\end{proof}

Generally, it should be clear that it a cardinal $\kappa$ is $\Pi^m_n$-indescribable, it is also $\Pi^{m'}_{n'}$-indescribable for every $m'<m, n'<n$. By the same line of thought, if a cardinal $\kappa$ satisfies property implied by $\Pi^m_n$-indescribability, it satisfies all properties implied by $\Pi^{m'}_{n'}$-indescribability for $m'<m, n'<n$, for example $\kappa$ is $\Pi^m_n$-indescribable for $m \geq 1, n \geq 0$, it is also an inaccessible cardinal.

% TODO pozorovani ze kdyz je $\kappa$ $\Pi$

\begin{theorem}\
If a cardinal $\kappa$ is $\Pi^1_1$-indescribable, then it is a Mahlo cardinal.
\end{theorem}

% todo kappa a ne v-kappa?
\begin{proof}
Assuming that $\kappa$ is $\Pi^1_1$-indescribable, we want to prove that every club set in $\kappa$ contains an inaccessible cardinal. 

Consider the following $\Pi^1_1$-sentence:
\begin{equation}
\begin{gathered}\label{eq:inac}
\forall P (\mbox{"$P$ is a function"} \et \exists x(x = dom(P) \lor \power{x} = dom(P)) \then\
\then \exists y(y = rng(P)))
\end{gathered}
\end{equation}
where $P$ is a type 2 variable and $x, y$ are type 1 variables, $rng(P)$ is defined in \ref{def:rng}, $dom(P)$ in \ref{def:dom} and "$P$ is a function" is a first-order formula defined in \ref{def:function}.
We will call this sentence \emph{Inac}, as in "inaccessible", because, given a cardinal $\mu$, the following holds if and only if $\mu$ is inaccessible:
\begin{equation}
\langle V_\mu, \in \rangle \models Inac
\end{equation}

So let's fix an arbitrary $C \subset \kappa$, club set in $\kappa$. We want to show that it contains an inaccessible cardinal. Since $C$ is a subset of $V_\kappa$, let's add it to the structure $\langle V_\kappa, \in \rangle$, turning it into $\langle V_\kappa, \in, C \rangle$. Then the following holds:
\begin{equation}
\langle V_\kappa, \in, C \rangle \models Inac \et \mbox{"$C$ in unbounded"}
\end{equation}
Note that this is correct, because, as we have noted just before introducing the statement now being proven, if $\kappa$ is $\Pi^1_1$-indescribable, it is also $\Pi^1_0$-indescribable. So $\kappa$ is itself inaccessible and therefore $\langle V_\kappa, \in, C \rangle \models Inac$. $C$ is obviously picked so that it is unbounded in $\kappa$\footnote{"$C$ in unbounded" is a first-order formula defined in \ref{def:unbounded_class}}.

Now because we have assumed that $\kappa$ is $\Pi^1_1$-indescribable and $Inac$ is a $\Pi^1_1$-formula, so $Inac \et \mbox{"$C$ in unbounded"}$ is equivalent to a $\Pi^1_1$-formula, there must be an ordinal $\alpha$ that satisfies
\begin{equation}
\langle V_\alpha, \in, C \cap V_\alpha \rangle \models Inac \et \mbox{"$C$ in unbounded"}
\end{equation}
which implies that $\alpha$ is inaccessible. 

To be finished, we need to verify that $\alpha \in C$. Since $\kappa = V_\kappa$ for inaccessible $\kappa$\footnote{TODO link -------- ?}, $C \cap V_\alpha = C \cap \alpha$, from unboundedness of $C \cap \alpha$ in $\alpha$, $\bigcup(C \cap \alpha) = \alpha$, which, together with the fact that $C$ is a club set in $\kappa$ and therefore closed in $\kappa$, yields that $\alpha \in C$.
\end{proof}

TODO asi jako Drake, pozn ze to jde i pro hyper-Mahlovy?

\begin{definition}{(Extension property)}
We say that a cardinal $\kappa$ has the \emph{extension property} iff for any $R \subseteq V_\kappa$ there is a~transitive set 
$X \neq V_\kappa$ and an $S \subseteq X$ such that 
$\langle V_\kappa, \in, R \rangle \prec \langle X, \in, S \rangle$
\end{definition}

%TODO co to znamena?

\begin{definition}{(Weakly compact cardinal)}\label{def:weakly_compact_extension}\\
We say that a cardinal $\kappa$ is \emph{weakly compact} iff it has the extension property.
\end{definition}

%\begin{definition}{(Weakly compact cardinal)}\label{def:weakly_compact_indescribable}
%We say that a cardinal $\kappa$ is \emph{weakly compact} iff it is $\Pi^1_1$-indescribable.
%\end{definition}

The above definitions are equivalent

\begin{theorem}
the following are equivalent:\\
\bce[(i)]
\item $\kappa$ is \emph{Weakly compact}.
\item $\kappa$ is $\Pi^1_1$-indescribable.
\ece
\end{theorem}

For a proof, see \cite{KanamoriBook}[Theorem 6.4]

% TODO def totalne nepopsatelny kardinal

% TODO asi nekde bude meritelny kardinal

{\color{red}
\begin{comment}
% balcar - stepanek strana 314, veta 5.10.
\begin{theorem}\label{th:refl_weakly_compact}
Let $\kappa$ be a~weakly compact cardinal. Then for every stationary set $S \subset \kappa$ there is an uncountable regular cardinal $\lambda < \kappa$ such that the set $S \cap \lambda$ is stationary in $\lambda$.
\end{theorem}
\begin{proof}
TODO % WATT??
\end{proof}

\

\end{comment}
}

\subsection{Measurable Cardinal}

TODO refaktorizovat fle:
\begin{definition}{(Ultrafilter)}\\
Given a set $X$, we say $U \subset \power{X}$ is an \emph{ultrafilter} iff all of the following hold:
\bce[(i)]
\item $\emptyset \not\in U$
\item $\forall x, y (\subset X \et x \subset y \et x \in U \then y \in U)$
\item $\forall x, y \in U (x \cap y) \in U$
\item $\forall x (x \subset X \then (x \in U \lor (X \setminus x) \in U))$
\ece
\end{definition}

\begin{definition}{($\kappa$-complete ultrafilter)}\\
We say that an ultrafilter $U$ is $\kappa$-complete iff
\end{definition}

\begin{definition}{(non-principal ultrafilter)}\\
TODO
\end{definition}

\begin{definition}{(Measurable Cardinal)}\\
Let $\kappa$ be a caridnal. We say is a \emph{measurable cardinal} iff it is an uncountable cardinal with a $\kappa$-complete, non-principal ultrafilter.
\end{definition}

\begin{theorem}
Let $\kappa$ be a cardinal. If $\kappa$ is a measurable cardinal then it is $\Pi^2_1$-indescribable.
\end{theorem}

\begin{theorem}
TODO Pod kazdym meritelnym kardinalem existuje ultrafiltr totalne nepopsatelnych, ktere tim padem nejsou sestrojitelne. VIZ VETA Z KANAMORIHO.
\end{theorem}

%\newpage
% =====================================================================================================================================

\subsection{The Constructible Universe} % Reflection and the constructible universe

% TODO reflektovat muzeme jenom kardinaly konzistentni s V=L, proc????

% TODO Plagiat -- prepsat a~vysvetlit

The constructible universe, denoted $L$, is a cumulative hierarchy of sets, presented by Kurt Gödel in his 1938 paper \emph{The Consistency of the Axiom of Choice and of the Generalised Continuum Hypothesis}. For a technical description, see below. Assertion of their equality, $V=L$, is called the \emph{axiom of constructibility}. The axiom implies GCH and therefore also AC and contradicts the existence of some of the large cardinals, our goal is to decide whether those introduced earlier are among them.

On order to formally establish this class, we need to formalize the notion of definability first. 
\begin{definition}
We say that a set $X$ is \emph{definable} over a model $\langle M, \in \rangle$ if there is a first-order formula $\varphi$ together with parameters $p_1, \ldots, p_n \in M$ such that
\begin{equation}
X = \{x: x \in M \et \langle M, \in \rangle \models \varphi(x, p_1, \ldots, p_n)\}
\end{equation}
\end{definition}

\begin{definition}{(The set of definable subsets)}\\
The following is a set of all definable subsets of a given set $M$, denoted Def($M$).
\begin{equation}
\begin{gathered}
Def(M) = \{\{y : x \in M \land \langle M, \in \rangle \models \varphi(y, u_1, \ldots, i_n) \} |\\
\mbox{ $\varphi$ is a~first-order formula, }p_1, \ldots, p_n \in M \}
\end{gathered}
\end{equation}
\end{definition}

We will use $Def(M)$ in the following construction in the way the powerset operation is used when constructing the usual Von Neumann's hierarchy of sets\footnote{For that reason, some authors use $\power^{\*}{M}$ instead of $Def(M)$, see section 11 of \cite{PinterBook} for one such example.}

Now we can recursively build $L$.
\begin{definition}{(The Constructible universe)}\label{def:constructible_universe}\\
\bce[(i)]
\item
\begin{equation}
L_0 \defeq  \emptyset
\end{equation}

\item
\begin{equation}
L_{\alpha+1} \defeq  Def(L_{\alpha})
\end{equation}
\item
\begin{equation}
L_{\lambda} = \bigcup_{\alpha < \lambda} L_{\alpha}\mbox{ If }\lambda\mbox{ is a~limit ordinal }
\end{equation}
\item
\begin{equation}\label{eq:def_l}
L = \bigcup_{\alpha\in Ord} L_{\alpha}
\end{equation}
\ece
\end{definition}

Note that while $L$ bears very close resemblance to $V$, the difference is, that in every successor step of constructing $V$, we take every subset of $V_\alpha$ to be $V_{\alpha+1}$, whereas $L_{\alpha+1}$ consists only of definable subsets of $L_\alpha$. Also note that $L$ is transitive.

In order to 

\begin{theorem}
Let $L$ be as in \ref{def:constructible_universe}.
\begin{equation}
L \models \sf{ZFC}
\end{equation}
\end{theorem}
For details, refer to Jech: \cite{JechBook}[Theorem 13.3].

\begin{definition}{(Constructibility)}\\
The axiom of constructibility say that every set is constructible. It is usually denoted as $L = V$.
\end{definition}

Without providing a proof, we will introduce two important results established by Gödel in TODO citace!

\begin{theorem}{(Constructibility $\then$ Choice)}
\begin{equation}
\sf{ZF} \models \mbox{\emph{Constructibility}} \then \mbox{\emph{Choice}} 
\end{equation}
\end{theorem}

\begin{definition}{(GCH)}\label{def:gch}\\
\emph{Generalized Continuum Hypothesis}, usually denoted \emph{GCH} for brevity, refers to the following statement:
\begin{equation}
\aleph_{n+1} = \power{\aleph_n}
\end{equation}
\end{definition}

\begin{theorem}{(Constructibility $\then$ Continuum Hypothesis)}\label{theorem:l_then_gch}
\begin{equation}
\sf{ZF} \models \mbox{\emph{Constructibility}} \then \mbox{\emph{GCH}} 
\end{equation}
\end{theorem}
It is worth mentioning that Gödel's proof of \emph{Construcibility} $\then$ \emph{GCH} featured the first formal use of a reflection principle. 
For the actual proofs, see for example TODO citace!! Kunen?

Since \emph{GCH} implies that $\kappa$ is a limit cardinal iff $\kappa$ is a strong limit cardinal for every $\kappa$, the distinctions between inaccessible and weakly inaccessible cardinals as well as between Mahlo and weakly Mahlo cardinals vanish.

% -------------------

Pokud $\kappa$ je nejaky kardinal ve $V$, je takovy i v $L$?

% TODO lemma: ord jsou ord v L (kunen?) "x is an ordinal" je delta_0

TODO lemma: card jsou card v L

\begin{theorem}{(Inaccessibility in $L$)}\label{theorem:inaccessible_in_l}\\
Let $\kappa$ be an inaccessible cardinal. Then $L \models \mbox{"$\kappa$ is inaccessible"}$.
\end{theorem}
\begin{proof}
We want to show that the following are all true for an inaccessible cardinal $\kappa$:
\bce[(i)] 
\item $L \models \mbox{"$\kappa$ is a cardinal"}$
\item $L \models \omega < \kappa$
\item $L \models \mbox{"$\kappa$ is regular"}$
\item $L \models \mbox{"$\kappa$ is limit"}$ . While inaccessible cardinals are strong limit cardinals, since \emph{GCH} holds in $L$, $L \models \mbox{"$\kappa$ is limit"}$ implies $L \models \mbox{"$\kappa$ is strong limit"}$.
\ece

Suppose $L \models \mbox{"$\kappa$ is not a cardinal"}$ holds, then there is a caridnal $\mu$, $\mu < \kappa$ and a function $f:\mu\then\kappa$, $f \in L$, such that $L\models \mbox{"$f:\mu\then\kappa$ is onto"}$. But since "$f$ is onto" is a $\Delta_0$ formula and $\Delta_0$ formulas are are absolute in transitive structures\footnote{see lemma \ref{def:delta_0_absoluteness}} and $L$ is a transitive class, $\mbox{"$f$ is onto"}^M \iff \mbox{"$f$ is onto"}$, this contradicts the fact that $\kappa$ is a cardinal.

Similarly, $\omega < \kappa$ is absolute in $L$.
% Since $\kappa>\omega$ (as an ordinal), this also yields that $L\models\mbox{"$\kappa$ is uncountable"}$.

%Regularity can be verified in a similar manner: since $\kappa$ is regular, every unbounded subset of $\kappa$ is cofinal, given one such set $y$, there is a function $f: Ord \then \kappa$ that is onto, let $x$ be the set of all . Since "$f$ is onto" is absolute, $L \models \mbox{"$rng(f)$ is unbounded in $\kappa$"}$. So $\kappa$ is regular 
TODO regularity!

Since $L\models GCH$ by \ref{theorem:l_then_gch}, it now suffices to show that $L\models\mbox{"$\kappa$ is a limit cardinal"}$. Since $\kappa$ is a limit cardinal, there is a limit ordinal $\lambda$ such that $\kappa = \aleph_\lambda$\footnote{In fact $\lambda = \kappa$ from inaccessibility, but we don't need right now for limitness.}.
Consider an ordinal $\alpha$ such that $\alpha<\lambda$. We know that ordinals are also "ordinals in $L$", so $\alpha \in \lambda$, thus $(\alpha \in \lambda)^L$, but also given an ordinal $\beta$ such that $\alpha < \beta < \lambda$, $(\alpha < \beta < \kappa)^L$. We have 
L \models (\forall\alpha<\lambda)(\exists \alpha<\beta<\lambda):\mbox{"$\mu$ is an ordinal."}

TODO JAK JE TO S ALEPHEM?
\end{proof}

\begin{theorem}{(Mahloness in $L$)}\label{theorem:mahlo_in_l}\\
Let $\kappa$ be a Mahlo cardinal. Then $L \models \mbox{"$\kappa$ is Mahlo"}$.
\end{theorem}
% http://math.stackexchange.com/questions/1791631/reference-mahlo-cardinals-remain-mahlo-in-l/1792486#1792486
% TODO citace webu?

\begin{proof}
Assume that there is a Mahlo cardinal and let's denote it $\kappa$. From the definition of Mahloness in \ref{def:mahlo_cardinal}, the set 
\begin{equation}
\{\alpha : \alpha \in \kappa \et \mbox{"$\alpha$ is inaccessible"}\}
\end{equation}
is stationary in $\kappa$ and $\kappa$ is also inaccessible.

Now let $T\defeq \{\alpha \in \kappa : L\models\mbox{"$\alpha$ is regular"}\}$. By the argument given above, any $\alpha$ that is regular in $V$is regular in $L$and hence $S\subseteq T$. Suppose that $L \models \mbox{"$\kappa$ is not Mahlo"}$. Then there is some $C\subseteq \kappa$ such that $L\models C$ is club in $\kappa$ and $C\cap T=\emptyset L\models$ C is club in $\kappa$ and $C\cap T=\emptyset$ . Being club is a $\Sigma_0$, property and hence $C\subseteq \kappa$ is club in $\kappa$. Since $S \subseteq T$, we have that $C\cap S=\emptyset$ and hence $S$ is not stationary in $V$. This is a contradiction and we therefore must have that $L \models \mbox{$T$ is stationary}$.

Thus, $\kappa$ remains Mahlo in $L$.


\end{proof}

%\begin{theorem}
%\begin{equation}
%Con(L + \exists \kappa (\kappa \mbox{" is a Weakly Inaccessible Cardinal Cardinal"})) then Con(L)
%\end{equation}
%\end{theorem}

\begin{theorem}
\begin{equation}
Con(L) -> Con(L + \exists \kappa (\kappa \mbox{" is a Measurable Cardinal"})) then Con(L)
\end{equation}
\end{theorem}
Ktera vera?

TODO vyska / sirka univerza

TODO co velky pismena ve jmenech kardinalu?

{\color{red}
\begin{comment}

TODO Plagiat -- prepsat a~vysvetlit
\begin{Fact}
The reflection -- constructed as explained in the previous paragraph (!!! preformulovat !!!) -- with second-order parameters for higher-order formulas (even of transfinite type) does not yield transcendence over $L$.
\end{Fact}

\end{comment}
}

TODO zduvodneni

\

TODO kratka diskuse jestli refl implikuje transcendenci na L - polemika, nazor - V=L a~slaba kompaktnost a~dalsi

\
