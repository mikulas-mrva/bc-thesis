\subsection{Notation and Terminology}
\subsubsection{The Language of Set Theory}
% TODO predpokladame splnovani a tak, z logiky? link na ucebnici kdyztak? 

We are about to define basic set-theoretical terminology on which the rest of this thesis will be built. For Chapter 2, the underlying theory will be the \emph{Zermelo –Fraenkel} set theory with the Axiom of Choice ($\sf{ZFC}$), a first-order set theory in the language $\mathscr{L} = \{=, \in\}$, which will be sometimes referred to as \emph{the language of set theory}. In Chapter 3\footnote{TODO bude jich vic? Chapter 4 taky?}, we shall always make it clear whether we are in first-order $\sf{ZFC}$ or second-order $\sf{ZFC}_2$, which will be precisely defined later in this chapter. When in second-order theory, we will usually denote type 1 variables, which are elements of the domain of discourse\footnote{co je "domain of discourse"?} by lower-case letters, mostly $u, v, w, x, y, z, p_1, p_2, p_3,  \ldots$ while type 2 variables, which represent $n$-ary relations of the domain of discourse for any natural number $n$, are usually denoted by upper-case letters $A, B, C, X, Y, Z$. Note that those may be used both as relations and functions, see the definition of a function below.\footnote{TODO ref?}

TODO uppercase $M$ is a set!

TODO "$M$ is a limit ordinal" je ve skutecnosti formule, nekam to sem napis!

The informal notions of \emph{class} and \emph{property} will be used throughout this thesis. They both represent formulas with respect to the domain of discourse. If $\varphi(x, p_1, \ldots, p_n)$ is a formula in the language of set theory, we call 
\begin{equation}
A = \{x : \varphi(x, p_1, \ldots, p_n)\}
\end{equation}
a class of all sets satisfying $\varphi(x, p_1, \ldots, p_n)$ in a sense that 
\begin{equation}
x \in A \iff \varphi(x, p_1, \ldots, p_n)
\end{equation}
One can easily define for classes $A$, $B$ the operations like $A \cap B$, $A \cup B$, $A \setminus C$, $\bigcup A$, but it is elementary and we won't do it here, see the first part of \cite{JechBook} for technical details. The following axioms are the tools by which decide whether particular classes are in fact sets. A class that fails to be considered a set is called a \emph{proper class}.
% taky diag -- class/property/formula je totez, formula je totiz trida n-tic ktere ji splnuji
\


\subsubsection{The Axioms}

\begin{definition}{(The existence of a set)}\label{def:existence_of_a_set}
\begin{equation}
\exists x (x = x)
\end{equation}
\end{definition}
The above axiom is usually not used because it can be deduced from the axiom of \emph{Infinity} (see below), but since we will be using set theories that omit \emph{Infinity}, this will be useful.

\begin{definition}{(Extensionality)}\label{def:extensionality}
\begin{equation}
\forall x, y, z ((z \in x \iff z \in y) \iff x = y)
\end{equation}
\end{definition}

\begin{definition}{(Specification)}\label{def:specification}\\
The following is a schema for every first-order formula $\varphi(x, p_1, \ldots, p_n)$ with no free variables other than $x, p_1, \ldots, p_n$.
\begin{equation}
\forall x, p_1, \ldots, p_n \exists y \forall z ( z \in y \iff ( z \in x \et \varphi(z, p_1, \ldots, p_n)))
\end{equation}
\end{definition}

We will now provide two definitions that are not axioms, but will be helpful in establishing some of the other axioms in a more intuitive way.
\begin{definition}{($x \subseteq y$, $x \subset y$)}
\begin{equation}
x \subseteq y \iff \forall z(z \in x \then z \in y)
\end{equation}
\begin{equation}
x \subset y \iff x \subseteq y \et x \neq y
\end{equation}
\end{definition}

\begin{definition}{(Empty set)}\label{def:emptyset}
\begin{equation}
\emptyset \defeq \{x : x \neq x\}
\end{equation}
\end{definition}
To make sure that $\emptyset$ is a set, note that there exists at least one set $y$ from \ref{def:existence_of_a_set}, then consider the following alternative definition.

\begin{equation}
\emptyset' \defeq \{x : \varphi(x) \et x \in y\}\mbox{ where $y$ $\varphi$ is the formula "$x \neq x$".}
\end{equation}
It should be clear that $\emptyset' = \emptyset$.\footnote{For details, see page 8 in \cite{JechBook}.}

Now we can introduce more axioms.
\begin{definition}{(Foundation)}\label{def:foundation}
\begin{equation}
\forall x (x \neq \emptyset \then \exists (y \in x) (\forall z \neg (z \in y \et z \in x)))
\end{equation}
\end{definition}

\begin{definition}{(Pairing)}\label{def:pairing}
\begin{equation}
\forall x, y \exists z \forall q (q \in z \iff q \in z \lor q \in y)
\end{equation}
\end{definition}

\begin{definition}{(Union)}\label{def:union}
\begin{equation}
\forall x \exists y \forall z (z \in y \iff \exists q( z \in q \et q \in x))
\end{equation}
\end{definition}

\begin{definition}{(Powerset)}\label{def:powerset}
\begin{equation}
\forall x \exists y \forall z (z \subseteq x \iff z \in y)
\end{equation}
\end{definition}

\begin{definition}{(Infinity)}\label{def:infinity}
\begin{equation}
\exists x (\forall y \in x)(y\cup\{y\} \in x)
\end{equation}
\end{definition}

Let us introduce a few more definitions that will make the two remaining axioms more comprehensible.
\begin{definition}{(Function)}\label{def:function}\\
Given arbitrary first-order formula $\varphi(x, y, p_1, \ldots, p_n)$, we say that $\varphi$ is a function iff
\begin{equation}\label{def:function_formula}
\forall x, y, z, p_1, \ldots, p_n (\varphi(x, y, p_1, \ldots, p_n) \et \varphi(x, z, p_1, \ldots, p_n) \then y = z)
\end{equation}
\end{definition}
When a $\varphi(x, y)$ is a function, we also write the following:
\begin{equation}
f(x) = y \iff \varphi(x, y)
\end{equation}
Note that this $f$ is in fact a formula TODO ???

TODO $f = \{(x, y) : \varphi(x, y)\}$ !!! f muze byt mnozina i trida!
\footnote{This can also be done for $\varphi$s with more than two free variables by either setting $f(x, p_1, \ldots p_n) = y \iff \varphi(x, y p_1, \ldots, p_n)$ or saying that $\varphi$ codes more functions, determined by the various parameters, so $f(x) = y \iff \varphi(x, y, t_1, \ldots, t_n)$ for given terms $t_1, \ldots, t_n$.}

\begin{definition}{(Dom(f))}\label{def:dom}\\
Let $f$ be a function. We read the following as "Dom(f) is the domain of f".
\begin{equation}
Dom(f) \defeq \{x : \exists y (f(x) = y)\}
\end{equation}
\end{definition}
We say "$f$ is a function on $A$", $A$ being a class, if $A = dom(f)$.

\begin{definition}{(Rng(f))}\label{def:rng}\\
Let $f$ be a function. We read the following as "Rng(f) is the range of f".
\begin{equation}
Rng(f) \defeq \{x : \exists y (f(x) = y)\}
\end{equation}
\end{definition}
We say that $f$ is i function into $A$, $A$ being a class, if $rng(f) \subseteq A$.

Note that \emph{Dom(f)} and \emph{Rng(f)} are not definitions in a strict sense, they are in fact definition schemas that yield definitions for every function $f$ given. Also note that they can be easily modified for $\varphi$ instead of $f$, with the only difference being the fact that it is then defined only for those $\varphi$s that are functions, which must be taken into account. This is worth noting as we will sometimes interchange the notions of \emph{function} and \emph{formula}.

\begin{definition}{(Function Defined For All Ordinals)}\label{def:function_dfao}\\
We say a function $f$ is \emph{defined for all ordinals}, this is sometimes written $f: Ord \then A$ for any class $A$, if $Dom(f) = Ord$.\
Alternatively,
\begin{equation}
(\forall \alpha \in Ord)(\exists y \in A)(f(\alpha) = y))
\end{equation}
\end{definition}

\begin{definition}{(Powerset)}\\
Given a set $x$, the \emph{powerset of $x$}, denoted $\power{x}$, is defined as follows:
\begin{equation}
\power{x} \defeq \{y: y \subseteq x\}
\end{equation}
\end{definition}

And now for the axioms.
\begin{definition}{(Replacement)}\label{def:replacement}\\
The following is a schema for every first-order formula $\varphi(x, p_1, \ldots, p_n)$ with no free variables other than $x, p_1, \ldots, p_n$.
\begin{equation}
"\varphi\mbox{ is a function}"\then \forall x \exists y \forall z (z \in y \iff (\exists q \in x)(\varphi(x, y, p_1, \ldots, p_n))
\end{equation}
\end{definition}

\begin{definition}{(Choice)}\label{def:choice}\\
This is also a schema. For every $A$, a family of non-empty sets\footnote{We say a class $A$ is a "family of non-empty sets" iff there is $B$ such that $A \subseteq \power{B}$}, such that $\emptyset \not\in S$, there is a function $f$ such that for every $x \in A$
\begin{equation}
f(x) \in x
\end{equation}
\end{definition}

We will refer the axioms by their name, written in italic type, e.g. \emph{Foundation} refers to the Axiom of Foundation. Now we need to define some basic set theories to be used in the article. There will be others introduce in Chapter 3, but those will usually be defined just by appending additional axioms or schemata to one of the following.

\begin{definition}{$(\sf{S})$}\label{def:s}\\
We call $\sf{S}$ a set theory with the following axioms:
\bce[(i)]
\item \emph{Existence of a set} (see \ref{def:existence_of_a_set})
\item \emph{Extensionality} (see \ref{def:extensionality})
\item \emph{Specification} (see \ref{def:specification})
\item \emph{Foundation} (see \ref{def:foundation})
\item \emph{Pairing} (see \ref{def:pairing})
\item \emph{Union} (see \ref{def:union})
\item \emph{Powerset} (see \ref{def:powerset})
\ece
\end{definition}

\begin{definition}{$(\sf{ZF})$}\label{def:zf}\\
We call $\sf{ZF}$ a set theory that contains all the axioms of the theory $\sf{S}$\footnote{With the exception of \emph{Existence of a set}} in addition to the following
\bce[(i)]
\item \emph{Replacement} schema (see \ref{def:replacement})
\item \emph{Infinity} (see \ref{def:infinity})
\ece
\end{definition}

\begin{definition}{$(\sf{ZFC})$}\label{def:zfc}\\
$\sf{ZFC}$ is a theory that contains all the axioms of $\sf{ZF}$ plus \emph{Choice} (\ref{def:choice}).
\end{definition}

\

\subsubsection{The Transitive Universe}
% $V$ a $V_\alpha$ odkazuji k Von Neumannove hierarchii (pro jistotu)
\begin{definition}{(Transitive class)}\label{def:transitivity}\\
We say a class $A$ is \emph{transitive} iff
\begin{equation}
\forall x(x \in A \then x \subseteq A)
\end{equation}
\end{definition}

\begin{definition}{Well Ordered Class}\label{def:well_ordering}
A class $A$ is said to be \emph{well ordered by $\in$} iff the following hold:
\bce[(i)]
\item $(\forall x \in A)(x \not\in x)$ (Antireflexivity)
\item $(\forall x, y, z \in A)(x \in y \et y \in z \then x \in z)$ (Transitivity)
\item $(\forall x, y \in A)(x = y \lor x \in y \lor y \in x)$ (Linearity)
\item $(\forall x)(x \subseteq A \et x \neq \emptyset \then (\exists y \in x)(\forall z \in x)(z = y \lor z \in y)))$ % (every nonempty subclass has a least element)
\ece
\end{definition}

\begin{definition}{(Ordinal number)}\label{def:ordinal}\\
A set $x$ is said to be an \emph{ordinal number}, also known as an \emph{ordinal}, if it is \emph{transitive} and \emph{well-ordered by $\in$}. 
\end{definition}
For the sake of brevity, we usually just say "$x$ is an \emph{ordinal}". 
Note that "$x$ is an ordinal" is a well-defined formula, since \ref{def:transitivity} is a formula and \ref{def:well_ordering} is in fact a conjunction of four formulas.
Ordinals will be usually denoted by lower case greek letters, starting from the beginning: $\alpha, \beta, \gamma, \ldots$.
Given two different ordinals $\alpha, \beta$, we will write $\alpha < \beta$ for $\alpha \in \beta$, see \cite{JechBook}{Lemma 2.11} for technical details.

\begin{definition}{(Successor Ordinal)}\label{def:successor_ordinal}\\
Consider the following operation
\begin{equation}
\beta + 1 \defeq \beta \cup \{\beta\}
\end{equation}
An ordinal $\alpha$ is called a \emph{successor ordinal} iff there is an ordinal $\beta$, such that $\alpha = \beta+1$
\end{definition}

\begin{definition}{(Limit Ordinal)}\label{def:limit_ordinal}\\
A non-zero ordinal $\alpha$\footnote{$\alpha \neq \emptyset$} is called a \emph{limit ordinal} iff it is not a successor ordinal.
\end{definition}

\begin{definition}{(Ord)}\label{def:ord}\\
\emph{The class of all ordinal numbers}, which we will denote $Ord$\footnote{It is sometimes denoted $On$, but we will stick to the notation in \cite{JechBook}} be the following class:
\begin{equation}
Ord \defeq \{x : x\mbox{ is an ordinal}\}
\end{equation}
\end{definition}

The following construction will be often referred to as the \emph{Von Neumann's Hierarchy}, sometimes also the \emph{Von Neumann's Universe}. %, the former referring more to the construction with the individual levels in mind, the latter referring more to the class $V$, but they can be interchanged with no confusion caused.

\begin{definition}{(Von Neumann's Hierarchy)}\label{def:von_neumann}\\
The \emph{Von Neumann's Hierarchy} is a collection of sets indexed by elements of $Ord$, defined recursively in the following way:
\bce[(i)]
\item 
\begin{equation}
V_0 = \emptyset
\end{equation}
\item 
\begin{equation}
V_{\alpha+1} = \power{V_\alpha}\mbox{ for any ordinal $\alpha$}
\end{equation}
\item
\begin{equation} 
V_\lambda = \bigcup_{\beta < \lambda} V_\beta \mbox{ for a limit ordinal $\lambda$}
\end{equation}
\ece
\end{definition}

\begin{definition}{(Rank)}\label{def:rank}\\
Given a set $x$, we say that the rank of $x$ (written as $rank(x)$) is the least ordinal $\alpha$ such that
\begin{equation}
x \in V_{\alpha+1}
\end{equation}
\end{definition}
Due to \emph{Regularity}, every set has a rank.\footnote{See chapter 6 of \cite{JechBook} for details.}

\begin{definition}{($\omega$)}\\
\begin{equation}
\omega \defeq \bigcap\{x : \mbox{"$x$ is a limit ordinal"})\}
\end{equation}
\end{definition}

\

\subsubsection{Cardinal Numbers}

\begin{definition}{(Cardinality)}\\
Given a set $x$, let the cardinality of $x$, written $|x|$, be defined as the smallest ordinal number such that there is an injective mapping from $x$ to $\alpha$.
\end{definition}
For formal details as well as why every set can be well-ordered assuming \emph{Choice}, see \cite{JechBook}.

\begin{definition}{(Aleph function)}\label{def:aleph}\\
Let $\omega$ be the set defined by \ref{def:omega}.
We will recursively define the function $\aleph$ for all ordinals.
\bce[(i)]
\item $\aleph_0 = \omega$
\item $\aleph_{\alpha+1}$ is the least cardinal larger than $\aleph_\alpha$\footnote{"The least cardinal larger than $\aleph_\alpha$" is sometimes notated as $\aleph_\alpha^{+}$}
\item $\aleph_\lambda = \bigcup_{\beta < \lambda}\aleph_\beta$ for a limit ordinal $\lambda$
\ece
\end{definition}

\begin{definition}{(Cardinal number)}\label{def:cardinal}\\
We say a set $x$ is a \emph{cardinal number}, usually called a \emph{cardinal}, if either $x \in \omega$, it is then called a \emph{finite cardinal}, 
there is an ordinal $\alpha$ such that $\aleph_\alpha = x$, then we call 
\end{definition}
Infinite cardinals will be notated by lower-case greek letters from the middle if the alphabet, e.g. $\kappa, \mu, \ni, \ldots$.\footnote{$\lambda$ is preferably used for limit ordinals, if it is ever used to denote an infinite cardinal, that should be contextually clear.}

\begin{definition}{(Cofinality of an ordinal)}\label{def:cofinality}\\ % a co https://math.berkeley.edu/~jhicks/links/SOTS/cskipper112613.pdf?
Let $\lambda$ be a limit ordinal. The \emph{cofinality} of $\lambda$, written $cf(\lambda)$, is the smallest limit ordinal $\alpha$, $\alpha \leq \lambda$, such that 
\begin{equation}
(\forall x \in \lambda)(\exists y \in \alpha)(x < y)
\end{equation}
\end{definition}\footnote{Cofinality is usually defined for arbitrary sets, but we won't use that in this thesis and the above definition is very convenient.}

\begin{definition}{(Regular Cardinal)}\label{def:regular_cardinal}\\
We say a cardinal $\kappa$ is regular iff $cf(\kappa) = \kappa$
\end{definition}

\begin{definition}{(Limit Cardinal)}\label{def:limit_ordinal}\\
We say that a cardinal $\kappa$ is a \emph{limit cardinal} if
\begin{equation}
(\exists \alpha \in Ord)(\kappa = \aleph_\alpha)
\end{equation}
\end{definition}

\begin{definition}{(Strong Limit Cardinal)}\label{def:limit_ordinal}\\
We say that an ordinal $\kappa$ is a \emph{strong limit cardinal} if it is a \emph{limit cardinal} and 
\begin{equation}
\forall \alpha (\alpha \in \kappa \then \power{\alpha} \in \kappa)
\end{equation}
\end{definition}

\begin{definition}{(Generalised Continuum Hypothesis)}\\
\begin{equation}
\aleph_{\alpha+1}=\power{\aleph_\alpha}
\end{equation}
\end{definition}
If \emph{GCH} holds (for example in Gödel's $L$, see chapter 3), the notions of a limit cardinal and a strong limit cardinal are equivalent.

\

\subsubsection{Relativisation and Absoluteness}
\begin{definition}{(Relativization)}\label{def:relativization}\\
Let $M$ be a class, $R \subseteq M\times M$ and let $\varphi(p_1, \ldots, p_n)$ be a first-order formula with no free variables besides $p_1, \ldots, _n$. 
The \emph{relativization of $\varphi$ to $M$ and $R$} is the formula, written as $\varphi^{M, R}(p_1, \ldots, p_n)$, defined in the following inductive manner:
\bce[(i)]
\item $(x \in y)^{M,R} \iff R(x, y)$
\item $(x = y)^{M,R} \iff x = y$
\item $(\neg \varphi)^{M,R} \iff \neg \varphi^{M,R}$
\item $(\varphi \et \psi)^{M,R} \iff \varphi^{M,R} \et \psi^{M,R}$
\item $(\varphi \lor \psi)^{M,R} \iff \varphi^{M,R} \lor \psi^{M,R}$
\item $(\varphi \then \psi)^{M,R} \iff \varphi^{M,R} \then \psi^{M,R}$
\item $(\exists x \varphi(x))^{M,R} \iff (\exists x \in M) \varphi^{M,R}(x)$
\item $(\forall x \varphi(x))^{M,R} \iff (\forall x \in M) \varphi^{M,R}(x)$
\ece
\end{definition}
When $R=\in\cap(M \times M)$, we usually write $\varphi^M$ instead of $\varphi^{M, R}$. When we talk about $\varphi^M(p_1, \ldots, p_n)$, it is understood that $p_1, \ldots, p_n \in M$.
We will also use $M \models \varphi(p_1, \ldots, p_n)$ and $\varphi^M(p_1, \ldots, p_n)$ interchangably.

\begin{definition}{(Absoluteness)}
Given a transitive class $M$, we say a formula $\varphi$ is \emph{absolute in $M$} if for all $p_1, \ldots, p_n \in M$
\begin{equation}
\varphi^M(p_1, \ldots, p_n) \iff \varphi(p_1, \ldots, p_n)
\end{equation}
\end{definition}

\begin{definition}{(Hierarchy of first-order formulas)}\\
\bce[(I)]
A first-order formula $\varphi$ is $\Delta_0$ iff it is logically equivalent to a first-order formula $\varphi'$ satisfying any of the following:
\bce[(i)]
\item $\varphi'$ contains no quantifiers
\item $y$ is a set, $\psi$ is a $\Delta_0$ formula, and $\varphi'$ is either $(\exists x \in y)\psi(y)$ or $(\forall x \in y)\psi(y)$.
\item $\psi_1, \psi_2$ are $\Delta_0$ formulas and $\varphi'$ is any of the following: $\psi_1 \lor \psi_2$, $\psi_1 \et \psi_2$, $\psi_1 \then \psi_2$, $\neg \psi_2$, 
\ece
\item If a formula is $\Delta_0$ it is also $\Sigma_0$ and $\Pi_0$
\item A formula $\varphi$ is $\Pi_n+1$ if it is logically equivalent to a formula $\varphi'$ such that $\varphi' = \forall x \psi$ where $\psi$ is a $\Sigma_n$-formula for any $n < \omega$.
\item A formula $\varphi$ is $\Sigma_n+1$ if it is logically equivalent to a formula $\varphi'$ such that $\varphi' = \forall x \psi$ where $\psi$ is a $\Pi_n$-formula for any $n < \omega$.
\ece
\end{definition}
Note that we can use the pairing function so that for $\forall p_1, \ldots, p_n \psi(p_1, \ldots, p_n)$, there a logically equivalent formula of the form $\forall x \psi'(x)$.

\begin{lemma}{($\Delta_0$ absoluteness)}\label{lemma:delta_0_abssoluteness}
Let $\varphi$ be a $\Delta_0$ formula, then $\varphi$ is absolute in any transitive class $M$.
\end{lemma}

\begin{proof}
This will be proven by induction over the complexity of a given $\Delta_0$ formula $\varphi$. Let $M$ be an arbitrary transitive class. Suppose, that 

Atomic formulas are always absolute by the definition of relativisation, see \ref{def:relativization}.
Suppose that $\Delta_0$ formulas $\psi_1$ and $\psi_2$ are absolute in $M$. Then from relativization, $(\psi_1 \et \psi_2)^M \iff \psi_1^M \et \psi_2^M$, which is, from the induction hypothesis, equivalent to $\psi_1 \et \psi_2$. The same holds for $\lor, \then, \neg$.

Suppose that a $\Delta_0$ formula $\psi$ is absolute in $M$. Let $y$ be a set and let $\varphi = (\exists x \in y) \psi(x)$. 
From relativization, $(\exists x \psi(x))^M \iff (\exists x \in M) \psi^M(x)$. Since the hypotheses makes it clear that $\psi^M \iff \psi$, we get $((\exists x \in y) \psi(x))^M \iff (\exists x \in y\cap M) \psi(x)$, which is the equivalent of $\varphi^M \iff \varphi$. The same applies to $\varphi = (\forall x \in y) \psi(x)$.
\end{proof}

% todo co Devlin -- p.27 -- downward a upward absoluteness?
\begin{lemma}{(Downward Absoluteness)}\label{lemma:downward_absoluteness}\\
Let $\varphi$ be a $\Pi_1$ formula and $M$ a transitive class. Then the following holds:
\begin{equation}
(\forall p_1, \ldots, p_n \in M)(\varphi(p_1, \ldots, p_n) \then \varphi(p_1, \ldots, p_n)^M
\end{equation}
\end{lemma}
\begin{proof}
Since $\varphi(p_1, \ldots, p_n)$ is $\Pi_1$, there is a $\Delta_0$ formula $\psi(p_1, \ldots, p_n, x)$ such that $\varphi = \forall x \psi(p_1, \ldots, p_n, x)$. From relativization and lemma \ref{lemma:delta_0_abssoluteness}, $\varphi^M(p_1, \ldots, p_n) \iff (\forall x \in M)\psi(p_1, \ldots, p_n, x)$.

Assume that for $p_1, \ldots, p_n \in M$ fixed, that $\forall x \psi(p_1, \ldots, p_n, x)$ holds, but $(\forall x \in M)\psi(p_1, \ldots, p_n, x)$ does not. 
Therefore $\exists x \neg \psi(p_1, \ldots, p_n, x)$, which contradicts $\forall x \psi(p_1, \ldots, p_n, x)$.
\end{proof}


\begin{lemma}{(Upward Absoluteness)}\label{lemma:upward_absoluteness}\\
Let $\varphi$ be a $\Sigma_1$ formula and $M$ a transitive class. Then the following holds:
\begin{equation}
(\forall p_1, \ldots, p_n \in M)(\varphi^M(p_1, \ldots, p_n) \then \varphi(p_1, \ldots, p_n)
\end{equation}
\end{lemma}
\begin{proof}
Since $\varphi(p_1, \ldots, p_n)$ is $\Sigma_1$, there is a $\Delta_0$ formula $\psi(p_1, \ldots, p_n, x)$ such that $\varphi = \exists x \psi(p_1, \ldots, p_n, x)$. From relativization and lemma \ref{lemma:delta_0_abssoluteness}, $\varphi^M(p_1, \ldots, p_n) \iff (\exists x \in M)\psi(p_1, \ldots, p_n, x)$.

Assume that for $p_1, \ldots, p_n \in M$ fixed, that $(\exists x \in M)\psi(p_1, \ldots, p_n, x)$ holds, but $\exists x \psi(p_1, \ldots, p_n, x)$ does not. This is an obvious contradiction.
\end{proof}


\subsubsection{More functions}


% \begin{definition}{(Function on a set)}
% We say function is 
% \end{definition}

\begin{definition}{(Strictly increasing function)}\label{def:increasing_function}\\
A function $f: Ord \then Ord$ is said to be \emph{strictly increasing} iff
\begin{equation}
\forall \alpha, \beta \in Ord (\alpha < \beta \then f(\alpha) < f(\beta)).
\end{equation}
\end{definition}

\begin{definition}{(Continuous function)}\label{def:continuous_function}\\
A function $f: Ord \then Ord$ is said to be \emph{continuous} iff
\begin{equation}
\alpha\mbox{ is limit} \then f(\lambda) = \bigcup_{\alpha < \lambda} f(\alpha).
\end{equation}
\end{definition}

\begin{definition}{(Normal Function)}\label{def:normal_function}\\
A function $f: Ord \then Ord$ is said to be \emph{normal} if it is \emph{strictly increasing} and \emph{continuous}.
\end{definition}

\begin{definition}{Fixed Point}\label{def:fixed_point}\\
We say $\alpha$ is a fixed point of ordinal function $f$ if $\alpha=f(\alpha)$.
\end{definition}

\begin{definition}{(Unbounded Class)}\label{def:unbounded_class}\\
We say a class $A$ is unbounded if 
\begin{equation}
\forall x (\exists y \in A) (x < y)
\end{equation}
\end{definition}

\begin{definition}{(Limit Point)}\label{def:limit_point}\\
Given a class $x \subseteq On$, we say that $\alpha \neq \emptyset$ is a limit point of $x$ iff 
\begin{equation}
\alpha = \bigcup(x \cap \alpha)
\end{equation}
\end{definition}

\begin{definition}{(Closed class)}\label{def:closed_class}\\
We say a class $A \subseteq Ord$ is closed iff it contains all of its limit points.
\end{definition}

\begin{definition}{(Club set)}\label{def:club_set}\\
For a regular uncountable cardinal $\kappa$, a set $x\ \subset\ \kappa$ is a \emph{closed unbounded} subset, abbreviated as a \emph{club set}, iff $x$ is both closed and unbounded in $\kappa$.
\end{definition}

\begin{definition}{(Stationary set)}\label{def:stationary_set}\\
For a regular uncountable cardinal $\kappa$, we say a set $A\ \subset\ \kappa$ is stationary in $\kappa$ iff it intersects every club subset of $\kappa$.
\end{definition}

\subsubsection{Structure, Substructure and Embedding}

Structures will be denoted $\langle M, \in, R \rangle$ where $M$ is a domain, $\in$ stands for the standard membership relation, it is assumed to be restricted to the domain\footnote{To be totally correct, we should write $\langle M, \in \cap M \times M, R \rangle$}, $R \subseteq M$ is a relation on the domain. When $R$ is not needed, we may as well only write $M$ instead of $\langle M, \in \rangle$.

\begin{definition}{(Elementary Embedding)}\label{def:elementary_embedding}\\
Given the structures $\langle M_1, \in, R \rangle$, $\langle M_2, \in, R \rangle$ and a one-to-one function $j: M_1 \then M_2$, we say $j$ is an \emph{elementary embedding} of $M_1$ into $M_2$, we write $j: M_1 \prec M_2$, when the following holds for every formula $\varphi(p_1, \ldots, p_n)$ and every $p_1, \ldots, p_n \in M_1$:
\begin{equation}
\langle M_1, \in, R \rangle \models \varphi(p_1, \ldots, p_n) \iff \langle M_2, \in, R \rangle  \models \varphi(j(p_1), \ldots, j(p_n))
\end{equation}
\end{definition}


\begin{definition}{(Elementary Substructure)}\label{def:elementary_substructure}\\
Given the structures $\langle M_1, \in, R \rangle$, $\langle M_2, \in, R \rangle$ and a one-to-one function $j: M_1 \then M_2$ such that $j: M_1 \prec M_2$, we say that $M_1$ is an \emph{elementary substructure} of $M_2$, denoted as $M_1 \prec M_2$, iff $j$ is an identity on $M_1$. In other words
\begin{equation}
\langle M_1, \in, R \rangle \models \varphi(p_1, \ldots, p_n) \iff \langle M_2, \in, R \rangle  \models \varphi(p_1, \ldots, p_n)
\end{equation}
\end{definition}

% While higher-order satisfaction relation for proper classes is unformalizable\footnote{TODO CITE KDE? Tarski nebo tak neco?},we can formalize satisfaction in a structure. For the rest of this chapter, let $D$ be a domain of such structure.
